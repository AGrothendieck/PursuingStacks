%% Anti-Copyright 2015 - the scrivener

\chapter{The Take-off (a letter to Daniel Quillen)}
\label{ch:I}
%\addcontentsline{toc}{chapter}{The Take-off (a letter to Daniel Quillen)}

\par\hfill Les Aumultes \ondate{19.2.}1983\namedpspage{L.1}{1}\par

Dear Daniel,

% 1
\hangsection{The importance of innocence.}\label{sec:1}%
Last year Ronnie Brown from Bangor sent me a heap of reprints and
preprints by him and a group of friends, on various foundational
matters of homotopical algebra. I did not really dig through any of
this, as I kind of lost contact with the technicalities of this kind
(I was never too familiar with the homotopy techniques anyhow, I
confess) -- but this reminded me of a few letters I had exchanged with
Larry Breen in 1975, where I had developed an outline of a program for
a kind of ``topological algebra'', viewed as a synthesis of
homotopical and homological algebra, with special emphasis on topoi --
most of the basic intuitions in this program arising from various
backgrounds in algebraic geometry. Some of those intuitions we
discussed, I believe, at IHES eight or nine years before, at a time
when you had just written up your nice ideas on axiomatic homotopical
algebra,\scrcomment{\textcite{Quillen1967}} published since in
Springer's Lecture Notes. I write you under the assumption that you
have not entirely lost interest for those foundational questions you
were looking at more than fifteen years ago. One thing which strikes
me, is that (as far as I know) there has not been any substantial
progress since -- it looks to me that an understanding of the basic
structures underlying homotopy theory, or even homological algebra
only, is still lacking -- probably because the few people who have a
wide enough background and perspective enabling them to feel the main
questions, are devoting their energies to things which seem more
directly rewarding. Maybe even a wind of disrepute for any
foundational matters whatever is blowing nowadays!  In this respect,
what seems to me even more striking than the lack of proper
foundations for homological and homotopical algebra, is the absence I
daresay of proper foundations for topology itself! I am thinking here
mainly of the development of a context of ``tame'' topology, which (I
am convinced) would have on the everyday technique of geometric
topology (I use this expression in contrast to the topology of use for
analysts) a comparable impact or even a greater one, than the
introduction of the point of view of schemes had on algebraic
geometry. The psychological drawback here I believe is not anything
like messiness, as for homological and homotopical algebra (as for
schemes), but merely the inrooted inertia which prevents us so
stubbornly from looking innocently, with fresh eyes, upon things,
without being dulled and imprisoned by standing habits of thought,
going with a familiar context -- \emph{too} familiar a context! The
task of working out the foundations of tame topology, and a
corresponding structure theory for ``stratified (tame) spaces'', seems
to me a lot more urgent and exciting still
than\namedpspage{L.1prime}{1'} any program of homological,
homotopical or topological algebra.

% 2
\hangsection{A short look into purgatory.}\label{sec:2}%
The motivation for this letter was the latter topic
however. Ronnie Brown and his friends are competent algebraists and
apparently strongly motivated for investing energy in foundational
work, on the other hand they visibly are lacking the necessary scope
of vision which geometry alone provides. They seem to me kind of
isolated, partly due I guess to the disrepute I mentioned before -- I
suggested to try and have contact with people such as yourself, Larry
Breen, Illusie and others, who have the geometric insight and who
moreover, may not think themselves too good for indulging in
occasional reflection on foundational matters and in the process help
others do the work which should be done.

At first sight it has seemed to me that the Bangor group had indeed
come to work out (quite independently) one basic intuition of the
program I had envisioned in those letters to Larry Breen -- namely
that the study of $n$-truncated homotopy types (of semisimplicial
sets, or of topological spaces) was essentially equivalent to the
study of so-called $n$-groupoids (where $n$ is any natural
integer). This is expected to be achieved by associating to any space
(say) $X$ its ``fundamental $n$-groupoid'' $\Pi_n(X)$, generalizing
the familiar Poincar\'e fundamental groupoid for $n=1$. The obvious
idea is that $0$-objects of $\Pi_n(X)$ should be the points of $X$,
$1$-objects should be ``homotopies'' or paths between points,
$2$-objects should be homotopies between $1$-objects, etc. This
$\Pi_n(X)$ should embody the $n$-truncated homotopy type of $X$, in
much the same way as for $n=1$ the usual fundamental groupoid embodies
the $1$-truncated homotopy type. For two spaces $X,Y$, the set of
homotopy-classes of maps $X\to Y$ (more correctly, for general $X,Y$,
the maps of $X$ into $Y$ in the homotopy category) should correspond
to $n$-equivalence classes of $n$-functors from $\Pi_n(X)$ to
$\Pi_n(Y)$ -- etc. There are very strong suggestions for a nice
formalism including a notion of geometric realization of an
$n$-groupoid, which should imply that any $n$-groupoid (or more
generally of an $n$-category) is relativized over an arbitrary topos
to the notion of an $n$-gerbe (or more generally, an $n$-stack), these
become the natural ``coefficients'' for a formalism of non-commutative
cohomological algebra, in the spirit of Giraud's thesis.

But all this kind of thing for the time being is pure heuristics -- I
never so far sat down to try to make explicit at least a definition of
$n$-categories and $n$-groupoids, of $n$-functors between these
etc. When I got the Bangor reprints I at once had the feeling that
this kind of work\namedpspage{L.2}{2} had been done and the homotopy category
expressed in terms of \oo-groupoids. But finally it appears this is
not so, they have been working throughout with a notion of
\oo-groupoid too restrictive for the purposes I had in mind (probably
because they insist I guess on strict associativity of compositions,
rather than associativity up to a (given) isomorphism, or rather,
homotopy) -- to the effect that the simply connected homotopy types
they obtain are merely products of Eilenberg-MacLane spaces, too bad!
They do not seem to have realized yet that this makes their set-up
wholly inadequate to a sweeping foundational set-up for homotopy. This
brings to the fore again to work out the suitable definitions for
$n$-groupoids -- if this is not done yet anywhere. I spent the
afternoon today trying to figure out a reasonable definition, to get a
feeling at least of where the difficulties are, if any. I am guided
mainly of course by the topological interpretation. It will be short
enough to say how far I got. The main part of the structure it seems
is expressed by the sets $F_i$ ($i\in\bN$) of $i$-objects, the source,
target and identity maps
\begin{align*}
  \cst s_1^i, \cst t_1^i &: F_i \to F_{i-1} \quad (i\ge 1) \\
  \cst k_1^i &: F_i \to F_{i+1} \quad (i\in\bN) \\
\intertext{and the symmetry map (passage to the inverse)}
  \inv_i &: F_i \to F_i \quad (i\ge 1),
\end{align*}
satisfying\marginpar{\emph{notation:}\\ $d_a=\cst k_1^i(a)$,\\ $\check u =
  \inv_i(u)$} some obvious relations: $\cst k_1^i$ is right inverse to
the source and target maps $\cst s_1^{i+1}, \cst t_1^{i+1}$, $\inv_i$
is an involution and ``exchanges'' source and target, and moreover for
$i\ge2$
\begin{align*}
  \cst s_1^{i-1} \cst s_1^i
  &= \cst s_1^{i-1} \cst t_1^i \;
    \Bigl(\eqdef \cst s_2^i : F_i \to F_{i-2} \Bigr) \\
  \cst t_1^{i-1} \cst s_1^i
  &= \cst t_1^{i-1} \cst t_1^i \;
    \Bigl(\eqdef \cst t_2^i : F_i \to F_{i-2} \Bigr) ;
\end{align*}
thus the composition of the source and target maps yields, for $0\le
j\le i$, just \emph{two} maps
\begin{equation*}
  \cst s_\ell^i, \cst t_\ell^i : F_i \to F_{i-\ell} = F_j \quad (\ell = i-j).
\end{equation*}
The next basic structure is the composition structure, where the
usual compoisition of arrows, more specifically of $i$-objects
($i\ge1$) $v \circ u$ (defined when $\cst t_1(u)=\cst s_1(v)$) must be
supplemented by the Godement-type operations $\mu * \lambda$ when
$\mu$ and $\lambda$ are ``arrows between arrows'', etc. Following this
line of thought, one gets the composition maps
\begin{equation*}
  (u,v) \mapsto (v *_\ell u) : (F_i, \cst s_\ell^i) \times_{F_{i-\ell}}
  (F_i,\cst s_\ell^i) \to F_i,
\end{equation*}
the composition of $i$-objects for $1\le \ell\le i$, being defined
when the $\ell$-target of $u$ is equal to the $\ell$-source of $v$,
and then we have
\begin{equation*}
  \left.\begin{aligned}
    \cst s_1^i(v *_\ell u) &=  \cst s_1^i(v) *_{\ell-1} \cst s_1^i(u) \\
    \cst t_1^i(v *_\ell u) &=  \cst t_1^i(v) *_{\ell-1} \cst s_1^i(u)
    \end{aligned}\right\}
  \quad \text{$\ell\ge2$ i.e. $\ell-1\ge1$}
\end{equation*}
and\namedpspage{L.2prime}{2'} for $\ell=1$
\begin{align*}
  \cst s_1(v *_1 u) &= s_1(u) \\
  \cst t_1(v *_1 u) &= t_1(v)
\end{align*}
(NB\enspace the operation $v *_1 u$ is just the usual composition $v \circ
u$).

One may be tempted to think that the preceding data exhaust the
structure of \oo-groupoids, and that they will have to be supplemented
only by a handful of suitable \emph{axioms}, one being
\emph{associativity} for the operation
$\overset{i}{\underset{\ell}{*}}$, which can be expressed essentially
by saying that that composition operation turns $F_i$ into the set of
arrows of a category having $F_{i-\ell}$ as a set of objects (with the
source and target maps $\cst s_\ell^i$ and $\cst t_\ell^i$, and with
identity map $\cst k_\ell^{i-\ell} : F_{i-\ell} \to F_i$ the
composition of the identity maps $F_{i-\ell} \to F_{i-\ell+1} \to
\dots \to F_{i-1} \to F_i$), and another being the Godement relation
\begin{equation*}
  (v' *_\alpha v) *_\nu (u' *_\alpha u) = (v' *_\nu u') *_\alpha (v
  *_\nu u)
\end{equation*}
(with the assumptions $1\le\alpha\le\nu$, and $u,u',v,v''in F_i$ and
\begin{equation*}
  \left\{\begin{aligned}
      \cst t_\alpha(u)=\cst s_\alpha(u') \\
      \cst t_\alpha(v)=\cst s_\alpha(v') \\
    \end{aligned}\right.\quad
  \cst t_\nu(u) = \cst s_\nu(v) = \cst s_\nu(v') = \cst t_\nu(u')
\end{equation*}
implying that both members are defined), plus the two relations
concerning the inversion of $i$-objects ($i\ge1$) $u \mapsto \check
u$,
\begin{multline*}
  u *_1 \check u = \id_{\cst t_1(u)} , \qquad \check u *_1 u =
  \id_{\cst s_1(u)} , \qquad
  (\check v *_\ell \check u) = \text{?} \quad(\ell\ge2)
\end{multline*}
It just occurs to me, by the way, that the previous description of
basic (or ``primary'') data for an \oo-groupoid is already incomplete
in some rather obvious respect, namely that the symmetry-operation
$\inv_i : u \mapsto \check u$ on $F_i$ must be complemented by $i-1$
similar involutions on $F_i$, which corresponds algebraically to the
intuition that when we have an $(i+1)$-arrow $\lambda$ say between two
$i$-arrows $u$ and $v$, then we must be able to deduce from it another
arrow from $\check u$ to $\check v$ (namely $u\mapsto\check u$ has a
``functorial character'' for variable $u$)? This seems a rather
anodine modification of the previous set-up, and is irrelevant for the
main point I want to make here, namely: that for the notion of
\oo-groupoids we are after, all the equalities just envisioned in this
paragraph (and those I guess which will ensure naturality by the
necessary extension of the basic involution on $F_i$) should be
replaced by ``homotopies'', namely by $(i+1)$-arrows between the two
members. These arrows should be viewed, I believe, as being part of
the data, they appear here as a kind of ``secondary'' structure. The
difficulty which appears now is to work out the natural coherence
properties\namedpspage{L.3}{3} concerning this secondary structure. The first thing I
could think of is the ``pentagon axiom'' for the associativity data,
which occurs when looking at associativities for the compositum (for
$\overset{i}{\underset{\ell}{*}}$ say) of four factors. Here again the
first reflex would be to write down, as usual, an \emph{equality} for
two compositions of associativity isomorphisms, exhibited in the
pentagon diagram. One suspects however that such equality should,
again, be replaced by a ``homotopy''-arrow, which now appears as a
kind of ``ternary'' structure -- before even having exhausted the list
of coherence ``relations'' one could think of with the respect to the
secondary structure! Here one seems caught at first sight in an
infinite chain of ever ``higher'', and presumably, messier structures,
where one is going to get hopelessly lost, unless one discovers some
simple guiding principle for shedding some clarity in the mess.

% 3
\hangsection{``Fundamental \texorpdfstring{\oo}{oo}-groupoids'' as
  objects of a ``model category''?}\label{sec:3}%
I thought of writing you mainly because I believe that,
if anybody, you should know if the kind of structure I am looking for
has been worked out -- maybe even \emph{you} did? In this respect, I
vaguely remember that you had a description of $n$-categories in terms
of $n$-semisimplicial sets, satisfying certain exactness conditions,
in much the same way as an ordinary category can be interpreted, via
its ``nerve'', as a particular type of semisimplicial set. But I have
no idea if your definition applied only for describing $n$-categories
with strict associativities, or not.

Still some contents in the spirit of your axiomatics of homotopical
algebra -- in order to make the question I am proposing more seducing
maybe to you! One comment is that presumably, the category of
\oo-groupoids (which is still to be defined) is a ``model category''
for the usual homotopy category; this would be at any rate one
plausible way to make explicit the intuition referred to before, that
a homotopy type is ``essentially the same'' as an \oo-groupoid up to
\oo-equivalence. The other comment: the construction of the
fundamental \oo-groupoid of a space, disregarding for the time being
the question of working out in full the pertinent structure on this
messy object, can be paraphrased in any model category in your sense,
and yields a functor from this category to the category of
\oo-groupoids, and hence (by geometric realization, or by
localization) also to the usual homotopy category. Was this functor
obvious beforehand? It is of a non-trivial nature only when the model
category is \emph{not} pointed -- as a matter of fact the whole
construction can be carried out canonically, in terms of a ``cylinder
object'' $I$ for the final object $e$ of the model category, playing
the role of the unit argument.\namedpspage{L.3prime}{3'}
It's high time to stop this letter -- please excuse me if it should
come ten or fifteen years too late, or maybe one year too early. If
you are not interested for the time being in such general nonsense,
maybe you know someone who is \ldots

\bigskip

Very cordially yours

\bigbreak

\presectionfill\ondate{20.2.}\namedpspage{L.4}{4}\par

I finally went on pondering about a definition of
\oo-groupoids, and it seems to me that, after all, the topological
motivation does furnish the ``simple guiding principle'' which
yesterday seemed to me to be still to be discovered, in order not to
get lost in the messiness of ever higher order structures. Let me try
to put it down roughly.

% 4
\hangsection{A bit of ordering in the mess of ``higher order
  structures''.}\label{sec:4}%
First I would like to correct somewhat the rather indiscriminate
description I gave yesterday of what I thought of viewing as
``primary'', secondary, ternary etc.\ structures for an
\oo-groupoid. More careful reflection conduces to view as the most
primitive, starting structure on the set of sets $F_i$ ($i\in\bN$), as
a skeleton on which progressively organs and flesh will be added, the
mere data of the source and target maps
\begin{equation*}
  \cst s_i, \cst t_i : F_i \rightrightarrows F_{i-1}
  \quad(i \ge 1),
\end{equation*}
which it will be convenient to supplement formally by corresponding
maps $\cst s_0, \cst t_0$ for $i=1$, from $F_0$ to
$F_{-1} \eqdef \text{one-element set}$. In a moment we will pass to a
universal situation, when the $F_i$ are replaced by the corresponding
``universal'' objects $\boldsymbol F_i$ in a suitable category stable
under finite products, where $F_{-1}$ will be the final element. For
several reasons, it is not proper to view the inversion maps
$\inv_i : F_i \to F_i$, and still less the other $i-1$ involutions on
$F_i$ which I at first overlooked, as being part of the primitive or
``skeletal'' structure. One main reason is that already for the most
usual $2$-groupoids, such as the $2$-groupoid whose $0$-objects are
ordinary ($1$-)groupoids, the $1$-objects being equivalences between
these (namely functors which are fully faithful and essentially
surjective), and the $2$-objects morphisms (or ``natural
transformations'') between such, there is \emph{not}, for an
$i$-object $f: C \to C'$, a natural choice of an ``inverse'' namely of
a quasi-inverse in the usual sense. And even assuming that such
quasi-inverse is chosen for every $f$, it is by no means clear that
such choice can be made involutive, namely such that
$(f\upvee)\upvee = f$ for every $f$ (and not merely $(f\upvee)\upvee$
isomorphic to $f$). The maps $\inv_i$ will appear rather, quite
naturally, as ``primary structure'', and they will not be involutions,
but ``pseudo-involutions'' (namely involutions ``up to homotopy''). It
turns out that among the various functors that we will construct, from
the category of topological spaces to the category of \oo-groupoids
(the construction depending on arbitrary choices and yielding a large
bunch of mutually non-isomorphic functors, which however are
``equivalent'' in a sense which will have to be made precise) -- there
are choices neater than others, and some of these will yield in the
primary structure maps $\inv_i$ which are actual involutions and
similarly for the other pseudo-involutions, appearing in succession as
higher order structure. The possibility of such neat and fairly
natural choices had somewhat misled me\namedpspage{L.4prime}{4'} yesterday.

What may look less convincing though at first sight, is my choice to
view as non-primitive even the ``degeneration maps'' $\cst k_1^i : F_i
\to F_{i+1}$, associating to every $i$-objects the $i+1$-object acting
as an identity on the former. In all cases I have met so far, these
maps are either given beforehand with the structure (of a $1$-category
or $2$-category say), or they can be uniquely deduced from the
axioms. In the present set-up however, they seem to me to appear more
naturally as ``primary'' (not as primitive) structure, much in the
same way as the $\inv_i$. Different choices for associating an
\oo-groupoid to a topological space, while yielding the same base-sets
$F_i$, will however (according to this point of view) give rise to
different maps $\cst k_1^i$. The main motivation for this point of
view comes from the fact that the mechanism for a uniform construction
of the chain of ever higher order structures makes a basic use of the
source and target maps only and of the ``transposes'' (see below), and
(it seems to me) not at all of the degeneration maps, which in this
respect rather are confusing the real picture, if viewed as
``primitive''.
The degeneration maps rather appear as typical cases of primary
structure, probably of special significance in the practical handling
of \oo-groupoids, but not at all in the conceptual machinery leading
up to the construction of the structure species of ``\oo-groupoids''.

Much in the same way, the composition operations
$\overset{i}{\underset{\ell}{*}}$ are viewed as primary, not as
primitive or skeletal structure. Their description for the fundamental
\oo-groupoid of a space -- for instance the description of composition
of paths -- depends on arbitrary choices, such as the choice of an
isomorphism (say) between $(I,1) \amalg_e (I,0)$ and $I$, where $I$ is
the unit interval, much in the same way as the notion of an inverse of
a path depends on the choice of an isomorphism of $I$ with itself,
exchanging the two end-points $0$ and $1$. The operations
$\overset{i}{\underset{2}{*}}$ of Godement take sense only once the
composition operations $*_1$ are defined -- they are ``secondary
structure'', and successively the operations $*_3, \ldots, *_i$ appear
as ternary etc.\ structure. This is correctly suggested by the
notations which I chose yesterday, where however I hastily threw all
the operations into a same pot baptized ``primary structure''!

% 5
\hangsection{Jumping over the abyss!}\label{sec:5}%
It is about time though to come to a tentative precise definition of
description of the process of stepwise introduction of an increasing
chain of higher order structure. This will be done by introducing a
canonical sequence of categories and functors
\[ C_0 \to C_1 \to C_2 \to \dots \to C_n \to C_{n+1} \to \cdots,\]
where $C_n$ denotes the category harbouring the ``universal''\namedpspage{L.5}{5}
partial structure of a would-be \oo-groupoid, endowed only with its
``structure of order $\le n$''. The idea is to give a direct inductive
construction of this sequence, by describing $C_0$, and the passage
from $C_n$ to $C_{n+1}$ $(n\ge0$), namely from an $n$-ary to
$(n+1)$-ary structure. As for the meaning of ``universal structure'',
once a given structure species is at hand, it depends on the type of
categories (described by the exactness properties one is assuming for
these) one wants to take as carriers for the considered structure, and
the type of exactness properties one assumes for the functor one
allows between these. The choice depends partly on the particular
species; if it is an algebraic structure which can be described say by
a handful of composition laws between a bunch of base sets (or
base-objects, when looking at ``realizations'' of the structure not
only in the category \Sets), one natural choice is to take categories
with finite products, and functors which commute to these. For more
sophisticated algebraic structures (including the structure of
category, groupoid or the like), which requires for the description of
data or axioms not only finite products, but also some fiber products,
one other familiar choice is to take categories with finite inverse
limits, and left exact functors. Still more sophisticated structures,
when the description of the structure in terms of base objects
requires not only some kind of inverse limits, but also more or less
arbitrary direct limits (such as the structure of a comodule over an
algebra, which requires consideration of tensor products over a ring
object\marginpar{or torsors under a group\ldots}), still more
stringent conditions will have to be imposed upon categories and
functors between these, for the structure to make a sense in these
categories, and the functors to transform a structure of this type in
one category into one of same type in another. In most examples I have
looked up, everything is OK taking categories which are topoi, and
functors between these which are inverse image functors for morphisms
of topoi, namely which are left exact and commute with arbitrary
direct limits. There is a general theorem for the existence of
universal structures, covering all these cases -- for instance there
is a ``classifying topos'' for most algebro-geometric structures,
whose cohomology say should be viewed as the ``classifying
cohomology'' of the structure species considered. In the case we are
interested in here, it is convenient however to work with the smallest
categories $C_n$ feasible -- which amounts to being as generous as
possible for the categories one is allowing as carriers for the
structure of an \oo-groupoid, and for the functors between these which
are expected to carry an \oo-groupoid into an \oo-groupoid. What we
will do is define ultimately a structure of an \oo-groupoid in a
category $C$, as a sequence of objects $F_i$ ($i\in\bN$), endowed with
some structure to be defined,\namedpspage{L.5prime}{5'} assuming merely that in $C$
finite products of the $F_i$ exist, plus certain finite inverse limits
built up with the $F_i$'s and the maps $\cst s_\ell^i$, $\cst
t_\ell^i$ between them (the iterated source and target maps). It
should be noted that the type of $\varprojlim$ we allow, which will
have to be made precise below, is fixed beforehand in terms of the
``skeletal'' or ``primitive'' structure alone, embodied by the family
of couples $(\cst s_i, \cst t_i)_{i\in\bN}$. This implies that the
categories $\bC_i$ can be viewed as having \emph{the same set of
  objects}, namely the objects $\boldsymbol F_i$ (written
in bold\scrcomment{was: \emph{underlined}} now to indicate their
universal nature, and including as was said before $\boldsymbol F_{-1} =
\text{final object}$), plus the finite products and iterated fiber
products of so-called ``standard'' type. While I am writing, it
appears to me even that the finite products here are of no use (so we
just drop them both in the condition on categories which are accepted
for harbouring \oo-groupoids, and in the set of objects of the
categories $C_i$). Finally, the common set of objects of the
categories $\bC_i$ is the set of ``standard'' iterated fiber
products of the $\boldsymbol F_i$, built up using only the primitive
structure embodied by the maps $\cst s_i$ and $\cst
t_i$ (which I renounce to underline!).
This at the same time gives, in principle, a precise definition of
$\bC_0$, at least up to equivalence -- it should not be hard anyhow
to give a wholly explicit description of $\bC_0$ as a small
category, having a countable set of objects, once the basic notion of
the standard iterated fiber-products has been explained.

Once $\bC_0$ is constructed, we will get the higher order
categories $\bC_1$ (primary), $\bC_2$ etc.\ by an inductive
process of \emph{successively adding arrows}. The category $\bC_\oo$
will then be defined as the direct limit of the categories
$\bC_n$, having the same objects therefore as $\bC_0$, with
\[\Hom_\oo(X, Y) = \varinjlim_n \Hom_n(X, Y)\]
for any two objects. This being done, giving a structure of
\oo-groupoid in any category $C$, will amount to giving a functor
\[\bC_\oo \to C\]
commuting with the standard iterated fiber-products. This can be
reexpressed, as amounting to the same as to give a sequence of objects
($F_i$) in $C$, and maps $s_\ell^i, t_\ell^i$ between these,
satisfying the two relations I wrote down yesterday (page \hyperref[p:L.2]2) (and
which of course have to be taken into account when defining $C_0$ to
start with, I forgot to say before), and such that
``standard''\footnote{I'll drop the qualification ``iterated''
  henceforth!} fiber-products defined in terms of these data should
exist in $C$, plus a bunch of maps between these fiber-products (in
fact, it will suffice to give such maps with target among the
$F_i$'s), satisfying certain relations embodied in the structure of
the category $\bC_\oo$. I am convinced that this bunch of maps
(namely the maps stemming from arrows in $\bC_\oo$) not only is
infinite, but cannot either be generated in the\namedpspage{L.6}{6} obvious
sense by a finite number, nor even by a finite number of infinite
series of maps such as $\cst k_\ell^i$, $\inv_i$,
$\overset{i}{\underset{\ell}{*}}$, the compatibility arrows in the
pentagon, and the like. More precisely still, I am convinced that none
of the functors $\bC_n \to \bC_{n+1}$ is an equivalence, which
amounts to saying that the structures of increasing order form a
strictly increasing sequence -- at every step, there is actual extra
structure added. This is perhaps evident beforehand to topologists in
the know, but I confess that for the time being it isn't to me, in
terms uniquely of the somewhat formal description I will make of the
passage of $\bC_n$ to $\bC_{n+1}$. This theoretically is all
that remains to be done, in order to achieve an explicit construction
of the structure species of an \oo-category (besides the definition of
standard fiber-products) -- without having to get involved, still less
lost, in the technical intricacies of ever messier diagrams to write
down, with increasing order of the structures to be added\ldots

% 6
\hangsection[The topological model: hemispheres building up the \dots]%
{The topological model: hemispheres building up the
  \texorpdfstring{\textup(}{(}tentative\texorpdfstring{\textup)}{)}
  ``universal \texorpdfstring{\oo-\textup(co\textup)}{oo-(co)}groupoid''.}%
\label{sec:6}%
In the outline of a method of construction for the structure species,
there has not been any explicit mention so far of the topological
motivation behind the whole approach, which could wrongly give the
impression of being a purely algebraic one. However, topological
considerations alone are giving me the clue both for the description
of the so-called standard fiber products, and of the inductive step
allowing to wind up from $\bC_n$ to $\bC_{n+1}$? The heuristics
indeed of the present approach is simple enough, and suggested by the
starting task, to define pertinent structure on the system of sets
$F_i(X)$ of ``homotopies'' of arbitrary order, associated to an
arbitrary topological space. In effect, the functors
\[ X \mapsto F_i(X)\]
are representable by spaces $D_i$, which are easily seen to be
$i$-disks. The source and target maps $s_\ell^i, t_\ell^i :
F_i(X) \rightrightarrows F_{i-1}(X)$ are transposed to maps, which I may
denote by the same letters,
\[s_\ell^i, t_\ell^i : D_{i-1} \rightrightarrows D_i.\]
Handling around a little, one easily convinces oneself that all the
main structural items on $F_*(X)$ which one is figuring out in
succession, such as the degeneracy maps $k_\ell^i$, the inversion maps
$\inv_i$, the composition $v\cdot u = v *_i u$ for $i$-objects, etc.,
are all transposed of similar maps which are defined between the cells
$D_i$, or which go from such cells to certain spaces, deduced from
these by gluing them together -- the most evident example in this
respect being the composition of paths, which is transposed of a map
from the unit segment $I$ into $(I,1)\amalg_e (I,0)$, having
preassigned values on the endpoints of $I$ (which correspond in fact
to the images of the two maps $s_1^1,t_1^1 : D_0
= \text{one point} \rightrightarrows D_1 =
I$).\namedpspage{L.6prime}{6'} In a more suggestive way, we could say from this
experiment that the family of discs $(D_i)_{i\in\bN}$, together with
the maps $s,t$ and a lot of extra structure which enters into the
picture step by step, is what we would like to call a
\emph{co-\oo-groupoid in the category} \cTop{} of topological spaces
(namely an \oo-groupoid in the dual category $\cTop\op$), and
that the structure of \oo-category on $F_*(X)$ we want to describe is
the transform of this co-structure into an \oo-groupoid, by the
contravariant functor from \cTop{} to \Sets{} defined by $X$. The
(iterated) amalgamated sums in \cTop{} which allow to glue together the
various $D_i$'s using the $s$ and $t$ maps between them, namely the
corresponding fibered products in $\cTop\op$, are indeed
transformed by the functor $h_X$ into fibered products of \Sets.
The suggestion is, moreover, that if we view our co-structure in \cTop{}
as a co-structure in the subcategory of Top, say $B_\oo$, whose
objects are the cells $D_i$ and the amalgamated sums built up with
these which step-wise enter into play, and whose arrows are all those
arrows which are introduced step-wise to define the co-structure, and
all compositions of these -- that this should be \emph{the universal
  structure} in the sense dual to the one we have been contemplating
before; or what amount to the same, that the corresponding
\oo-groupoid structure in the dual category $B_\oo\op$ is
``universal'' -- which means essentially that it is none other than
the universal structure in the category $\bC_\oo$ we are
after. Whether or not this expectation will turn out to be correct (I
believe it is\footnote{\alsoondate{25.2.} But no longer now and I do not really
  care! Cf.\ p.\ \hyperref[p:L.10]{10}.}), we should be aware that, while the successive
introduction of maps between the cells $D_i$ and their ``standard''
amalgamated sums (which we will define precisely below) depends at
every stage on arbitrary choices, the categories $\bC_n$ and their
limit $\bC_\oo$ do not depend on any of these choices; assuming the
expectation is correct, this means that up to (unique) isomorphism,
the category $B_\oo$ (and each of the categories $B_n$ of which it
appears as the direct limit) is independent of those choices -- the
isomorphism between two such categories transforming any one choice
made for the first, into the corresponding choice made for the
second. Also, while this expectation was of course the crucial
motivation leading to the explicit description of $\bC_0$ and of
the inductive step from $\bC_n$ to $\bC_{n+1}$, this description
seems to me a reasonable one and in any case it makes a formal sense,
quite independently of whether the expectation proves valid or not.

% 7
\hangsection{Gluing hemispheres: the ``standard amalgamations''.}%
\label{sec:7}%
Before pursuing, it is time to give a more complete description of the
primitive structure on $(D_i)$, as embodied by the maps $s,t$, which I
will now denote by\footnote{\emph{NB}\enspace it is more natural to consider
  $\varphi^+$ as ``target'' and $\varphi^-$ as ``source''.}
\[\varphi_i^+,\varphi_i^- :  D_i \rightrightarrows D_{i+1}. \]
It appears that these maps are injective, that their images make
up\namedpspage{L.7}{7} the boundary $S_i = \dot D_{i+1}$ of $D_{i+1}$, more
specifically these images are just two ``complementary'' hemispheres
in $S_i$, which I will denote by $S_i^+$ and $S_i^-$. The kernel of
the pair $(\varphi_i^+, \varphi_i^-)$ is just $S_{i-1} = \dot D_i$,
and the common restriction of the maps $\varphi_i^+, \varphi_i^-$ to
$S_{i-1}$ is an isomorphism
\[ S_{i-1} \simeq S_i^+ \cap S_i^-.\]
This $S_{i-1}$ in turn decomposes into the two hemispheres $S_{i-1}^+,
S_{i-1}^-$, images of $D_{i-1}$. Replacing $D_{i+1}$ by $D_i$, we see
that the $i$-cell $D_i$ is decomposed into a union of $2i+1$ closed
cells, one being $D_i$ itself, the others being canonically isomorphic
to the cells $S_j^+,S_j^-$ ($0\le j\le i$), images of $D_j\to D_n$ by
the iterated morphisms
\[\varphi_{n,j}^+,\varphi_{n,j}^- :  D_j \rightrightarrows D_n. \]
This is a cellular decomposition, corresponding to a partition of
$D_n$ into $2n+1$ open cells $D_n$, $S_j^+ = \varphi_{n,j}^+(D_j)$,
$S_j^- = \varphi_{n,j}^-(D_j)$. For any cell in this decomposition,
the incident cells are exactly those of strictly smaller dimension.

When introducing the operation $\overset{n}{\underset{\ell}{*}}$ with
$\ell=n-j$, it is seen that this corresponds to choosing a map
\[ D_n \to (D_n, S_j^+) \amalg_{D_j} (D_n, S_j^-),\]
satisfying a certain condition *, expressing the formulas I wrote down
yesterday for $s_1$ and $t_1$ of $u *_\ell v$ -- the formulas
translate into demanding that the restriction of the looked-for map of
$D_n$ to its boundary $S_{n-1}$ should be a given map (given at any
rate, for $\ell\ge2$, in terms of the operation $*_{\ell-1}$, which
explains the point I made that the $*_\ell$-structure is of order just
above the $*_{\ell-1}$-structure, namely (inductively) is of order
$\ell$ \ldots). That the extension of this map of $S_{n-1}$ to $D_n$
does indeed exist, comes from the fact that the amalgamated sum on the
right hand side is contractible for obvious reasons.

This gives a clue of what we should call ``standard'' amalgamated sums
of the cells $D_i$. The first idea that comes to mind is that we
should insist that the space considered should be contractible,
excluding amalgamated sums therefore such as
\[
\begin{tikzcd}[row sep=tiny]
  & \bullet \ar[dr] \\ \bullet \ar[ur] \ar[rr] & & \bullet
\end{tikzcd} \quad\text{or}\quad
\begin{tikzcd}
  \bullet \ar[r, bend left] \ar[r, bend right] & \bullet
\end{tikzcd}\]
which are circles. This formulation however has the inconvenience of
not being directly expressed in combinatorial terms. The following
formulation, which has the advantage of being of combinatorial nature,
is presumably equivalent to the former, and gives (I expect) a large
enough notion of ``standardness'' to yield for the corresponding
notion of \oo-category enough structure for whatever one will ever
need. In any case, it is\namedpspage{L.7prime}{7'} understood that the ``amalgamated
sum'' (rather, finite $\varinjlim$) we are considering are of the most
common type, when $X$ is the finite union of closed subsets $X_i$,
with given isomorphisms
\[ X \simeq D_{n(i)},\]
the intersection of any two of these $X_i\cap X_j$ being a union of
closed cells both in $D_{n(i)}$ and in $D_{n(j)}$. (This implies in
fact that it is either a closed cell in both, or the union of two
closed cells of same dimension $m$ and hence isomorphic to $S_m$, a
case which will be ruled out anyhow by the triviality condition which
follows.\footnote{False, [picture of disk divided in three]}) The
triviality or ``standardness'' condition is now expressed by demanding
that the set of indices $I$ can be totally ordered, i.e., numbered in
such a way that we get $X$ by successively ``attaching'' cells
$D_{n(i)}$ to the space already constructed, $X(i-1)$, by a map from a
sub-cell of $D_{n(i)}$, $S_j^\xi \to X(i-1)$ ($\xi\in\{\pm1\}$), this
map of course inducing an isomorphism, more precisely \emph{the}
standard isomorphism, $\varphi_{n(i)}^\xi : S_j^\xi \simeq D_j$ with
$S_j^\xi$ one of the two corresponding cells $S_j^+, S_j^-$ in some
$X_{i'}\simeq D_{n(i')}$. The dual translation of this, in terms of
fiber products in a category $C$ endowed with objects $F_i$
($i\in\bN$) and maps $s_1^i,t_1^i$ between these, is clear: for a
given set of indices $I$ and map $i\mapsto n(i) : I \to \bN$, we
consider a subobject of $\prod_{i\in I} D_{n(i)}$, which can be
described by equality relations between iterated sources and targets
of various components of $u=(u_i)_{i\in I}$ in $P$, the structure of
the set of relations being such that $I$ can be numbers, from $1$ to
$N$ say, in such a way that we get in succession $N-1$ relations on
the $N$ components $u_i$ respectively ($2\le i\le N$), every relation
being of the type $f(u_i) = g(u_i')$, with $f$ and $g$ being iterated
source of target maps, and $i'<i$. (Whether source or target depending
in obvious way on the two signs $\xi,\xi'$.)

\bigbreak

\presectionfill\ondate{21.2.}\namedpspage{L.8}{8}\par

% 8
\hangsection{Description of the universal \emph{primitive} structure.}%
\label{sec:8}%
Returning to the amalgamated sum $X = \bigcup_i X_i$, the cellular
decompositions of the components $X_i \simeq D_{n(i)}$ define a
cellular decomposition of $X$, whose set of cells with incidence
relation forms a finite ordered set $K$, finite union of a family of
subsets $(K_i)_{i\in I}$, with given isomorphisms
\[ f_i : K_i \simeq \cst J_{n(i)} \quad (i\in I),\]
where for every index $n\in\bN$, $\cst J_n$ denotes the ordered set of
the $2n+1$ cells $S_j^\xi$ ($0\le j\le n-1$, $\xi\in\{\pm1\}$), $D_n$
of the pertinent cellular decomposition of $D_n$. We may without loss
of generality assume there is no inclusion relation between the $K_i$,
moreover the standardness condition described above readily translates
into a condition on this structure $K$, $(K_i)_{i\in I}$, $(f_i)_{i\in
  I}$, and implies that for $i,i'\in I$, $K_i\to K_{i'}$, is a
``closed'' subset in the two ordered sets $K_i,K_{i'}$ (namely
contains with any element $x$ the elements smaller than $x$), and
moreover isomorphic (for the induced order) to some $\cst J_n$. Thus
the category $B_0$ can be viewed as the category of such ``standard
ordered sets'' (with the extra structure on these just said), and the
category $C_0$ can be defined most simply as the dual category
$B_0\op$. (NB\enspace the definition of morphisms in $B_0$ is clear I guess
\ldots) I believe the category $B_0$ is stable under amalgamated sums
$X \amalg_Z Y$, provided however we insist that the empty structure
$K$ is \emph{not} allowed -- otherwise we have to restrict to
amalgamated sums with $Z \ne \emptyset$. It seems finally more
convenient to exclude the empty structure in $B_0$, i.e.\ to exclude
the final element from $\bC_0$, for the benefit of being able to
state that $\bC_0$ (and all categories $\bC_n$\footnote{this is
  obviously false, see P.S.\ p.12.}) are stable under amalgamated sums,
and that the functors $\bC_n\to C$ (and ultimately $\bC_\oo\to
C$) we are interested in are just those commuting to arbitrary binary
amalgamated sums (without awkward reference to the objects $\boldsymbol F_i$
and the iterated source and target maps between them\ldots).

% 9
\hangsection[The main inductive step: just add coherence arrows! The \dots]%
{The main inductive step: just add coherence arrows! The abridged
  story of an \texorpdfstring{\textup(}{(}inescapable and
  irrelevant\texorpdfstring{\textup)}{)} ambiguity}\label{sec:9}%
The category $\bC_0$ being fairly well understood, it remains to
complete the construction by the inductive step, passing from $\bC_n$
to $\bC_{n+1}$. The main properties I have in mind therefore, for the
sequence of categories $\bC_n$ and their limit $\bC_\oo$, are the
following two.
\begin{enumerate}[label=(\Alph*)]
\item\label{it:8.A} For any $K\in\Ob(\bC_\oo)$ ($=\Ob(\bC_0)$), and any two
  arrows in $\bC_\oo$
  \[ f,g : K \rightrightarrows F_i,\]
  with $i\in\bN$, and such that either $i=0$, or the equalities
  \begin{equation}\label{eq:8.1}
    s_1^i\, f = s_1^i\, g , \quad
    t_1^i\, f = t_1^i\, g\tag{1}
  \end{equation}
  hold (case $i\ge1$), there exists $h : K \to F_{i+1}$ such that
  \begin{equation}\label{eq:8.2}
    s_1^{i+1}\, h = f, \quad
    t_1^{i+1}\, h = g.\tag{2}
  \end{equation}
\item\label{it:8.B}
  For any $n\in\bN$, the category $\bC_{n+1}$ is deduced from $\bC_n$
  by keeping the same objects, and just adding new arrows $h$ as
  in \ref{it:8.A}, with $f,g$ arrows in $\bC_n$.
\end{enumerate}
The\namedpspage{L.8prime}{8'} expression ``deduces from'' in \ref{it:8.B} means that we are
adding arrows $h: K\to F_i$ (each with preassigned source and target
in $\bC_n$), with as ``new axioms'' on the bunch of these uniquely
the two relations \eqref{eq:8.2} of \ref{it:8.A}, the category $\bC_{n+1}$
being deduced from $\bC_n$ in an obvious way, as the solution of a
universal problem within the category of all categories where binary
amalgamated products exist, and ``maps'' between these being functors
which commute to those fibered products. In practical terms, the
arrows of $\bC_{n+1}$ are those deduced from the arrows in $\bC_n$ and
the ``new'' arrows $h$, by combining formal operations of
composing arrows by $v \circ u$, and taking (binary) amalgamated
products of arrows.\footnote{This has to be corrected -- amalgamated
  sums exist in $\bC$, only -- and \emph{those} should be
  respected.}

NB\enspace Of course the condition \eqref{eq:8.1} in \ref{it:8.A} is necessary for the
existence of an $h$ satisfying \eqref{eq:8.2}. That it is sufficient
too can be viewed as an extremely strong, ``universal'' version of
coherence conditions, concerning the various structures introduced on
an \oo-groupoid. Intuitively, it means that whenever we have two ways
of associating to a finite family $(u_i)_{i\in I}$ of objects of an
\oo-groupoid, $u_i\in F_{n(i)}$, subjected to a standard set of
relations on the $u_i$'s, an element of some $F_n$, in terms of the
\oo-groupoid structure only, then we have automatically a ``homotopy''
between these built in in the very structure of the \oo-groupoid,
provided it makes at all sense to ask for one (namely provided
condition \eqref{eq:8.1} holds if $n \ge 1$). I have the feeling
moreover that conditions \ref{it:8.A} and \ref{it:8.B} (plus the
relation $\bC_\oo = \varinjlim \bC_n$) is all what will be ever
needed, when using the definition of the structure species, -- plus of
course the description of $\bC_0$, and the implicit fact that the
categories $\bC_n$ are stable under binary fiber products and the
inclusion functors commute to these.\footnote{Inaccurate; see above}
Of course, the category which really interests us is $\bC_\oo$, the
description of the intermediate $\bC_n$'s is merely technical --
the main point is that there should exist an increasing sequence
$(\bC_n)$ of subcategories of $\bC_\oo$, having the same objects (and
the ``same'' fiber-products), such that $\bC_\oo$ should be the
limit (i.e., every arrow in $\bC_\oo$ should belong to some $\bC_n$),
and such that the passage from $\bC_n$ to $\bC_{n+1}$
should satisfy \ref{it:8.B}. It is fairly obvious that these
conditions alone do by no means characterize $\bC_\oo$ up to
equivalence, and still less the sequence of its subcategories $\bC_n$.
The point I wish to make though, before pursuing with a proposal
of an explicit description, is that \emph{this ambiguity is in the
  nature of things}. Roughly saying, two different mathematicians,
working independently on the conceptual problem I had in mind, assming
they both wind up with some explicit definition, will almost certainly
get non-equivalent definitions -- namely with non-equivalent
categories of (set-valued, say) \oo-groupoids! And, secondly and as
importantly,\namedpspage{L.9}{9} that \emph{this ambiguity however is an
  irrelevant one}. To make this point a little clearer, I could say
that a third mathematician, informed of the work of both, will readily
think out a functor or rather a pair of functors, associating to any
structure of Mr.\ X one of Mr.\ Y and conversely, in such a way that
by composition of the two, we will associate to an $X$-structure ($T$
say) another $T'$, which will not be isomorphic to $T$ of course, but
endowed with a canonical \oo-equivalence (in the sense of Mr.\ X) $T
\underset{\oo}{\simeq} T'$, and the same on the Mr.\ Y side. Most
probably, a fourth mathematician, faced with the same situation as the
third, will get his own pair of functors to reconcile Mr.\ X and Mr.\
Y, which very probably won't be equivalent (I mean isomorphic) to the
previous one. Here however, a fifth mathematician, informed about this
new perplexity, will probably show that the two $Y$-structures $U$ and
$U'$, associated by his two colleagues to an $X$-structure $T$, while
not isomorphic alas, admit however a canonical \oo-equivalence between
$U$, and $U'$ (in the sense of the $Y$-theory). I could go on with a
sixth mathematician, confronted with the same perplexity as the
previous one, who winds up with another \oo-equivalence between $U$
and $U'$ (without being informed of the work of the fifth), and a
seventh reconciling them by discovering an \oo-equivalence between
these equivalences. The story of course is infinite, I better stop
with seven mathematicians, a fair number nowadays to allow themselves
getting involved with foundational matters \ldots
There should be a mathematical statement though resuming in finite
terms this infinite story, but in order to write it down I guess a
minimum amount of conceptual work, in the context of a given notion of
\oo-groupoids satisfying the desiderata \ref{it:8.A} and \ref{it:8.B}
should be done, and I am by no means sure I will go through this, not
in this letter anyhow.

% 10
\hangsection{Cutting down redundancies -- or: ``l'embarras du
  choix''.}\label{sec:10}%
Now in the long last the explicit description I promised of $\bC_{n+1}$
in terms of $\bC_n$. As a matter of fact, I have a handful
to propose! One choice, about the widest I would think of, is: for
every pair $(f,g)$ in $\bC_n$ satisfying condition \eqref{eq:8.1}
of \ref{it:8.A}, add one new arrow $h$. To avoid set-theoretic
difficulties though, we better first modify the definition of $\bC_0$
so that the set of its objects should be in the universe we are
working in, preferably even it be countable. Or else, and more
reasonably, we will pick one $h$ for every isomorphism class of
situations $(f,g)$ in $\bC_n$. Another restriction to avoid too
much redundancy -- this was the first definition actually that flipped
to my mind the day before yesterday -- is to add a \emph{new} $h$ only
when there is no ``old'' one, namely in $\bC_n$, serving the same
purpose. Then it came to my mind that there is a lot of redundancy
still, thus there would be already\namedpspage{L.9prime}{9'} infinitely many
operations standing for the single operation $v \overset{i}{\circ} u$
say, which could be viewed in effect in terms of an arbitrary
$n$-sequence ($n\ge2$) of ``composable'' $i$-objects $u_1=u, u_2=v,
u_3, \dots, u_n$. The natural way to meet this ``objection'' would be
to restrict to pairs $(f,g)$ which cannot be factored non-trivially
through another objects $K'$ as
\[\begin{tikzcd}
  K \ar[r] & K' \ar[r, shift left, "f'"] \ar[r, shift right, swap,
  "g'"] & F_i \end{tikzcd}.\]
But even with such restrictions, there remain a lot of redundancies --
and this again seems to me in the nature of things, namely that there
is no really natural, ``most economic'' way for achieving condition
\ref{it:8.A}, by a stepwise construction meeting condition
\ref{it:8.B}. For instance, in $\bC_1$ already we will have not
merely the compositions $v \overset{i}{\circ} u$, but at the same time
simultaneous compositions
\begin{equation}
  \label{eq:10.star}
  u_n \circ u_{n-1} \circ \dots \circ u_1 \tag{*}
\end{equation}
for ``composable'' sequences of $i$-objects ($i\ge1$), without
reducing this (as is customary) to iteration of the binary composition
$v \circ u$.
Of course using the binary composition, and more generally iteration
of $n'-ary$ compositions with $n'<n$ (when $n\ge3)$, we get an
impressive bunch of operations in the $n$ variables $u_1, \dots, u_n$,
serving the same purpose as \eqref{eq:10.star}. All these will be tied
up by homotopies in the next step $\bC_2$. We would like to think
of this set of homotopies in $\bC_2$ as a kind of ``transitive
system of isomorphisms'' (of associativity), now the transitivity
relations one is looking for will be replaced by homotopies again
between compositions of homotopies, which will enter in the picture
with $\bC_3$, etc. Here the infinite story is exemplified by the
more familiar situation of the two ways in which one could define a
``$\otimes$-composition with associativity'' in a category, starting
either in terms of a binary operation, or with a bunch of $n$-ary
operations -- with, in both cases the associativity isomorphisms being
an essential part of the structure. Here again, while it is generally
(and quite validly) felt that the two points of view are equivalent;
and both have their advantages and their drawbacks, still it is not
true, I believe, that the two categories of algebraic structures
``category with associative $\otimes$-operation'', using one or the
other definition, are equivalent.\footnote{even not in the compoid???
  context.} Here the story though of the relation between the two
notions is a finite one, due to the fact that it is related to the
notion of $2$-categories or $2$-groupoids, instead of \oo-groupoids as
before\ldots

Thus I don't feel really like spending much energy in cutting down
redundancies, but prefer working with a notion of \oo-groupoid which
remains partly indeterminate, the main features being embodied in the
conditions \ref{it:8.A} and \ref{it:8.B} and in the description of
$\bC_0$, without other specification.\namedpspage{L.10}{10}

% 11
\hangsection[Returning to the topological model (the canonical
functor \dots]%
{Returning to the topological model \texorpdfstring{\textup(}{(}the
  canonical functor from spaces to
  ``\texorpdfstring{\oo}{oo}-groupoids''\texorpdfstring{\textup)}{)}.}%
\label{sec:11}%
One convenient way for constructing a category $\bC_\oo$ would be
to define for every $K,L\in\Ob(\bC_0) = \Ob(B_0)$ the set
$\Hom_\oo(K,L)$ as a subset of the set $\Hom(\abs L,\abs K)$ of
continuous maps between the geometric realizations of $L$ and $K$ in
terms of gluing together cells $D_i$, the composition of arrows in
$\bC_\oo$ being just composition of maps. This amounts to defining
$\bC_\oo$ as the dual of a category $\bB_\oo$ of topological
descriptions. It will be sufficient to define for every cell $D_n$ and
every subset $\Hom_\oo(D_n,\abs K)$ of $\Hom(D_n,\abs K)$, satisfying
the two conditions:
\begin{enumerate}[label=(\alph*)]
\item\label{it:11.a}
  stability by compositions $D_n \to \abs K \to \abs{K'}$, where $K\to
  K'$ is an ``allowable'' continuous map, namely subjected only to the
  condition that its restriction to any standard subcell $D_{n'}
  \subset \abs K$ is again ``allowable'', i.e., in $\Hom_\oo$.
\item\label{it:11.b}
  Any ``allowable'' map $S_n \to \abs K$ (i.e., whose restrictions to
  $S_n^+$ and $S_n^-$ are allowable) extends to an allowable map
  $D_{n+1}\to\abs K$.
\end{enumerate}

Condition \ref{it:11.a} merely ensures stability of allowable maps
under composition, and the fact that $\bB_\oo$ (endowed with the
allowable maps as morphisms) has the correct binary amalgamated sums,
whereas \ref{it:11.b} expresses condition \ref{it:11.a} on $\bC_\oo$.
These conditions are satisfied when we take as $\Hom_\oo$
subsets defined by tameness conditions (such as piecewise linear for
suitable piecewise linear structure on the $D_n$'s, or differentiable,
etc.). The condition \ref{it:11.b} however is of a subtler nature in
the topological interpretation and surely not met by such sweeping
tameness requirements only! Finally, the question as to whether we can
actually in this way describe an ``acceptable'' category $\bC_\oo$,
by defining sets $\Hom_\oo$, namely describing $\bC_\oo$ in terms
of $\bB_\oo$, seems rather subsidiary after all. We may think of
course of constructing stepwise $\bB_\oo$ via subcategories $\bB_n$,
by adding stepwise new arrows in order to meet condition
\ref{it:11.b}, thus paraphrasing condition \ref{it:8.B} for passage
from $\bC_n$ to $\bC_{n+1}$ -- but it is by no means clear that
when passing to the category $\bB_{n+1}$ by composing maps of $\bB_n$
and ``new'' ones, and using amalgamated sums too, there might not
be some undesirable extra relations in $\bB_{n+1}$, coming from the
topological interpretation of the arrows in $\bC_{n+1}$ as maps. To
say it differently, universal algebra furnishes us readily with an
acceptable sequence of categories $\bC_n$ and hence $\bC_\oo$,
and by the universal properties of the $\bC_n$ in terms of
$\bC_{n+1}$, we readily get (using arbitrary choices) a contravariant
functor $K \mapsto \abs K$ from $\bC_\oo$ to the category of
topological spaces (i.e., a co-\oo-groupoid in \cTop), but it is by no
means clear that this functor is faithful -- and it doesn't really
matter after all!\namedpspage{L.10prime}{10'}

% 12
\hangsection{About replacing spaces by objects of a ``model
  category''.}\label{sec:12}%
I think I really better stop now, except for one last comment. The
construction of a co-\oo-groupoid in \cTop, giving rise to the
fundamental functor
\[ \cTop \longrightarrow (\text{\oo-groupoids}),\]
generalizes, as I already alluded to earlier, to the case when \cTop{}
is replaced by an arbitrary ``model category'' $M$ in your sense. Here
however the choices occur not only stepwise for the primary,
secondary, ternary etc.\ structures, but already for the primitive
structures, namely by choice of objects $D_i$ ($i\in\bN$) in $M$, and
source and target maps $D_i \rightrightarrows D_{i+1}$. These choices
can be made inductively, by choosing first for $D_0$ the final object,
or more generally any object which is fibrant and trivial (over the
final objects), $D_{-1}$ being the initial object, and defining
further $S_0 = D_0 \amalg_{D_{-1}} D_0 = D_0 \amalg D_0$ with obvious
maps $\psi_0^+,\psi_0^- : D_0 \to S_0$, and then, if everything is
constructed up to $D_n$ and $S_n = (D_n,\varphi_{n-1}^+)
\amalg_{D_{n-1}} (D_n,\varphi_{n-1}^-)$, defining $D_{n+1}$ as any
fibrant and trivial object together with a cofibrant map
\[ S_n \to D_{n+1}, \]
and $\varphi_n^+, \varphi_n^-$ as the compositions of the latter with
$\psi_n^+,\psi_n^-: D_n \rightrightarrows S_n$.
Using this and amalgamated sums in $M$, we get our functor
\[ \bB_0 = \bC_0\op \to M , \quad K\mapsto \abs K_M,\]
commuting with amalgamated sums, which we can extend stepwise through
the $\bC_n\op$'s to a functor $\bB_\oo = \bC_\oo\op \to M$,
provided we know that the objects $\abs K_M$ of $M$ ($K\in\Ob\bC_0$)
obtained by ``standard'' gluing of the $D_n$'s in $M$, are again
fibrant and trivial -- and I hope indeed that your axioms imply that,
via, say, that if $Z \to X$ and $Z\to Y$ are cofibrant and $X,Y,Z$ are
fibrant and trivial, then $X \amalg_Z Y$ is fibrant and trivial\ldots

Among the things to be checked is of course that when we localize the
category of \oo-groupoids with respect to morphisms which are ``weak
equivalences'' in a rather obvious sense (NB\enspace the definition of the
$\Pi_i$'s of an \oo-groupoid is practically trivial!), we get a
category equivalent to the usual homotopy category \Hot. Thus we get a
composed functor
\[ M \to (\text{\oo-groupoids}) \to \Hot,\]
as announced. I have some intuitive feeling of what this functor
stands for, at least when $M$ is say the category of semisimplicial
sheaves, or (more or less equivalently) of $n$-gerbes or \oo-gerbes on
a given topos: namely it should correspond to the operation of
``integration'' or ``sections'' for $n$-gerbes (more generally for
$n$-stacks) over a topos -- which is indeed \emph{the} basic operation
(embodying non-commutative cohomology objects of the topos) in
``non-commutative homological algebra''.

I guess that's about it for today. It's getting late and time to go to
bed! Good night.

\bigbreak

% 13
\presectionfill\ondate{22.2.}1983\namedpspage{L.11}{11}\par

\hangsection[An urgent reflection on proper names: ``Stacks''
  and \dots]%
  {An urgent reflection on proper names: ``Stacks'' and
  ``coherators''.}\label{sec:13}%
It seems I can't help pursuing further the reflection I started with
this letter! First I would like to come back upon terminology. Maybe
to give the name of $n$-groupoids and \oo-groupoids to the objects I
was after is not proper, for two reasons: a) it conflicts with a
standing terminology, applying to structure species which are
frequently met and deserve names of their own, even if they turn out
to be too restrictive kind of objects for the use I am having in mind
-- so why not keep the terminology already in use, especially for
two-groupoids, which is pretty well suited after all; b) the structure
species I have in mind is not really a very well determined one, it
depends as we saw on choices, without any one choice looking
convincingly better than the others -- so it would be a mess to give
an unqualified name to such structure, depending on the choice of a
certain category $\bC_\oo = \bC$. I have been thinking of the
terminology $n$-stack and \oo-stack (stack = ``champ'' in French), a
name introduced in Giraud's book (he was restricting to champs =
1-champs), which over a topos reduced to a one-point space reduces in
his case to the usual notion of a category, i.e., $1$-category. Here
of course we are thinking of ``stacks of groupoids'' rather than
arbitrary stacks, which I would like to call (for arbitrary order
$n\in\bN$ or $n=\oo$) $n$-Gr-stack -- suggesting evident ties with the
notion of Gr-categories, we should say Gr-$1$-categories, of Mme Hoang
Xuan Sinh. One advantage of the name ``stack'' is that the use it had
so far spontaneously suggests the extension of these notions to the
corresponding notions over an arbitrary topos, which of course is what
I am after ultimately. Of course, when an ambiguity is possible, we
should speak of $n$-$\bC$-stacks -- the reference to $\bC$
should make superfluous the ``Gr'' specification. Thus $n$-$\bC$-stacks
are essentially the same as $n$-$\bC$-stacks over the
final topos, i.e., over a one point space. When both $\bC$ and
``Gr'' are understood in a given context, we will use the terminology
$n$-stack simply, or even ``stack'' when $n$ is fixed throughout. Thus
it will occur that in certain contexts ``stack'' will just mean a
usual groupoid, in others it will mean just a category, but when $n=2$
it will not mean a usual $2$-groupoid, but something more general,
defined in terms of $\bC$.

The categories $\bC=\bC_\oo$ described before merit a name too
-- I would like to call them ``coherators'' (``coh\'ereurs'' in
French). This name is meant to suggest that $\bC$ embodies a
hierarchy of coherence relations, more accurately of coherence
``homotopies''. When dealing with stacks, the term
\emph{$i$-homotopies} (rather than $i$-objects or $i$-arrows) for the
elements of the $i$\textsuperscript{th} component $F_i$ of a stack
seems to me the most suggestive -- they will of course be denoted
graphically by arrows,\namedpspage{L.11prime}{11'} such as $h: f\to g$ in the
formulation of \ref{it:8.A} yesterday. More specifically, I will call
coherator any category \emph{equivalent} to a category $\bC_\oo$ as
constructed before. Thus a coherator is stable under binary fiber
products,\footnote{false, see PS} moreover the $F_i$ are recovered up
to isomorphism as the indecomposable elements of $\bC$ with respect
to amalgamation. However, in a category $\bC_\oo$, the objects
$\boldsymbol F_i$ have non-trivial automorphisms -- namely the ``duality
involutions'' and their compositions (the group of automorphisms of
$\boldsymbol F_i$ should turn out to be canonically isomorphic to
$(\pm1)^i$),\footnote{false, see below} in other words by the mere
category structure of a coherator we will not be able to recover the
objects $\boldsymbol F_i$ in $\bC$ up to unique isomorphism. Therefore, in
the structure of a coherator should be included, too, the choice of
the basic indecomposable objects $\boldsymbol F_i$ (one in each isomorphism
class), and moreover the arrows $\cst s_1^i,\cst t_1^i : \boldsymbol F_i
\rightrightarrows \boldsymbol F_{i-1}$ for $i\ge1$ (a priori, only the
\emph{pair} $(\cst s_1^i,\cst t_1^i)$ can be described intrinsically
in terms of the category structure of $\bC$, once $\boldsymbol F_i$ and
$\boldsymbol F_{i-1}$ are chosen\ldots). But it now occurs to me that we
don't have to put in this extra structure after all -- while the
$\boldsymbol F_i$'s separately do have automorphisms, the system of objects
$(\boldsymbol F_i)_{i\in\bN}$ and of the maps $(\cst s_1^i)_{i\ge1}$ and $(\cst
t_1^i)_{i\ge1}$ has only the trivial automorphism (all this of course
is heuristics, I didn't really prove anything -- but the structure of
the full subcategory of a $\bC_\oo$ formed by the objects $\boldsymbol F_i$ seems
pretty obvious\ldots). To finish getting convinced that the
mere category structure of a coherator includes already all other
relevant structure, we should still describe a suitable intrinsic
filtration by subcategories $\bC_n$. We define the $\bC_n$
inductively, $\bC_0$ being the ``primitive structure'' (the arrows
are those deducible from the source and target arrows by composition
and fiber products), and $\bC_{n+1}$ being defined in terms of
$\bC_n$ as follows: add to $\Fl \bC_n$ all arrows of $\bC$ of
the type $h: K \to \boldsymbol F_i$ ($i\ge1$) such that $\cst s_1^i\, h$ and
$\cst t_1^i\,h$ are in $\bC_n$, and the arrows deduced from the
bunch obtained by composition and fibered products.\footnote{This may
  give however \emph{too large} a category $\bC_{n+1}$.} In view of
these constructions, it would be an easy exercise to give an intrinsic
characterization of a coherator, as a category satisfying certain
internal properties.

I was a little rash right now when making assertions about the
structure of the group of automorphisms of $\boldsymbol F_i$ -- I forgot that
two days ago I pointed out to myself that even the basic operation
$\inv_i$ upon $\boldsymbol F_i$ need not even be involutions!\footnote{Nor
  even automorphisms} However, I just checked that if in the inductive
construction of coherators $\bC_\oo$ given yesterday, we insist on
the most trivial irredundancy condition (namely that we don't add a
``new'' homotopy $h: f\to g$ when there is already an old one), then
any morphism $h: \boldsymbol F_i \to \boldsymbol F_i$ such that $\cst s\,f=\cst s$
and $\cst t\,f=\cst t$, is the identity -- and that\namedpspage{L.12}{12}
implies inductively that an automorphism of the system of $\boldsymbol F_i$'s
related by the source and target maps $\cst s_1^i,\cst t_1^i$ is the
identity. Thus it is correct after all, it seems, that the category
structure of a coherator implies all other structure relevant to
us.\footnote{Maybe false; it is safer to give moreover the subcategory
  $\bC_0$.}

I do believe that the description given so far of what I mean by a
coherator, namely something acting like a kind of pattern in order to
define a corresponding notion of ``stacks'' (which in turn should be
the basic coefficient objects in non-commutative homological algebra,
as well as a convenient description of homotopy types) embodies some
of the essential features of the theory still in embryo that wants to
be developed. It is quite possible of course that some features are
lacking still, for instance that some extra conditions have to be
imposed upon $\bC$, possibly of a very different nature from mere
irredundancy conditions (which, I feel, are kind of irrelevant in this
set-up). Only by pushing ahead and working out at least in outline the
main aspects of the formalism of stacks, will it become clear whether
or not extra conditions on $\bC$ are needed. I would like at least
to make a commented list of these main aspects, and possibly do some
heuristic pondering on some of these in the stride, or afterwards. For
today it seems a little late though -- I have been pretty busy with
non-mathematical work most part of the day, and the next two days I'll
be busy at the university. Thus I guess I'll send off this letter
tomorrow, and send you later an elaboration (presumably much in the
style of this unending letter) if you are interested. In any case I
would appreciate any comments you make -- that's why I have been
writing you after all! I will probably send copies to Ronnie Brown,
Luc Illusie and Jean Giraud, in case they should be interested (I
guess at least Ronnie Brown is). Maybe the theory is going to take off
after all, in the long last!

\bigskip

Very cordially yours

\bigskip

PS (25.2.)\enspace I noticed a rather silly mistake in the notes of two days
ago, when stating that the categories $\bC_n$ admit fiber products:
what is true is that the category $\bC_0$ has fiber products (by
construction, practically), and that these are fiber products also in
the categories $\bC_n$ (by construction equally), i.e., that the
inclusion functors $\bC_0 \to \bC_n$ commute to fiber
products. Stacks in a category $C$ correspond to functors $\bC_\oo\to
C$ whose restriction to $\bC_0$ commutes with fiber
products. I carried the mistake along in the yesterday notes -- it
doesn't really change anything substantially. I will have to come back
anyhow upon the basic notion of a coherator\ldots

%%% Local Variables:
%%% mode: latex
%%% TeX-master: "main.tex"
%%% End:
