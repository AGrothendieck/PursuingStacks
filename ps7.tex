%% Anti-Copyright 2015 - the scrivener

\chapter{Linearization of homotopy types (2)}
\label{ch:VII}

\noindent\hfill\ondate{22.10.}\pspage{555}\par

% 133
\hangsection{Birth of Suleyman.}\label{sec:133}%
Again nearly a week has passed by without writing any notes -- the
tasks and surprises of life took up almost entirely my attention and
my energy. It were days again rich in manifold events -- most
auspicious one surely being the birth, three days ago, of a little
boy, Suleyman, by my daughter. The birth took place at ten in the
evening, in the house of a common friend, in a nearby village where my
daughter had been awaiting the event in quietness. It came while
everyone in the house was in bed, the nearly five years old girl
sleeping next to her mother giving birth. The girl awoke just after
the boy had come out, and then ran to tell Y (their hostess) she got a
little brother. When I came half an hour later, the little girl was
radiant with joy and wonder, while telling me in whispers, sitting
next to her mother and to her newborn brother, what had just
happened. As usual with grownups, I listened only distractedly,
anxious as I was to be useful the best I could. There wasn't too much
by then I could do, though, as the mother knew well what was to be
done and how to do it herself -- presently, tie the chord twice, and
then cut it in-between. ``Now you are on your own, boy!'' she told him
with a smile when it was done. A little later she and Y helped the
baby to a warm bath -- and still later, while the girl was asleep next
to her brother and Y in her room, the mother took a warm bath herself
in the same basin, to help finishing with the labors. When her mother
arrived a couple of hours later, everybody in the house was asleep
except me in the room underneath, awaiting her arrival while taking
care of some fallen fruit which had been gathered that very day.

This birth I feel has been beneficial, a blessing I might say, in a
number of ways. The very first one I strongly perceived, was about
Suleyman's sister, who had so strongly participated in his birth. This
girl has been marked by conflict, and her being is in a state of
division, surely as strongly as any other child of her age. However,
after this experience, if ever her time should come to bear a child
and give birth, she will do so with joy and with confidence, with no
secret fear or refusal interfering with her labors and with the act of
giving birth -- this most extraordinary act of all, maybe, which a
human being is allowed to accomplish; this unique privilege of woman
over man, this blessing, carried by so many like a burden and like a
curse\dots\enspace And there are other blessing too in this birth
which I perceive more or less clearly, and surely others still which
escape my conscious attention.\pspage{556}

I suspect that potentially, in every single instance anew, the act of
giving birth, and the sudden arrival and presence of a newborn, are a
blessing, carry tremendous power. In many cases, however, the greatest
part (if not all) of this beneficial power is lost, dispersed through
the action of crispation and fear (including the compulsory medical
``mise en scène'', from which it gets over harder to get rid in our
well-to-do countries\dots). A great deal could be said on this matter
in general, while my own leanings at present would be, rather, to
ponder about the manifold personal aspects and meaning of the
particular, manifold event I have just been involved in. But to yield
fully now to this leaning of mine would mean to stop short with the
mathematical investigation and with these notes. The drive carrying
this investigation is very much alive, though, and I have been feeling
it was time getting back to work -- to this kind of work, I mean.

This brings me to somewhat more mundane matters -- such as the
beginning of school and teaching duties. This year I am in charge of
preparing the ``concours d'agrégation'',\scrcomment{these are the
  civil service competitive examinations for high school (lycée)
  teachers\dots} something which, I was afraid, would be rather
dull. Rather surprisingly, the first session of common work on a
``problème d'agrègue'' wasn't dull at all -- the problem looked
interesting, and two of the three students who turned up so far were
interested indeed, and the atmosphere relaxed and friendly. It looks
as though I was going again to learn some mathematics through
teaching, or at any rate to apply things the way I have known and
understood them (in a somewhat ``highbrow'' way, maybe) to the more
down-to-earth vision going with a particular curriculum (here the
``programme d'agrègue'') and corresponding virtuosity tests --
something rather far, of course, from my own relation to mathematical
substance! Besides these arpeggios, we started a microseminar with
three participants besides me, on the Teichmüller groupoids; I expect
one of the participants is going to take a really active interest in
the stuff. I have been feeling somewhat reluctant to start this
seminar while still involved with the homotopy story, which is going
to keep me busy easily for the next six months still, maybe even
longer. Sure enough, the little I told us about some of the structures
and operations to be investigated, while progressively gaining view of
them again after a long oblivion -- and as discovering them again
hesitatingly, while pulling them out of the mist by bribes and bits --
this little was enough to revive the special fascination of these
structures, and all that goes with them.\pspage{557} It seems to me
there hasn't been a single thing in mathematics, including
motives,\scrcomment{unreadable margin note}
which has exerted such a fascination upon me. It will be hard, I'm
afraid, to carry on a seminar on such stuff, and go on and carry to
their (hopefully happy!) end those ponderings and notes on homotopical
algebra, so-called -- and a somewhat crazy one too at times, I am
afraid! We'll see what comes out of all this! The night after the
seminar session at any rate, and already during the two hours drive
home, the Teichmüller stuff was brewing anew in my head. Maybe it
would have gone on for days, but next day already several things came
up demanding special attention, last not least being the birth of
Suleyman\dots 

\bigbreak\scrcomment{the typescript says ``26.8.'' but it
  must be a misprint!}

\noindent\hfill\ondate{26.10.}\par

% 134
\hangsection[Back to linearization in the modelizer \Cat: another
\dots]{Back to linearization in the modelizer
  \texorpdfstring{\Cat}{(Cat)}: another handful of questions around
  Kan-Dold-Puppe.}\label{sec:134}%
It is time to come back to the ``review'' of linearization of
(ordinary) homotopy types, and the homology and cohomology formalism
in the context of the modelizer \Cat, which has been pushed aside now
for over two months, for the sake of that endless digression on
schematization of homotopy types. The strong tie between these two
strains of reflection has been the equal importance in both of the
linearization process. Technically speaking though, linearization, as
finally handled in the previous chapter (via the so-called
``integrators''), looks pretty much different from the similar
operation in the schematic set-up, due (at least partly) to the choice
we made of models for expressing schematic homotopy types, namely
taking semisimplicial scheme-like objects; which means, notably,
working with $\Simplex$ rather than more general test categories, and
relying heavily on the Kan-Dold-Puppe relationship. In the discrete
set-up, it turned out (somewhat unexpectedly) that the latter can be
replaced, when working with more general categories $A$ than
$\Simplex$ (which need not even be test categories or anything of the
kind), by the $\mathrm Lp_!\supab$ operation (when $p:A\to e$ is the
map from $A$ to the final object $e$ in \Cat), computable in terms of
a choice of an ``integrator'' for $A$. More specifically, recalling
that
\begin{equation}
  \label{eq:134.1}
  p_!\supab : \Ahatab \to \Ab\tag{1}
\end{equation}
is defined as the left adjoint of the inverse image functor
\[p^*:\Ab\to\Ahatab,\]
associating to every abelian group the corresponding ``context''
abelian\pspage{558} presheaf on $A$, and
\begin{equation}
  \label{eq:134.2}
  \LH_\bullet = \mathrm Lp_!\supab : \D^-(\Ahatab) \to \D^-\Ab, \quad
  \text{inducing $\D_\bullet(\Ahatab)\to\D_\bullet\Ab$}\tag{2}
\end{equation}
is its left derived functor (computed using projective resolution of
complexes bounded above in \Ahatab), composing \eqref{eq:134.2} with
the tautological inclusion functor
\[\Ahatab\hookrightarrow\D_\bullet(\Ahatab)\hookrightarrow
  \D^-(\Ahatab),\]
we get a canonical functor
\begin{equation}
  \label{eq:134.3}
  \mathrm Lp_!\supab \restrto \Ahatab:\Ahatab\to \D_\bullet\Ab,\tag{3}
\end{equation}
which may be expressed in terms of an integrator $L_\bullet^B$ for
$B=A\op$, i.e., a projective resolution of the constant presheaf
$\bZ_B$ in \Bhatab, as the composition
\begin{equation}
  \label{eq:134.4}
  \Ahatab\to\Ch_\bullet\Ab\to\D_\bullet\Ab,\tag{4}
\end{equation}
where the first arrow is
\begin{equation}
  \label{eq:134.5}
  F\mapsto F *_\bZ L_\bullet^B : \Ahatab\to\Ch_\bullet\Ab,\tag{5}
\end{equation}
and the second is the canonical localization functor. The functors
\eqref{eq:134.2} and \eqref{eq:134.3} may be viewed as the (total)
homology functors of $A$, with coefficients in complexes of abelian
presheaves, resp.\ in abelian presheaves simply. When we focus
attention on the latter, we may introduce in \Ahatab{} the set of
arrows which become isomorphisms under the total homology functor
\eqref{eq:134.3}, let's call them ``\emph{abelian weak equivalences}''
in \Ahatab, not to be confused with the notion of quasi-isomorphism
for a map between complexes in \Ahatab. Let's denote by
\begin{equation}
  \label{eq:134.6}
  \HotabOf_A = (W_A\supab)^{-1}\Ahatab\tag{6}
\end{equation}
the corresponding localized category of \Ahatab, where $W_A\supab$
denotes the set of abelian weak equivalences in \Ahatab. Thus, the
choice of an integrator $L_\bullet^B$ for $A$ (i.e., a cointegrator
for $B$) gives rise to a commutative diagram of functors
\begin{equation}
  \label{eq:134.7}
  \begin{tabular}{@{}c@{}}
    \begin{tikzcd}[baseline=(O.base)]
      \Ahatab \ar[r]\ar[d] & \Ch_\bullet\Ab\ar[d] \\
      \HotabOf_A \ar[r] &
      |[alias=O]| {\ooalign{$\D_\bullet\Ab$\cr$\D_\bullet\Ab\eqdef\HotabOf$\hidewidth\cr}}
    \end{tikzcd}\qquad\qquad,
  \end{tabular}\tag{7}
\end{equation}
where the lower horizontal arrow is defined via \eqref{eq:134.3}
(independently of the choice of $L_\bullet^B$), the vertical arrows
being the localization functors. Beware that even when $A=\Simplex$
and $L_\bullet^B$ is the usual, ``standard''\pspage{559} integrator
for $\Simplex$, the upper horizontal arrow in \eqref{eq:134.7} is
\emph{not} the Kan-Dold-Puppe equivalence of categories, it has to be
followed still by the ``normalization'' operation. Thus, \emph{we
  certainly should not expect in any case the functor \eqref{eq:134.5}
  to be an equivalence -- however, we suspect that when $A$ is a test
  category} (maybe even a weak test category would do it), \emph{then
  the lower horizontal arrow in \eqref{eq:134.7}}
\begin{equation}
  \label{eq:134.8}
  \HotabOf_A \to \D_\bullet\Ab=\HotabOf \tag{8}
\end{equation}
\emph{is an equivalence of categories}. Whenever true, for a given
category $A$, this statement looks like a reasonable substitute (on
the level of the relevant derived categories) for the Kan-Dold-Puppe
theorem, known in the two cases $\Simplex$ and $\Globe$.

There are however still two important extra features in the case
$A=\Simplex$, which deserve to be understood in the case of more
general $A$. One is that a map $u:F\to G$ in \Ahatab{} is in
$W_A\supab$ (i.e., is an ``abelian weak equivalence'') if{f} it is in
$W_A$, i.e., if{f} it is a weak equivalence when forgetting the
abelian structures. In terms of a final object $e$ in $A$ (when such
object exists indeed),\footnote{it is enough that $A$ be $1$-connected
  instead of having a final object} viewing the categories $A_{/F}$
(for $F$ in \Ahatab) pointed by the zero map $e\to F$, which yields
the necessary base-point $e_F$ for defining homotopy invariants
$\pi_i(F)$ for $i\ge0$, the relationship just considered between
abelian weak equivalence and weak equivalence, will follow of course
whenever we have functorial isomorphisms
\begin{equation}
  \label{eq:134.9}
  \pi_i(F) \eqdef \pi_i(A_{/F},e_F) \simeq \mathrm H_i(A,F),\tag{9}
\end{equation}
which are known to exist indeed in the case $A=\Simplex$. It should be
noted that in section~\ref{sec:92}, when starting (in a somewhat
casual way) with some reflections on ``abelianization'', we introduced
a category (denoted by $\HotabOf_A$) by localizing \Ahatab{} with
respect to the maps which are weak equivalences (when forgetting the
abelian structures), whereas it has now become clear that, in case the
latter should not coincide with the ``abelian weak equivalences''
defined in terms of linearization or homology, it is the category
\eqref{eq:134.6} definitely which is the right one. Still, the
question of defining isomorphisms \eqref{eq:134.9} when $A$ has a
final object, and whether the equality
\begin{equation}
  \label{eq:134.10}
  W_A\supab = \forg_A^{-1}(W_A)\tag{10}
\end{equation}
holds, where
\[\forg_A:\Ahatab\to\Ahat\]
is the ``forgetful functor'', should be settled for general $A$.

The\pspage{560} other ``extra feature'' is about the relationship of
$W_A\supab$ with the notion of homotopism. In case $A=\Simplex$, a map
in \Ahatab{} is a weak equivalence (or equivalently, an abelian weak
equivalence) if{f} it is a homotopism when forgetting the abelian
structures -- this follows from the well-known fact that
semisimplicial groups are Kan complexes. There is of course also a
notion of homotopism in the stronger abelian sense -- a particular
case of the notion of a homotopism between semisimplicial objects in
an arbitrary category (here in \Ab). The Kan-Dold-Puppe theory implies
that if $F$ and $G$ in \Simplexhatab{} have as values projective
\bZ-modules, then a map $F\to G$ in \Simplexhatab{} is a weak
equivalence if{f} it is an ``abelian'' homotopism; and likewise, two
maps $u,v: F\rightrightarrows G$ are equal in $\HotabOf_\Simplex$ if{f}
they are homotopic (in the strict, \emph{abelian} sense of the
word). The corresponding statements still make sense and are true,
when replacing \bZ{} by any ground ring $k$, working in
$\Simplex_k\adjuphat$ rather than $\Simplex_\bZ\adjuphat =
\Simplexhatab$ -- and more generally still, when working in
\[\Simplex\adjuphat_\scrM = \bHom(\Simplexop, \scrM),\]
where \scrM{} is any abelian category. Now, replacing $\Simplex$ by an
arbitrary object $A$ in \Cat, these statements still make sense, it
would seem, provided we got on \Ahat{} a suitable ``homotopy
structure'', more specifically, a suitable ``homotopy interval
structure'' (in the sense of section~\ref{sec:51}). In
section~\ref{sec:97} (p.~\ref{p:355}) we reviewed the three standard
homotopy interval structures which may be introduced on a category
\Ahat, and their relationships -- the main impression remaining was
that, in case $A$ is a \emph{contractor} (cf.\ section~\ref{sec:95}),
those three structures coincide, and may be defined also in terms of a
\emph{contractibility structure} on \Ahat{} (section~\ref{sec:51}),
the latter giving rise (via the general construction of
section~\ref{sec:79}) to the usual notion of weak equivalence $W_A$ in
\Ahat. Thus, we may hope that the feature just mentioned for
$\Simplexhat$ may be valid too for \Ahat, whenever $A$ is a
contractor. In case $A$ is only a test category, then the most
reasonable homotopy interval structure on \Ahat, I would think, which
possibly may still yield the desired ``extra feature'', is the one
defined in terms of the set $W_A$ of weak equivalences in \Ahat{} as
in section~\ref{sec:54} (namely $h_{W_A}$).

% 135
\hangsection[Proof of ``integrators are abelianizators'' (block
against \dots]{Proof of ``integrators are abelianizators''
  \texorpdfstring{\textup(}{(}block against homology
  receding?\texorpdfstring{\textup)}{)}.}\label{sec:135}%
This handful of questions (some of which we met with before) is mainly
a way of coming into touch again with the abelianization story, which
has been becoming somewhat remote during the previous two months. I am
not sure I am going to\pspage{561} make any attempt, now or later, to
come to an answer. The last one, anyhow, seems closely related to the
formalism of closed model structure on categories \Ahat, and the
proper place for dealing with it would seem to be rather next chapter
VII,\scrcomment{I guess AG thought this was still
  Part~\ref{ch:VI}. Since there was never a second volume of PS, we
  refer to \textcite{Cisinski2006} for a discussion of closed model
  structures on presheaf topoi, now known as \emph{Cisinski model
    structures}\dots} where such structures are going to be
studied. What I would like to do, however, here-and-now, is to
establish at last the long promised relationship ``integrators are
abelianizators'', which we have kept turning around and postponing
ever since section~\ref{sec:99}, when those integrators were finally
introduced, mainly for this purpose (of furnishing us with
``abelianizators'').

First of all, I should be more outspoken than I have been before, in
defining the ``\emph{abelianization functor}'' (or ``\emph{absolute
  Whitehead functor}''):
\begin{equation}
  \label{eq:135.1}
  \bWh: \Hot\eqdef\scrW^{-1}\Cat \to \HotabOf
  \eqdef\D_\bullet\Ab,\tag{1} 
\end{equation}
without any use of the semisimplicial machinery which, at the
beginning of our reflections, had rather obscured the picture
(section~\ref{sec:92}). Defining such a functor amounts to defining a
functor
\begin{equation}
  \label{eq:135.2}
  \LH_\bullet:\Cat\to\D_\bullet\Ab,\tag{2}
\end{equation}
or ``total homology functor'', which should take weak equivalences
into isomorphisms. For an object $A$ in \Cat, we define
\begin{equation}
  \label{eq:135.3}
  \LH_\bullet(A) = \LH_\bullet(A,\bZ_A) =
  \mathrm L{p_A}_!\supab(\bZ_A),\tag{3} 
\end{equation}
where $\bZ_A$ is the constant abelian presheaf on $A$ with value \bZ,
and
\[p_A : A\to e\]
is the map to the final object of \Cat. We have to define the
functorial dependence on $A$. More generally, for pairs
\[(A,F), \quad\text{with $A$ in \Cat, $F$ in \Ahatab,}\]
the expression
\[\LH_\bullet(A,F)\quad\text{in $\D_\bullet\Ab$}\]
is functorial with respect to the pair $(A,F)$, where a map
\[(A,F) \to (A',F')\]
is defined to be a pair $(f,u)$, where
\begin{equation}
  \label{eq:135.4}
  f: A\to A', \quad
  u: F\to f^*(F'),\tag{4}
\end{equation}
the composition of maps being the obvious one. To see that such a pair
defines a map
\begin{equation}
  \label{eq:135.5}
  \LH_\bullet(f,u): \LH_\bullet(A,F) \to \LH_\bullet(A',F'),\tag{5}
\end{equation}
we\pspage{562} use
\[p_A = p_{A'} \circ f,\]
hence
\begin{equation}
  \label{eq:135.star}
  (p_A)_!\supab \simeq (p_{A'})_!\supab \circ f_!\supab,
  \quad\text{and}\quad
  \mathrm L(p_A)_!\supab \simeq \mathrm L(p_{A'})_!\supab
  \circ \mathrm Lf_!\supab,\tag{*}
\end{equation}
taking into account that $f_!\supab$ maps projectives to
projectives. Hence, we get
\[\LH_\bullet(A,F)\eqdef \mathrm L(p_A)_!\supab(F) \simeq
  \mathrm L(p_{A'})_!\supab(\mathrm Lf_!\supab(F)) \eqdef
  \LH_\bullet(A',\mathrm Lf_!\supab(F)).\]
This, in order to get \eqref{eq:135.5}, we need only define a map in
$\D^-({A'}\uphat\subab)$
\begin{equation}
  \label{eq:135.6}
  \mathrm Lf_!\supab(F)\to F',\tag{6}
\end{equation}
which will be obtained as the composition
\begin{equation}
  \label{eq:135.6prime}
  \mathrm Lf_!\supab(F)\to f_!\supab(F) \to F',\tag{6'}
\end{equation}
where the first map in \eqref{eq:135.6prime} is the canonical
augmentation maps towards the $\mathrm H_0$ object, and where the
second corresponds to $u$ in \eqref{eq:135.4} by adjunction. This
defines the map \eqref{eq:135.5}, and compatibility with compositions
should be a tautology. Hence the functor \eqref{eq:135.2}. To get
\eqref{eq:135.1}, we still have to check that when
\[f:A\to A'\]
is a weak equivalence, then
\[\LH_\bullet(f): \LH_\bullet(A) \to \LH_\bullet(A')\]
is an isomorphism in $\D_\bullet\Ab$. It amounts to the same to check
that for any object $K^\bullet$ in $\D^+\Ab$, the corresponding map
between the $\RHom$'s with values in $K^\bullet$ is an isomorphism in
$\D^+\Ab$. But the latter map can be identified with the map for
\emph{cohomology}
\[\RH^\bullet(A', K_{A'}^\bullet) \to \RH^\bullet(A, K_A^\bullet),\]
with coefficients in the constant complex of presheaves defined by
$K^\bullet$ on $A'$ and on $A$, which is an isomorphism, by the very
definition of weak equivalences in \Cat{} via cohomology.

Now, the statement ``an integrator is an abelianizator'' may be
rephrased rather evidently, without any reference to a given
integrator, as merely the commutativity, up to canonical isomorphism,
of the following diagram for a given $A$ in \Cat:\pspage{563}
\begin{equation}
  \label{eq:135.7}
  \begin{tabular}{@{}c@{}}
    \begin{tikzcd}[baseline=(O.base),column sep=large]
      \Ahat \ar[r,"\varphi_A"]\ar[d,"\Wh_A"'] &
      \Hot\ar[d,"{\bWh}"] \\
      \Ahatab \ar[r,"{\LH_\bullet(A,{-})}"'] &
      |[alias=O]| {\ooalign{$\Hotab$\cr$\Hotab\eqdef\D_\bullet\Ab$\hidewidth\cr}}
    \end{tikzcd}\qquad\qquad,
  \end{tabular}\tag{7}
\end{equation}
or equivalently, of the corresponding diagram where the categories
\Ahat, \Ahatab{} are replaced by the relevant localizations:
\begin{equation}
  \label{eq:135.7prime}
  \begin{tabular}{@{}c@{}}
    \begin{tikzcd}[baseline=(O.base)]
      \HotOf_A \ar[r]\ar[d] &
      \HotOf \ar[d] \\
      \HotabOf_A \ar[r] &
      |[alias=O]| \HotabOf
    \end{tikzcd}.
  \end{tabular}\tag{7'}
\end{equation}
The left vertical arrow in \eqref{eq:135.7} is of course the trivial
abelianization functor in \Ahat:
\[\Wh_A:\Ahat\to\Ahatab, \quad
  X\mapsto \bZ^{(X)}=\bigl(a\mapsto \bZ^{(X(a))}\bigr).\]
Going back to the definitions, the commutativity of \eqref{eq:135.7}
up to isomorphism, means that for $X$ in \Ahat, there is a canonical
isomorphism
\begin{equation}
  \label{eq:135.8}
  \LH_\bullet(A_{/X},\bZ) \simeq\LH_\bullet(A, \bZ^{(X)}),\tag{8}
\end{equation}
functorial with respect to $X$. To define \eqref{eq:135.8}, let
\[f : A'\eqdef A_{/X} \to A\]
be the canonical functor, then we get (using \eqref{eq:135.star} of
the previous page, with the roles of $A$ and $A'$ reversed)
\[\LH_\bullet(A',\bZ_{A'}) \simeq \LH_\bullet(A, \mathrm
  Lf_!\supab(\bZ_{A'})),\]
and the relation \eqref{eq:135.8} follows from the more precise
relation
\begin{equation}
  \label{eq:135.9}
  \mathrm Lf_!\supab(\bZ_{A'}) \tosim \bZ^{(X)}.\tag{9}
\end{equation}
To get \eqref{eq:135.9}, we remark that we have the tautological
relation
\begin{equation}
  \label{eq:135.9prime}
  f_!\supab(\bZ_{A'}) \simeq \bZ^{(X)},\tag{9'}
\end{equation}
hence the map \eqref{eq:135.9}. To prove it is an isomorphism amounts
to proving
\[\mathrm L_if_!\supab(\bZ_{A'}) = 0\quad\text{for $i>0$,}\]
but we have indeed
\begin{equation}
  \label{eq:135.10}
  \mathrm L_if_!\supab = 0\quad
  \text{for $i>0$, i.e., $f_!\supab$ is \emph{exact},}
  \tag{10}
\end{equation}
not only right exact, a rather special feature, valid for a
localization\pspage{564} functor like $f:A_{/X}\to A$, namely for a
functor which is fibering (in the sense of the theory of ``fibered
categories'') and has discrete fibers. It comes from the explicit
description of $f_!\supab$, as
\begin{equation}
  \label{eq:135.11}
  f_!\supab(F) = \Bigl( a\mapsto \bigoplus_{u\in X(a)} F(a)_u\Bigr),\tag{11}
\end{equation}
where, for a group object $F$ in $\Ahat_{/X}\simeq(A_{/X})\uphat$, and
$u$ in $X(a)$ (defining therefore an object of $A_{/X}$) $F(a)_u = $
fiber of $F(a)$ at $u\in X(a)$ is the corresponding abelian
group. I'll leave the proof of \eqref{eq:135.11} to the reader, it
should be more or less tautological.

Once the whole proof is written down, it looks so simple that I feel
rather stupid and can't quite understand why I have turned around it
for so long, rather than writing it down right away more than three
months ago! The reason surely is that I have been accustomed so
strongly to expressing everything via cohomology rather than homology,
that there has been something like a block against doing work
homologically, when it is homology which is involved. This block has
remained even after I took the trouble of telling myself quite
outspokenly (in section~\ref{sec:100}) that homology was just as
important and meaningful as cohomology, and more specifically still
(in section~\ref{sec:103}) that the proof I had in mind first, via
Quillen's result about $A\simeq A\op$ in \Hot{} and via cohomology,
was an ``awkward'', an ``upside-down'', one. The review on
abelianization I went into just after made things rather worse in a
sense, as there I took great pains to make the point that homology and
cohomology were just one and the same thing (so why bother about
homology!). Still, I \emph{did} develop some typically homologically
flavored formalism with the $*_k$ operation, and I hope that at the
end that block of mine is going to recede\dots

\bigbreak

\noindent\hfill\ondate{27.10.}\pspage{565}\par

% 136
\hangsection[Preliminary perplexities about a full-fledged ``six
\dots]{Preliminary perplexities about a full-fledged ``six operations
  duality formalism'' within
  \texorpdfstring{\Cat}{(Cat)}.}\label{sec:136}%
I just spent a couple of hours, after reading the notes of last night,
trying to get a better feeling of the basic homology operation in the
context of the basic localizer \Cat, namely taking the left derived
functor $\mathrm Lf_!\supab$ of the ``unusual'' direct image functor
for abelian presheaves, associated to a map
\[f: A \to B\]
in \Cat. This led me to read again the notes of section~\ref{sec:92},
when I unsuspectingly started on an ``afterthought, later gradually
turning into a systematic reflection on abelianization. With a
distance of nearly four months, what strikes me most now in these
notes is awkwardness of the approach followed at start, when yielding
to the reflex of laziness of describing abelianization of homotopy
types via the semisimplicial grindmill. The uneasiness in these notes
is obvious throughout -- I kind of knew ``au fond''\scrcomment{``au
  fond'' -- at the bottom -- deep down} that dragging in
the particular test category $\Simplex$ was rather silly. In
section~\ref{sec:100} only, does it get clear that the best
description for abelianization, with the modelizer \Cat, is via the
unusual direct image $p_!\supab$ corresponding to the projection
\[p=p_A:A\to e\]
of the ``model'' $A$ in \Cat{} to the final object (formulæ
\eqref{eq:100.11} and \eqref{eq:100.12} page~\ref{p:359}), by applying
$p_!\supab$ to a ``cointegrator'' $L_\bullet^A$ for $A$, namely to a
projective resolution of $\bZ_A$ (where $A$ is written $B$ by the way,
as I had been led before to replace a given $A$ in \Cat{} by its
``dual'' or opposite $B=A\op$, bound as I was for interpreting
``integrators'' for $A$ in terms of ``cointegrators'' for $B$\dots);
and in the next section the step is finally taken (against the
``block''!) to write
\[p_!\supab(L_\bullet^B)=\mathrm Lp_!\supab(\bZ_B),\]
which inserts abelianization into the familiar formalism of (left)
derived functors of standard functors. The reasonable next thing to do
was of course what I finally did only yesterday, namely check the
commutativity of the diagram \eqref{eq:135.7} of p.~\ref{p:563},
namely compatibility of this notion of abelianization (or homology)
with Whitehead's abelianization, when working with models coming from
\Ahat, $A$ any object in \Cat. This by the way, when applied to the
case $A=\Simplex$, gives at once the equivalence of the intrinsic
definition of abelianization, with the semisimplicial one we started
with -- provided we remark that in this case, $\mathrm Lp_!\supab$
(rather, its restriction to $\Simplexhatab$) may be equally
interpreted as the Kan-Dold-Puppe\pspage{566} functor (more
accurately, the composition of the latter with the localization
functor $\Ch_\bullet\Ab\to\D_\bullet\Ab=\HotabOf$). The equivalence of
the two definitions of abelianization is mentioned on the same
p.~\ref{p:369}, somewhat as a chore I didn't really feel then to dive
into. Besides the ``block'' against homology, the picture was being
obscured, too, by the ``computational'' idea I kept in mind, of
expressing $\mathrm Lp_!\supab(F)$ for $F$ in \Ahatab{} in terms of an
``integrator'' for $A$, i.e., as $F *_\bZ L_\bullet^B$ -- whereas I
should have known best myself that for establishing formal properties
relating various derived functors, the particular approaches used for
``computing'' them more or less elegantly are wholly irrelevant\dots

One teaching I am getting out of all this, is that when expressing
abelianization, or presumably any other kind of notion or operation of
significance for homotopy types, one should be careful, for any
modelizer one chooses to work in, to dig out the description which
fits smoothly those particular models. Clearly, when working with
semisimplicial models, the description via tautological abelianization
$\Wh_\Simplex$ and using Kan-Dold-Puppe is the best. When working
within the modelizer \Cat, though, making the detour through
\Simplexhat{} is awkward and makes us just miss the relevant
facts. Once we got this, it gets clear, too, what to do when
$\Simplex$ is replaced by any other object $A$ in \Cat{} when taking
models in \Ahat{} (never minding even whether \Ahat{} is indeed a
``modelizer''): namely, apply $\Wh_A$, and take total homology!

This brings to my mind another example, very similar indeed. Some time
after I got across Thomason's nice paper, showing that \Cat{} is a
closed model category (see comments in section~\ref{sec:87}), I got
from Tim Porter another reprint of
Thomason's,\scrcomment{\textcite{Thomason1979}} where he gives the
description of ``homotopy colimits'' (or ``integration'', as I call
it) within the modelizer \Cat, in terms of the total category
associated to a fibered category (compare section~\ref{sec:69}). There
he grinds through a tedious, highly technical proof, whereas the
direct proof when describing ``integration'' as the left adjoint
functor to the tautological ``inverse image'' functor, is more or less
tautological, too. The reason for this awkwardness is again that,
rather than being content to work with the models as they are,
Thomason refers to the
Bousfield-Kan\scrcomment{\textcite{BousfieldKan1972}} description of
colimits in the semisimplicial set-up which he takes as his definition
for colimits. I suspect that \Cat{} is the one modelizer most ideally
suited for expressing the ``integration'' operation, and that the
Bousfield-Kan description is just the obvious, not-quite-as-simple one
which can be deduced\pspage{567} from the former, using the relevant
two functors between \Cat{} and \Simplexhat{} which allow to pass from
one type of models to another. (Sooner or later I should check in
Bousfield-Kan's book whether this is so or not\dots)\enspace Thomason,
however, did the opposite, and it is quite natural that he had to pay
for it by a fair amount of sweat! (Reference of the paper: Homotopy
colimits in the category of small categories, Math.\ Proc.\ Cambridge
Philos.\ Soc.\ (1979), 85, p.~91--109.)

The proof written down yesterday for compatibility of abelianization
of homotopy types with the Whitehead abelianization functor within a
category \Ahat, still goes through when replacing abelianization by
$k$-linearization, with respect to an arbitrary ground ring $k$ (not
even commutative). It wasn't really worth while, though, to introduce
a ring $k$, as the general result should follow at once from the case
$k=\bZ$, by the formula
\begin{equation}
  \label{eq:136.1}
  \mathrm Lp_!^k(k_A) \simeq \mathrm Lp_!^\bZ(\bZ_A) \Lotimes_\bZ
  k,\tag{1} 
\end{equation}
where in the left-hand side we are taking the left derived functor for
the functor
\[p_!^k : A\uphat_k \to (\AbOf_k)\]
generalizing $p_!\supab = p_!^\bZ$ (with $p: A\to e$ as above), and
where in the right-hand side we are using the ring extension functor
\[\Lotimes_\bZ : \D_\bullet\Ab\to\D_\bullet(\AbOf_k)\]
for the relevant derived categories. This reminds me of the need of
developing a more or less exhaustive formulaire around the basic
operations
\begin{equation}
  \label{eq:136.2}
  \mathrm Lf_!, \quad f^*, \quad \mathrm Rf_*,\tag{2}
\end{equation}
including the familiar one for the two latter, valid more generally
for maps between ringed topoi, and including also a ``\emph{projection
  formula}'' generalizing \eqref{eq:136.1}. Such a formula will be no
surprise, surely, to a reader familiar with a duality context (such as
étale cohomology, or ``coherent'' cohomology of noetherian schemes,
say), where a formalism of the ``four variances'' $f_!$, $f^*$, $f_*$,
$f^!$ and the ``two internal operations'' $\Lotimes$ and $\RbHom$ can
be developed -- it would seem that the formal properties of the triple
\eqref{eq:136.2}, together with the two internal operations just
referred to, are very close (for the least) to those of the slightly
richer one in duality set-ups, including equally an ``unusual inverse
image'' $f^!$, right adjoint to $\mathrm Rf_*$ (denoted sometimes
simply by $f_*$). This similarity\pspage{568} is a matter of course,
as far as the two last among the functors \eqref{eq:136.2}, together
with the two internal operations, are concerned, as in both contexts
(homology and cohomology formalism within \Cat{} on the one hand, and
the ``sweeping duality formalism'' on the other) the formulaire
concerning these four operations
\[f^*, \quad \mathrm Rf_*, \quad {\Lotimes}, \quad \RbHom\]
is no more, no less than just the relevant formulaire in the context
of arbitrary (commutatively) ringed topoi, and maps between such. The
common notation $f_!$ (occurring in $\mathrm Lf_!$ in the \Cat{}
context, in $\mathrm Rf_!$ in the ``duality'' context) is a very
suggestive one, for the least, and I am rather confident that most
reflexes (concerning formal behavior of $f_!$ with respect to the
other operations) coming from one context, should be OK too in the
other. Whether this is just a mere formal analogy, or whether there is
a deeper relationship between the two kinds of contexts, I am at a
loss at present to say. It doesn't seem at all unlikely to me that
among arbitrary maps in \Cat, one can single out some, by some kind of
``finite type'' condition, for which a functor
\begin{equation}
  \label{eq:136.star}
  f^! : \D^+(B\uphat_k) \to \D^+(A\uphat_k)\tag{*}
\end{equation}
can be defined (where $f:A\to B$), right adjoint to the familiar
$\mathrm Rf_*$ functor, so that \eqref{eq:136.2} can be completed to a
sequence of \emph{four} functors
\begin{equation}
  \label{eq:136.3}
  \mathrm Lf_!, \quad f^*, \quad \mathrm Rf_*, \quad f^!,\tag{3}
\end{equation}
forming a sequence of mutually adjoint functors between derived
categories, in the usual sense (the functor immediately to the right
of another being its right adjoint). I faintly remember that Verdier
worked out such a formalism within the context of discrete or
profinite groups, or both, in a Bourbaki talk he gave, this being
inspired by the similar work he did within the context of usual
topological spaces. In the latter, the finiteness condition required
for a map
\[f:X\to Y\]
of topological spaces to give rise to a functor $f^!$ (between, say,
the derived categories of the categories of abelian sheaves on $Y$ and
$X$) is mainly that $X$ should be locally embeddable in a product
$Y\times \bR^d$ -- a rather natural condition indeed! As we are using
\Cat{} as a kind of algebraic paradigm for the category of topological
spaces, this last example for ``sweeping duality'' makes it rather
plausible that something of the same kind should exist indeed in
\Cat{} -- and likewise for the context of groups (which we may view as
just particular cases of\pspage{569} models in \Cat).

I am sorry I was a bit confused, when describing $f^!$ as a right
adjoint to $\mathrm Rf_*$ -- I was thinking of the analogy with a map
of schemes, or spaces, which is not only ``of finite type'' in a
suitable sense, but moreover \emph{proper} -- in which case, in those
duality contexts, $\mathrm Rf_*$ is canonically isomorphic with the
functor denoted by $f_!$ or $\mathrm Rf_!$. Otherwise, the
(``non-trivial'') ``duality theorem'' will assert, rather, that the
pair
\[\mathrm Rf_!,\quad f^!\]
is a pair of adjoint functors, just as is the pair
\[f^*,\quad \mathrm Rf_*\quad\text{(or simply $f_*$).}\]
But even when $f$ is assumed to be proper, the sequence
\eqref{eq:136.3} isn't a sequence of adjoint functors in the standard
duality contexts, namely $\mathrm Rf_!$ is by no means left adjoint to
$f^*$, i.e., $f^*$ isn't isomorphic to $f^!$ (except in extremely
special cases, practically I would think only étale maps are OK, which
in the context of \Cat{} would correspond to maps in \Cat{} isomorphic
to a map $A_{/X}\to A$ for $X$ in \Ahat, namely maps which are
fibering with discrete fibers). This \emph{does} make an important
discrepancy indeed, between the two kinds of contexts -- and increases
my perplexity, as to whether or not one should expect a ``four
variance duality formalism'' to make sense in \Cat. If so, presumably
the $f_!$ or $\mathrm Rf_!$ it will involve (perhaps via a suitable
notion of ``proper'' maps in \Cat, as already referred to earlier
(section~\ref{sec:70})) will be different after all from the $\mathrm
Lf_!$ we have been working with lately, embodying homology
properties. But so does $\mathrm Rf_!$ too, in a rather strong sense,
via the ``duality theorem''!

In the various duality contexts, a basic part is played by the three
particular classes of maps: proper maps, smooth maps, and immersions,
and factorizations of maps into an immersion followed by either a
proper, or a smooth map. In the context of Cat, there is a very
natural way indeed to define the three classes of maps, as we'll see
in the next chapter, presumably -- so natural indeed, that it is hard
to believe that there may be any other reasonable choice! One very
striking feature (already mentioned in section~\ref{sec:70}) is that
the two first notions are ``dual'' to each other in the rather
tautological sense, namely that a map $f:A\to B$ in \Cat{} is proper
if{f} the corresponding map $f\op:A\op\to B\op$ is smooth (whereas the
notion of an immersion is\pspage{570} autodual). How does this fit
with the expectation of developing a ``four variance'' duality
formalism within \Cat? It rather heightens perplexity at first sight!
Proper maps include cofibrations (in the sense of category theory, not
in Kan-Quillen's sense!); dually, smooth maps include
fibrations. Consequently, maps which are moth smooth and proper
include bifibrations, and hence are not too uncommon. Now, how strong
a restriction is it for a map $f:A\to B$ in \Cat{} to factor into
\[f=p\circ i,\]
where $i$ is an immersion $A\to A'$ (namely, a functor identifying $A$
to a full subcategory of $A'$, whose essential image includes, with
any two objects $x,y$, any other $z$ which is ``in between'': $x\to
z\to y$), and $p$ is both proper and smooth (a bifibration, say)?

I start feeling like a battle horse scenting gunpowder again -- still,
I don't think I'll run into it. Surely, there is something to be
cleared up, and perhaps once again a beautiful duality formalism with
the six operations and all will emerge out of darkness -- but this
time I will not do the pulling. Maybe someone else will -- if he isn't
discouraged beforehand, because the big-shots all seem kind of blasé
with ``big duality'', derived categories and all that. As for my
present understanding, I feel that the question isn't really about
homotopy models, or about foundations of homotopy and cohomology
formalism -- at any rate, that I definitely don't need this kind of
stuff, for the program I have been out for. I shouldn't refrain, of
course, to pause on the way every now and then and have a look at the
landscape, however remote or misty -- but I am not going to forget I
am bound for a journey, and that the journey should not be an unending
one\dots

\bigbreak

\noindent\hfill\alsoondate{27.10.}\par

% 137
\hangsection[Looking for the relevant notions of properness and
\dots]{Looking for the relevant notions of properness and smoothness
  for maps in \texorpdfstring{\Cat}{(Cat)}. Case of ordered sets as a
  paradigm for cohomology theory of conically stratified
  spaces.}\label{sec:137}%
It occurred to me that I have been a little rash yesterday, when
asserting that the notions of ``smoothness'' and ``properness'' for
maps in \Cat{} which I hit upon last Spring is the only ``reasonable''
one. Initially, I referred to these notions by the names
``cohomologically smooth'', ``cohomologically proper'', as a measure
of caution -- they were defined by properties of commutation of base
change to formation of the Leray sheaves $\mathrm R^if_*$ (i.e.,
essentially, to ``cointegration''), which were familiar to me for
smooth resp.\ proper maps in the context of schemes, or ordinary
topological spaces. These\pspage{571} cohomological counterparts of
smoothness and properness fit very neatly into the homology and
cohomology formalism, and I played around enough with them, last
Spring as well as more than twenty years ago when developing étale
cohomology of schemes, for there being no doubt left in my mind that
these notions are relevant indeed. However, I was rather rash
yesterday, while forgetting that these cohomological versions of
smoothness and properness are considerably weaker than the usual
notions. Thus, in the context of schemes over a ground field, the
product of any two schemes is cohomologically smooth over its factors
-- or equivalently, any scheme over a field $k$ is cohomologically
smooth over $k$! Similarly, in the context of \Cat, any object in
\Cat, namely any small category, is both cohomologically smooth and
proper over the final object $e$ (as it is trivially ``bifibered''
over $e$). On the other hand, it isn't reasonable, of course, to
expect any kind of Poincaré-like duality to hold for the cohomology
(with twisted coefficients, say) of an arbitrary object $A$ in
\Cat. To be more specific, it is easy to see that in many cases, the
functor
\[\mathrm Rf_* : \D^+(\Ahatab)\to\D^+\Ab\]
does not admit a right adjoint (which we would call $f^!$). For
instance, when $A$ is discrete, then $f^!$ exists (and may then be
identified with $f^*$) if{f} $A$ is moreover \emph{finite} -- a rather
natural condition indeed, when we keep in mind the topological
significance of the usual notion of properness! This immediately
brings to mind some further properties besides base change properties,
which go with the intuitions around properness -- for instance, we
would expect for proper $f$, the functors $f_*$ and $\mathrm Rf_*$ to
commute to filtering direct limits, and the same expectation goes with
the assumption that $\mathrm Rf_*$ should admit a right adjoint. This
exactness property is not satisfied, of course, when $A$ is discrete
infinite. We now may think (still in case of target category equal to
$e$) to impose the drastic condition that the category $A$ is
finite. Such restriction however looks in some respects too weak, in
others too strong. Thus, it will include categories defined by finite
groups, which goes against the rather natural expectation that
properness + smoothness, or any kind of Poincaré duality, should go
with \emph{finite cohomological dimension}. On the other hand, there
are beautiful infinite groups (such as the fundamental group of a
compact surface, or of any other compact variety that is a $K(\pi,1)$
space\dots) which satisfy Poincaré duality.

These\pspage{572} reflections make it quite clear that the notions of
properness and of smoothness for maps in \Cat, relevant for a duality
formalism, have still to be worked out. Two basic requirements to be
kept in mind are the following: \namedlabel{rem:137.1}{1)}\enspace for
a proper map $f: A\to B$, and any ring of coefficients $k$, the
functor
\[\mathrm Rf_* : \D^+(A\uphat_k)\to\D^+(B\uphat_k)\]
should admit a right adjoint $f^!$, and
\namedlabel{rem:137.2}{2)}\enspace for a smooth map $f$ factored as
$f=g\circ i$, with $g$ proper and $i$ an ``open immersion'', the
composition $f^!\eqdef i^*\circ g^!$ should be expressible as
\begin{equation}
  \label{eq:137.5}
  f^!: K^\bullet \mapsto f^*(K)\Lotimes_k T_f(k)[d_f],\tag{5}
\end{equation}
where $T_f(k)$ is a presheaf of $k$-modules on $A$ locally isomorphic
to the constant presheaf $k_A$ ($T_f(k)$ may be called the
\emph{orientation sheaf} for $f$, with coefficients in $k$), and $d_f$
is a natural integer (which may be called the \emph{relative
  dimension} of $A$ over $B$, or of $f$). (For simplicity, I assume in
\ref{rem:137.2} that $A$ is connected, otherwise $d_f$ should be
viewed as a function on the set of connected components of
$A$.)\enspace This again should give the correct relationship, for $f$
as above, between the (for the time being hypothetical) $\mathrm Rf_!$
($\eqdef \mathrm Rg_*\circ\mathrm Lf_!$) and our anodyne $\mathrm
Lf_!$, for an argument $L_\bullet$ in $\D^\bullet(A\uphat_k)$ say:
\begin{equation}
  \label{eq:137.6}
  \mathrm Rf_!(L_\bullet) \simeq \mathrm Lf_!(L_\bullet \otimes
  T_f^{-1})[-d_f],\tag{6} 
\end{equation}
where the left-hand side is just $\mathrm Rf_*(L_\bullet)$, if we
assume moreover $f$ to be proper.

This precise relationship between the two possible versions of an
$f_!$ operation between derived categories, namely $\mathrm Lf_!$
embodying homology, defined for any map $f$ in \Cat, and
$\mathrm Rf_!$ embodying ``cohomology with proper supports'', defined
for a map that may be factored as $g\circ i$ with $g$ ``proper'' and
$i$ an (open, if we like) immersion, relation valid if $f$ is moreover
assumed to be ``smooth'', greatly relaxes yesterday's perplexity,
coming from a partial confusion in my mind between the operations
$\mathrm Rf_!$ and $\mathrm Lf_!$. (Beware the notation $\mathrm Rf_!$
is an abuse, as it doesn't mean at all anything like the right derived
functor of the functor $f_!$!)\enspace At the same time, I feel a lot
less dubious now about the existence of a ``six operations'' duality
formalism in the \Cat{} context -- I am pretty much convinced, now,
that such a formalism exists indeed. The main specific work ahead is
to get hold of the relevant notions of proper and smooth maps. The
demands we have on\pspage{573} these, besides the relevant base change
properties, are so precise, one feels, that they may almost be taken
as a definition! Maybe even the ``almost'' could be dropped -- namely
that a comprehensive axiomatic set-up for the duality formalism could
be worked out, in a way applicable to the known instances as well as
to the presently still unknown one of \Cat, by going a little further
still than Deligne's exposition in
SGA~5\scrcomment{\textcite[Dualité]{SGA4andhalf}} (where the notions
of ``smooth'' and ``proper'' maps were supposed to be given
beforehand, satisfying suitable properties). Before diving into such
axiomatization game, one should get a better feeling, though, through
a fair number of examples (not all with $e$ as the target category
moreover), of how the proper, the smooth and the proper-and-smooth
maps in \Cat{} actually look like. Here, presumably, Verdier's work in
the context of discrete infinite groups should give useful clues.

Other important clues should come from the opposite side so to say --
namely ordered sets. Such a set $I$, besides defining in the usual way
a small category and hence a topos, may equally be viewed as a
topological space, more accurately, the topos it defines may be viewed
as being associated to a topological space, admitting $I$ as its
underlying set (cf.\ section~\ref{sec:22}, p.~\ref{p:18}). This
topological space is noetherian if{f} the ordered set $I$ is -- for
instance if $I$ is finite. In such a case, an old algebraic geometer
like me will feel in known territory, which maybe is a delusion,
however -- at any rate, I doubt the duality formalism for topological
spaces (using factorizations of maps $X\to Y$ via embeddings in spaces
$Y\times\bR^d$) makes much sense for such non-separated
spaces. However, as we saw in section~\ref{sec:22}, when $I$ satisfies
some mild ``local finiteness'' requirement (for instance when $I$ is
finite), we may associate to it a \emph{geometrical realization} $\abs
I$, which is a locally compact space (a compact one indeed if $I$ is
finite) endowed with a ``conical subdivision'' (index by the opposite
ordered set $I\op$), hence canonically triangulated via the
``barycentric subdivision''. The homotopy type of this space is
canonically isomorphic to the homotopy type of $I$, viewed as a
``model'' in \Cat. What is more important here, is that a (pre)sheaf
of sets (say) on the category $I$ may be interpreted as being
essentially the same as a sheaf of sets on the geometric realization
$\abs I$ \emph{which is locally constant} (\emph{and hence constant})
\emph{on each of the \emph{``open''} strata or ``cones'' of $I$}. This
description then carries over to sheaves of $k$-modules. The ``clue''
I had in mind is that \emph{within the context of locally finite
  ordered sets, the looked-for ``six operations duality formalism''
  should be no more, no less than the accurate reflection of the same
  formalism within the context\pspage{574} of \textup(locally
  compact\textup) topological spaces}, as worked out by Verdier in or
of his Bourbaki talks -- it being understood that when applying the
latter formalism to spaces endowed with conical stratifications, maps
between these which are compatible with the stratifications (in a
suitable sense which should still be pinned down), and to sheaves of
modules which are compatible with the stratifications in the sense
above, these will give rise (via the ``six operations'') to sheaves
satisfying the same compatibility. (NB\enspace when speaking of
``sheaves'', I really mean complexes of sheaves $K^\bullet$, and the
compatibility condition should be understood as a condition for the
ordinary sheaves of modules $\mathrm H^i(K^\bullet)$.)\enspace This
remark should allow to work out quite explicitly, in purely algebraic
(or ``combinatorial'') terms, the ``six operations'' in the context of
locally finite ordered sets, at any rate.

This interpretation suggests that an ordered set $I$ should be viewed
as a ``proper'' object of \Cat{} if{f} $I$ is finite. In the same
vein, whenever $I$ is ``locally proper'', namely locally finite, and
moreover its topological realization $\abs I$ is a ``smooth''
topological space in the usual sense, namely is a topological
\emph{variety} (for which it is enough that for every $x$ in $I$, the
topological realization
\begin{equation}
  \label{eq:137.7}
  \abs{I_{>x}}\tag{7}
\end{equation}
of the set of elements $y$ such that $y>x$ should be a sphere), we
would consider $I$ as a ``smooth'' object in \Cat. If $A$ is any
object in \Cat{} and $I$ is an ordered set which is finite resp.\
``smooth'' in the sense above, we will surely expect $A\times I\to A$
to be ``proper'' resp.\ ``smooth'' for the duality formalism we wish
to develop in \Cat.

It should be kept in mind that for the algebraic interpretation above
to hold, for sheaves on a conically stratified space $\abs I$ in terms
of an ordered set $I$, we had to take on the indexing set $I$ for the
strata the order relation \emph{opposite} to the inclusion relation
between (closed) strata -- otherwise, the correct interpretation of
sheaves constant on the open strata is via \emph{covariant} functors
$I\to\Sets$, i.e., \emph{presheaves on $I\op$} (not $I$). At any rate,
as there is a canonical homeomorphism
\begin{equation}
  \label{eq:137.8}
  \abs I\simeq\abs{I\op}\tag{8}
\end{equation}
respecting the canonical barycentric subdivisions of both sides,
notions for $I$ (such as properness, or smoothness) which are
expressed as intrinsic properties of the corresponding topological
space $\abs I$ (independently of its subdivision) are autodual -- they
hold for $I$ if{f} they do for $I\op$.\pspage{575} This is in sharp
contrast with the more naive notions of ``cohomological'' properness
and smoothness via base change operations, which are interchanged by
duality (which was part of yesterday's perplexities, now straightening
out\dots).

I feel I should be a little more outspoken about the relevant notion
of ``proper maps'' between ordered sets, which should be the algebraic
expression of the geometric notion alluded to above, of a map between
conically stratified spaces being ``compatible with the conical
stratifications''. In order for the corresponding direct image functor
for sheaves to take sheaves compatible with the stratification above
to sheaves of same type below, we'll have to insist that the image by
$f$ of a strata above should be a whole strata below. Now, it is clear
that any map
\begin{equation}
  \label{eq:137.9}
  f:I\to J\tag{9}
\end{equation}
between ordered sets takes flags into flags, and hence induces a map
\begin{equation}
  \label{eq:137.10}
  \abs f:\abs I \to \abs J\tag{10}
\end{equation}
between the geometric realizations, compatible with the barycentric
triangulations and thus taking simplices into simplices. But even when
$I$ and $J$ are finite (hence ``proper''), the latter map does not
always satisfy the condition above. Thus, when $I$ is reduced to just
one point, hence \eqref{eq:137.9} is defined by the image $j\in J$ of
the latter, the corresponding map \eqref{eq:137.10} maps the unique
point of $\abs I$ to the barycenter of the stratum $C_j$ in $\abs J$,
which is a stratum of $\abs J$ if{f} $C_j$ is a minimal stratum, i.e.,
$j$ is a \emph{maximal} element in $J$. A natural algebraic condition
to impose upon $f$, in order to ensure that $\abs f$ satisfy the
simple geometric condition above, is that for any $x$ in $I$, and any
$y'$ in $J$ such that
\[y'>y \eqdef f(x),\]
there should exist an $x'$ in $I$ satisfying
\begin{equation}
  \label{eq:137.11}
  x'>x \quad\text{and}\quad f(x')=y'.\tag{11}
\end{equation}
This condition strongly reminds us of the valuative criterion for
properness in the context of schemes, where the relation $y'\ge y$ or
$y\to y'$, say, should be interpreted as meaning that $y'$ is a
\emph{specialization} of $y$. However, in the valuative criterion for
properness (for a map of preschemes of finite type over a noetherian
prescheme say), if one wants actual properness\pspage{576} indeed
(including separation of $f$, not just that $f$ is universally
closed), one has to insist that the $x'$ above should be unique: every
specialization $y'$ of $y=f(x)$ lifts to a \emph{unique}
specialization $x'$ of $x$. If we applied this literally in the
present context, this would translate into the condition that $\abs f$
should map \emph{injectively} each closed stratum $C_x$ of $\abs I$ --
which would exclude such basic maps as the projection $I\to e$ to just
one point!

One may wonder why trouble about the analogy with algebraic geometry
and any extra condition on $f$ besides the one we got. The point,
however, is that we would like the map
\begin{equation}
  \label{eq:137.12}
  C_x=\abs{I_{\ge x}} \to C_y=\abs{I_{\ge y}}\tag{12}
\end{equation}
between corresponding strata induced by $\abs f$ to be
``cohomologically trivial'' in a suitable sense, not only surjective
-- in analogy,\footnote{analogy: aspheric fibers! \scrcommentinline{I
    \emph{think} that's what the footnote says; it's hard to
    read\dots}} say, with the usual notion of maps between
triangulated spaces; if we don't have some condition of this type, we
will have no control over the structure of the direct image and the
higher direct images $\mathrm R^i\abs f_*$ of an ``admissible'' sheaf
upstairs. I didn't really analyze the situation carefully, in terms of
what we are after here in the context of those stratified geometric
realizations. I feel pretty sure, though, that there the general
notion of ``cohomological properness'' which I worked out last Spring
fits in just right, to give the correct answer. The criterion I
obtained (necessary and sufficient for the relevant compatibility
property of $\mathrm R^if_*$ with arbitrary base change $J'\to J$ in
\Cat) reads as follows: let
\begin{equation}
  \label{eq:137.13}
  I(f,x,x') =\set{x'\in I}{\text{$x'>x$ and $f(x')=y'$}}\tag{13}
\end{equation}
be the subset of $I$ satisfying the conditions \eqref{eq:137.11}
above. Instead of demanding only that this set be non-empty, or going
as far as demanding that it should be reduced to just one point, we'll
demand that this set should be \emph{aspheric}. (When we are concerned
with cohomology with commutative coefficients only, presumably it
should be enough to demand only that this set be \emph{acyclic} --
which should be enough for the sake of a mere duality
formalism\dots)\enspace In more geometric terms, this should mean, I
guess, that the inverse image, by the map \eqref{eq:137.12} above
between closed strata, of any closed stratum below, should be
aspheric. I doubt this condition holds under the mere assumption that
the sets \eqref{eq:137.13} be non-empty, i.e., \eqref{eq:137.12} be
surjective -- I didn't sit down, though, to try and make an example.

Thus,\pspage{577} we see that when working with the notion of
properness of maps, even for such simple gadgets as finite ordered
sets, which should be viewed as ``proper'' (or ``compact'') objects by
themselves, this notion is far from being a wholly trivial one -- for
instance, it does not hold true that any map between such ``proper''
objects is again ``proper''. This now seems to me just a mathematical
``fact of life'', which we may not disregard when working with finite
ordered sets, say, in view of expressing in algebraic terms some
standard operations in the cohomology theory of sheaves on topological
spaces, endowed with suitable conical stratifications. The fault,
surely, is not with the notion of conical stratification itself, which
may be felt by some as being ad hoc, awkward and what not. I know the
notion is just right -- but at any rate, even when working with strata
which are perfect topological cells (so that nobody could possibly
object to them), exactly the same facts of life are there -- not every
map between such cellular decompositions, mapping cells \emph{onto}
cells, will fit into a ``combinatorial'' description when it comes to
describing the standard operations of the cohomology of sheaves, for
sheaves compatible with the stratifications\dots

To sum up, it seems to me that definitely, there is a very rich
experimental material available already at present, in order to come
to a feeling of what duality is like in the context of \Cat, and for
developing some of the basic intuitions needed for working out,
hopefully, ``the'' full-fledged duality theory which should hold in
\Cat. Before leaving this question, I would like to point out still
one other property connected with the intuitions around ``properness''
-- more generally, around maps ``of finite type'' in a suitable sense,
which sometimes may translate into: factorizable as a composition
$f=g\circ i$, with $g$ proper and $i$ an immersion. This is about
stability of ``constructibility'' or ``finiteness'' conditions for
(complexes of) sheaves of $k$-modules, with respect to the standard
operations $\mathrm Rf_*$, $\mathrm Rf_!$, $f^!$ (stability by $f^*$
being a tautology in any case). Such stability of course, whenever it
holds, is an important feature, for instance for making ``virtual''
calculations in suitable ``Grothendieck groups'' (where Euler-Poincaré
type invariants may be defined). It should be recalled, however, that
the six fundamental operations in the duality formulaire, as well as
nearly all of the formulaire itself, make sense (and formulæ hold
true) without any finiteness conditions on the complexes of sheaves we
work with, except just boundedness conditions on the degrees of those
complexes.

\begin{remarks}
  1)\enspace One\pspage{578} of the (rather few) instances in the
  duality formulaire where finiteness conditions are clearly needed,
  is the so-called ``biduality theorem'', when taking $\RbHom$'s with
  values in a so-called ``dualizing complex'', which here should be an
  object
  \[R^\bullet\quad\text{in $\D^{\mathrm b}(A\uphat_k)$}\]
  (for given $A$ in \Cat{} and given ring of coefficients $k$, for
  instance $k=\bZ$). The question arises here, for any given $A$ and
  $k$, a)\enspace whether there exists a dualizing complex $R^\bullet$
  (which, as usual, will be unique up to dimension shift and ``twist''
  by an invertible sheaf of $k$-modules on $A$), and b)\enspace can
  such dualizing complex be obtained as
  \[R^\bullet=p^!(k_e),\]
  where $p:A\to e$ is the projection to the final object of \Cat? It
  should be easy enough, once the basic duality formalism is written
  down for ordered sets, as contemplated above, to show that when $A$
  is a finite ordered set (or only locally finite of finite
  combinatorial dimension), then $p^!(e_k)$ is indeed a dualizing
  complex. To refer to something more ``en vogue'' at present than
  those poor dualizing complexes, it is clear, from what I heard from
  Illusie and Mebkhout about the ``complexe d'intersection'' for
  stratified spaces, that this complex can be described also in the
  context of locally finite ordered sets (if I got the stepwise
  construction of this complex right). As the construction here
  corresponds to stratifications where the strata are by no means
  even-dimensional, I am not too sure, though, if the complex obtained
  this way is really relevant -- it isn't a topological invariant of
  the topological space at any rate, independently of its
  stratification -- a bad point indeed. Too bad!

  2)\enspace Here is a rather evident example showing that the
  asphericity condition on \eqref{eq:137.13} is not automatic. Take
  $I$ with a smallest element $x$ (hence $C_x=\abs I$), and
  $J=\Simplex_1=(0\to 1)$. To give a map $I\to J$ amount to give
  $I_0=f^{-1}(0)\subset I$, which is any open subset of $I$ (i.e.,
  such that $a\in I_0$, $b\le a$ implies $b\in I_0$). If we take
  $I_0=\{x\}$, then properness of $f$ is equivalent with
  $I_1=I\setminus I_0=I\setminus\{x\}$ being aspheric, while the
  weaker condition contemplated first (the sets \eqref{eq:137.13}
  non-empty) means only that $I_1\ne\emptyset$. Now, $I_1$ may of
  course be taken to be any (finite say) ordered set -- the
  construction made amounting to taking the cone over the map
  $\abs{I_1}\to e$. In this examples, all fibers of $\abs f:\abs I\to
  \abs J$ are (canonically) homeomorphic to $I_1$, except the fiber at
  the ``barycenter'' $1$, which\pspage{579} is reduced to a point --
  visibly not a very ``cellular'' behavior when $\abs{I_1}$ isn't
  aspheric! At any rate, as the sheaves $\mathrm R^i\abs f_*(F)$ (for
  $F$ a constant sheaf above, say) may be computed fiberwise, we see
  that if $I_1$ isn't aspheric, these sheaves (which are constant on
  $\abs J\setminus\{1\}$) are \emph{not} going to be constant on $\abs
  J\setminus\{0\}$, as they should if we want an algebraic paradigm of
  operations like $\mathrm R\abs f_*$ in terms of sheaves on finite
  ordered sets.

  In this example we could take $\abs I$ to be a perfect $n$-cell
  ($n\ge1$), hence $I_1$ is an $(n-1)$-sphere, whereas $\abs J$ isn't
  really a (combinatorial) $1$-cell, as its boundary has just
  \emph{one} point $0$, instead of two. The $1$-cell structure
  corresponds to the ordered set (\emph{opposite} to the ordered set
  formed by the two endpoints and the dimension $1$ stratum)
  \[J =
    \begin{tikzcd}[row sep=tiny,column sep=small,cramped]
      & y\ar[dl]\ar[dr] & \\
      0 & & 1
    \end{tikzcd}.\]
  For any ordered set $I$, to give a map $f:I\to J$ amounts to giving
  the two subsets
  \[I_0=f^{-1}(0), \quad I_1=f^{-1}(1),\]
  subject to the only condition of being open and disjoint. In terms
  of strata of $\abs I$, this means that we give two sets $I_0,I_1$ of
  strata, containing with any stratum any smaller one, and having no
  stratum in common. The condition that the sets \eqref{eq:137.13}
  should be non-empty says that any point in $I$ which is neither in
  $I_0$ nor $I_1$ admits majorants in both -- or geometrically, any
  stratum which isn't in $I_0$ nor in $I_1$ meets both $\abs{I_0}$ and
  $\abs{I_1}$, i.e., contains strata which are in $I_0$ and strata
  which are in $I_1$. Even when $\abs I$ is a combinatorial $2$-cell,
  i.e., a polygonal disc, this condition does not imply asphericity of
  the sets \eqref{eq:137.13} (not even $0$-connectedness). To see
  this, we take the set \eqref{eq:137.13} where $x$ is the dimension
  $2$ stratum, i.e., the smallest element of $I$, mapped to $y$ (the
  smallest in $J$), and $y'$ either $0$ or $1$. The reader will easily
  figure out on a drawing the structure of the map $\abs f$, as I just
  did myself: the fibers at the endpoints of the segment $\abs J$ are,
  as given, discrete with cardinal $m$, the fiber at a point different
  from the endpoints and from their barycenter are disjoint sums of
  $m$ segments (hence homotopic to the former fibers), whereas the
  fiber at the barycenter is the union of $m$ segments meeting in
  their\pspage{580} common middle, hence is contractible. Thus, the
  $\mathrm R^0\abs f_*$ of a constant sheaf on the disc $\abs I$ is by
  no means constant on the open, dimension one stratum of $\abs J$.

  These examples bring to my mind that for any map $f:I\to J$ between
  finite ordered sets, possibly submitted to the mild restriction that
  the sets \eqref{eq:137.13} should be non-empty, the homotopy types
  of those ordered sets \eqref{eq:137.13} (for the order relation
  induced by $I$) should be exactly the homotopy types of the fibers
  of the maps \eqref{eq:137.12} between strata. Thus, asphericity of
  these ordered sets should express no more, no less than the
  contractibility of those fibers. This latter condition is exactly
  what is needed in order to ensure stability by $\mathrm R\abs f_*$
  of the notion of complexes of sheaves compatible with the
  stratifications. 
\end{remarks}

\bigbreak

\noindent\hfill\ondate{28.10.}\par

% 138
\hangsection[Niceties and oddities: $\mathrm Rf_!$ commutes to
\dots]{Niceties and oddities: \texorpdfstring{$\mathrm Rf_!$}{Rf!}
  commutes to colocalization, not localization.}\label{sec:138}%
Yesterday and the day before, I got involved in that unforeseen
digression around the foreboding of a ``six operations duality
formalism'' in \Cat, and suitable notions for smoothness and (more
important still) of properness for a map in \Cat. This digression
wholly convinced me that the usual duality formalism should hold in
\Cat{} too. Working this out in full detail should be a most pleasant
task indeed, and presumably the best, or even the only way for gaining
complete mastery of the cohomology formalism within \Cat{} or, what
more or less amounts to the same, for topoi which admit sufficiently
many projective objects. In the previous two sections, I referred to
such duality formalism as one concerned with sheaves of $k$-modules,
for any fixed ring $k$ -- but from the example of étale duality for
schemes, say, it is likely that instead of fixing a ring $k$, we may
as well take objects $A$ in \Cat{} endowed with an arbitrary sheaf of
rings $\scrO_A$ (which we'll only have to suppose commutative when
concerned with the two internal operations $\Lotimes$ and $\RbHom$),
and taking maps of such ringed objects as the basic maps. In the
present context, the usual ``six operations'' in duality theory will
be enriched, however, by still another one, namely $\mathrm Lf_!$,
defined for any map $f$ between such ringed objects (not to be
confused with the $\mathrm Rf_!$ operation, defined only under
suitable finiteness assumptions, such as ``properness'', for the
underlying map in \Cat), whose relationship to the other operations
should be understood and added to the standard duality
formulaire. One\pspage{581} such formula, namely the precise relation
between $\mathrm Lf_!$ and $\mathrm Rf_!$ for a smooth $f$, was given
yesterday (p.~\ref{p:572} \eqref{eq:137.6}).

The day before, I was out for trying to get a better understanding of
the $\mathrm Rf_!$ operation (including the case of non-constant
sheaves of rings for the modules we work with). This operation still
remains unfamiliar to me, very unlike my old friend $\mathrm Rf_*$ --
there is a number of things about it which are not quite clear yet in
my mind, even when just taking the functor $f_!$ between modules,
before taking a left derived functor. For instance, for general rings
of operators $\scrO_A$ and $\scrO_B$, when $f$ is a ringed map
\begin{equation}
  \label{eq:138.1}
  f: (A,\scrO_A) \to (B,\scrO_B),\tag{1}
\end{equation}
it doesn't seem that formation of $f_!$ commutes to restriction of
rings of operators (to the constant rings $\bZ_A$ and $\bZ_B$, say),
namely that for a given $\scrO_A$-module $F$, $f_!(F)$ may be
interpreted as just ${f_0}_!\supab(F)$ with suitable operations of
$\scrO_B$ on the latter, where
\begin{equation}
  \label{eq:138.2}
  f_0:A\to B\tag{2}
\end{equation}
is the map in \Cat{} underlying $f$. When reducing to a suitable
``universal'' case, this may be expressed by saying that for given map
$f_0$ and given abelian sheaf $F$ on $A$, if we define
\[\scrO_A = \bEnd_\bZ(F), \quad \scrO_B=f_*(\scrO_A),\]
there doesn't seem to be a natural operation of $\scrO_B$ upon
${f_0}_!\supab(F)$; all we can say, it seems, is that the ring of
global sections of $\scrO_B$ operates on ${f_0}_!\supab(F)$. More
generally, reverting to the general case of a map $f$ of ringed
objects in \Cat, the ring of global sections of $\scrO_B$ operates on
${f_0}_!\supab(F)$. When $\scrO_B$ is a constant sheaf of rings $k_B$,
this implies of course that $k$ operates on this abelian sheaf on $B$,
from which will follow by an obvious argument that with this structure
of a sheaf of $k$-modules, ${f_0}_!\supab(F)$ may indeed be identified
with $f_!(F)$.

Going over to $\mathrm Lf_!$ which we would like to express via
$\mathrm L{f_0}_!\supab$, the situation is worse, as we still have to
check (granting $\scrO_B$ to be constant) that for $F$ a projective
$\scrO_A$-module, we got
\begin{equation}
  \label{eq:138.3}
  \mathrm L_i{f_0}_!\supab(F)=0\quad\text{for $i>0$.}\tag{3}
\end{equation}
This isn't always true, even when $B$ is the final object in \Cat, and
$A$ has a final object, hence $\scrO_A$ is a projective module
over\pspage{582} itself, and \eqref{eq:138.3} reads
\[\mathrm H_i(A,\scrO_A) =0 \quad\text{for $i>0$,}\]
which isn't always true. (NB\enspace if it were for any commutative
ring $\scrO_A$ on $A$, it would be too for any abelian sheaf $M$ on
$A$, as we see by taking $\scrO_A=\bZ_A\otimes M$, hence $A$ would be
homological dimension $0$, a drastic restriction, indeed, even when
$A$ has a final object.)

When however $\scrO_A$ is equally a constant sheaf of rings, say
$\scrO_A=k'_A$, then the relation \eqref{eq:138.3} holds for any
projective module on $A$. We need only check it for
\[F=\scrO_A^{(a)} = {k'}^{(a)},\]
with $a$ in $A$, then \eqref{eq:138.3} follows from
\begin{equation}
  \label{eq:138.4}
  \mathrm L_i {f_0}_!\supab( M^{(a)} )=0 \quad
  \text{for $i>0$, $M$ in \Ab,}\tag{4}
\end{equation}
To check \eqref{eq:138.4}, we take a projective resolution of
$M^{(a)}$, by using a projective resolution $L_\bullet$ of $M$ in
\Ab{} and taking $L_\bullet^{(a)}$ (using the fact that the functor
\[L\mapsto L^{(a)}:\Ab\to\Ahatab\]
is exact). We then get
\[\mathrm L{f_0}_!(M^{(a)}) = {f_0}_! (L_\bullet^{(a)}) =
  L_\bullet^{(b)} \simeq M^{(b)},\]
where $b=f_0(a)$, and where the last equality stems from exactness of
$L\mapsto L^{(b)}$. (NB\enspace the relation \eqref{eq:138.4}
generalizes a standard acyclicity criterion in the homology theory of
discrete groups\dots)

Thus, we get finally that in case both rings $\scrO_A$ and $\scrO_B$
are constant, that formation of $\mathrm Lf_!$ commutes to restriction
of operator rings (provided the rings to which we are restricting are
constant too -- say they are just the absolute $\bZ_A$ and
$\bZ_B$). Presumably, a little more care should show the similar
result for locally constant rings.

Reverting to the case \eqref{eq:138.1} of a general map between ringed
objects in \Cat, our inability, for a given $\scrO_A$-module $F$ on
$A$, to define an operation of $\scrO_B$ upon ${f_0}_!\supab$ (while
there is an operation of the ring $\Gamma(B,\scrO_B)$ upon it), is
tied up with this difficulty, that formation of ${f_0}_!\supab$, and a
fortiori of $f_!$ for arbitrary rings $\scrO_A$ and $\scrO_B$, does
not commute to ``localization'' (as $f_*$ and $\mathrm Rf_*$ does),
namely to base change of the type\pspage{583}
\begin{equation}
  \label{eq:138.5}
  B_{/b}\to B,\tag{5}
\end{equation}
where $b$ is a given object in $B$, and $B_{/b}$ designates as usual
the category of all ``objects over $b$'' in $B$, i.e., of all arrows
in $B$ with target $b$. This gives rise to the cartesian square in
\Cat{}
\begin{equation}
  \label{eq:138.6}
  \begin{tabular}{@{}c@{}}
    \begin{tikzcd}[baseline=(O.base)]
      A_{/b}\ar[d]\ar[r] & A\ar[d] \\
      B_{/b}\ar[r] & |[alias=O]| B
    \end{tikzcd},
  \end{tabular}\tag{6}
\end{equation}
where $A_{/b}$ is the category of all pairs
\[\text{$(a,u)$ with $u:f_0(a)\to b$,}\]
which may be identified equally with the category $A_{/f_0^*(b)}$. The
commutation property we have in mind is a tautology for $f_*$, and it
follows for $\mathrm Rf_*$, because the inverse image by $A_{/b}\to A$
of an injective module on $A$ is an injective module on $A_{/b}$. The
latter fact is true, more generally, for any ``localization map'', of
the type $A_{/X}\to A$, with $X$ in \Ahat, i.e., any map which is
fibering with discrete fibers. Thus, the commutation property for
$\mathrm Rf_*$ is valid more generally for any base change of the type
\[B_{/Y}\to B,\]
with $Y$ in \Bhat{} (not necessarily in $B$). More generally still, it
can be shown to hold for any map
\[B'\to B\]
which is fibering (not necessarily with discrete fibers).

On the other side of the mirror, when taking $f_!$ and its left
derived functor, already the former definitely does \emph{not} commute
to base change of the type \eqref{eq:138.5}, i.e., to ``localization
on the base'' -- something a little hard to get accustomed to! The
base changes which will do here, are those of the dual type
\begin{equation}
  \label{eq:138.5prime}
  \preslice Bb\to B,\tag{5'}
\end{equation}
we may call them maps of ``colocalization'' on the base $B$. It gives
rise to a cartesian diagram in \Cat{} dual to \eqref{eq:138.6}
\begin{equation}
  \label{eq:138.6prime}
  \begin{tabular}{@{}c@{}}
    \begin{tikzcd}[baseline=(O.base)]
      \preslice Ab\ar[d]\ar[r] & A\ar[d] \\
      \preslice Bb\ar[r] & |[alias=O]| B
    \end{tikzcd},
  \end{tabular}\tag{6'}
\end{equation}
where\pspage{584} now $\preslice Ab$ is the category of pairs
\[\text{$(a,u)$ with $u:b\to f_0(a)$.}\]
We'll have to assume now that both shaves of rings $\scrO_A$,
$\scrO_B$ are constant, and correspond to the same ring $k$. Thus,
modules on $A$ and $B$ are just contravariant functors from these
categories to the category $\AbOf_k$ of $k$-modules, and accordingly,
$f_!$ (left adjoint to the composition functor $f^*$) may be computed
by a well-known formula, involving direct limits on the categories
$\preslice Ab$:
\begin{equation}
  \label{eq:138.7}
  f_!(F)(b) \simeq\varinjlim_{\preslice Ab} F(a),\tag{7}
\end{equation}
where the limit in the second member is relative to the
composition\scrcomment{It seems to me that we're forgetting
  that $F$ is contravariant. Should be easily fixed by inserting ${}\op$'s,
  though\dots}
\[\preslice Ab\to A\xrightarrow F{} \AbOf_k.\]
(NB\enspace This formula is dual to the formula for $f_*$, the right
adjoint of $f^*$, closer to intuition -- to mine at any rate --
because the $\varprojlim$ in
\[f_*(F)(b) \simeq \varprojlim_{A_{/b}} F(a)\]
may be ``visualized'' as the set of ``sections'' of $F$ over
$A_{/b}$.)\enspace From this formula \eqref{eq:138.7} follows at once
commutation of $f_!$ with colocalization. To get the corresponding
result for $\mathrm Lf_!$, we have only to use the fact that the
relevant inverse image functor (corresponding to $\preslice Ab\to A$)
transforms projective modules into projective modules. As the map
$\preslice Bb\to B$ is a \emph{cofibering functor with discrete
  fibers}, so is the map
\[h:A'=\preslice Ab\to A\]
deduced by base change. Now, for any such functor $h:A'\to A$ between
small categories, the inverse image functor
\[h^*:\Ahat \to {A'}\uphat, \quad\text{or}\quad
  h_k^*:A\uphat_k \to {A'}\uphat_k\]
carries indeed projectives into projectives. This statement is dual
formally to the corresponding statement for injectives, valid when we
make on $h$ the dual assumption of being \emph{fibering with discrete
  fibers} -- in the latter case the (well-known) proof comes out
formally from the fact that the left adjoint functor $h_!$ or $h_!^k$
carries monomorphisms into monomorphisms -- a fact that we used in
section~\ref{sec:135} in the form $\mathrm Lh_!^k=h_!^k$ (in case
$k=\bZ$). In the present case, the proof is essentially the dual one
-- as a matter of fact, as there are enough\pspage{585} projectives,
the statement about $h^*$ or $h_k^*$ taking projectives into
projectives is equivalent with the right adjoint $h_*$ or $h_*^k$ (for
sheaves of sets, resp.\ sheaves of $k$-modules) transforming
epimorphisms into epimorphisms, which in the case of $k$-modules can
be written equally under the equivalent form
\begin{equation}
  \label{eq:138.8}
  \mathrm R^ih_*=0 \quad\text{for $i>0$.}\tag{8}
\end{equation}
Now, this exactness property for $h_*$, in the case when $h$ is
cofibering with discrete fibers, is I guess well known (it is
well-known to me at any rate), and comes from the specific computation
of $h_*(F)$ for $F$ in ${A'}\uphat$, valid whenever $h$ is
\emph{cofibering} (with arbitrary fibers)
\begin{equation}
  \label{eq:138.9}
  h_*(F)(x) \simeq \Gamma(A'_x, F\restrto A'_X)\quad
  \text{for $x$ in $A$,}\tag{9}
\end{equation}
where $A'_X$ is the fiber of $A'$ over $x$ (a category not to be
confused with $A'_{/x}$, the two being closely related,
however\dots). In case the fibers of $h$ are discrete, the right-hand
side of \eqref{eq:138.9} may be written as a product, hence the
formula
\begin{equation}
  \label{eq:138.10}
  h_*(F)(x) \simeq \prod_{\text{$x'$ in $A'_X$}} F(x'),\tag{10}
\end{equation}
(which may be viewed as the dual of the formula \eqref{eq:135.11}
p.~\ref{p:564}). As in the category of sets (and hence also in
$\AbOf_k$) a product of epimorphisms is again an epimorphism, the
result we want follows indeed.

\begin{remarks}
  1)\enspace The results just given, as well as their proofs,
  concerning inverse images of injectives or projectives, are valid
  not only in the case of a common constant sheaf of rings on $A$ and
  $A'$, but more generally for any sheaf of rings $\scrO_A$ on $A$,
  when taking on $A'$ the ``induced'' sheaf of rings
  \begin{equation}
    \label{eq:138.11}
    \scrO_{A'} = h^*(\scrO_A).\tag{11}
  \end{equation}
  However, it doesn't seem that the result about commutation of
  $\mathrm Lf_!$ to colocalization is valid under the corresponding
  assumption $\scrO_A=f^*(\scrO_B)$, without assuming moreover
  $\scrO_B$ to be constant (hence $\scrO_A$ too), because already for
  the functor $f_!$ itself for modules it doesn't seem that
  commutation will hold.

  2)\enspace I should correct as silly mistake I made at the very
  beginning of this section, when rashly stating that the functor
  $\mathrm Lf_!$ may be defined for \emph{any} map \eqref{eq:138.1}
  between ringed objects in \Cat. I was thinking of the fact that for
  any ringed object $(A,\scrO_A)$ in \Cat, there are indeed enough
  projectives in the category of $\scrO_A$-modules --\pspage{586}
  thus, the modules
  \begin{equation}
    \label{eq:138.12}
    \scrO_A^{(a)}, \quad\text{for $a$ in $A$,}\tag{12}
  \end{equation}
  are clearly projective, and there are ``sufficiently many''. Thus,
  any additive functor from $\Mod(\scrO_A)$ to an abelian category
  admits a total left derived functor. However, it is not always true,
  for a map \eqref{eq:138.1} of small ringed categories, that the
  corresponding inverse image functor for modules
  \[G\mapsto f^*(G) = {f_0}^*(G) \otimes_{{f_0}^*(\scrO_B)} \scrO_A\]
  admits a left adjoint $f_!$, or what amounts to the same, that this
  functor (which is right exact) is left exact and commutes to small
  products. It isn't even true, necessarily, when we assume $A$ and
  $B$ to be the final category! Left exactness of $f^*$ just means
  flatness of $\scrO_A$ over $\scrO_B$, i.e., that for any $a$ in $A$,
  $\scrO_A(a)$ is flat as a module over $\scrO_B(b)$, where
  $b=f(a)$. As for commutation to small products, it amounts (together
  with the first condition) to the still more exacting condition that
  $\scrO_A(a)$ should be a \emph{projective module of finite type over
    $\scrO_B(b)$}, for any $a$ in $A$. This condition is so close to
  the condition $\scrO_A={f_0}^*(\scrO_B)$ (already considered in
  \eqref{eq:138.11} above), that for a bird's eye view as we are
  aiming at here, we may as well assume this slightly stronger
  condition! Anyhow, as noticed before, to get commutation of $f_!$
  with restriction of scalars and with colocalization, even this
  assumption isn't enough, apparently, and we'll have to assume
  moreover that $\scrO_B$ (hence also $\scrO_A$) is constant, or for
  the least, locally constant.

  This brings us back to the case when a fixed ring $k$ is given, and
  when we are working with categories of presheaves of $k$-modules --
  a situation studied at length in chapter~\ref{ch:V}. In case the
  target category $B$ is the final category, hence $\mathrm Lf_!$ is
  just (absolute) total homology of $A$, this then may be computed
  nicely, using an ``integrator'' for $A$, namely a projective
  resolution of $k_{A\op}$ in $(A\op)\uphat_k$. This is more or less
  where we ended up by the end of chapter~\ref{ch:V}, when developing
  a nicely autodual homology-cohomology set-up, replacing the category
  of $k$-modules $\AbOf_k$ by a more-or-less arbitrary abelian
  category. I was about to go on and carry through a similar treatment
  in the ``relative'' case, namely for an arbitrary map $f$ in \Cat{}
  (but then I got caught unsuspectingly by that unending digression on
  schematic homotopy types, finally making up a whole chapter by
  itself). Maybe it is still worthwhile to come back to
  this\pspage{587} without necessarily grinding through a complete
  formulaire for the five main operations we got so far (namely
  $\mathrm Lf_!$, $f^*$, $\mathrm Rf_*$, $\Lotimes$, $\RbHom$). Not
  later than two pages ago or so, we were faced again with two visible
  dual statements, one about $\mathrm Lf_!$, the other about $\mathrm
  Rf_*$ -- and feeling silly not to be able to merely deduce one from
  the other!
\end{remarks}

\bigbreak

\noindent\hfill\ondate{29.10.}\par

% 139
\hangsection[Retrospective on ponderings on abelianization, and on
\dots]{Retrospective on ponderings on abelianization, and on coalgebra
  structures in \texorpdfstring{\Cat}{(Cat)}.}\label{sec:139}%
Last night I still pondered a little about the $\mathrm Lf_!$
operation, and did some more reading in the notes of
chapter~\ref{ch:V} of about three months ago, which had been getting a
little distant in my mind. In those notes a great deal of emphasis
goes with the notions of integrators and cointegrators -- as a matter
of fact, that whole chapter sprung from an attempt to understand the
meaning of certain ``standard complexes'' associated with standard
test categories such as $\Simplex$, which then led us to the notions
of an integrator (via the intermediate one of an
``abelianizator''). This in turn brought us to the $*_k$ formalism,
expressing most conveniently the relationship between categories such
as $A\uphat_k$ and $(A\op)\uphat_k$ ($k$ any commutative ring), and
its various avatars. I was so pleased with this formalism and its
``computational'' flavor, that in my enthusiasm I subsumed under it
the dual treatment of homology and cohomology for an object $A$ of
\Cat{} (with coefficients in a complex of \scrM-valued presheaves,
\scrM{} being an abelian category satisfying some mild conditions), in
section~\ref{sec:108}, as a particular case of the total derived
functors $F_\bullet \Last_k L_\bullet'$ and
$\RHom_k(L_\bullet,K^\bullet)$ (with values in the derived category of
\scrM). The main point here was using \emph{projective resolutions} of
the argument $L_\bullet$ in $B\uphat_k$ or $A\uphat_k$ (where
$B=A\op$), which in the most important case was just the constant
sheaf of rings $k_B$ or $k_A$ -- rather than resolve the argument
$F_\bullet$ (projectively) or $K^\bullet$ (injectively), namely the
coefficients for homology or cohomology. It was the enthusiasm of the
adept of a game he just discovered -- I was going to try it out for
the next step, namely relative homology and cohomology $\mathrm Lf_!$
and $\mathrm Rf_*$, with $f$ any map in \Cat{} -- but then I got
caught by the more fascinating schematization game. Coming back now
upon the rather routine matter of looking up a comprehensive mutually
dual treatment of $\mathrm Lf_!$ and $\mathrm Rf_*$, it doesn't seem
that the formalism of integrators and cointegrators is going to be of
much help. To be more specific, in order to compute (or simply define)
$\mathrm Lf_!(F_\bullet)$\pspage{588} or $\mathrm Rf_*(K^\bullet)$,
for a general map
\[f:A\to B\]
in \Cat, and $F_\bullet\in\D^-(A\uphat_k)$,
$K^\bullet\in\D^+(A\uphat_k)$, I do not see any means of bypassing
projective resolutions of $L_\bullet$, injective ones of
$K^\bullet$. Taking the more familiar case of $\mathrm
Rf_*(K^\bullet)$, one natural idea of course would be to take a
projective resolution $L_\bullet^A$ of $k_A$ (i.e., a cointegrator for
$A$), and write tentatively
\begin{equation}
  \label{eq:139.1}
  \mathrm Rf_*(K^\bullet) \overset{\text{?}}{\simeq}
  f_*(\bHom_k^{\bullet\bullet}(L_\bullet^A,K^\bullet).\tag{1}
\end{equation}
The (misleading) reflex inducing us to write down this formula, is
that this formula looks as if it were to boil down to the similar
(correct) formula for the maps $A_{/b}\to e$, when taking the
localizations on $B$,
\[B_{/b}\to B.\]
For this intuition to be correct, it should be true that the
restriction (or ``localization'') of $L_\bullet^A$ to $A_{/b}$ (which
is of course a resolution of the constant sheaf $k$ on $A_{/b}$) is
indeed a cointegrator on $A_{/b}$, namely that its components are
still projective. This, however, we suspect, will hold true only under
very special assumptions -- as in general, it is inverse image by
\emph{colocalization} (not by localization) that takes projectives
into projectives. Thus, I don't expect a relation \eqref{eq:139.1} to
hold, except under most exacting conditions on $f$ and $K^\bullet$,
which I didn't try to pin down. To take an example, assume $A$ has a
final object, hence $k_A$ is projective and we may take
$L_\bullet^A=k_A$, then \eqref{eq:139.1} reads (when $K^\bullet=K$ is
reduced to degree zero)
\[\mathrm Rf_*(K) \simeq f_*(K), \quad
  \text{i.e., $\mathrm R^if_*(K)=0$ for $i>0$,}\]
which need not hold true even if $K$ is constant ($=k_A$ say). For
instance, we may start with an arbitrary object $A_0$ of \Cat{} and
add a final object $e_1$ to get $A$ (intuitively, it is the ``cone''
over $A_0$), which is mapped into the cone over $e$,
$B=\Simplex_1=(0\to 1)$, in the obvious way. Then for any sheaf $K$ on
$A$, with restriction $K_0$ to $A_0$, we get
\[\mathrm R^if_*(K)_0\text{ (fiber at $0$) } = \mathrm H^i(A_0,K_0),\]
which needs not be $0$ for $i>0$.

There is a big blunder at the end of section~\ref{sec:101}, where the
formulæ \eqref{eq:101.12}, \eqref{eq:101.12prime}, \eqref{eq:101.13}
(p.~\ref{p:377}) (supposed to be ``essentially trivial''), are false
for essentially the same kind of reason. The formulæ ran into the
typewriter as a matter of course, as\pspage{589} they looked just the
same as familiar ones from the standard duality formulaire (with
$\mathrm Lf_!$, $f^*$ replaced by $\mathrm Rf_!$, $f^!$). The first
one reads, in case as above when $A$ has a final object $\varepsilon$
\begin{equation}
  \label{eq:139.2}
  \mathrm Rf_*(f^*(K)) \overset{\text{?}}{\simeq}
  \bHom_k(f_!(k_A),K),\tag{2}
\end{equation}
But we have 
\[f_!(k_A)=f_!(k^{(\varepsilon)}) = k^{(b)} , \quad
  \text{where $b=f_!(\varepsilon)=f(\varepsilon)$,}\]
and if $f$ takes final object into final object, we thus get
$f_!(k_A)=k_B$, and the right-hand side of \eqref{eq:139.2} is just
$K$, and hence \eqref{eq:139.2} implies
\[\mathrm R^if_*(f^*(K)) = 0 \quad \text{for $i>0$,}\]
which, however, needs not be true even for $K=k_B$, as we saw with the
previous example.

To come back to the general $\mathrm Lf_!$, $\mathrm Rf_*$ formalism,
it seems it can't be helped, we'll have to do the usual silly thing
and just resolve the argument involved, projectively in one case,
injectively in the other. Even in the ``absolute case'' when $B=e$, we
couldn't help it, either taking such resolutions, when it comes to
working with the internal operations $\RbHom_k(L_\bullet,K^\bullet)$
or $F_\bullet\Lotimes_k L_\bullet'$, as we saw already in
section~\ref{sec:108} (which should have tuned down a little my
committedness to integrators, but it didn't!).

Once this got clear, in order to turn the homological algebra mill,
all we still need is a handy criterion for existence of ``enough''
projectives or injectives in categories of the type
\[ \AhatM = \bHom(A\op, \scrM),\]
with \scrM{} an abelian category. In section~\ref{sec:109} we got such
a criterion (prop.~\ref{prop:109.4} p.~\ref{p:433}) -- the simplest
one can imagine: it is sufficient that (besides the stability under
direct or inverse limits, needed anyhow in order for the functor $f_!$
resp.\ $f_*$ from \AhatM{} to \BhatM\pspage{590} to exist) that in
\scrM{} there should be enough projectives resp.\ injectives! This
gives what is needed, surely, in order to grind through a mutually
dual treatment of relative homology $\mathrm Lf_!$ and cohomology
$\mathrm Rf_*$. I have the feeling that the little work ahead, for
defining the basic operations and working out the relevant formulaire
(including ``cap products''), is a matter of mere routine, and I don't
really expect any surprise may come up. Therefore, I don't feel like
grinding through this, and rather will feel free to use whenever
needed the most evident formulæ, as the adjunction formulæ between the
three functors
\[ \mathrm Lf_!, \quad f^*, \quad \mathrm Rf_*,\]
transitivity isomorphisms for a composition of maps, possibly also
various projection formulæ -- trying however to be careful with base
change questions, notably for $\mathrm Lf_!$, and not repeat the same
blunders!

In retrospect, the main role for me of the reflections of
chapter~\ref{ch:V} on abelianization has been to become a little more
familiar with the \emph{homology} formalism in \Cat, namely
essentially with the $\mathrm Lf_!$ operation, which has been more or
less a white spot in my former experience, centered rather upon
cohomology. An interesting, still somewhat routine byproduct of these
ponderings has been the careful formulation of the duality
relationship around the pairings between sheaves and co-sheaves,
namely the operations $*_k$ and $\oast_k$. Granting this, the game
with integrators and cointegrators boils down to the standard reflex,
of taking projective resolutions of the ``unit'' sheaf or cosheaf,
$k_A$ or $k_{A\op}$ -- which are the most obvious objects at hand of
all, in our coefficient categories.

The one idea which seems to me of wider scope and significance in this
whole reflection on abelianization, is the ``further step in
linearization'', whereby the models in \Cat{} are replaced by their
$k$-additive envelope, endowed moreover with their natural
\emph{diagonal map} (section~\ref{sec:109}). The psychological effect
of this discovery has been an immediate one -- it triggered at once
the reflection on schematization of chapter~\ref{ch:VI}. This
reflection, apparently, took me into a rather different direction from
those ``$k$-coalgebra structures in \Cat'', which had reawakened and
made more acute the feeling that homotopy types should make sense
``over any ground ring''. Coming back now to the homology and
cohomology formalism within \Cat, it remains\pspage{591} a very
striking fact indeed for me, that as far as I can see at present, all
basic operations in the (commutative) homology and cohomology
formalism in \Cat, and their basic properties and interrelations,
should make sense for these linearized objects, which therefore may be
viewed as more perfect carriers still than small categories for
embodying the relevant formalism. To what extent this feeling is
indeed justified, cannot of course be decided beforehand -- only
experience can tell. For instance, does the ``six operations duality
formalism'' contemplated lately for \Cat{} (sections~\ref{sec:136},
\ref{sec:137}), which has still to be worked out, carry over to this
wider, linearized set-up? This will become clearer when the relevant
notions in \Cat{} are understood, so that it will become a meaningful
question whether for a map in \Cat, the property of being proper, or
smooth, or an immersion, may be read off in terms of properties of the
corresponding map of coalgebra structures in \Cat. And what about
subtler types of cohomology operations, such as the Steenrod
operations (which, I remember, may be defined in the context of
cohomology with coefficients in general sheaves of $\bF_p$-modules)?

When concerned with homology and cohomology formalism in \Cat,
involving general sheaves of coefficients, not merely locally constant
ones, we are leaving, strictly speaking, the waters of a reflection on
``homotopy models''. The objects of \Cat{} now are no longer viewed as
mere models for homotopy types, but rather, each one as defining a
topos, with the manifold riches it carries; a richness similar to the
one of a topological space, almost all of which is being stripped off
when looking at the mere homotopy type -- including even such basic
properties as dimension, compactness, smoothness, cardinality and the
like. When passing from an object in \Cat{} to the corresponding
coalgebra structure in \Cat, much of this richness is preserved --
maybe everything, indeed, which can be expressed in terms of
\emph{commutative} sheaves of coefficients. As for the realm of
non-commutative cohomology formalism (which is supposed to be the main
these of these notes, with overall title ``Pursuing Stacks''!), it
doesn't seem, not at first sight at any rate, that much of this could
be read off the enveloping coalgebra $P=\Add_k(A)$ of a given object
$A$ in \Cat{} -- except of course in the case when $A$ can be
recovered in terms of $P$, maybe as the category of ``exponential''
pairs $(x,u)$, where $x$ is in $P$ and
\[ u:\delta(x) \simeq x \otimes_k x\]
an isomorphism.

\clearpage

\noindent\hfill\ondate{4.11.}\pspage{592}\par

% 140
\hangsection{The meal and the guest.}\label{sec:140}%
Life keeps pushing open the doors of that well-tempered hothouse of my
mathematical reflections, as a fresh wind and often an impetuous one,
sweeping off the serene quietness of abstraction, -- a breath rich
with the manifold fragrance of the world we live in. This is the world
of conflict, weaving around each birth and each death and around the
lovers' play alike -- it is the world we all have been born into
without our choosing. I used to see it as a stage -- the stage set for
our acting. Our freedom (rarely used indeed!) includes choosing the
role we are playing, possibly changing roles -- but not choosing or
changing the stage. It doesn't seem the stage ever changes during the
history of mankind -- only the decors kept changing. More and more,
however, over the last years, I have been feeling this world I am
living in, the world of conflict, somehow as a \emph{meal} -- a meal
of inexhaustible richness. Maybe the ultimate fruit and meaning of all
my acting on that stage, is that parts of that meal, of that richness,
be actually eaten, digested, assimilated -- that they become part of
the flesh and bones of my own being. Maybe the ultimate purpose of
conflict, so deeply rooted in every human being, is to be the raw
material, to be eaten and digested and changed into understanding
about conflict. Not a collective ``understanding'' (I doubt there is
such a thing!), written down in textbooks or sacred books or whatever,
nor even something expressed or expressible in words necessarily --
but the kind of immediate knowledge, rather, the walker has about
walking, the swimmer about swimming, or the suckling about milk and
mother's breasts. My business is to be a learner, not a teacher --
namely to allow this process to take place in my being, letting the
world of conflict, of suffering and of joy, of violence and of
tenderness, enter and be digested and become knowledge about myself.

I am not out, though, to write a ``journal intime'' or meditation
notes, so I guess I better get back to the thread of mathematical
reflection where I left it, rather than write allusively about the
events of these last days, telling me about life and about myself
through one of my children.

\bigbreak

\noindent\hfill\ondate{12.11.}\par

Maybe at times I like to give the impression, to myself and hence
others, that I am the easy learner of things of life, wholly
relaxed,\pspage{593} ``cool'' and all that -- just keen for learning,
for eating the meal and welcome smilingly whatever comes with its
message, frustration and sorrow and destructiveness and the softer
dishes alike. This of course is just humbug, an image
d'Épinal\scrcomment{the Épinal prints being proverbial for a naive
  depiction, showing only good aspects\dots} which at whiles I'll kid
myself into believing I am like. Truth is that I am a hard learner,
maybe as hard and reluctant as anyone. At any rate, the inbuilt
mechanisms causing rejection of the dishes unpalatable to my wholly
conditioned, wholly ego-controlled taste, are as much present in me
now as they have ever been in my life, and as much as in anyone else I
know. This interferes a lot with the learning -- it causes a
tremendous amount of friction and energy dispersion (wholly unlike
what happens when I am learning mathematics, say, namely discovering
things about any kind of substance which my own person and ego is not
part of\dots). If there has been anything new appearing in my life, it
is surely not the end of this process of dispersion, or the end of
inertia, closely related to dispersion. It is something, rather, which
causes learning to take place all the same -- be it the hard way, as
it often happens, very much like the troubled digestion of one who
took a substantial meal reluctantly or in a state of nervousness, of
crispation. Once one is through with the digestion, though, the food
one ate is transformed into flesh and muscles, blood and bones and the
like, just as good and genuine as if the meal had been taken relaxedly
and with eager appetite, as it deserved. What really counts for the
process of assimilation to be able to take place, is that in a certain
sense the food, palatable or not, be \emph{accepted} -- not vomited,
or just kept in the bowels like a foreign body, sometimes for
decennaries.\scrcomment{decennary = decade} The remarkable fact I come
to know through experience, is that even after having been kept thus
inertly for a lifetime, a process of digestion and assimilation may
still come into being and transform the obtrusive stuff into living
substance.

During the last week I have been sick for a few days -- a
grippe\scrcomment{grippe = the flu} I might say, but surely a case of
troubled digestion too. It seems though I am through now -- till the
next case at any rate! I daresay life has been generous with me for
these last three months, while I haven't even taken the trouble to
stop with the mathematical nonsense for any more than a week or
two. This week, too, I still did some mathematical scratchwork, still
along the lines of abelianization, which keeps showing a lot richer
than suspected.

%%% Local Variables:
%%% mode: latex
%%% TeX-master: "main.tex"
%%% End:
