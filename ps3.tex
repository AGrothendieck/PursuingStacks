%% Anti-Copyright 2015 - the scrivener

\chapter{Grinding my way towards canonical modelizers}
\label{ch:III}

\noindent\hfill\ondate{27.3.}\pspage{89}\par

% 45
\hangsection[It's burning again! Review of some ``recurring striking
\dots]{It's burning again! Review of some ``recurring striking
  features'' of modelizers and standard modelizing functors.}%
\label{sec:45}%
The review yesterday of the various ``test notions'', turning around
test categories and test functors, turned out a lot longer than
expected, so much so to have me get a little weary by the end -- it
was clear though that this ``travail d'intendance'' was necessary, not
only not to get lost in a morass of closely related and yet definitely
distinct notions, but also to gain perspective and a better feeling of
the formal structure of the whole set-up. As has been the case so
often, during the very work of ``grinding through'', there has
appeared this characteristic feeling of getting close to something
``burning'' again, something very simple-minded surely which has kept
showing up gradually and more and more on all odds and ends, and which
still is escaping, still elusive. These is an impressive bunch of
things which are demanding pressingly more or less immediate
investigation -- still I can't help, I'll have to try and pin down
some way or other this ``burning spot''.

There seem to be recurring striking features of the modelizers met
with so far -- namely essentially \Cat{} and the elementary modelizers
\Ahat{} and possibly their ``weak'' variants. In all of them, these is
a very strong interplay between the following notions, which seem to
be the basic ones and more or less determine each other mutually:
\emph{weak equivalences} (which define the modelizing structure of the
given modelizer $M$), \emph{aspheric objects} (namely such that
$x\to e_M$ is a weak equivalence), \emph{homotopy intervals}
$(I,\delta_0,\delta_1)$, and last not least, the notion of a test
functor $A \to M$, where $A$ is a test category (or more generally
weak test functors of weak test categories into $M$). The latter so
far have been defined only when $M=\Cat$, and initially they were
viewed as being mainly more handy substitutes to $j_A$, for getting a
model-preserving functor $\Cat\to\Ahat$ quasi-inverse to the
all-important model-preserving functor $i_A:\Ahat\to\Cat$,
$F\mapsto A_{/F}$. I suspect however that their role is a considerably
more basic one than just computational convenience -- and this reminds
me of the analogous feeling I had, when first contemplating using such
a thing as (by then still vaguely conceived) ``test categories'', for
investigating ways of putting modelizing structures on categories $M$
such as categories of algebraic structures of some kind or
other. (Cf.\ notes of \hyperref[date:7.3.]{March 7}, and more
specifically par.\ \ref{sec:26} -- this was the very day, by the way,
I first had this feeling of being ``burning''\ldots)

Test\pspage{90} categories seem to play a similar role here as (the
spectra of) discrete valuation rings in algebraic geometry -- they can
be mapping into anywhere, to ``test'' what is going on there -- here
it means, they can be sent into any modelizer $(M,W)$ (at least among
the ones which we feel are the most interesting), by ``test functors''
$i:A\to M$ giving rise to a model-preserving functor $i^*:M\to\Ahat$,
allowing comparison of $M$ with an elementary modelizer $\Ahat$. As
for the all-encompassing basic modelizer \Cat, it seems to play the
opposite role in a sense, at least with respect to elementary
modelizers, \Ahat, which all admit modelizing maps
$i_A:\Ahat\to\Cat$. As a matter of fact, for given test category $A$,
i.e., a given elementary modelizer \Ahat, I see for the time being
just \emph{one} way to get a modelizing functor from it to \Cat,
namely just the canonical $i_A$. There is also a striking difference
between the exactness properties of the functors
\begin{equation}
  \label{eq:45.1}
  i^*:M\to\Ahat\tag{1}
\end{equation}
one way, which commute to inverse limits, and the functors
\begin{equation}
  \label{eq:45.2}
  i_A:\Ahat\to\Cat\tag{2}
\end{equation}
in the other direction, commuting to direct limits. Another difference
is that we should not expect that the left adjoint $i_!$ to $i^*$ be
model preserving too (with the exception of the very special case when
$M=\Cat$ and $i:A\to M=\Cat$ is the canonical functor $i_A\restrto A$,
which appears as highly non-typical in this respect), whereas the
right adjoint $j_A=i_A^*$ of $i_A$ is model preserving, this $i_A$ is
part of a pair $(i_A,j_A)$ of model preserving adjoint functors.

Of course, we may want to compare directly an arbitrary modelizer $M$
to \Cat{} by sending it into \Cat{} by a modelizing functor
$M\to\Cat$; we get quite naturally such a functor (for any given
choice of test-functor $i:A\to M$)
\begin{equation}
  \label{eq:45.3}
  i_Ai^*: M \to \Cat,\tag{3}
\end{equation}
but this functor is not likely any more to commute neither to direct
nor inverse limits, even finite ones -- and it isn't too clear that
for a modelizer $M$ which isn't elementary, we have much chance to get
a modelizing functor to \Cat{} which is either left or right
exact. However, the functors \eqref{eq:45.3} we've got, whenever
modelizing and if $M$ is a \emph{strict} modelizer (namely $M\to
W_M^{-1}M\simeq\Hot$ commutes with finite products), will commute to
finite products ``up to weak equivalence''. Also the functors $i^*$,
although not right exact definitely, have a tendency to commute to
sums, and hence the same will hold (not only up to weak equivalence)
for \eqref{eq:45.3}.

As for getting a modelizing functor $\Cat\to M$, for a modelizer
$M$\pspage{91} which isn't elementary, in view of having a standard
way for describing a given homotopy type (defined by an object $C$ in
\Cat) by a ``model'' in $M$, depending functorially on $C$, there
doesn't seem to be any general process for finding one, even without
any demand on exactness properties, except of course when $M$ is
supposed to be elementary; in this case $M=\Ahat$ we get the functors
\begin{equation}
  \label{eq:45.4}
  i^*:\Cat\to\Ahat\tag{4}
\end{equation}
associated to test functors $A\to\Cat$, which can be viewed as a
particular case of \eqref{eq:45.1}, applied to the case
$M=\Cat$. Using such functors \eqref{eq:45.4}, we see that the
question of finding a modelizing functor
\begin{equation}
  \label{eq:45.star}
  \varphi: \Cat\to M,\tag{*}
\end{equation}
for a more or less general $M$, is tied up with the question of
finding such a functor from an elementary modelizer \Ahat{} into $M$
\begin{equation}
  \label{eq:45.starstar}
  \psi:\Ahat\to M.\tag{**}
\end{equation}
More specifically, if we got a $\psi$, we deduce a $\varphi$ by
composing with $i^*$ in \eqref{eq:45.4}, and conversely, if we got a
$\varphi$, we deduce a $\psi$ by composing with the canonical functor
$i_A$ in \eqref{eq:45.2}. Maybe it's unrealistic to expect modelizing
functors \eqref{eq:45.star} or \eqref{eq:45.starstar} to exist for
rather general $M$. (Which modelizers will turn out to be really ``the
interesting ones'' will appear in due course presumably\ldots) There
is one interesting case though when we got such functors, namely when
\[M=\Spaces\]
is the category of topological spaces, and taking for $\psi$ one of
the manifold avatars of ``geometric realization functor'', associated
to a suitable functor
\begin{equation}
  \label{eq:45.starstarstar}
  r:A\to\Spaces\tag{***}
\end{equation}
by taking the canonical extension $r_!$ to \Ahat, commuting with
direct limits. This is precisely the ``highly non-typical'' case, when
we get a pair of adjoint functors $r_!,r^*$
\begin{equation}
  \label{eq:45.5}
  \begin{tikzcd}
    \Ahat\ar[r, shift left, "r_!"] & \Spaces\ar[l, shift left, "r^*"]
  \end{tikzcd}\tag{5}
\end{equation}
which are \emph{both} modelizing. The situation here mimics very
closely the situation of the pair $(i_A,j_A=i_A^*)$ canonically
associated to the elementary modelizer \Ahat, with the ``basic modelizer''
\Cat{} being replaced by \Spaces, which therefore can be considered as
another ``basic modelizer'' of sorts. In this case the corresponding
functor
\begin{equation}
  \label{eq:45.6}
  r_!r^* : \Cat\to\Spaces
  \tag{6}
\end{equation}
mimics the functor $i_Ai^*$ of \eqref{eq:45.3} (where on the left hand
side $M$ is taken to be just \Cat, and on the right \Cat{} as the
basic modelizer is replaced by its next best substitute \Spaces). Here
as in \eqref{eq:45.3}, the modelizing functor we got is neither left
nor right exact, it has a tendency\pspage{92} though to commute to
sums, as usual.

I wouldn't overemphasize the capacity of \Spaces{} to serve the
purpose of a ``basic modelizer'' as does \Cat, despite the attractive
feature of more direct (or at any rate, more conventional) ties with
so-called ``topological intuition''. One drawback of \Spaces{} is the
relative sophistication of the structure species ``topological
spaces'' it corresponds to, which is by no means an ``algebraic
structure species'', and fits into algebraic formalisms only at the
price of detours. More seriously still, or rather as a reflection of
this latter feature, only for some rather special elementary
modelizers \Ahat, namely rather special test categories $A$, do we get
a geometric realization functor $r_!:\Ahat\to\Spaces$ which can be
view as part of a pair of mutually adjoint modelizing functors,
mimicking the canonical pair $(i_A,j_A)$; still less does there seem
to be anything like a really canonical choice (although some choices
are pretty natural indeed, dealing with the standard test categories
such as $\Simplex$ and its variants). At any rate, it is still to be
seen whether there exists such a pair $(r_!,r^*)$ for some rather
general class of test categories -- this is one among the very many
things that I keep pushing off, as more urgent matters are calling for
attention\ldots

To sum up the outcome of these informal reflections about various
types of modelizing functors between modelizers, the two main types
which seem to overtower the whole picture, and are likely to be the
essential ones for a general understanding of homotopy models, are the
two types \eqref{eq:45.1} and \eqref{eq:45.2} above. The first one
$i^*$ is defined in terms of an arbitrary modelizer $M$. The second
$i_A$, with opposite exactness properties to the previous one, is
canonically attached to any test category, and maps the corresponding
elementary modelizer \Ahat{} into the basic modelizer \Cat, without
any reference to more general types of modelizers $M$. The right
adjoint of the latter, which is still model preserving, is in fact of
the type \eqref{eq:45.1} again, for the canonical test functor
$A\to\Cat$ induced by $i_A$, namely $a\mapsto A_{/a}$.

This whole reflection was of course on such an informal level, that
there was no sense at that stage to bother with distinctions between
weaker or stricter variants of the test-notion. Maybe it's about time
now to start getting a little more specific.\pspage{93}

% 46
\hangsection[Test functors with values in any modelizer: an
\dots]{Test functors with values in any modelizer: an observation,
  with an inspiring ``silly question''.}\label{sec:46}%
First thing to do visibly is to define the notion of a test-functor
\[ i : A \to M,\]
where $M$ is any modelizer. Thus $M$ is endowed with a subset
$W_M\subset\Fl(M)$, i.e., a notion of weak equivalence, satisfying the
``mild saturation conditions'' of p.\ \ref{p:59}, and moreover we
assume that $W_M^{-1}M$ is equivalent to \Hot{} -- but the choice of
an equivalence, or equivalently, of the corresponding localization
functor
\[ M \to \Hot,\]
is not given with the structure. (If we admit the ``inspiring
assumption'', there is no real choice, as a matter of fact -- but we
don't want to use this in a technical sense, but only as a guide and
motivation.)

Let's start with the weak variant -- we assume $A$ to be a weak test
category, and want to define what it means that $i$ is a \emph{weak
  test functor}. In all this game, it is understood that in case
$M=\Cat$, the notions we want to define (of a weak test functor and of
a test functor) should reduce to the ones we have pinpointed in
yesterday's notes.

The very first idea that comes to mind, is to demand merely that the
corresponding $i^*$ \eqref{eq:45.1}
\[ i^*: M\to\Ahat\]
should be modelizing, which means (I recall)
\begin{enumerate}[label=\alph*)]
\item $W_M= (i^*)^{-1}(W_A)$.
\item The induced functor $W_M^{-1}M \to W_A^{-1}\Ahat$ is an
  equivalence.
\end{enumerate}

This, I just checked, does correspond to the definition we gave
yesterday (p.\ \ref{p:85}), when $M=\Cat$. There is a very interesting
extra feature though in this special case, which appears kind of ``in
between the lines'' in the ``\hyperref[thm:keyresult]{key result}'' on
p.\ \ref{p:61}, and which I want now to state in the more general
set-up.

As usual in related situations, the notion of weak equivalence in $M$
gives rise to a corresponding notion of ``aspheric'' elements in $M$
-- namely those for which the unique map
\[x \to e_M\]
is a weak equivalence. We assume now the existence of a final object
$e_M$ in $M$, and will assume too, if necessary, that it's the image
in the localization $W_M^{-1}M= H_M$ is equally a final object. Thus,
if $x$ in $M$ is aspheric, its image in $H_M$ is a final object, and
the converse holds provided as assume $W_M$ strongly saturated, namely
any map in $M$ which becomes an isomorphism in $H_M$ is a weak
equivalence.\footnote{\alsoondate{29.3.} This assumption will be verified if there
  exists a weak test functor $i:A\to M$.}

I\pspage{94} can now state the ``interesting extra feature''.
\begin{observation}
  For a functor $i:A\to M$ of a weak test category $A$ into the
  modelizer $M$ \textup(with final object $e_M$, giving rise to the
  final object in $H_M=W_M^{-1}M$\textup), and in the special case
  when $M=\Cat$, the following conditions are equivalent:
  \begin{enumerate}[label=(\roman*),font=\normalfont]
  \item\label{it:46.i}
    $i^*$ transforms weak equivalence into weak equivalences,
    i.e., induces a functor $H_M\to H_\Ahat$.
  \item\label{it:46.ii}
    $i^*$ transforms aspheric objects into aspheric objects.
  \item\label{it:46.iii}
    $i$ is a weak test functor, namely $W_M=(i^*)^{-1}(W_\Ahat)$
    \textup(a stronger version of \textup{\ref{it:46.i}}\textup)
    \emph{and} the induced functor $H_M\to H_\Ahat$ is an \emph{equivalence}.
  \end{enumerate}
\end{observation}

Here the obvious implications are of course
\[ \text{\ref{it:46.iii}} \Rightarrow \text{\ref{it:46.i}} \Rightarrow
\text{\ref{it:46.ii},}\]
the second implication coming from the fact that $i^*$ is compatible
with final objects, and that $e_\Ahat$ is aspheric. Of course,
\ref{it:46.ii} means that for any aspheric $x$ in $M$, $A_{/i^*(x)}$
is aspheric in \Cat. In case $M=\Cat$, and when moreover the elements
$i(a)$ in \Cat{} have final objects (a condition I forgot to include
in the statement of the observation above, sorry), this condition was
seen to imply \ref{it:46.iii} (cf.\ ``\hyperref[thm:keyresult]{key result}'' on
p.\ \ref{p:61}, \hyperref[it:key.a.iv]{(a~iv)} $\Rightarrow$
\hyperref[it:key.a.ii]{(a~ii)} -- indeed, it is even enough to check
that for any $C$ \emph{with final element} in \Cat, $i^*(C)$ is
aspheric. The proof moreover turns out practically trivial, in terms
of the usual \ref{lem:asphericitycriterion} for a functor between
categories. So much so that the really amazing strength of the
statement, which appears clearly when looked at in a more general
setting, as I just did, was kind of blurred by the impression of
merely fastidiously grinding through routine equivalences. We got
there at any rate quite an interesting class of functors between
modelizers (an elementary and the basic one, for the time being), for
which the mere fact that the functor be compatible with weak
equivalences, or only even take aspheric objects into aspheric ones,
implies that the functor in modelizing, namely that the functor
$H_M\to H_\Ahat$ it induces (and the very \emph{existence} of this
functor was all we demanded beforehand!) is actually an
\emph{equivalence of categories}.

The question that immediately comes to mind now, is if this ``extra
feature'' is indeed an extremely special one, strongly dependent on
the assumption $M=\Cat$ and the categories $i(a)$ having final objects
-- or if it may not have a considerably wider significance. This
suggests the still more general questions, involving two modelizers
$M,M'$, neither of which needs by elementary or by \Cat{}
itself:\pspage{95}
\begin{question}\label{q:naivequestion}
  Let
  \[f : M\to M'\]
  be a functor between modelizers $(M,W)$ and $(M',W')$, assume if
  needed that $f$ commute with inverse limits, or even has a left
  adjoint, and that inverse limits (and direct ones too, as for that!)
  exist in $M,M'$. Are there some natural conditions we can devise for
  $M$ and $M'$ (which should be satisfied for elementary modelizers
  and for the basic modelizer \Cat), plus possibly some mild extra
  conditions on $f$ itself, which will ensure that whenever $f$
  transforms weak equivalences into weak equivalences, or even only
  aspheric objects into aspheric objects, $f$ is model-preserving,
  i.e., $W_M = f^{-1}(W_{M'})$ \emph{and} the induced functor
  $H_f:H_M\to H_{M'}$ on the localizations is an equivalence of categories?
\end{question}

Maybe it's a silly question, with pretty obvious negative answer -- in
any case, I'll have to find out! The very first thing to check is to
see what happens in case of a functor
\[i^*:M\to\Ahat,\]
where \Ahat{} is a weak elementary modelizer, and where $M$ is either
\Cat{} or another weak elementary modelizer $B\uphat$, $i^*$ in any
case being associated of course to a functor
\[i:A\to M,\]
with a priori no special requirement whatever on $i$. In case
$M=\Cat$, this means looking up in the end the question we have
postponed for quite a while now, namely of how to rescue the
``\hyperref[thm:keyresult]{key result}'' of p.\ \ref{p:61}, when
dropping the assumption that the categories $i(a)$ (for $a$ in $A$)
have final objects. We finally got a strong motivation for carrying
through a generalization, if this is indeed feasible.

\bigbreak
\noindent\hfill\ondate{30.3.}\par

% 47
\hangsection[An approach for handling \Cat-valued test functors, and
\dots]{An approach for handling \texorpdfstring{\Cat}{(Cat)}-valued
  test functors, and promise of a ``key result'' revised. The
  significance of contractibility.}\label{sec:47}%
It had become clear that the most urgent thing to do now was to come
to a better understanding of test functors with values in \Cat, when
dropping the assumption that the categories $i(a)$ have final objects,
and trying to replace this (if it should turn out that something
\emph{is} needed indeed) by a kind of assumption which should make
sense when \Cat{} is replaced by a more or less arbitrary modelizer
$M$. I spent a few hours pondering over the situation, and it seems to
me that in the case at least when $A$ is a \emph{strict}, namely when
\Ahat{} is totally aspheric, there is now a rather complete
understanding of the situation, with a generalization of the
``\hyperref[thm:keyresult]{key result}'' of p.\ \ref{p:61} which seems
to be wholly satisfactory.

The\pspage{96} basic idea of how to handle the more general situation,
namely how to compare the categories $A_{/i^*(C)}=A_{/C}$ and $C$, and
show (under suitable assumptions) that there is a canonical
isomorphism between their images in the localized category
$W_\Cat^{-1}\Cat=\Hot$, was around since about the moment I worked out
the ``key result''. It can be expressed by a diagram of ``maps'' in
\Cat
\begin{equation}
  \label{eq:47.1}
    \begin{tikzcd}[cramped]
      A_{/C} \ar[r] & A_{\sslash C} & \\
      & A\times C \ar[u]\ar[r] & C\quad,
    \end{tikzcd}
  \tag{1}
\end{equation}
where $A_{\sslash C}$ is the fibered category over $A$, associated to
the functor
\[ A\op\to\Cat, \quad a\mapsto\bHom(i(a),C).\]
Here one should be careful with the distinction between the \emph{set}
\[ \Hom(i(a),C) = \Ob \bHom(i(a),C),\]
and the \emph{category} $\bHom(i(a),C)$, both depending
bi-functorially on $a$ in $A$ and $C$ in \Cat. The former (as a
presheaf on $A$ for fixed $C$) gives rise to $A_{/C}$, a fibered
category over $A$ with \emph{discrete} fibers, whereas the latter
gives rise to $A_{\sslash C}$, which is fibered over $A$ with fibers
that need not be discrete. Identifying a set with the discrete
category it defines, we get a canonical functor
\[ \Hom(i(a),C) \to \bHom(i(a),C),\]
which is very far from being an equivalence nor even a weak
equivalence; being functorial for varying $a$, it gives rise to the
first map in \eqref{eq:47.1}. The second is deduced from the canonical
functor
\[C \to \bHom(i(a),C),\]
identifying $C$ with the full subcategory of \emph{constant} functors
from $i(a)$ to $C$. This map is functorial in $a$, and gives rise
again to a cartesian functor between the corresponding fibered
categories over $A$, the first one (which corresponds to the constant
functor $A\op\to\Cat$ with value $C$) is just $A\times C$ fibered over
$A$ by $\pr_1$, hence the second arrow in \eqref{eq:47.1}. The third
arrow is just $\pr_2$.

It seems that, with the introduction of $A_{\sslash C}$, this is the
first time since the beginning of these reflections that we are making
use of the \emph{bicategory structure} of \Cat, namely of the notion
of a morphism or map or ``homotopy'' (the tie with actual homotopies
will be made clear below), between two ``maps'' namely (here) functors
$C'\rightrightarrows C$. There is of course a feeling that such a
notion of homotopy should make sense in a more or less arbitrary
modelizer $M$, and that the approach displayed by the diagram
\eqref{eq:47.1} may well generalize to mere general
situations\pspage{97} still, with \Cat{} replaced by such an $M$.

In the situation here, the work will consist in devising handy
conditions on $i: A\to\Cat$ and $A$ that will ensure that all three
maps in \eqref{eq:47.1} are weak equivalences, for any choice of
$C$. This will imply that the corresponding maps in \Hot{} are
isomorphisms, hence a canonical isomorphism between the images in
\Hot{} of $A_{/C}$ and $C$, which will imply that a)\enspace the
functor $i_Ai^*$ from \Cat{} to \Cat{} carries weak equivalences into
weak equivalences, and hence induces a functor
\[\Hot\to\Hot,\]
and b)\enspace that this functor is isomorphic to the identity
functor. If moreover $A$ is a weak test category, and therefore the
functor
\[W_A^{-1}\Ahat\to\Hot\]
induced by $i_A$ is an equivalence, it will follow that the functor
\[\Hot\to W_A^{-1}\Ahat\]
induced by $i^*$ is equally an equivalence, namely that $i^*$ is
indeed a test-functor.

For the map
\begin{equation}
  \label{eq:47.2}
  A\times C\to C
  \tag{2}
\end{equation}
to be a weak equivalence for any $C$ in \Cat, it is necessary and
sufficient that $A$ be aspheric (cf.\ par.\ \ref{sec:40}, page
\ref{p:69}), a familiar condition on $A$ indeed! For handling the map
\begin{equation}
  \label{eq:47.3}
  A\times C\to A_{\sslash C}\quad\text{associated to $C\to\bHom(i(a),C)$,}
  \tag{3}
\end{equation}
we'll use the following easy result (which I'll admit for the time
being):
\begin{proposition}
  Let $F$ and $G$ be two categories over a category $A$\kern1pt, and $u:F\to
  G$ a functor compatible with projections. We assume
  \begin{enumerate}[label=\alph*),font=\normalfont]
  \item\label{it:47.a}
    For any $a$ in $A$\kern1pt, the induced map on the fibers
    \[u_a : F_a \to G_a\]
    is a weak equivalence.
  \item\label{it:47.b}
    Either $F$ and $G$ are both cofibering over $A$ and $u$ is
    cocartesian, or $F$ and $G$ are fibering and $u$ is cartesian.
  \end{enumerate}
  Then $u$ is a weak equivalence.
\end{proposition}
This shows that a sufficient condition for \eqref{eq:47.3} to be a
weak equivalence, is that the functors $C\to\bHom(i(a),C)$ be a weak
equivalence, for any $a$ in $A$ and $C$ in \Cat. We'll see in the next
section\footnote{\alsoondate{9.4.} Actually, this is done only a lot later, on page
  \ref{p:121} and (for the converse) on page \ref{p:143}.} that this
amounts to demanding that the objects $i(a)$ in \Cat{} should be
``contractible'', in the most concrete sense of this expression, which
is actually stronger than just asphericity. (Earlier in these notes I
was a little floppy with the terminology, by using a few times the
word ``contractible''\pspage{98} as synonymous to ``aspheric'' as in
the context of topological spaces, or CW spaces at any rate, the two
notions do indeed coincide, finally I came to use rather the word
``aspheric'' systematically, as it fits nicely with the notion of an
aspheric morphism of topos\ldots) The most evident example of
contractible categories are the categories with final object. Thus I
have the strong feeling that the condition of contractibility of the
objects $i(a)$ in \Cat{} is ``the right'' generalization of the
assumption made in the ``key result'', namely that the $i(a)$ have
final elements. Also, it seems now likely that the numerous cases of
statements, when to check some property for arbitrary $C$, it turned
out to be enough to check it for $C$ with a final object, may well
generalize to more general cases, with \Cat{} replaced by some $M$ and
the reduction is from arbitrary $C$ in $M$ to contractible ones.

The next thing to do is to develop a little the notion of
contractibility of objects and of homotopies between maps, and to get
the criterion just announced for \eqref{eq:47.3} to be a weak
equivalence for any $C$. After this, handling the question of the
first map in \eqref{eq:47.1}
\begin{equation}
  \label{eq:47.4}
  A_{/C} \to A_{\sslash C}
  \tag{4}
\end{equation}
being a weak equivalence for any $C$, in terms of the
\ref{lem:asphericitycriterion} for a functor, will turn out pretty
much formal, and we'll finally be able to state a new version of the
``key result'' about test functors $i:A\to\Cat$, with this time twice
as many equivalent formulations of the same property. On n'arr\^ete
pas le progr\`es!

\bigbreak
\noindent\hfill\ondate{31.3.}\par

% 48
\hangsection[A journey through abstract homotopy notions (in terms
\dots]{A journey through abstract homotopy notions
  \texorpdfstring{\textup(}{(}in terms of a set
  \texorpdfstring{$W$}{W} of ``weak
  equivalences''\texorpdfstring{\textup)}{)}.}\label{sec:48}%
It's time now to develop some generalities about homotopy classes of
maps, the relation of homotopy between objects of a modelizer, and the
corresponding notion of contractibility. For the time being, it will
be enough to start with any category $M$, endowed with a set
$W\subset\Fl(M)$ of arrows (the ``weak equivalences''), satisfying the
mild saturation assumptions \ref{it:37.a}\ref{it:37.b}\ref{it:37.c} of
par.\ \ref{sec:37} (p.\ \ref{p:59}). On $M$ we'll assume for the time
being that there exists (at least one) homotopy interval
$\bI = (I,\delta_0,\delta_1)$ in $M$ (loc.\ sit.) which implies also
that $M$ has a final object, which I denote by $e_M$ or simply $e$,
and equally an initial element $\varnothing_M$. I'm not too sure yet
whether we'll really need that the latter be strict initial element,
as required in the definition of a homotopy interval on page
\ref{p:59} (it was used in the generalities of pages \ref{p:59} and
\ref{p:60} only for the corollary on the Lawvere
element\ldots). Whether or not will appear soon enough! I'll assume it
till I am forced to. To be safe, we'll assume on the other hand that
$M$ admits binary products.

Let\pspage{99} $X,Y$ be objects of $M$, and
\[ f,g: X\rightrightarrows Y\]
two maps in $M$ from $X$ to $Y$. One key notion constantly used lately
(but so far only when $f$ is an identity map, and $g$ a ``constant''
one -- which is equally the case needed for defining contractibility)
is the notion of an \emph{\bI-homotopy from $f$ to $g$}, namely a map
$X\times I\xrightarrow h Y$ making commutative the diagram
\[\begin{tikzcd}[baseline=(O.base),sep=small]
  & X\times I \ar[dd,"h"] & \\
  X \ar[ur,"\id_X\times\delta_0"] \ar[dr,swap,"f"] & &
  X \ar[ul, swap, "\id_X\times\delta_1"] \ar[dl,"g"] \\
  & |[alias=O]| Y &
\end{tikzcd}.\]
Let's first restate the ``\hyperref[lem:homotopylemma]{homotopy lemma}'' of page \ref{p:60}
in a slightly more complete form:
\begin{homotopylemmareformulated}\label{lem:hlr}
  Assume $f$ and $g$ are \bI-homotopic. Then:
  \begin{enumerate}[label=\alph*),font=\normalfont]
  \item\label{it:48.hlr.a}
    $\gamma(f)=\gamma(g)$, where
    \[\gamma: M \to W^{-1}M=H_M\]
    is the canonical functor
  \item\label{it:48.hlr.b}
    If $f$ is a weak equivalence, so is $g$ \textup(and conversely of course,
    by symmetry of the roles of $\delta_0$ and $\delta_1$\textup).
  \item\label{it:48.hlr.c}
    Assume $f$ is an isomorphism, and $g$ constant -- then $X\to e$
    and $Y\to e$ are $W$-aspheric.
  \end{enumerate}
\end{homotopylemmareformulated}

It is important to notice that the relation ``$f$ is \bI-homotopic to
$g$'' in $\Hom(X,Y)$ is not necessarily symmetric nor transitive, and
that it depends on the choice of the homotopy interval
$\bI=(I,\delta_0,\delta_1)$. Thus the symmetric relation from
$\bI$-homotopy is $\check\bI$-homotopy, where $\check\bI$ is the
homotopy interval ``opposite'' to \bI{} (namely with
$\delta_0,\delta_1$ reversed). The example we are immediately
interested in is $M=\Cat$, with $W=W_\Cat$ the usual notion of weak
equivalence. A homotopy interval is just an aspheric small category
$I$, endowed with two \emph{distinct} objects $e_0,e_1$. (The
condition $e_0\ne e_1$ just expresses the condition $e_0\sand
e_1=\varnothing$ on homotopy intervals -- if it were not satisfied,
\bI-homotopy would just mean equality of $f$ and $g$\ldots) For the
usual choice $I=\Simplex_1=(e_0\to e_1)$, an \bI-homotopy from $f$ to
$g$ is just a morphism between functors $f\to g$ -- the $\bI$-homotopy
relation between $f$ and $g$ is the existence of such a morphism, it
is a transitive, and generally non-symmetric relation. If we take $I$
to be a category with just two objects $e_0$ and $e_1$, equivalent to
the final category, an \bI-homotopy between $f$ and $g$ is just an
isomorphism from $f$ to $g$ -- the \bI-homotopy relation now is both
transitive and symmetric, and it is a lot more restrictive than the
previous one. If we take $I$ to be the barycentric
subdivision\pspage{100} of $\Simplex_1$, which can also be interpreted
as an amalgamated sum of $\Simplex_1$ with itself, namely
\[I = \begin{tikzcd}[baseline=(A.base),row sep=-7pt,column sep=small]
  e_0 \ar[dr] & \\ & |[alias=A]| e_2 \\ e_1\ar[ur] &
\end{tikzcd},\]
an \bI-homotopy from $f$ to $g$ is essentially a triple $(k,u,v)$,
with $k:X\to Y$ and $u:f\to k$ and $v:g\to k$ maps in $\bHom(X,Y)$;
this time, the relation of \bI-homotopy is symmetric, but by no means
transitive. Returning to general $M$, it is customary to introduce the
equivalence relation in $\Hom(X,Y)$ generated by the relation of
\bI-homotopy -- we'll say that $f$ and $g$ are \emph{\bI-homotopic in
  the wide sense}, and we'll write
\[ f \Isim g,\]
if they are equivalent with respect to this relation. As seen from the
examples above (where $M=\Cat$), this relation still depends on the
choice of the homotopy interval \bI. Let's first look at what we can
do for fixed \bI, and then how what we do depends on \bI.

If $f$ and $g$ are \bI-homotopic, then so are their composition with
any $Y\to Z$ or $T\to X$. This implies that the relation $\Isim$ of
\bI-homotopy in the wide sense is compatible with compositions. If we
denote by
\[\Hom(X,Y)_\bI\]
the quotient set of $\Hom(X,Y)$ by the equivalence relation of
\bI-homotopy in the wide sense, we get composition between the sets
$\Hom(X,Y)_\bI$, and hence a structure of a category $M_\bI$ having
the same objects as $M$, and where maps from $X$ to $Y$ are
\bI-homotopy classes (in the wide sense -- this will be understood
henceforth when speaking of homotopy classes) of maps from $X$ to $Y$
in $M$. Two objects $X,Y$ of $M$, i.e., of $M_\bI$ which are
isomorphic as objects of $M_\bI$ will be called
\emph{\bI-homotopic}. This means also that we can find maps in $M$
(so-called \emph{\bI-homotopisms } -- namely $M_\bI$-isomorphisms)
\[ f:X\to Y, \quad g:Y\to X\]
such that we get \bI-homotopy relations in the wide sense
\[gf \Isim \id_X,\quad fg \Isim\id_Y.\]
We'll say $X$ is \emph{\bI-contractible} if $X$ is \bI-homotopic to
the final object $e_M=e$ of $M$ (which visibly is also a final object
of $M_\bI$), i.e., if $X$ is a final object of $M_\bI$. In terms of
$M$, this means that there exists a section $f$ of $X$ over $e$, such
that $fp_X$ is \bI-homotopic in the wide sense to $\id_X$ (where $p_X$
is the unique map $X\to e$). In fact, if there is such a section $f$,
any other section will do too.

From the homotopy lemma \ref{it:48.hlr.a} it follows that the canonical
functor\pspage{101} $M\to W^{-1}M=H_M$ factors into
\[ M \to M_\bI \to H_M = W^{-1}M,\]
and from \ref{it:48.hlr.b} it follows that if $f,g: X\rightrightarrows Y$
are in the same \bI-class, then $f$ is in $W$ if{f} $g$ is, hence by
passage to quotient a subset
\[ W_\bI \subset \Fl(M_\bI)\]
of the set of arrows in $M_\bI$, namely a notion of weak equivalence
in $M_\bI$. It is evident from the universal property of $H_M$ that
the canonical functor $M_\bI\to H_M$ induces an isomorphism of
categories
\[H_{M_\bI} = W_\bI^{-1}M_\bI \tosim H_M=W^{-1}M.\]

It's hard at this point not to expect that $W_\bI$ should satisfy the
same mild saturation conditions as $W$, so let's look into this in the
stride (even though I have not had any use of this so far). Condition
\ref{it:37.b} of saturation, namely that if $f,g$ are composable and
two among $f,g,gf$ are in $W$, so is the third, carries over
trivially. Condition \ref{it:37.a}, namely the tautological looking
condition that $W$ should contain all isomorphisms, makes already a
problem, however. It is OK though if $W$ satisfies the following
saturation condition, which is a strengthening of condition
\ref{it:37.c} of page \ref{p:59}:
\begin{enumerate}[label=\alph*'),start=3]
\item\label{it:48.cprimeone}
  Let $f:X\to Y$ and $g:Y\to X$ such that $gf\in W$ and $fg\in W$,
  then $f,g\in W$.
\end{enumerate}
This condition \ref{it:48.cprimeone} carries over to $M_\bI$
trivially. This suggests to introduce a strengthening of the ``mild
saturation conditions'', which I intend henceforth to call by the name
of ``saturation'', reserving the term of ``\emph{strong saturation}''
to what I have previously referred to occasionally as ``saturation''
-- namely the still more exacting condition that $W$ consists of
\emph{all} arrows made invertible by the localization functor $M\to
H_M=W^{-1}M$, or equivalently, by some functor $M\to H$. Thus, we'll
say $W$ is \emph{saturated} if{f} it satisfies the following:
\begin{enumerate}[label=\alph*')]
\item\label{it:48.aprime}
  For any $X$ in $M$, $\id_X\in W$.
\item\label{it:48.bprime}
  Same as \ref{it:37.b} before: if two among $f,g,gf$ are in $W$, so
  is the third.
\item\label{it:48.cprime}
  If $f:X\to Y$ and $g:Y\to X$ are such that $gf,fg\in W$, then
  $f,g\in W$.
\end{enumerate}
Each of these conditions carries over from $W$ to a $W_\bI$ trivially.

I can't help either having a look at the most evident exactness
properties of the canonical functor $M\to M_\bI$: Thus one immediately
sees that for two maps
\[f,g : X \rightrightarrows Y=Y_1\times Y_2,\]
with components $f_i,g_i$ ($i\in\{1,2\}$), $f$ and $g$ are
\bI-homotopic in the wide sense if{f} so are $f_i$ and $g_i$ (for
$i\in\{1,2\}$). The analogous statement is valid for maps into any
product object $Y=\prod Y_i$ on a finite set of\pspage{102} indices. The
dual statement so to say, when $X$ is decomposed as a sum $X=\coprod
X_i$, is valid too, provided taking products with $I$ is distributive
with respect to finite direct sums. Thus we get that $M\to M_\bI$
commutes with finite products, and with finite sums too provided they
are distributive with respect to multiplication with any object (or
with $I$ only, which would be enough).

The notion of \bI-homotopy in the wide sense between
$f,g:X\rightrightarrows Y$ can be interpreted in terms of strict
$\bI'$-homotopy with variable $\bI'$, as follows, provided we make
some mild extra assumptions on $(M,W)$, namely:
\begin{enumerate}[label=\alph*)]
\item\label{it:48.a}
  (Just for memory) $M$ is stable under finite products.
\item\label{it:48.b}
  $M$ is stable under amalgamated sums $I \amalg_e J$ under the final
  object $e$ ($I$ and $J$ endowed with sections over $e$).
\item\label{it:48.c}
  If moreover $I$ and $J$ are aspheric over $e$, then so is $I\amalg_e J$.
\end{enumerate}

Conditions \ref{it:48.b} and \ref{it:48.c} give a means of
constructing new homotopy intervals \bK, by amalgamating two homotopy
intervals \bI{} and \bJ, using as sections of \bI{} and \bJ{} for
making the amalgamation, either $\delta_0$ or $\delta_1$, which gives
four ways of amalgamating -- of course we take as endpoints of the
amalgamated interval, the sections over $e$ coming from the two
endpoints of \bI{} (giving rise to $\delta_0$ for \bK) and \bJ{}
(giving rise to $\delta_1$ for \bK) which have not been ``used up'' in
the amalgamation. Maybe the handiest convention is to define
\emph{the} amalgamated interval $\bI lor \bJ$, without any ambiguity
of choice, as being
\[ \bI \lor \bJ = (I,\delta_1^\bI) \amalg_e
(J,\delta_0^\bJ)\qquad
\parbox[t]{0.33\textwidth}{\raggedright endowed with the two
  sections coming from $\delta_0^\bI:e\to I$, $\delta_1^\bI:e\to
  J$,}\]
and defining the three other choices in terms of this operation, by
replacing one or two among the summands \bI, \bJ{} by the ``opposite''
interval $\check\bI$ or $\check\bJ$. The operation of amalgamation of
intervals, and likewise of homotopy intervals, just defined, is
clearly associative up to a canonical isomorphism, and we have a
canonical isomorphism of intervals
\[ (\bI\lor \bJ)\check{\phantom{\mathrm i}} \simeq \check\bI \lor \check\bJ.\]
I forgot to check, for amalgamation of homotopy intervals, the
condition \ref{cond:HIb} of page \ref{p:59}, namely $e_0 \sand e_1 =
\varnothing_M$ (which has not so far played any role, anyhow). To get
this condition, we'll have to be slightly more specific in condition
\ref{it:48.b} above on $M$ of existence of the relevant amalgamations,
by demanding (as suggested of course by the visual intuition of the
situation) that $I$ and $J$ should become subobjects of the
amalgamation $K$, and their intersection should be reduced to the
tautological part $e$ of it. More relevant still for the use we have
in mind is to demand that those amalgamations should commute to
products by an arbitrary element $X$ of $M$.\pspage{103} This I'll
assume in the interpretation of $\Isim$ in terms of strict
homotopies. Namely, let
\[\Comp(\bI)\]
by the set of all homotopy intervals deduced from \bI{} by taking
amalgamations of copies of \bI{} and $\check\bI$, with an arbitrary
number $n\ge1$ of summands. Thus we get just \bI{} and $\check\bI$ for
$n=1$, four intervals for $n=2$, \ldots, $2^n$ intervals for $n$
arbitrary. It is now immediately checked that for $f,g$ in
$\Hom(X,Y)$, the relation $f\Isim g$ is equivalent to the existence of
$\bK\in\Comp(\bI)$, such that $f$ and $g$ be \bK-homotopic (in the
strict sense).

\begin{remark}
  The saturation conditions
  \ref{it:48.aprime}\ref{it:48.bprime}\ref{it:48.cprime} on $W$ are
  easily checked for the usual notion of weak equivalence for
  morphisms of topoi, and hence also in the categories \Cat{} and in
  any topos, and therefore in any category \Ahat{} (where it boils
  down too to the corresponding properties on $W_\Cat$, as
  $W_A=i_A^{-1}(W_\Cat)$). Thus it seems definitely reasonable
  henceforth to take these as the standard notion of saturation
  (referring to its variants by the qualifications ``mild'' or
  ``strong''). On the other hand, the stability conditions
  \ref{it:48.a}\ref{it:48.b}\ref{it:48.c} on $(M,W)$ are satisfied
  whenever $M$ is a topos, with the usual notion of weak equivalence
  -- the condition \ref{it:48.c} being a consequence of the more
  general Mayer-Vietoris type statement about amalgamations of topoi
  under closed embeddings of such (cf.\ lemma on page \scrcommentinline{?}). The same
  should hold in \Cat, with a similar Mayer-Vietoris argument -- there
  is a slight trouble here for applying the precedent result on
  amalgamation of topoi, because a section $e\to C$ of an object $C$
  of \Cat, namely an embedding of the one-point category $e=\Simplex_0$
  into $C$ by choice of an object of $C$, does not correspond in
  general to a closed embedding of topoi. (In geometrical language, we
  get a ``point'' of the topos \Chat{} defined by $C$, but a point
  need not correspond to a subtopos, let alone a closed one\ldots)
  This shows the asphericity criterion for amalgamation of topoi, and
  hence also for amalgamation of categories, has not been cut out with
  sufficient generality yet. As this whole $\Comp(\bI)$ story is just
  a digression for the time being, I'll leave it at that now.
\end{remark}

More important than amalgamation of intervals, is to compare the
notions of homotopy defined in terms of a homotopy interval \bI, to
the corresponding notions for another interval, \bJ. Here the natural
idea first is to see what happens if we got a morphism of intervals
(compatible with endpoints, by definition)
\[ \bJ\to\bI.\]
It\pspage{104} is clear then, for $f,g\in\Hom(X,Y)$, that any
\bI-homotopy from $f$ to $g$ gives rise to a \bJ-homotopy; hence if
$f$ and $g$ are \bI-homotopic, they are \bJ-homotopic, and hence the
same for homotopy in the wide sense. We get thus a canonical functor
\[ M_\bI \to M_\bJ\]
which is the identity on objects, entering into a cascade of canonical
functors
\[ M \to M_\bI \to M_\bJ \to H,\]
where $H=H_M=W^{-1}M$ can be viewed as the common localization of $M,
M_\bI, M_\bJ$ with respect to the notion of weak equivalences in these
categories. We may view $M_\bJ$ as a closer approximation to $H$ than
$M_\bI$. There is of course an evident transitivity relation for the
functors corresponding to two composable morphisms of intervals
\[ \bK\to\bJ\to\bI.\]
\begin{remark}
  In order to get that $\Isim$ implies $\Jsim$, it is sufficient to
  make a much weaker assumption than existence of a morphism of
  intervals $\bJ\to\bI$ -- namely it suffices to assume that the two
  sections $\delta_i^\bI: e\to I$ are \bJ-homotopic. More generally,
  let $\sim$ be an equivalence relation in $\Fl(M)$, compatible with
  compositions and with cartesian products (this is the case indeed
  for $\Jsim$), and let $\bI=(I,\delta_0,\delta_1)$ any object $I$ of
  $M$ (not necessarily aspheric over $e$) endowed with two sections
  over $e$, such that $\delta_0\sim\delta_1$. Then the interval \bI{}
  gives rise to an equivalence relation $\Isim$ in $\Fl(M)$, whose
  definition is quite independent of $W$ -- and a priori, if $f\Isim
  g$ and $f\in W$, this need not imply $g\in W$. However, the
  condition $\delta_0\sim\delta_1$ implies immediately that the
  relation $\Isim$ implies the relation $\sim$. When the latter is
  $\Jsim$, we get moreover that $W$ is the inverse image of a set of
  arrows in $M_\bI$, i.e., $f\Isim g$ and $f\in W$ implies $g\in W$.
\end{remark}

An interesting particular case is the one when we can find a homotopy
interval $\bI_0$ in $M$, which has the property that for any other
homotopy interval \bI{} in $M$, its structural sections satisfy
\[\delta_0^\bI \Izsim \delta_1^\bI.\]
This implies that the homotopy relation $\Izsim$ is implies by all
other similar relations $\Isim$, i.e., it is the coarsest among all
relations $\Isim$. We may then view $\bI_0$ as a
``\emph{fundamental}'' or ``\emph{characteristic}'' \emph{homotopy
  interval} in $M$, in the sense that the relation $f \Wsim g$ in the
sense below, namely existence of a homotopy interval \bI{} such that
$f \Isim g$, is equivalent to $f\Izsim g$, i.e., can be check using
the one and unique $\bI_0$. In the case of $M=\Cat$, we get readily
$\Simplex_1$ as a characteristic homotopy interval.\pspage{105} More
specifically, if $\delta_0,\delta_1:e \to I$ are two sections of an
object $I$ of \Cat, i.e., two objects $e_0,e_1$ of the small category
$I$, then these are $\Simplex_1$-homotopic if{f} $e_0,e_1$ belong to the
same connected component of $I$, which is automatic if $I$ is
$0$-connected, and a fortiori if $I$ is aspheric. This accounts to a
great extent, it seems, for the important role $\Simplex_1$ is playing
in the homotopy theory of \Cat, and consequently in the whole
foundational set-up I am developing here, using \Cat{} as the basic
``modelizer''.

It is not clear to me for the time being whether it is reasonable to
expect in more or less any modelizer $(M,W)$ the existence of a
characteristic homotopy interval (provided of course a homotopy
interval exists). This is certainly the case for the elementary
modelizers met so far. Presumably, I'll have to come back upon this
question sooner or later.

We'll now see that the set of equivalence relations $\Isim$ on
$\Fl(M)$, indexed by the set of homotopy intervals \bI, is ``filtrant
d\'ecroissant'', namely that for two such relations $\Isim$ and
$\Jsim$, there is a third ``wider'' one, $\Ksim$, implied by both. It
is enough to construct a \bK, endowed with morphisms
\[ \bK\to\bI,\quad \bK\to\bJ\]
of homotopy intervals. Indeed, there is a universal choice, namely the
category of homotopy intervals admits binary products -- we'll take
thus
\[ \bK = \bI \times \bJ,\]
where the underlying object of $\bK$ is just $I\times J$, endowed with
the two sections $\delta_i^\bI \times \delta_i^\bJ$ ($i\in\{0,1\}$).

As usual, we'll denote by $\Wsim$ or simply $\sim$ the equivalence
relation on $\Fl(M)$, which is the limit or union of the the
equivalence relations $\Isim$ -- in other words
\[f\Wsim g\quad\text{if{f} exists \bI, a homotopy interval in $M$,
  with $f\Isim g$.}\]
For $X,Y$ in $M$, we'll denote by
\[ \Hom(X,Y)_W\]
the quotient of $\Hom(X,Y)$ by the previous equivalence relation. This
relation is clearly compatible with compositions, and hence we get a
category $M_W$, having the same objects as $M$, which can be equally
viewed as the filtering limit of the categories $M_\bI$,
\[ M_W = \varinjlim_\bI M_\bI,\]
where we may take as indexing set for the limit the set of all \bI's,
preordered by $\bI\le\bJ$ if{f} there exists a morphism of intervals
$\bJ\to\bI$ (it's the preorder relation opposite to the usual one on
the\pspage{106} set of objects of a category\ldots).

We now get canonical functors
\[ M \to M_\bI \to M_W \to H,\]
where $H$ can again be considered as the common localization of the
three categories $M$, $M_\bI$ (any \bI), $M_W$ with respect to the
notion of weak equivalence in each. It is clear that the set of weak
equivalences in $M_W$, say $\overline W$, is saturated provided $W$
is. Also, the canonical functor $M\to M_W$ commutes with finite
products, and also with finite sums provided formation of such sums in
$M$ commutes with taking products with any fixed element of $M$.

A map $f:X\to Y$ is called a \emph{homotopy equivalence} or a
\emph{homotopism} (with respect to $W$) if it is an isomorphism in
$M_W$, namely if there exists $g:Y\to X$ such that
\[gf\Wsim\id_X, \quad fg\Wsim\id_Y.\]
This implies that $f$ and $g$ are weak equivalences. If such an $f$
exists, namely if $X$ and $Y$ are isomorphic objects of $M_W$, we'll
say they are \emph{$W$-homotopic}, or simply \emph{homotopic}. This
implies that there exist weak equivalences $X\to Y$ and $Y\to X$, but
the converse of course isn't always true. If $Y$ is the final object
$e_M$, we'll say $X$ is \emph{$W$-contractible} (or simply
\emph{contractible}) instead of $W$-homotopic to $Y=e$. In case there
exists a characteristic homotopy interval $\bI_0$, all these
$W$-notions boil down to the corresponding $\bI_0$-notions considered
before.
\begin{remark}
  If occurs to me that in all what precedes, we never made any use
  really of the existence of a homotopy interval! The only notion we
  have effectively been working with, it seems to me, is the notion of
  ``weak homotopy interval'', by which I mean a triple
  $\bI=(I,\delta_0,\delta_1)$ satisfying merely the condition that $I$
  be aspheric over $e$ (which is the condition \ref{cond:HIc} on page
  \ref{p:59}). Such an \bI{} always exists of course, we need only
  take $I=e$ itself! In case however an actual homotopy interval (with
  $\Ker(\delta_0,\delta_1)=\varnothing_M$) does exist, to make sure that
  the notion of $W$-homotopy obtained by using all weak homotopy
  intervals is the same when using only actual homotopy intervals, we
  should be sure that for any $I$ aspheric over $e$, any two sections
  $\delta_0,\delta_1$ of $I$ over $e$ are $W$-homotopic in the initial
  meaning. This is evidently so when $M=\Cat$ and $W=W_\Cat$ (indeed,
  it is enough that $I$ be $0$-connected, instead of\pspage{107} being
  aspheric), and in the case when $M$ is one among the usual
  elementary modelizers. The question of devising notions which still
  make sense when there is no homotopy interval in $M$ isn't perhaps
  so silly after all, if we remember that the category of
  semi-simplicial face complexes (namely without degeneracies) is
  indeed a modelizer, but it hasn't hot a homotopy interval. However,
  for the time being I feel it isn't too urgent to get any more into this.
\end{remark}

% 49
\hangsection{Contractible objects. Multiplicative intervals.}\label{sec:49}%
I would like now to elaborate a little on the notion of a contractible
element $X$ in $M$, which (I recall) means an object, admitting a
section $e_0:e\to X$, such that the constant map
\[c = p_Xe_0 : X\to e\to X\]
is homotopic to $\id_X$, i.e., there exists a homotopy interval $I$
such that $\id_X\Isim c$ (\bI-homotopy in the wide sense).

If $X$ is contractible, then the constant map $c:X\to X$ is a weak
equivalence (as it is homotopic to $\id_X$ which is) and hence by the
saturation condition \ref{it:48.cprime} (in fact the mild saturation
condition \ref{it:37.c} suffices) it follows that $p_X:X\to e$ is a
weak equivalence. In fact, one would expect it is even universally so,
i.e., that $p_X$ is an \emph{aspheric} map, as a plausible
generalization of the homotopy lemma \ref{it:48.hlr.c} above (which we
have used already a number of times as our main asphericity criterion
in elementary \Ahat\ldots). The natural idea to prove asphericity of
$p_X$, namely that for any base change $S\to e$, the projection
$X_S=X\times S\to S$ is a weak equivalence, is to apply the precedent
criterion to $X_S$, viewed as an element of $M_{/S}$. As the base
change functor commutes with products, it is clear indeed that for two
maps $f,g:X\rightrightarrows Y$ in $M$, \bI-homotopy of $f$ and $g$
will imply $\bI_S$-homotopy of $f_S$ and $g_S$, with evident
definition of base change for an ``interval''. On the other hand,
asphericity of \bI{} over $e$ implies tautologically asphericity of
$\bI_S$ over $S$, the final object of $M_{/S}$, which is all we need
to care about to get the result that $X_S\to S$ is a weak
equivalence. (The condition \ref{cond:HIb} on homotopy intervals is
definitely misleading here, as it would induce us to put the extra
condition that the base change functor is compatible with initial
elements, which is true indeed if $\varnothing_M$ is strict, but here a
wholly extraneous condition\ldots)

Thus \emph{$X$ contractible implies $X$ aspheric over $e$}. It is
well-known that the converse isn't true, already for $M=\Sssets$, or
$M=\Spaces$ with the usual notions of weak equivalences, taking
aspheric complexes which are not Kan complexes, or ``aspheric'' spaces
(in the sense of singular cohomology) which are not CW-spaces. The
same examples show, too, that even a good honest homotopy interval
need not be contractible,\pspage{108} contrarily to what one would
expect from the intuitive meaning of an ``interval''. In those two
examples, the condition of getting two ``disjoint'' sections of the
aspheric $I$ over $e$ looks kind of trivial -- kind of unrelated to
the question of whether or not $I$ is actually contractible, and not
only aspheric. What comes to mind here is to look for \emph{contractible
  homotopy intervals}, namely homotopy intervals
$(I,\delta_0,\delta_1)$ (including condition \ref{it:37.b} that
$\Ker(\delta_0,\delta_1)=\varnothing_M$) such that $I$ is not only
aspheric over $e$, but even contractible. Existence of contractible
homotopy intervals seems a priori a lot stronger than just existence
of homotopy intervals, but in the cases I've met so far, the two
apparently coincide. As a matter of fact, the first choice of homotopy
interval that comes to mind, in all those examples, \emph{is} indeed a
contractible one -- so much so that till the notes today I was
somewhat confused on this matter, and was under the tacit impression
that homotopy intervals are all contractible, and trivially so! But
probably, as usual with most evidently false impressions, underneath,
something correct should exist, worth being explicitly stated.

First thing that comes to mind is here to look at the simplest case of
an aspheric interval \bI{} which is contractible -- namely when \bI{}
is even \bI-contractible, and more specifically still, when there
exists an \bI-homotopy (an ``elementary'', not a ``composed'' one!)
from $\id_I$ to one of the two constant maps of $I$ into itself,
defined by the two sections $\delta_0,\delta_1$ -- say by $\delta_1$,
to be specific. Such a homotopy is a map
\[ h : I\times I\to I\]
in $M$, having the two properties expressed symbolically by
\[ h(e_0,x)=x, \quad h(e_1,x)=e_1,\]
where $x$ may be viewed as any ``point'' of $I$ with ``values'' in an
arbitrary parameter object $T$ of $M$, i.e., $x:T\to I$, and $e_0,e_1$
are the constant maps $T\to I$ defined by $\delta_0,\delta_1$. If we
view $h$ as a composition law on $I$, these relations just mean that
$e_0$ acts as a left unit, and $e_1$ as a left zero element. Such
composition law in an interval $I$, a would-be homotopy interval as a
matter of fact, has been repetitively used before, and this use
systematized in the ``\hyperref[lem:comparisonlemmaforHI]{comparison lemma
  for homotopy intervals}'' (page \ref{p:60}); there we found that if
$I$ is an object of $M$ with such a composition law, and if there
exists an \emph{aspheric} interval \bJ{} and a morphism of intervals
from \bJ{} into \bI, then $I$ is equally aspheric. This we would now
see as coming from the fact that $I$ being \bI-contractible is
\bJ-contractible, and a fortiori (as \bJ{} is aspheric over $e$)
aspheric over $e$. In any case, we see that when $I$ is endowed with a
composition law as above, then $I$ is aspheric over $e$ if{f} it is
contractible, and equivalently if{f} $\delta_0$ and $\delta_1$ are
homotopic.

In\pspage{109} case when there exists a Lawvere element $L_M$ in $M$,
for instance if $M$ is a topos, this element is automatically endowed
with an idempotent composition law, coming from intersection of
subobjects, and in case $M$ admits a strict initial object, $L_M$ is
endowed with two canonical sections which are indeed ``disjoint'',
corresponding respectively to the ``full'' and the ``empty''
subobjects. Then (as already noticed) previously in the
\hyperref[cor:ofcomparisonlemmaforHI]{corollary} of the comparison
lemma) $M$ admits a homotopy interval if{f} $L_M$ is such an interval,
namely is aspheric over $e_M$. But we can now add that in this case,
there exists even a \emph{contractible} homotopy interval, namely
$L_M$ itself. It is even a ``\emph{strict homotopy interval}'', namely
one admitting a composition $L\times L\to L$ having the properties
above (where $\delta_0$ and $\delta_1$, for the first time, play
asymmetric roles!). Thus when a Lawvere element exists in $M$, there
is an equivalence between the three properties we may expect from a
pair $(M,W)$, with respect to homotopy intervals, namely:
\[ (\exists\text{homotopy int.}) \Leftarrow
(\exists\text{contractible hom.\ int.}) \Leftarrow
(\exists\text{strict hom.\ int.}).\]
It seems that in all cases I've in mind at present, when there exists
a homotopy interval, there exists even a strict one. The only case
besides topoi (where it is so, because of the existence of a Lawvere
element) which I have looked up so far, is the case of \Cat{} and
various full subcategories, all containing $\Simplex_1$ which is indeed
a strict homotopy interval, as it represents the presheaf on \Cat{}
\[ C\mapsto\text{set of all cribles in $C$.}\]
If we take the choice $I=$ two-point category equivalent to final one,
this also is a strict homotopy interval, as it represents the functor
\[ C\mapsto\text{set of all full subcategories of $C$,}\]
hence again an intersection law. The first homotopy interval though is
a lot more important than the second, because the first one is
``characteristic'', namely sufficient for checking the homotopy
relation between any two maps in \Cat, whereas the second isn't. A
``\emph{perfect}'' homotopy interval would be one which is both
\emph{strict} (hence contractible) and \emph{characteristic}.

\bigbreak
\noindent\hfill\ondate{2.4.}\par

% 50
\hangsection[Reflection on some main impressions. The foresight of
\dots]{Reflection on some main impressions. The foresight of an
  ``idyllic picture'' \texorpdfstring{\textup(}{(}of would-be
  ``canonical
  modelizers''\texorpdfstring{\textup)}{)}.}\label{sec:50}%
While writing the notes last time, and afterwards while pondering a
little more about the matter, a few impressions came gradually into
the fore. One was about the interplay of four basic ``homotopy
notions'' which more or less mutually determine each other, namely the
\emph{homotopy relation} between maps, the notion of \emph{homotopy
  interval}, the notion of homotopy equivalences or \emph{homotopisms}
(which has formal analogy to weak equivalences\pspage{110} the was it
is handled), and the notion of \emph{contractible objects}. Another
impression was about the dependence of these notions upon a
preliminary notion of ``weak equivalence'', namely upon
$W \subset \Fl(M)$, being a rather loose one. Thus the construction of
homotopy notions in terms of a given interval \bI{} (including the
category $M_\bI$ with the canonical functor $M\to M_\bI$) is valid for
any interval in any category $M$ with final object and binary products
(instead of binary products, it is even enough that $I$ be
``squarable'', namely all products $X\times I$ exist in $M$). As for
the $W$-homotopy notions, they depend on $W$ via the corresponding
notion of $W$-asphericity over $e$, which is at first sight \emph{the}
natural condition to impose upon an interval \bI, in order for the
corresponding \bI-homotopy notions to fit nicely with $W$ (as
expressed in the homotopy lemma). But then we noticed that a much
weaker condition than asphericity on \bI{} suffices -- namely that the
two sections $\delta_0,\delta_1$ of $I$ over $e$ be $W$-homotopic,
which means essentially that the ``points'' of $I$ they define can be
``joined'' by a finite chain of $W$-aspheric intervals mapping into
$I$. This strongly suggests (in view of the main application we have
in mind, namely to the study of modelizers) that the natural condition
to impose upon intervals, in most contexts of interest to us, will be
merely $0$-connectedness. But this notion is \emph{intrinsic to the
  category} $M$, irrespective again of the choice of any $W$; and
therefore the corresponding homotopy notions in $M$ will turn out (in
the cases at least of greatest interest to us) to be equally intrinsic
to the category $M$. On the other hand (and here fits in the third
main impression that peeled out two day ago), the work carried through
so far in view of the ``observation'' and the (naive) ``question'' of
last week (pages \ref{p:94} and \ref{p:95}) strongly suggests that in
the nicest modelizers (including \Cat{} and the elementary modelizers,
presumably), the notion of weak equivalence $W$ can be described in
terms of the homotopy notions, more specifically in terms of the
notion of contractible objects (when exactly and how should appear in
due course). Thus it will follow that for those modelizers, the
modelizing structure $W$ itself is uniquely determined in terms of the
intrinsic category structure -- thus any equivalence between the
underlying categories of any two such modelizers should automatically
be model-preserving! It will be rather natural to call the modelizers
which fit into this idyllic picture \emph{canonical modelizers}, as
their modelizing structure $W$ is indeed canonically determined by the
category structure. Next thing then would be to try to gain an overall
view of how to get ``all'' canonical modelizers, if possible in as
concrete terms as the overall view we got upon elementary modelizers
\Ahat{} in terms of the corresponding test categories $A$.\pspage{111}

% 51
\hangsection{The four basic ``pure'' homotopy notions with
  variations.}\label{sec:51}%
\renewcommand*{\thesubsection}{\Alph{subsection})}%
First thing though I would like to do now, is to elaborate ``from
scratch'' on the four basic homotopy notions and their interplay, much
in the style of a ``fugue with variations'' I guess, and without
interference of a pregiven notion $W$ of weak equivalence --
relationship with a $W$ will be examined only after the intrinsic
homotopy notions and their interrelations are well understood.

We start with a category $M$, without for the time being any specific
assumptions on $M$. The strongest we're going to introduce, I guess,
is existence of finite products, and incidently maybe finite sums and
fiber products. In the cases we have in mind, $M$ is a ``large''
category, therefore it doesn't seem timely here introduce $M\uphat$
and the embedding of $M$ into $M\uphat$.

The most trivial implications between the four basic homotopy notions
are symbolized by the plain arrows in the diagram below, the somewhat
more technical ones by dotted arrows. It is understood these notions
correspond to a given ``homotopy structure'' on $M$, symbolized by the
letter $h$, and which (in the most favorable cases) may be described
at will in terms of any one of the four notions. I'll first describe
separately each of these basic notions, and afterwards the
relationships symbolized by the arrows in the diagram. I recall that
in \emph{interval} in $M$ is just an object $I$, endowed with two
subobjects $e_0,e_1$ which are final objects of $M$, or equivalently,
with two sections $\delta_0,\delta_1$ of $I$ over a fixed final object
$e_M=e$ of $M$. I definitely want to forget entirely for the time
being about any condition of the type $e_0\sand e_1=\varnothing_M$
(initial object of $M$) -- we may later refer to these as
``\emph{separated}'' intervals (namely the endpoints $e_0,e_1$ are
``separated''). We'll denote by $\Int(M)$ the set of all intervals in
$M$, by $\bInt(M)$ the corresponding category (the notion of a
morphism of intervals being the obvious one). Now here's the
\namedlabel{fig:organigram}{organigram}:
\begin{equation}
  \label{eq:51.D}
  \begin{tikzcd}[arrows=Rightarrow]
    \begin{tabular}{@{}c@{}}1) homotopy relation\\
      $R_h\subset\Fl(M)\times\Fl(M)$
    \end{tabular} \ar[d]\ar[r] &
    \begin{tabular}{@{}c@{}}2) homotopism\\
      $W_h\subset\Fl(M)$
    \end{tabular} \ar[l, bend right=10, dashed, swap, "1"]\ar[d] \\
    \begin{tabular}{@{}c@{}}3) homotopy intervals\\
      $\Sigma_h\subset\Int(M)$
    \end{tabular} \ar[u, bend left=25, dashed, "2"] &
    \begin{tabular}{@{}c@{}}4) contractible objects\\
      $C_h\subset\Ob(M)$
    \end{tabular} \ar[l, dashed, swap, "3"]
  \end{tikzcd}
  \tag{D}
\end{equation}

% A
\subsection[Homotopy relation between maps]{Homotopy relation between maps.}
\label{subsec:51.A}
As\pspage{112} a \emph{type of structure}, a homotopy relation between
maps in $M$ is a subset
\[R_h\subset\Fl(M)\times\Fl(M),\]
namely a relation in the set $\Fl(M)$ or arrows of $M$, the
\emph{basic assumption} being that whenever $f$ and $g$ are
``homotopic'' arrows, then they have the same source, and the same
target. Thus, the data $R_h$ is equivalent to giving a ``homotopy
relation'' in any one of the sets $\Hom(X,Y)$, with $X$ and $Y$
objects in $M$. The relevant \emph{saturation condition} is twofold:
\begin{enumerate}[label=\alph*)]
\item\label{it:51.A.a}
  the relation $R_h$ is an equivalence relation, or equivalently, the
  corresponding relations in the sets $\Hom(X,Y)$ are equivalence
  relations;
\item\label{it:51.A.b}
  stability under composition: if $f$ and $g$ are homotopic, then so
  are $vf$ and $vg$, and so are $fu$ and $gu$, for any arrow $v$ or
  $u$ such that the relation makes sense.
\end{enumerate}

When these conditions are satisfied, we'll say we got a \emph{homotopy
  relation} between maps of $M$. This relation between $f$ and $g$
will be denoted by a symbol like
\[ f\hsim g.\]
If the basic assumption is satisfied, but not the saturation
condition, there is an evident way of ``saturating'' the given
relation, getting one $\overline R_h$ which is saturated, i.e., a homotopy
relation in $M$ (in fact, the smallest one containing $R_h$).

Given a homotopy relation $R_h$, we denote by
\[ \Hom(X,Y)_h\]
the corresponding quotient sets of the set $\Hom(X,Y)$, they compose
in an evident way, so as to give rise to a category $M_h$ having the
same objects as $M$, and to a canonical functor
\[ M\to M_h\]
which is the identity on objects, and surjective on arrows. We may
view thus $M_h$ as a \emph{quotient category} of $M$, having the same
objects as $M$. Clearly, $R_h\mapsto M_h$ is a bijective
correspondence between the set of homotopy relations in $M$, and the
set of quotient categories of $M$ satisfying the aforesaid
property. By abuse of language, we may even consider that considering
a homotopy relation in $M$, amounts to the same as giving a functor
$M\to M_h$ from $M$ which is bijective on objects and surjective on
arrows.

When we got a homotopy relation $R_h$ in $M$, we deduce a notion of
\emph{homotopisms}
\[ W_h\subset\Fl(M),\]
namely those arrows in $M$ which become isomorphisms in $M_h$. Also
we\pspage{113} deduce a notion of \emph{homotopy interval}, i.e.,
\[\Sigma_h\subset\Int(M),\]
namely those intervals \bI{} in $M$ such that the two marked sections
be homotopic. (NB\enspace $\Sigma_h$ is non-empty if and only if $M$ has a
final element, in this case $\Sigma_h$ contains all intervals such
that $\delta_0=\delta_1$ -- which we may call \emph{trivial}
intervals, for instance the \emph{final interval} with $I=e$\ldots)
This notion of a homotopy interval is considerably wider than the one
we have worked with so far, however it is clearly the right one in the
context of pure homotopy notions. To avoid any confusion, we better
call this notion by the name of \emph{weak homotopy intervals} --
funnily there won't be any unqualified ``homotopy intervals'' in our
present set-up of ``pure'' homotopy notions!

The two prescriptions above account for two among the plain arrows in
our \ref{fig:organigram}.

We'll often make use of an \emph{accessory assumption} on $R_h$, which
can be expressed by demanding that \emph{the canonical functor $M\to
  M_h$ commute to binary products}, in case we assume already such
products exist in $M$. This can be expressed also by the property that
for two maps
\[f,g: X \rightrightarrows Y_1\times Y_2,\]
$f$ and $g$ are homotopic if{f} so are $f_i$ and $g_i$ ($i\in\{1,2\}$)
(where the ``only if'' part is satisfied beforehand anyhow). This
implies too that in $M_h$ binary products exist, and that $f\sim f'$,
$g\sim g'$ implies $f\times g\sim f'\times g'$. On the other hand, it
is trivial that if $M$ admits a final object, this is equally a final
object of $M_h$ (and hence, under the accessory assumption on $R_h$,
the functor $M\to M_h$ commute to finite products).

% B
\subsection[Homotopisms, and homotopism structures]{Homotopisms.}
\label{subsec:51.B}
As a type of structure on $M$, we got just a subset
\[W_h\subset\Fl(M),\]
without any basic assumption to make. The natural \emph{saturation
  condition} is just the strong saturation for a subset of $\Fl(M)$,
which can be expressed by stating that $W_h$ can be obtained as the
set of arrows made invertible by some functor from $M$ into a category
$M'$, or equivalently, by the localization functor
\[M\to W_h^{-1}M.\]
We may refer to a strongly saturated $W_h$ as a ``\emph{homotopism
  structure}'' (or ``homotopy equivalence structure'') in $M$ -- but
as in the case \ref{subsec:51.A}, we'll have soon enough to make
pretty strong extra assumptions. Maybe we should, at the very least,
demand for the notion of homotopy structure that the canonical functor
above, which is bijective on objects in any case, should be moreover
\emph{surjective on arrows} -- thus I'll take this\pspage{114} as a
\emph{basic assumption} after all. This assumption makes sense of
course independently of any saturation condition. If $W_h$ is not
strongly saturated, then denoting by $\overline W_h$ the subset of $\Fl(M)$
of all arrows made invertible by the canonical functor, this will now
be a strongly saturated set of arrows (in fact the smallest one
containing $W_h$, and giving rise to the same localized category, and
hence satisfying the basic assumption too) -- thus $\overline W_h$ is
indeed a homotopism structure on $M$. When $M$ admits a final object,
this will equally be a final object of $W_h^{-1}M$. We may now define
in terms of $W_h$ the notion of contractible objects in $M$, forming a
subset
\[C_h \subset \Ob(M),\]
as those objects $X$ in $M$ such that the projection $p_X:X\to e$ is
in $W_h$, or equivalently, such that $X$ is a final object in the
localized category $W_h^{-1}M$. This accounts for the third plain
arrow of the \ref{fig:organigram}.

We'll now dwell a little more on the first dotted arrow, namely the
description of a homotopy relation
\[R_h \subset \Fl(M)\times\Fl(M)\]
in terms of $W_h$: the natural choice here is to define $f,g\in\Fl(M)$
to be \emph{homotopic} (or $W_h$-homotopic, if ambiguity may arise)
if{f} their images in the category $W_h^{-1}M$ are equal. This
relation between maps in $M$ clearly satisfies the basic assumption on
source and target, as well as the saturation condition -- it is
therefore a ``homotopy relation'' in $M$, namely the one associated to
$W_h^{-1}$, viewed as a quotient category of $M$. It is clear that we
recover $W_h$ from $R_h$, consequently, by the process described in
\ref{subsec:51.A}.

To make the relationship between the notions 1) and 2) still clearer,
let's denote respectively by
\[ \Hom_1(M),\Hom_2(M)\]
the set of all homotopy relations, resp.\ of homotopism notions, in
$M$. We got maps\footnote{\alsoondate{3.4.} There \emph{is no map} $r_{21}$, only
  $r_{12}$, see correction in \S\ref{sec:52}.}
\[\begin{tikzcd}[cramped]
  \Hom_1(M) \ar[r, shift left, "r_{21}"] &
  \Hom_2{M} \ar[l, shift left, "r_{12}"]
\end{tikzcd},\]
and the relevant fact here is that
\[r_{12}:\Hom_2\to\Hom_1\]
\emph{is injective, and admits $r_{21}$ as a left inverse.} Thus, we
may view $\Hom_2$ as a subset of $\Hom_1$, i.e., the structure of a
``homotopism notion'' on $M$ as a particular case of the structure of
a ``homotopy relation'' on $M$. Namely, a structure of the latter type
can be described in terms of a notion of homotopism in $M$, if{f} the
canonical functor $M\to M_h$ it gives to\pspage{115} is a localization
functor.

For a general $R_h\in\Hom_1(M)$, if we consider the corresponding
$W_h$ ($=r_{21}(R_h)$) in $\Hom_2(M)$, it is clear that the canonical
functor $M\to M_h$ of \ref{subsec:51.A} factors into
\[ M \to W_h^{-1} \to M_h,\]
and $R_h$ ``is in $\Hom_2(M)$'', i.e., $R_h=r_{12}(W_h)$, if{f} the
second functor
\[ W_h^{-1}M \to M_h\]
(which is anyhow bijective on objects and surjective on arrows) is an
isomorphism, or equivalently, faithful. Here is a rather direct
\emph{sufficient} condition on $R_h$ for this to be
so,\footnote{\alsoondate{3.4.} See \S\ref{sec:52} for a
  \scrcommentinline{?} of this rash statement!} namely:
\begin{description}
\item[\namedlabel{cond:51.C.12}{C$_{12}$})] If $f,g:X\to Y$ are homotopic,
there exists a homotopism $X'\to X$, two sections $s_0,s_1$ of $X'$
over $X$, and a map $h:X'\to Y$, such that
\[f_0=hs_0, \quad f_1=hs_1.\]
\end{description}
\begin{remark}
  Intuitively, we are thinking of course of $X'$ as a product $X\times
  I$, where $\bI=(I,e_0,e_1)$ is a weak homotopy interval, and
  $s_0,s_1$ are defined in terms of $e_0,e_1$. In Quillen's somewhat
  different set-up, $X'$ is referred to as a ``cylinder object for
  $X$'', suitable for defining the ``left homotopy relation''
  associated to a given $W_h$. The condition \ref{cond:51.C.12} is not
  autodual, we could state a dual sufficient condition in terms of a
  ``path object for $Y$'', namely a homotopism $Y\to Y'$ endowed with
  two retractions $t_0,t_1$ upon $Y$ -- but we don't have any use for
  this in the present set-up, which (as for as the main emphasis is
  concerned) is by no means autodual, as is Quillen's.
\end{remark}
The condition \ref{cond:51.C.12} above can be viewed equally as a
condition on a $W_h\in\Hom_2(M)$.

We may interpret the set $\Hom_2(M)$ of homotopism notions in $M$ as
the set of all quotient categories $M_h$ of $M$, having the same
objects as $M$, and such that moreover the canonical functor $M\to
M_h$ be a localizing functor. As in \ref{subsec:51.A}, the relevant
``\emph{accessory assumption}'' on $W_h$ (a particular case indeed of
the corresponding one for $R_h$) is that this functor commute to
products. I don't see any simple computational way though to express
this condition directly in terms of $W_h$, as previously in terms of
$R_h$. I would only like to notice here a consequence of this
assumption (I doubt it is equivalent to it), namely that \emph{the
  cartesian product of two homotopisms is again a homotopism} -- which
implies, for instance,\pspage{116} that the product of a finite family
of contractible objects of $M$ is again contractible.

% C
\subsection[Homotopy interval structures]{Weak homotopy intervals.}
\label{subsec:51.C}
We assume $M$ stable under finite products. The type of structure
we've in view is a set of intervals in $M$,
\[\Sigma_h\subset\Int(M),\]
called the ``weak homotopy intervals''. No basic assumption on this
set, it seems; the natural ``\emph{saturation condition}'' is the
following:
\begin{description}
\item[\namedlabel{cond:51.Sat.3}{(Sat 3)}] Any interval
  $\bI=(I,\delta_0,\delta_1)$ in $M$, such that the sections
  $\delta_0,\delta_1$ of $I$ be $\Sigma_h$-homotopic (see below), is
  in $\Sigma_h$.
\end{description}

The assumption on \bI{} means, explicitly, that there exists a finite
chain of sections of $I$
\[s_0=\delta_0, s_1, \dots, s_N=\delta_1,\]
joining $\delta_0$ to $\delta_1$, and for two consecutive
$s_i,s_{i+1}$ an interval \bJ{} in $\Sigma_h$, and a map of intervals
from \bJ{} or $\check\bJ$ to $(I,s_i,s_{i+1})$, i.e., a map $J\to I$,
mapping the two given sections of $J$, one into $s_i$, the other into
$s_{i+1}$ (without specification which is mapped into which).

The significance of this saturation condition becomes clear in terms
of the second dotted arrow of the \ref{fig:organigram}. Namely, in terms of any
subset $\Sigma_h$ of $\Int(M)$, we get a corresponding homotopy
relation between maps, say $R_h$, which is the equivalence relation in
$\Fl(M)$ generated by the ``elementary'' homotopy relation (with
respect to $\Sigma_h$) between maps $f,g$ in $M$, namely the relation $R_0$
\[ f \mathrel{\underset{R_0}{\sim}} g
\;\xLeftrightarrow{\textup{def}}\;
\parbox[t]{0.6\textwidth}{there exists
  \bI{} in $\Sigma_h$, and an \bI-homotopy from $f$ to $g$.}\]
The corresponding equivalence relation $R_h$ in $\Fl(M)$ is already
saturated, namely stable under compositions, moreover it satisfies
condition \ref{cond:51.C.12} above -- thus we may view this homotopy
relation as defined in terms of a homotopisms notion -- thus in fact
the second dotted arrow should go from 3) to 2) rather than from 3) to
1)! Now, if we look at the subset $\Sigma_h$ of $\Int(M)$
defined in terms of $R_h$ as in \ref{subsec:51.A} (namely the
set of ``homotopy intervals with respect to $R_H$''), we get
\[ R_h \subset \overline R_h,\]
and the equality holds if{f} $R_h$ satisfies \ref{cond:51.Sat.3}? At
the same time, in case of arbitrary $R_h$, we get the construction of
its saturation, $\overline R_h$, which may of course be described
alternatively as the smallest saturated subset of $\Int(M)$ containing
$R_h$.

We'll call \emph{weak homotopy interval structures} on $M$, any
set\pspage{117} $\Sigma_h$ of intervals in $M$, satisfying the
saturation condition above. The set of all such structures on $M$ will
be denoted by $\Hom_3(M)$, thus we get two embeddings
\[ \Hom_3(M) \hookrightarrow \Hom_2(M) \hookrightarrow \Hom_1(M),\]
in such a way that a weak homotopy interval structure on $M$ may be
viewed also as a particular case of a homotopism structure on $M$, and
a fortiori as a particular case of a homotopy relation on $M$. Of
course, the homotopy relations or homotopism structures on $M$ we'll
ultimately be interested in, are those stemming from weak homotopy
interval structures on $M$. Recall that $M$ admits finite products,
and these structures satisfy automatically the accessory assumption,
namely commutation of the canonical functor $M\to M_h$ to finite
products.

It is immediate that if we start with a homotopy relation $R_h$, the
corresponding $\Sigma_h$ as defined in \ref{subsec:51.A} is
saturated. Thus, the canonical embedding $r_{13}$ of $\Hom_3$ into
$\Hom_1$ admits a canonical left inverse $r_{31}$, the restriction of
which to $\Hom_2$ is a canonical left inverse $r_{32}$ of the natural
embedding $r_{23}$ of $\Hom_3$ into $\Hom_2$.

% D
\subsection[Contractibility structures]{Contractibility structures.}
\label{subsec:51.D}
(We still assume $M$ admits finite products.) As a type of structure,
it is a set of objects of $M$
\[ C_h\subset \Ob(M),\]
without any ``basic assumption'' on $C_h$ it seems. These objects will
be called the \emph{contractible} objects. Sorry, there \emph{is} a
basic assumption here I just overlooked, namely every $X$ in $C_h$
should have at least one section (thus I better assume beforehand, as
in \ref{subsec:51.C}, that $M$ has a final object $e$). To get the
natural saturation condition on $C_h$, we'll make use of the third
dotted arrow in the \ref{fig:organigram}, by associating to $C_h$ the set
$\Sigma_h$ of ``contractible intervals'', namely intervals
$\bI=(I,\delta_0,\delta_1)$ such that $I$ is in $C_h$. Of course in
general there is no reason that $\Sigma_h$ should be saturated, never
mind -- it defines anyhow (as seen in \ref{subsec:51.C}) a homotopism
notion in $M$, and hence (as seen in \ref{subsec:51.B}) a notion of
contractible objects, i.e., another subset $\overline C_h$ of
$\Ob(M)$. Now it occurs to me that it is by no means clear that the
latter contains $C_h$, which brings near the necessity of a more
stringent \emph{basic assumption} on $C_h$, namely for the very least
\[ C_h \subset \overline C_h\]
(this will imply that any $X$ in $C_h$ has indeed a section over $e$,
as this is automatically the case for $W_h$-contractible objects). The
saturation condition (Sat~4) will of course be equality
\[ C_h = \overline C_h,\]
and\pspage{118} for general $C_h$ (satisfying the basic assumption
$C_h\subset\overline C_h$), $\overline C_h$ can be viewed as the
``saturation'' of $C_h$, namely the smallest saturated subset of
$\Ob(M)$ satisfying the basic assumption, or in other words, the
smallest \emph{contractibility structure} on $M$ such that the objects
in $C_h$ are contractible.

It may be worth while to state more explicitly the basic assumption
here, and the saturation condition on $C_h$.
\begin{description}
\item[\namedlabel{cond:51.Bas.4}{(Bas~4)}]
  For any $X$ in $C_h$, we can find a finite sequence of maps from $X$
  to $X$,
  \[ f_0=\id_X,f_1,\dots,f_N=c_s,\]
  joining the identity map of $X$ to a constant map $c_s$ (defined by
  some section $s$ of $X$), in such a way that two consecutive maps
  $f_i,f_{i+1}$ are $C_h$-homotopic in the strict sense, namely we can
  find $Y_i$ in $C_h$ and two sections $\delta_0^i$ and $\delta_1^i$
  of $Y_i$ over $e$, and a map
  \[ h_i : Y_i\times X\to X,\]
  such that
  \[ h_i\circ(\delta_0^i\times\id_X)=f_i,\quad
  h_i\circ(\delta_1^i\times\id_X)=f_{i+1}.\]
\item[\namedlabel{cond:51.Sat.4}{(Sat~4)}]
  \emph{Any} object $X$ in $M$ satisfying the condition just stated is
  in $C_h$.
\end{description}

The third dotted arrow can be viewed as denoting an embedding of
$\Hom_4(M)$ (the set of all contractibility structures on $M$) into
$\Hom_3(M)$, we finally get a cascade of three inclusions
\[\Hom_4(M) \hookrightarrow
\Hom_3(M) \hookrightarrow
\Hom_2(M) \hookrightarrow
\Hom_1(M),\]
in terms of which a contractibility structure on $M$ can be viewed a a
particular case of any of the three types of homotopy structures on
$M$ considered before.

If we start with a homotopism structure $W_h$ on $M$, and consider the
corresponding set $C_h$ of contractible objects of $M$ (namely objects
$X$ such that $X\to e$ is in $W_h$), it is pretty clear that $C_h$
satisfies the saturation condition \ref{cond:51.Sat.4}, but by no
means clear that it satisfies the basic assumption
\ref{cond:51.Bas.4}, even in the special case when we assume moreover
that $W_h$ comes from a weak homotopy interval structure $\Sigma_h$ on
$M$. The trouble comes from the circumstance that there is no reason
in general that the contractibility of an object $X$ of $M$ can be
described in terms of a sequence of elementary homotopies between maps
$f_i:X\to X$ (joining $\id_X$ to a constant map) involving weak
homotopy intervals $\bI_i$ \emph{which are moreover contractible}. I
doubt this is always so, and there doesn't come either any plausible
extra condition on $\Sigma_h$ which may ensure this, except precisely
that $\Sigma_h$ can be generated (through saturation)\pspage{119} by
the subset $\Sigma_{hc}$ of its contractible elements, which is just
another way of saying that this $\Sigma_h\in\Hom_3(M)$ comes already
from a contractibility structure $C_h\in\Hom_4(M)$! Thus, definitely
the uniformity of formal relationships between successively occurring
notions seems broken here, namely there does not seem to be any
natural retraction $r_{43}$ of $\Hom_3(M)$ onto the subset
$\Hom_4(M)$. For the least, if there is such a retraction, its
definition should be presumably a somewhat more delicate one than the
first that comes to mind. I will not pursue this matter any further
now, as it is not clear if we'll need it later.

It is clear that for any weak homotopy interval structure $\Sigma_h$
on $M$, $\Sigma_h$ is stable under the natural notion of finite
products of intervals (in the sense of the category structure of
$\bInt(M)$). We saw already that this is handy, as the consideration
of products of intervals allows to show that the family of homotopy
relations $\Isim$ in $\Fl(M)$, for variable \bI{} in $\Sigma_h$, is
``filtrant d\'ecroissant'', so we get the relation $\hsim$ as the
filtering direct limit or union of the more elementary relations
$\Isim$. Similarly, if $C_h\subset\Ob(M)$ is a contractibility
structure on $M$, $C_h$ is stable under finite products.
\begin{remark}
  From the way we've been working so far with homotopy notions, it
  would seem that we're only interested here in homotopy notions which
  stem from a structure in $\Hom_3(M)$, namely which can be described
  in terms of a notion of weak homotopy intervals. The focus on
  contractibility has set in only lately, and it is too soon to be
  sure whether we'll be working only with homotopy structures on $M$
  which can be described in terms of a contractibility notion, namely
  which are in $\Hom_4(M)$. In the cases I've had in mind so far, it
  turns out, it seems that the homotopy notions dealt with do come
  from a structure in $\Hom_4(M)$, i.e., from a contractibility
  structure.
\end{remark}

% E
\subsection[Generating sets of homotopy intervals. Two standard ways
of generating multiplicative intervals. Contractibility of
\texorpdfstring{$\Hom(X,Y)$}{Hom(X,Y)}'s]{Generating sets of weak
  homotopy intervals. Contractors.}\label{subsec:51.E}
Let $\Sigma_h$ be a weak homotopy interval structure on
$M$.\footnote{This implies we assume $M$ stable under finite
  products.} A subset $\Sigma_h^0$ is called \emph{generating}, if
$\Sigma_H$ is just its saturation (cf.\ \ref{subsec:51.C} above),
i.e., for any \bI{} in $\Sigma_h$, the two endpoints can be joined by
a chain as in \ref{cond:51.Sat.3}, involving only intervals in $\Sigma_h^0$.
This implies that all homotopy notions dealt with so far can be
checked directly in terms of intervals in $\Sigma_h^0$. We've met a
particular case of this before, when $\Sigma_h^0$ is reduced to just
one element \bI{} -- we then called \bI{} ``characteristic'', but
``\emph{generating weak homotopy interval}'' now would seem the more
appropriate expression. Even when\pspage{120} there should not exist
such a generating interval, the natural next best assumption to make
is the existence of a generating set $\Sigma_h^0$ which is ``small''
(namely an element of the ``universe'' we are working in). The case of
a finite generating set of intervals reduces to the case of a single
one though, by just taking the product of those intervals.

An interesting case is when the generating set $\Sigma_h^0$ consists
of contractible objects of $M$. Such a generating set exists if{f} the
structure considered $\Sigma_h$ comes from a contractibility
structure. About the best we could hope for is the existence of a
single \emph{generating contractible weak homotopy interval} \bI. If
we got any interval \bI{} in $M$, this can be viewed as a generating
contractible weak homotopy interval for a suitable homotopy structure
on $M$ (then necessarily unique) if{f} the identity map of $I$ can be
joined to a constant one by a chain of maps, such that two consecutive
ones are tied by an \bI-homotopy or an $\check\bI$-homotopy. The most
evident way to meet this condition is by a \emph{one-step chain} from
$\id_I$ to the constant map defined by one of the endpoints,
$\delta_1$ say. This brings us back to structure of a composition law
\[ I\times I\to I\]
in $I$, having $e_0$ as a left unit and $e_1$ as a left zero
element. Let's call an interval, endowed with such a composition law,
a \emph{contractor} in $M$. Thus starting from a contractor in $M$ is
about the nicest way to define a homotopy structure in $M$, as a
matter of fact the strongest type of such a structure -- namely a
contractibility structure, admitting a \emph{generating contractible
  weak homotopy interval} (and better still, admitting a
\emph{generating contractor}).

Of course, starting with the weakest kind of homotopy structure on
$M$, namely just a homotopy relation $R_h\in\Hom_1(M)$, if \bI{} is a
homotopy interval which is moreover endowed with a structure of a
contractor, i.e., if it is a contractor such that the end-point
sections $\delta_0,\delta_1$ are homotopic, then $I$ is automatically
contractible (never mind if it is generating or not).

It seems to me that the homotopy structures I've looked at so far
(such as \Cat) and various standard elementary modelizers \Ahat) are
not only contractibility structures, but they all can be defined by a
single contractor each.

Besides the ``\emph{basic contractor}'' $\Simplex_1$ in \Cat, there are
two general ways I've met so far for getting contractors. One has been
made explicit in these notes a number of times, namely the
\emph{Lawvere element} $L_M$ in $M$\pspage{121} if it exists, and if
moreover $M$ has a \emph{strict} initial element,
$\varnothing_M$. Recall that $L_M$ represents the contravariant functor
on $M$
\[X\mapsto\text{set of all subobjects of $X$,}\]
and that the ``full'' and ``empty'' subobjects of $X$, for variable
$X$, define two sections $\delta_0$ and $\delta_1$ of $L_M$. I forgot
to state the extra condition that in $M$ fibered products exist
(intersection of two subobjects would be enough); then the
intersection law endows $L_M$ with a structure of a contractor
$\bL_M$, admitting $\delta_0$ as a unit and $\delta_1$ as a zero
element. Moreover, it is clear that $\bL_M$ as an interval is
separated, i.e., $\Ker(\delta_0,\delta_1)=\varnothing_M$. More precisely
still, $\bL_M$ can be viewed as a \emph{final object of the category
  of all separated intervals in $M$}, namely for any such interval,
there is a \emph{unique} map of intervals
\[\bI\to\bL_M.\]
This implies that if $M$ is endowed with a homotopy structure, such
that there exists a weak homotopy interval which is separated, then
$\bL_M$ is such an interval, and it is moreover contractible. It is
doubtful though, even if we can find a \emph{generating} contractor
for the given homotopy structure on $M$, that the Lawvere contractor
is generating too.

Here now is a second interesting way of getting contractors. We assume
that $M$ admits finite products (as usual). Let $X$ be an objects, and
assume the object $\Hom(X,X)$, representing the functor
\[Y \mapsto \Hom(X\times Y,X) = \Hom_Y(X_Y,X_Y),\]
exists in $M$. (NB\enspace $X_Y$ denotes $X\times Y$, viewed as an object of
$M_{/Y}$.) Composition of endomorphisms of $X_Y$ clearly endow this
functor with an associative composition law, admitting a two-sided
unit, which I call $e_0$. Notice that sections of
\[\bI=\bHom(X,X)\]
can be identified with maps $X\to X$, and the section corresponding to
$\id_X$ is of course the two-sided unit. On the other hand, if $X$
admits sections, i.e., admits ``constant'' endomorphisms, it is
clear that the corresponding sections of \bI{} are left zero
elements. If we choose a section of $X$, \bI{} becomes a
contractor. Its interest lies in the following
\begin{proposition}
  Assume finite products exist in $M$, and $M$ endowed with a homotopy
  structure.\footnote{\scrcommentinline{?}} Let $X$ be an object of $M$ endowed with a
  section $e_X$, and suppose the object $\bHom(X,X)$ exists, hence a
  contractor \bI{} as seen above. The following two conditions are
  equivalent:
  \begin{enumerate}[label=\alph*),font=\normalfont]
  \item\label{it:51.E.a}
    $X$ is contractible.
  \item\label{it:51.E.b}
    \bI{} is contractible \textup(or, equivalently as seen above,
    \bI{} is a weak homotopy interval, namely the two endpoints are homotopic\textup).
  \end{enumerate}
  Moreover,\pspage{122} this condition implies the following two:
  \begin{enumerate}[label=\alph*),font=\normalfont,resume]
  \item\label{it:51.E.c}
    For any object $Y$ in $M$, if $\bHom(Y,X)$ exists, it is contractible.
  \item\label{it:51.E.d}
    For any $Y$ in $M$, if $\bHom(X,Y)$ exists, the natural map
    \[ Y \mapsto \bHom(X,Y)\]
    \textup(identifying $Y$ to the ``subobject of constant maps from
    $X$ to $Y$''\textup) is a homotopism.
  \end{enumerate}
\end{proposition}

The equivalence of \ref{it:51.E.a} and \ref{it:51.E.b} is just a
tautological translation of contractibility and homotopy relations in
terms of weak homotopy intervals \bJ{} (cf.\ cor.\ \hyperref[cor:51.E.2]{2}
below). That \ref{it:51.E.c} and \ref{it:51.E.d} follow comes from the
fact that the monoid object $I=\bEnd(X)=\bHom(X,X)$ operates on the
left on $\bHom(Y,X)$, on the right on $\bHom(X,Y)$, and the following:
\begin{corollarynum}\label{cor:51.E.1}
  Let \bI{} be a weak homotopy interval, assume the underlying $I$
  ``operates'' on an object $H$, namely we are given a map
  \[ h:I\times H\to H\quad(\text{``operation'' of $I$ on $H$})\]
  satisfying the relations \textup(where $h(u,f)$ is written simply
  $u\cdot f$\textup{):}
  \[ e_0 \cdot f = f, \quad e_1\cdot (e_1 \cdot f) = e_1 \cdot f,\]
  namely $e_0$ acts as the identity and $e_1$ acts as an idempotent
  $p$ on $H$ \textup(a very weak associativity assumption indeed if
  $I$ is a contractor, as $e_1\cdot e_1=e_1$\textup). Assume the image
  of $p$, i.e., $\Ker(\id_H,p)$ exists, let $H_0$ be the corresponding
  subobject of $H$, and
  \[p_0:H\to H_0\]
  the map induced by $p$. Then $p_0$ is a homotopism \textup(and hence
  the inclusion $i:H_0\to H$, which is a section of $p_0$, is a
  homotopism too\textup).
\end{corollarynum}

Because of the saturation property \ref{it:48.cprime} on homotopies, it
is enough to check that $p=ip_0$ is a homotopism (as $p_0i=\id_H$
already is one), and for this it is enough to see it is homotopic to
the identity map of $H$. But a homotopy between the two is realized by
$h$, qed.

The argument for equivalence of \ref{it:51.E.a} and \ref{it:51.E.b}
above can be generalized as follows:
\begin{corollarynum}\label{cor:51.E.2}
  Let $M$ be as before, and $X$ and $Y$ objects such that $H =
  \bHom(X,Y)$ exists in $M$. Let $f,g:X\rightrightarrows Y$ be two
  maps, which we'll identify to the corresponding sections of
  $H$. Then $f$ and $g$ are homotopic maps if{f} they give rise to
  homotopic sections of $H$.
\end{corollarynum}

% F
\subsection{The canonical homotopy structure: preliminaries on
  \texorpdfstring{$\pi_0$}{pi0}.}\label{subsec:51.F}
In order to simplify life, I will in this section make the following
assumptions on $M$ (which presumably, except for the first, could
be\pspage{123} considerably weakened, but these will be sufficient):
\begin{enumerate}[label=\alph*)]
\item\label{it:51.F.a}
  Finite products exist in $M$ (``pour m\'emoire'').
\item\label{it:51.F.b}
  Arbitrary sums exist in $M$, they are ``disjoint'' and ``universal''
  (which implies that $M$ has a \emph{strict} initial object).
\item\label{it:51.F.c}
  Every object in $M$ is isomorphic to a direct sum of $0$-connected ones.
\end{enumerate}

I recall an object is called \emph{$0$-connected} if it is a)
``non-empty'', i.e., non-isomorphic to $\varnothing_M$, and b)
connected, i.e., any decomposition of it into a sum of two subobjects
is trivial (namely, one is ``empty'', the other is ``full''). Also,
under the assumption \ref{it:51.F.b}, well-known standard arguments
show that for any object $X$, a decomposition of $X$ into a direct sum
of $0$-connected components is essentially unique (namely the
corresponding set of subobjects of $X$ is unique), if it exists. The
subobjects occurring in the sum are called the \emph{connected
  components} of $X$, and the set of connected components of $X$ is
denoted as usual by $\pi_0(X)$. We thus get a natural functor
\[ X \mapsto \pi_0(X), \quad M \to \Sets,\]
This can be equally described as a \emph{left adjoint} to the functor
\[ E \mapsto E_M, \quad \Sets \to M,\]
associating to a set $E$ the corresponding ``\emph{constant object}''
$E_M$ of $M$, sometimes also denoted by the product symbol $E\times
e_M = E\times e$ ($e$ the final object of $M$), namely the direct sum
in $M$ of $E$ copies of $e$. The adjunction formula is
\[\Hom_M(X,E_M) \simeq \Hom_\Sets(\pi_0(X),E).\]
The adjunction relation implies that the functor $\pi_0$ commutes with
all direct limits which exist in $M$, and in particular (and trivially
so) to direct sums.

I'll finally assume also, to fix the ideas:
\begin{enumerate}[label=\alph*),resume]
\item\label{it:51.F.d}
  The final object of $M$ is $0$-connected, i.e., $\pi_0(e)=$
  one-point set.
\end{enumerate}

Empty objects of $M$, on the other hand, are of course characterized
by the condition
\[\pi_0(\varnothing_M)=\emptyset.\]

Finally, I'll make in the end a very strong assumption on $M$, which
however is satisfied more or less trivially in the cases we are
interested in, when $M$ is a would-be modelizer:
\begin{enumerate}[label=\alph*),resume]
\item\label{it:51.F.e}
  (Total $0$-asphericity of $M$): the product of two $0$-connected
  objects of $M$ is again $0$-connected.
\end{enumerate}

This is clearly equivalent to the condition
\begin{enumerate}[label=\alph*'),start=5]
\item\label{it:51.F.eprime}
  The functor $\pi_0:M\to\Sets$ commutes to finite products.
\end{enumerate}
\begin{remarks}
  The\pspage{124} crucial assumptions here seem to be \ref{it:51.F.b}
  (which allows definition of a $\pi_0$ functor, and topological
  intuition tied up with connectedness to enter into play), and
  \ref{it:51.F.e}, which implies that with respect to cartesian
  products, the usual intuitive background for connectedness, rooted
  in the example of $M = (\text{topological spaces})$ is indeed
  valid. This condition is clearly stronger than \ref{it:51.F.d},
  which is a mere condition for convenience in itself (otherwise, a
  decomposition of $e$ into connected components would mean a
  corresponding decomposition of $M$ as a product category, and
  everything could be looked at ``componentwise''). As for
  \ref{it:51.F.e} it could probably be dispensed with, by still
  defining $\pi_0(X)$ as a strict pro-set. In the case of modelizers
  anyhow, such generalization seems quite besides the point.

  The condition \ref{it:51.F.e}, however innocent-looking in terms of
  topological intuition, seems to me an extremely strong condition
  indeed. I suspect that in case $M$ is a topos, it is equivalent to
  total asphericity. In any case, if $M$ is the topos associated to a
  locally connected topological space, we've seen time ago that the
  condition \ref{it:51.F.e} implies $X$ is irreducible, and hence
  totally aspheric. In view of this exactingness of \ref{it:51.F.e},
  I'll not use it unless specifically stated.
\end{remarks}

\bigbreak
\noindent\hfill\ondate{3.4.}\par

% 52
\hangsection{Inaccuracies rectified.}\label{sec:52}%
Before pursuing the review of ``pure homotopy notions'' begun in
yesterday's notes, I would like to correct some inaccuracies which
flew in when looking at the relationship between the two first basic
homotopy notions, namely the notion of a homotopy relation, and the
notion of a homotopism structure. As usual, the provisional image I
had in mind was still somewhat vague, while the reasonable
expectations came out more clearly through the process of writing
things down (including factual inaccuracies!).

The two notions clearly correspond to two kinds of ways of
constructing new categories $M'$ in terms of a given one, and a
functor
\[ M \to M'\]
which is bijective on objects, and has the property moreover that for
any category $C$, the corresponding functor
\[\bHom(M',C) \to \bHom(M,C)\]
is a fully faithful embedding in the strict sense, namely injective
on\pspage{125} objects. One way is to take for $M'$ any quotient
category of $M$, by an equivalence relation which is the discrete one
on objects -- thus it corresponds just to an equivalence relation $R$
in $\Fl(M)$, compatible with the source and target maps and with
compositions. The other is to take as $M'$ a localization with respect
to some $W\subset\Fl(M)$, and if we take $W$ strongly saturated we get
a bijective correspondence between these $M'$ and the set of strongly
saturated subsets of $\Fl(M)$.

It wouldn't be any more reasonable to call an arbitrary $R$ as above,
corresponding to an arbitrary quotient category $M'$ with the same
objects as $M$, a ``homotopy relation'' on $M$ (as I did though
yesterday), as it would be to call an arbitrary strongly saturated
$W\subset\Fl(M)$ a ``homotopism structure'' on $M$ (as I nearly did
yesterday, but then rectified in the stride). \emph{The characteristic
  flavor of homotopy theory comes in, when we get an $M'$ which is
  \emph{both} a quotient category and a localization of $M$.} Thus
neither approach, via $R$ or via $W$, is any more contained in the
other, then the converse. We should regard the \emph{homotopy
  structure} on $M$ to be embodied in the basic functor
\[M\to M',\]
which is a description where no choice yet is made between the two
possible descriptions of $M'$, either by an $R$, or by a $W$. If we
describe $M'$ in terms of $R$, the extra assumption to make on $R$ for
calling it a ``\emph{homotopy relation}'', is that the canonical
functor
\[ M \to M/R\]
should be localizing. Alternatively, describing $M'$ by $W$, the extra
assumption to make on $W$ (as we did yesterday) in order to call $W$ a
\emph{homotopism structure} on $M$, is that the canonical localization
functor
\[ M \to W^{-1}M\]
be essentially a passage-to-quotient functor, namely surjective on
arrows (as we know already it is bijective on objects). Thus the set
of all homotopy relations on $M$ is in one-to-one correspondence with
the set of all homotopism structures on $M$, and if we denote these
sets (in accordance with yesterday's provisional notations)
$\Hom_1(M)$ and $\Hom_2(M)$, we get thus a \emph{bijective}
correspondence
\[ \Hom_1(M) \leftrightarrow \Hom_2(M).\]
The set of homotopy structures $\Hom(M)$ on $M$ may be either defined
as the usual quotient set defined by the previous two-member system of
transitive bijections between sets, or more substantially, as a set of
isomorphism classes of categories $M'$ ``under M'', i.e., endowed with
a functor $M\to M'$, and subject to the following two extra
conditions:
\begin{enumerate}[label=\alph*)]
\item\label{it:52.a}
  The\pspage{126} functor $M\to M'$ is bijective on objects,
  surjective on arrows.
\item\label{it:52.b}
  The functor $M\to M'$ is a localization functor (it will be so then,
  in view of \ref{it:52.a}, in the strict sense, namely $M'$ will be
  $M$-isomorphic, not only $M$-equivalent, to a localization $W^{-1}M$).
\end{enumerate}

However, the question arises whether it is possible to define such a
homotopy structure on $M$ in terms of an arbitrary $R$, i.e., an
arbitrary quotient category having the same objects (let $Q(M)=Q$ be
the set of all such $R$'s) or in terms of an arbitrary localization of
$M$, or what amounts to the same, in terms of an arbitrary strongly
saturated $W\subset\Fl(M)$ (let's call $L(M)=L$ the set of all such
$W$'s). The first thing that comes to mind here, is that we got two
natural maps
\begin{equation}
  \label{eq:52.1}
  \begin{tikzcd}[cramped]
    Q \ar[r, bend left, "r"] &
    L \ar[l, bend left, "s"]
  \end{tikzcd}
  \tag{1}
\end{equation}
between $Q$ and $L$, which are defined by the observation that
whenever we have a functor $i:M\to M'$, injective on objects, it
defines both an $R\in Q$ (namely $f\Rsim g$ if{f} $i(f)=i(g)$) and a
$W\in L$ (namely $f\in W$ if{} $i(f)$ is invertible). For defining
$r(R)$ resp.\ $s(W)$, we apply this to the case when $M'=M/R$ resp.\
$W^{-1}M$. Let's look a little at the two cases separately.

Start with $R$ in $Q$, we get $W=r(R)$,
\[ W = \set[\big]{f\in\Fl(M)}{\text{$i(f)$ invertible}},\]
where
\[ i : M\to M/R\]
is the canonical functor, we thus get a canonical functor (compatible
with the structures ``under $M$'')
\begin{equation}
  \label{eq:52.2}
  \alpha_R : M_W \to M_R,
  \tag{2}
\end{equation}
where for simplicity $I$ write
\[ M_R = M/R,\quad M_W = W^{-1}M.\]
We may define $W$ as the largest element in $L$ (for the natural order
relation in $L$, namely inclusion of subsets of $\Fl(M)$) such that a
functor \eqref{eq:52.2} exists (compatible with the functors from
$M$ into both sides) -- such a functor of course is unique (by the
preliminaries on functors $M\to M'$ made at the beginning). In terms of
\eqref{eq:52.1}, we can say that $R$ is actually a homotopy relation
(let's call $Q_0(M)=Q_0$ the subset of $Q$ of all such relations)
if{f} \eqref{eq:52.2} is an isomorphism, or equivalently (as it is
clearly bijective on objects, surjective on arrows) if{f} it is
\emph{injective on arrows}, i.e., \emph{faithful}.

Conversely, start now with $W$ in $L$, we get $R=s(W)$,
\[ R = \set[\big]{(f,g)\in\Fl(M)\times\Fl(M)}{i(f)=i(g)},\]
where\pspage{127} now
\[ i : M\to W^{-1}M=M_W\]
is the canonical functor defined in terms of $W$; we thus get a
canonical functor of categories ``under $M$''
\begin{equation}
  \label{eq:52.3}
  \beta=\beta_W : M_R \to M_W,
  \tag{3}
\end{equation}
as a matter of fact, $R$ is the largest element in $Q$ (for the
inclusion relation of subsets of $\Fl(M)\times\Fl(M)$) for which a
functor \eqref{eq:52.3} exists (then necessarily unique, as
before). We may say that $W$ is a homotopism structure on $M$, i.e.,
$W\in L_0$ (where $L_0$ is the subset of $L$ of all homotopism
structures on $M$) if{f} the functor \eqref{eq:52.3} is an
isomorphisms, or equivalently (as it is clearly bijective on objects,
injective on arrows) if{f} it is \emph{surjective on arrows}.

We may describe $Q_0$ and $L_0$ in a purely set-theoretic way, in
terms of the system $(r,s)$ of maps in \eqref{eq:52.1}, by the formula
(which is just a translation of the definitions of $Q_0$ and $L_0$)
\begin{align*}
  Q_0 &= \set[\big]{q\in Q}{sr(q)=q} \\
  L_0 &= \set[\big]{\ell\in L}{rs(\ell)=\ell} , 
\end{align*}
and we can describe formally the pair of subsets $(Q_0,L_0)$ of $Q,L$
as the largest pair of subsets, such that $r$ and $s$ induce between
$Q_0$ and $L_0$ bijections inverse of each other. In the general
set-theoretic set-up, it is by no means clear, and false in general,
that $r$ maps $Q$ into $L_0$ or $L$ into $Q_0$ (thus both $Q_0$ and
$L_0$ may well be empty, whereas $Q$ and $L$ are not). Thus it is not
clear at all that starting with an arbitrary $R\in Q$, the
corresponding $W=i(R)$ is a homotopism structure, and it is easily
seen that this is \emph{not} so in general, contrarily to what I
hastily stated in yesterday's notes. (Take $M$ which just one object,
therefore defined by a monoid $G$, and $M_R$ corresponds to a quotient
monoid $G'$, we may take $G'=$ unit monoid, thus $W$ is $G$ itself,
and $M_W$ the enveloping group $\overline G$ of $G$ -- the map
$\overline G\to G'$ need not be injective!) In the opposite direction,
starting with an arbitrary $W$ in $L$, the corresponding $R$ need not
be a homotopy relation. (If $M$ is reduced to a point, this amounts to
saying that if we localize a monoid $G$ with respect to a subset
$W\subset G$, by making invertible the elements in $W$, the
corresponding map $G\to \overline G$ need no be surjective!)

What we may say, though, is that if we start with a pair
\[(R,W) \in Q\times L\]
such that the \emph{two} functors \eqref{eq:52.2}, \eqref{eq:52.3}
exist, i.e., such that
\[ W\le r(R)\quad\text{and}\quad R\le s(W),\]
then $(R,W)\in Q_0\times L_0$, i.e., $R$ is a homotopy relation and
$W$ is a homotopism\pspage{128} structure, and the two are
associated. This comes from the fact that both compositions of the
functors \eqref{eq:52.2}, \eqref{eq:52.3} must be the identity
functors (being compatible with the ``under $M$'' structure), hence
$\alpha$ and $\beta$ are isomorphisms, which shows both $R\in Q_0$ and
$W\in L_0$, and that $R$ and $W$ are associated. In terms of the
set-theoretic situation \eqref{eq:52.1}, this may be described by
using the order relations on $Q$ and $L$, and the fact that $r$ and
$s$ are monotone maps, and satisfy moreover
\begin{equation}
  \label{eq:52.star}
  sr(q)\le q, \quad rs(\ell)\le\ell\quad (\text{any $q\in Q, \ell\in L$}),
  \tag{*}
\end{equation}
which implies that the set $C_0$ of pairs $(q,\ell)$ of associated
elements of $Q_0,L_0$ can be described also as
\[ C_0 = \set[\big]{(q,\ell)\in Q\times L}{\ell\le r(q),\; q\le
  s(\ell)}.\]

Thus it doesn't seem evident to get a homotopy structure on $M$, just
starting with an $R\in Q$ or a $W\in L$, without assuming beforehand
that $R$ is a homotopy relation, or $W$ a homotopism structure. The
condition \ref{cond:51.C.12} on page \ref{p:115} may be viewed as a
condition on a pair $(R,W)\in Q\times L$, and it clearly implies
\[ R \le s(W);\]
if we assume moreover $W=r(R)$ we get $R\le s(r(R))$ and hence, in
view of the first inequality \eqref{eq:52.star} above,
\[ R = sr(R), \quad\text{i.e.,}\quad R\in Q_0,\]
i.e., $R$ is a homotopy relation, as asserted somewhat quickly
yesterday.

The only standard way for getting homotopy structures in a general
category $M$ which I can see by now, is in terms of an arbitrary set
$\Sigma_h$ of intervals in $M$ (assuming only that $M$ admits finite
products). As soon as the focus gets upon intervals for describing
homotopy structures, the situation becomes typically non-autodual --
in contrast to Quillen's autodual treatment of homotopy relations (via
``left'' and ``right'' homotopies, involving respectively ``cylinder''
and ``path'' objects). This is in keeping with the highly non-autodual
axiom on universal disjoint sums in $M$, which we finally introduced
by the end of yesterday's reflection.

To come back however upon the relationships between the four basic
``homotopy notions'' introduced in yesterday's notes, I would now
rather symbolize these relations in the following diagram of maps
between sets\pspage{129}
\[\begin{tikzpicture}[commutative diagrams/every diagram]
  \matrix[matrix of math nodes, name=m, commutative diagrams/every cell] {
    Q(M) & L(M) \\
    Q_0=\Hom_1(M) & \Hom_2(M)=L_0 \\
    \Hom_3(M) & \Hom_4(M) \\};
  \path[commutative diagrams/.cd, every arrow, every label]
  (m-1-1) edge[bend left=10] node {$r$} (m-1-2)
  (m-1-2) edge[bend left=10] node {$s$} (m-1-1)
  (m-2-1) edge[commutative diagrams/hook] (m-1-1)
  (m-2-1) edge[<->] node {$\sim$} (m-2-2)
  (m-2-2) edge[commutative diagrams/hook] (m-1-2)
  (m-3-1) edge[commutative diagrams/hook] (m-2-1)
  (m-3-2) edge[commutative diagrams/hook] (m-2-2)
  (m-3-2) edge[commutative diagrams/hook] (m-3-1);
  \path[commutative diagrams/.cd, every arrow, every label]
  (m-1-1) edge[controls=+(190:15mm) and +(150:10mm)] (m-3-1.north west);
%    (m-1-1.west) edge[<-,controls=+(180:1cm) and +(180:8mm)] (m-3-1.west)
  \node (P1) [right=6pt,text width=4cm,align=left] at (m-2-2.east) {("homotopy
    structures" on $M$, defined in terms of homotopy relations, or
    homotopism structures)};
  \draw[decorate,decoration={brace,mirror,amplitude=6pt},yshift=0pt]
    (P1.north west) -- (P1.south west);
  \node (P2) [below,text width=3cm,align=left,xshift=3mm] at (m-3-1.south) {(weak
    homotopy interval structures)};
  \node (P3) [below,text width=3cm,align=left,xshift=5mm] at (m-3-2.south) {(contractibility
    structures)};
  \node[right,yshift=-3mm] at (P3.east) {,};
\end{tikzpicture}\]
which\footnote{D\'eployer le diagramme, trop tass\'e dans les deux
  dimensions. \scrcommentinline{I think I've fixed that; I also reversed the brace for
  clarity.}} in terms of the preceding reflections, and yesterday's, is
self-explanatory. The vertical arrow from $Q(M)$ to $\Hom_3(M)$ is the
canonical retraction -- in terms of the latter, and its composition
with $s:L(M)\to Q(M)$, there \emph{are} ways after all to associate to
any $R$ in $Q(M)$ or $W$ in $L(M)$ a homotopy structure on $M$,
provided only $M$ admits finite products, by using weak homotopy
intervals. If $h$ is the homotopy structure thus defined, we get a
priori a functor
\[ M_h \to M_R=M/R\quad\text{resp.}\quad M_h\to M_W=W^{-1}M,\]
provided we assume $R$ resp.\ $W$ satisfy the ``accessory
assumption'', namely that the corresponding functor $M\to M'$ (where
$M'$ is either $M_R$ or $M_W$) commute with finite products.

The main fact to remember from this whole discussion, it seems to me,
is that there are not really \emph{four}, but only \emph{three}
essentially distinct types of structure (among yesterday's) we may
consider upon $M$ as ``homotopy-flavored'' structures, namely
\[\parbox[c]{0.345\textwidth}{\raggedright
  homotopy structures $\Hom_1(M)\leftrightarrow\Hom_2(M)$}
\Leftarrow
\parbox[c]{0.26\textwidth}{\raggedright
  weak homotopy interval structures $\Hom_3(M)$}
\Leftarrow
\parbox[c]{0.30\textwidth}{\raggedright
  contractibility structures $\Hom_4(M)$.}
\]
It would seem at present that the homotopy structures that naturally
come up in our present ``modelizer story'' are all of the strictest
type, and even describable in terms of just one \emph{generating
  contractible weak homotopy interval} (I would like to drop the
qualification ``weak'', definitely when a contractibility assumption
comes in!), and even a \emph{generating contractor}, with commutative
idempotent composition law.\pspage{130}

% 53
\hangsection[Compatibility of a functor $u: M\to N$ with a
homotopy \dots]{Compatibility of a functor \texorpdfstring{$u: M\to N$}{u:
    M->N} with a homotopy structure on
  \texorpdfstring{$M$}{M}.}\label{sec:53}%
Before pursuing yesterday's reflection about the $\pi_0$-functor and
its relation to homotopy structures on $M$, it seems more convenient
to interpolate some more or less obvious ``functorialities'' on the
homotopy notions just developed. They all seem to turn around the
relationship of such notions in $M$ with a more or less arbitrary
functor
\[ u : M\to N,\]
for the time being I am not making any special assumption on $u$. In
terms of the four ways we got for describing a homotopy structure in
$M$, we get four corresponding natural conditions of ``compatibility''
of $u$ with a given homotopy structure $h$, namely:
\begin{description}
\item[\namedlabel{it:53.i}{(i)}]
  If $f \hsim g$ in $M$, i.e. $(f,g)\in R_h$, then $u(f)=u(g)$.
\item[\namedlabel{it:53.iprime}{(i')}]
  If $\bI=(I,\delta_0,\delta_1)\in\Sigma_h$ is a weak homotopy
  interval, then $u(\delta_0)=u(\delta_1)$.
\item[\namedlabel{it:53.ii}{(ii)}]
  If $f\in W_h$, i.e., $f$ is a homotopism, then $u(f)$ is an isomorphism.
\item[\namedlabel{it:53.iiprime}{(ii')}]
  If $X\in C_h$ is a contractible object, then $u(X)$ is a final
  object (NB\enspace here we assume that $u(e_M)$ is a final object in $N$).
\end{description}
(NB\enspace It is understood implicitly, whenever dealing with intervals and
with contractible objects, that $M$ admits finite products.)

The conditions \ref{it:53.i} and \ref{it:53.ii} are clearly
equivalent, and equivalent to the requirement that $u$ factors into
\begin{equation}
  \label{eq:53.1}
  M \to M_h \to N,\tag{1}
\end{equation}
where $M\to M_h$ is the canonical functor of $M$ into the
corresponding \emph{homotopy-types} category. We also have the
tautological implications \ref{it:53.i} $\Rightarrow$
\ref{it:53.iprime} and \ref{it:53.ii} $\Rightarrow$
\ref{it:53.iiprime}. Moreover we have (trivially) \ref{it:53.iiprime}
$\Rightarrow$ \ref{it:53.iprime} whenever the homotopy structure on
$M$ is a weak homotopy interval structure, and \emph{moreover} the
functor $u$ commutes to finite products. All these implications are
summarized in the diagram
\[\begin{tikzcd}[baseline=(O.base),math mode=false,%
  arrows=Rightarrow]
  \ref{it:53.i} \ar[r,Leftrightarrow]\ar[d] &
  \ref{it:53.ii} \ar[d] \\
  \ref{it:53.iprime} \ar[u,bend left,"*\;\;" pos=0.45, "$\Hom_3$"] &
  |[alias=O]| \ref{it:53.iiprime} \ar[l,swap,"$\Hom_4$"]
\end{tikzcd},\]
where the symbol $\Hom_3$ or $\Hom_4$ indicates that the implication
qualified by it is valid provided we assume that the homotopy
structure on $M$ is in $\Hom_3$ (namely is defined in terms of weak
homotopy intervals) resp.\ in $\Hom_4$ (namely is a contractibility
structure), and where (*) denotes the extra assumption of commutation
of $u$ with finite products.

We'll\pspage{131} say the functor $u$ is \emph{compatible with the
  homotopy structure} $h$ on $M$, if it satisfied the equivalent
conditions \ref{it:53.i}, \ref{it:53.ii}, i.e., if it factors as in
\eqref{eq:53.1} above. In case $u$ commutes with finite products, and
if either the homotopy structure $h$ can be described in terms of weak
homotopy intervals, or in terms of contractible objects, the
compatibility of $u$ with $h$ can be checked correspondingly, either
by \ref{it:53.iprime}, or by \ref{it:53.iiprime}.

\clearpage

% 54
\hangsection[Compatibility of a homotopy structure with a set $W$ of
\dots]{Compatibility of a homotopy structure with a set
  \texorpdfstring{$W$}{W} of ``weak equivalences''. The homotopy
  structure \texorpdfstring{$h_W$}{h-W}.}\label{sec:54}%
An important particular case is the one when
\[N = W^{-1}M\]
is a localization on $M$ by a set of arrows in $M$
\[W\subset \Fl(M).\]
We'll say that the homotopy structure $h$ on $M$ is \emph{compatible
  with} $W$, if it is with the canonical functor $M\to W^{-1}M$. If
$W$ is \emph{strongly saturated}, this is most readily expressed by
the condition that $W_h\subset W$, i.e., any homotopism is in $W$
(i.e., ``any homotopism is a weak equivalence'', if the elements of
$W$ are named ``weak equivalences''); in case $M$ admits finite
products and the localization functor commutes to these (e.g., the
case $(M,W)$ is a \emph{strict} modelizer), and if moreover $h$ is a
contractibility structure, it is sufficient to check that for any
contractible $X$, the projection $X\to e$ is in $W$.

If we don't assume or know beforehand that $W$ is strongly saturated,
but just saturated say, we may still introduce a more stringent
compatibility condition, by saying that the homotopy structure $h$ and
$W$ are \emph{strictly compatible} if $W_h\subset W$. Using the
saturation condition \ref{it:48.cprime} on $W$, it is easily seen that
in the case when $h$ is a contractibility structure, then $W_h\subset
W$ (strict compatibility) is equivalent to: for contractible $X$, the
projection $X\to e$ is ``universally in $W$'', or (as we'll say)
\emph{$W$-aspheric}. Indeed, to deduce from this that any homotopism
$f:X\to Y$ is in $W$, we are reduced to checking that any endomorphism
of either $X$ or $Y$ which is homotopic to the identity map, is in
$W$. Now this will follow from the assumption, and the following
\begin{proposition}\label{prop:54}
  Assume the homotopy structure $h$ on $M$ can be defined by a
  generating set $\Sigma_h^0$ of weak homotopy intervals
  $\bI=(I,\delta_0,\delta_1)$ which are \emph{$W$-aspheric} \textup(i.e., $I$
  $W$-aspheric over $e$\textup), where $W\subset\Fl(M)$ is any saturated
  subset \textup(in fact, mildly saturated is enough\textup). Then $W$ is the
  inverse image by the canonical functor $M\to M_h=\overline M$ of a
  subset $\overline W\subset\Fl(\overline M)$, i.e., if $f,g$ are
  homotopic arrows in $M$, if one is in $W$, so is the other. Moreover
  \textup(if $W$ is saturated\textup) $W_h\subset W$, i.e., $h$ and
  $W$ are strictly compatible.
\end{proposition}

The\pspage{132} first statement is just the
``\hyperref[lem:hlr]{homotopy lemma}'' part \ref{it:48.hlr.b} (page
\ref{p:99}), the second follows by the argument sketched above.

We're about back now to the context we started with three days ago
(par.\ \ref{sec:48}, page \ref{p:98} and following), where we started
with a $W$ (viewed as a notion of ``weak equivalence''), and in terms
of $W$ constructed various homotopy notions -- namely those, we would
now say, corresponding to the homotopy structure defined by the set of
all intervals \bI{} in $M$ which are $W$-aspheric (i.e., $I$ is
$W$-aspheric over $e$\footnote{\alsoondate{11.4.} By which we mean that $I\to e$ is
  ``universally in $W$''. The terminology used here for
  ``$W$-aspheric'' is highly ambiguous, cf.\ discussion p.\
  \ref{p:181} and following.}). As a matter of fact, we were a little
stricter still, by restricting to intervals which are moreover
``disjoint'' (and which we called ``homotopy intervals'' relative to
$W$), but this restriction now definitely appears as awkward and
artificial. I will henceforth call \emph{homotopy intervals} (with
respect to $W$), any interval (not necessarily a separated one) which
is $W$-aspheric. Let $h_W$ be the corresponding homotopy structure on
$M$, which is a weak homotopy interval structure admitting the set of
all $W$-homotopy intervals as a generating set of weak homotopy
intervals. (Clearly, there will be many weak homotopy intervals for
this structure, which are far from being $W$-aspheric, i.e., far from
being homotopy intervals.) Of course, as stated in the preceding
proposition, $h_W$ and $W$ are strictly compatible, i.e.,
\[W_{h_W} \subset W,\]
i.e., any $h_W$-homotopism is in $W$ (i.e., is a ``weak
equivalence''). As a matter of fact, the definition of homotopy
notions in terms of $W$ we gave in loc.\ sit.\ were just the widest
one we could think of by that time, which would ensure the
``compatibility'' of these notions with $W$, in a sense which wasn't
technically clear (not even definable at that point) as it is now, but
however reasonably clear in terms of mathematical ``bon sens''.
At present though the question arises rather naturally whether the
homotopy structure $h_W$ we selected ``au flair'' by that time is
indeed the best one, namely the widest one, we could get. More
explicitly, this means whether the homotopy structure $h_W$ is the
\emph{widest} (in terms of the natural order relation considered in
the previous paragraph) among all those which are compatible with $W$
in the strict sense $W_{h_W}\subset W$. Now this is certainly not so,
if we are not a little more specific about restricting to homotopy
structure definable in terms of a weak homotopy interval
structure. For instance, if we take for $W$ a homotopism structure on
$M$, compatible with products, corresponding to a homotopy structure
$h$, to say $h_W$ is the ``best'' would imply that $W$ itself can be
described in terms of weak homotopy intervals\pspage{133} which is not
always the case. (Take for instance $M$ to be an abelian category, say
projective complexes of modules and quasi-isomorphism between these;
in this case, more generally whenever $M$ is a ``zero objects'' namely
one which is both initial and final, any interval in $M$ is trivial,
i.e., $\delta_0=\delta_1$, and hence any weak homotopy interval
structure on $M$ is trivial, namely $W_h$ is reduced to
isomorphisms\ldots)

Thus the more reasonable question here is whether any homotopy
structure $h$ on $M$, definable in terms of a weak homotopy interval
structure, and such that $W_h\subset W$, satisfies $h\le
h_W$. Clearly, for such an $h$, any weak homotopy interval \bI{} (for
$h$) satisfies $u(\delta_0)=u(\delta_1)$, where $u:M\to W^{-1}M$ is
the canonical functor (indeed, it is enough for this that
$W_h\subset\overline W$ instead of $W_h\subset W$, where $\overline W$
is the strong saturation of $W$), and conversely, if $W=\overline W$
and if moreover $u$ \emph{commutes to finite products}. On the other
hand, $h\le h_W$ means that any $\bI\in\Sigma_h$ is in $\Sigma_{h_W}$,
which also means that its endpoint sections $\delta_0,\delta_1$ are
$h_W$-homotopic, namely may be joined by a chain of sections, any two
consecutive of which are related by some \bJ-homotopy, where \bJ{} is
a \emph{$W$-aspheric interval}. Thus we get the:
\begin{proposition}
  Let $W$ a saturated set of arrows in $M$ \textup($M$ stable under
  finite products\textup), hence a corresponding homotopy structure
  $h_W$ on $M$, defined in terms of $W$-aspheric
  intervals\footnote{\alsoondate{11.4.} Cf.\ note on preceding page.\label{fn:54.star}} in $M$ as a
  generating set of weak homotopy intervals for $h_W$. Consider the
  following conditions:
  \begin{enumerate}[label=(\roman*),font=\normalfont]
  \item\label{it:54.i}
    $h_W$ is the widest of all homotopy structures $h$ on $M$ which
    are
    \begin{enumerate}[label=(\alph*),font=\normalfont]
    \item\label{it:54.i.a} strictly compatible with $W$, i.e., such
      that $W_h\subset W$ and moreover
    \item\label{it:54.i.b}
      definable in terms of a weak homotopy interval structure.
    \end{enumerate}
  \item\label{it:54.ii}
    For any object $I$ of $M$ and two sections $\delta_0,\delta_1$ of
    $I$ such that $u(\delta_0)=u(\delta_1)$ \textup(where $u:M\to
    W^{-1}M$ is the canonical functor\textup), $\delta_0$ and
    $\delta_1$ are $h_W$-homotopic, namely can be joined by a chain of
    elementary homotopies as above, involving
    \emph{$W$-aspheric}\footref{fn:54.star} intervals \bJ.
  \end{enumerate}
  Then \textup{\ref{it:54.ii}} implies \textup{\ref{it:54.i}}, and conversely if $W$ is
  strongly saturated and moreover $u$ commutes to finite products.
\end{proposition}

In connection with the $\pi_0$-functor, we are going to get pretty
natural conditions in terms of $0$-connectedness for ensuring
\ref{it:54.ii} and hence \ref{it:54.i}, which should apply I guess to
all ``reasonable'' modelizers $(M,W)$. It would thus seem that in
practical terms, the definition of $h_W$ is the best, in all cases of
actual interest to us. Of course, in case $W$ is strongly saturated
and $u$ commutes with finite products, the widest $h$ is the one whose
weak homotopy intervals are triples $(I,\delta_0,\delta_1)$ satisfying
$u(\delta_0)=u(\delta_1)$, which we could have used instead of just
$W$-aspheric intervals, which are\pspage{134} also the intervals such
that $I\to e$ is in $W$ (and hence universally so in terms of the
assumption made of compatibility of $u$ with products).  The trouble
with working with this $h$, rather than with $h_W$ as above, is
twofold though: a)\enspace The assumptions of strong saturation on $W$
and compatibility with products are not so readily verified in the
cases of interest to us, and the second moreover is not always
satisfied, e.g., there are test categories which are not strict, i.e.,
elementary modelizers which are not strict; b)\enspace the condition
$u(\delta_0)=u(\delta_1)$ is not readily verified in terms of $W$
directly, whereas the condition of $W$-asphericity is -- still more so
if $W$ is compatible with products and hence $I\to e$ is $W$-aspheric
just means it is in $W$.

% 55
\hangsection{Maps between homotopy structures.}\label{sec:55}%
Let's now look at ``\emph{morphisms}'' between categories endowed with
homotopy structures, $(M_1,h_1)$ and $(M_2,h_2)$ say. The natural
definition here is to take as morphisms between these homotopy
structures the functors $u:M_1\to M_2$ that give rise to a commutative
square of functors
\[\begin{tikzcd}[baseline=(O.base)]
  M_1 \ar[r,"u"]\ar[d] & M_2\ar[d] \\
  \overline M_1 \ar[r,"\overline u"] & |[alias=O]| \overline M_2
\end{tikzcd},\]
where the vertical arrows are the canonical functors into the
respective homotopy-types categories, and $\overline u$ a suitable
functor, necessarily unique.
The existence of $\overline u$ can be expressed at will in terms of
the $\Hom_1$ or $\Hom_2$ structures, namely as
\begin{enumerate}[label=(\roman*)]
\item\label{it:55.i}
  $f \underset{h_1}\sim g$ implies $u(f) \underset{h_2}\sim u(g)$,
\end{enumerate}
or as
\begin{enumerate}[label=(\roman*),resume]
\item\label{it:55.ii}
  $f\in W_{h_1}$ implies $u(f)\in W_{h_2}$.
\end{enumerate}
These conditions, when $M_1$ and $M_2$ have final objects and these
are respected by $u$, imply that $u$ transforms weak homotopy
intervals into weak homotopy intervals, and homotopisms into
homotopisms. Conversely, \emph{if $u$ commutes with finite products},
and if $h_1$ can be defined by a weak homotopy interval structure
(respectively, by a contractibility structure), then for $u$ to be a
morphism of homotopy structures, it is (necessary and) sufficient that
$u$ carry weak homotopy intervals (resp.\ contractible objects) into
same.

In case $h_1$ is described in terms of a generating set
$\Sigma_{h_1}^0$ of weak homotopy intervals, and if \emph{$u$ commutes
  with finite products}, the most economic way often to express that
$u$ is a morphism of homotopy structures, is by the condition that for
any \bI{} in $\Sigma_{h_1}^0$, $u(\bI)$ be a weak homotopy\pspage{135}
interval in $M_2$, namely $u(\delta_0) \underset{h_2}\sim
u(\delta_1)$. If we assume moreover that the intervals \bI{} in
$\Sigma_{h_1}^0$ are contractible, the previous condition is
equivalent to $u(I)$ being a contractible object of $M_2$ for any
\bI{} in $\Sigma_{h_1}^0$. The case I am mainly thinking of, of
course, is the one when $h_1$ can be described by a single generating
weak homotopy interval, possibly even contractible, or even by a
(generating) contractor. In the latter case, because of commutation of
$u$ with finite products, $u(\bI)$ will be equally a contractor -- and
contractible for $h_2$ if{f} $u$ is a morphism of homotopy structures.

In the precedent paragraphs, I forgot to mention the reduction of the
corresponding compatibility conditions (of a functor $u:M\to N$, or of
a saturated $W\subset\Fl(M)$) with a homotopy structure $h$, when the
latter is defined in terms of just a generating set $\Sigma_h^0$ of
weak homotopy intervals, possibly reduced to a single one, and
moreover \emph{$u$ or $W$ is ``compatible with finite products''}. In
the first case, it is enough to check $u(\delta_0)=u(\delta_1)$ for
any \bI{} in $\Sigma_h^0$ -- and if \bI{} is contractible, this
amounts to demanding $u(\bI)$ is a final object of $N$. In the second
case it is enough for strict compatibility, i.e., $W_h\subset W$, to
check that for any \bI{} in $\Sigma_h^0$, $I\to e$ is in $W$ (this
condition is also necessary, if any \bI{} in $\Sigma_h^0$ is
contractible). Even if $W$ is not supposed compatible with finite
products, namely $M\to W^{-1}M$ does not commute with finite products,
it is still sufficient for $W_h\subset W$ that all \bI's in
$\Sigma_h^0$ be \emph{$W$-aspheric} (as stated in
\hyperref[prop:54]{prop}.\ (p.\ \ref{p:131})), and this is necessary
too, if the \bI's are contractible.

\bigbreak

\noindent\hfill\ondate{4.4.}\par

% 56
\hangsection[Another glimpse upon canonical modelizers. Provisional
\dots]{Another glimpse upon canonical modelizers. Provisional working
  plan -- and recollection of some questions.}\label{sec:56}%
This seemingly endless review of generalities on homotopy notions is
getting a little fastidious - and still I am not quite through yet I
feel. One main motivation for embarking on this review was one strong
impression which grew out of the reflections of now just one week ago
(paragraph \ref{sec:48}), namely that the interesting ``test
functors'' from a test category $A$ into a modelizer $(M,W)$ are those
which factor through the full subcategory $M\subc$ of
\emph{contractible objects} of $M$. The presumable extra condition to
put on a functor $A\to M\subc$ to correspond to an actual test functor
$i$ from $A$ to $M$ are strikingly weak, such as $i^*(\bI)$ should be
aspheric over $e_\Ahat$ under the assumption we got a contractible
generating homotopy interval $\bI$ in $M$. In any case, if $M\subc^0$
is any full subcategory of $M\subc$ which gives rise (by taking
intervals in $M\subc^0$) to a family of homotopy intervals which
\emph{generates} the homotopy structure $h_W$ on $M$ associated to
$W$, it should\pspage{136} suffice, if $M$ is a ``canonical
modelizer'', that the asphericity condition on $i^*(\bI)$ should be
verified for any \bI{} in $M\subc^0$; this will presumably turn out in
due course as part of the definition (still ahead) of a ``canonical''
modelizer. Now, the most evident way to meet this condition, is to
take for $i$ a \emph{fully faithful} functor \emph{whose image
  contains} $M\subc^0$, or what amounts essentially to the same, any
full subcategory of $M$ containing $M\subc^0$! As we would like though
$A$ to be ``small'', this will be feasible only if $M\subc^0$ is small
-- hence the significance of the condition of a small generating
family of weak homotopy intervals for $h_W$ (which will imply that we
can find such a family with \emph{contractible} intervals, provided
only the homotopy structure $h_W$ can be described by a
contractibility structure, as we did indeed assume). Just as the
homotopy structure $h_W$ of $M$ was defined in terms of $W$,
conversely the notion of weak equivalences $W$ should be recoverable
in terms of the homotopy structure, and more specifically in terms of
the subcategory $M\subc$ and the small ``generating'' subcategories
$M\subc^0$ of $M\subc$, which we may now as well denote by $A$, by
taking the inclusion functor $i:A\to M$ of such an $A$, hence a
functor
\[i^*:M\to\Ahat,\]
and taking
\[W=(i^*)^{-1}(W_\Ahat).\]
Of course, we'll have still to check under which general conditions
upon a pair $(M,W)$ of a category and a saturated set of arrows $W$,
or rather, upon a pair $(M,M\subc)$ of a category endowed with a
contractibility structure $M\subc=C_h$ (where we think of $h$ as an
$h_W$), is it true that the saturated set of arrows
\[W(A) = (i^*)^{-1}(W_\Ahat)\subset\Fl(M)\]
in $M$ does not depend upon the choice of the full small
homotopy-generating subcategory $A$ of $M\subc$ (if restrictive
conditions are needed indeed). It may be reasonable to play safe, to
restrict at first to subcategories $A$ which are stable under finite
products in $M$, which will ensure that $A$ is a \emph{strict} test
category, i.e., \Ahat{} is a \emph{strict} elementary modelizer,
namely \Ahat{} is totally aspheric. But such restriction -- as well as
to test functors which are fully faithful -- should be a provisional
one, as ultimately we want of course to be able to use test categories
such as $\Simplex$ for ``testing'' rather general (canonical)
modelizers, whereas $\Simplex$ is by no means stable under products, nor
embeddable \emph{faithfully} in modelizers such as \Spaces{} say.

This\pspage{137} expectation of $W$ to be recoverable in terms of the
corresponding homotopy structure $h=h_W$ on $M$ takes its full
meaning, when joined with another one, namely that the latter can be
canonically described in terms of the category structure of $M$ and
the corresponding notion of $0$-connectedness. This latter expectation
is extremely strongly grounded, and I'll come back to it
circumstantially very soon I think (I started on it two days ago, but
then it got too late to take it to the end, and yesterday was spent on
some formal digressions\ldots). The two ``expectations'' put together,
when realized by carefully cutting out the suitable notions, should
imply that the structure of any ``canonical modelizer'' is indeed
determined ``canonically'' in terms of its category structure alone.

To come back to the relationship between test categories and
categories of the type $M\subc^0$, the idea which has been lurking
lately is that possibly, test categories can be viewed as no more, no
less, as categories endowed with a homotopy structure (necessarily
unique) which is a contractibility structure, and for which \emph{all
  objects are contractible}. At any rate, there must be a very close
relationship between the two notions, which I surely want to
understand. But as it would be quite unreasonable to restrict the
notion of a test category (and its weak and strong variants) to
categories admitting finite products, this shows that for a
satisfactory understanding of the above relationship, we should be
able to work with contractibility structures, and presumably too with
weak homotopy interval structures, in categories $A$ where we do not
make the assumption of stability under products. Thus in the outline
of the last few days, I still wasn't general enough it would seem!
This situation reminds me rather strongly of the early stages when
developing the language of sites, and restricting to sites where
fiber-products exist -- this seems by then a very weak and natural
assumption indeed, before it appeared (first to Giraud, I believe)
that it was quite an awkward and artificial restriction indeed, which
had to be overcome in order to work really at ease\ldots

All this now gives a lot of interesting things to look up in the short
run! I'll make a provisional plan of work, as follows:
\begin{enumerate}[label=\alph*)]
\item\label{it:56.a}
  Relation between a homotopy structure and the $\pi_0$ functor, and
  description of the so-called canonical homotopy structures.
\item\label{it:56.b} Write down in the end the ``key result'' on test
  functors $A\to\Cat$ which is overripe since the reflections of four
  days ago (par.\ \ref{sec:47}).  Presumably,\pspage{138} this will
  yield at the same time an axiomatic characterization of $W_\Cat$,
  namely of the notion of weak equivalence for functors between
  categories.
\item\label{it:56.c}
  At this point, we could go on and try and carry through the similar
  characterization for test functors $A\to\Bhat$, where $A$ and $B$
  are both test categories. There are also some generalities to
  develop about ``morphisms'' between test categories, which is ripe
  too for quite a while and cannot be pushed off indefinitely -- here
  would be the right moment surely. If the expected ``key result'' for
  test functors $A\to\Bhat$ carries through nicely it could presumably
  be applied at once in order to study general test functors $A\to M$,
  and thus get the clues for cutting out ``the'' natural notion of a
  canonical modelizer, which ``was in the air'' since the ``naive
  question'' of par.\ \ref{sec:46} (page \ref{p:95}).
\item\label{it:56.d}
  However, there is another approach to canonical modelizers which is
  just appearing, via the idea (described above) of associating
  canonically a notion of ``weak equivalence'' $W$ to a homotopy
  structure of type $\Hom_4$, i.e., to a contractibility structure,
  subject possibly to some restrictions. This ties in, as explained
  above, with a closer look at the relationship between test
  categories, and ``coarse'' contractibility structures (where all
  objects are contractible).
\end{enumerate}

It would seem unreasonable to push off \ref{it:56.a} and \ref{it:56.b}
any longer now -- so I'll begin with these. I am hesitant however
between \ref{it:56.c} and \ref{it:56.d} -- with a feeling that the
later approach \ref{it:56.d} may well turn out to be technically the
most expedient one. Both have to be carried through anyhow, and the
two together should give a rather accurate picture of what canonical
modelizers are about.

On the other hand, there are still quite a bunch of questions which
have been waiting for investigation -- for instance the list of
questions of nearly three weeks ago (page \ref{p:42}). Among the six
questions stated there, four have been settled, or are about to be
settled through the previous program (if it works out), questions
\ref{it:32.4} and \ref{it:32.6} remain, the first one being about
\Ahat{} being a closed model category, and about the homotopy
structure of \Cat. There are a number of more technical questions too,
for instance I did not finish yet my review of the ``standard'' test
categories and never wrote down the proof that \Simplexf{} (simplices
with face operations and no degeneracies) is indeed a weak test
category. But for the time being, all these questions appear as
somewhat marginal with respect to the strong focus the reflection has
been gradually taking nearly since the very\pspage{139} beginning --
namely an investigation of modelizers and, more specifically, the
gradual unraveling of a notion of ``canonical modelizer''. I certainly
feel like carrying this to the end at once, without any digressions
except when felt relevant for the main focus at present. As for
choosing precedence between \ref{it:56.c} and \ref{it:56.d}, it is
still time to decide, when we're through with \ref{it:56.a} and \ref{it:56.b}!

% 57
\hangsection[Relation of homotopy structures to $0$-connectedness and
\dots]{Relation of homotopy structures to
  \texorpdfstring{$0$}{0}-connectedness and
  \texorpdfstring{$\pi_0$}{pi-0}. The canonical homotopy structure
  \texorpdfstring{$h_M$}{hM} of a category
  \texorpdfstring{$M$}{M}.}\label{sec:57}%
\textbf{Relation of homotopy structures to $0$-connectedness and to
  \piz.}\enspace Here we're resuming the reflection started in par.\
\ref{sec:51} \ref{subsec:51.F} (page \ref{p:122}). All we did there
was to introduce some conditions on a category $M$, namely
\ref{it:51.F.a} to \ref{it:51.F.e} (page \ref{p:123}), and introduce
the functor
\[\pi_0 : M \to \Sets.\]
As before, we'll assume now $M$ satisfies conditions \ref{it:51.F.a}
to \ref{it:51.F.d} (the first three, or rather \ref{it:51.F.b} and
\ref{it:51.F.c}, are all that is needed for defining the functor
$\pi_0$), and will not assume \ref{it:51.F.e}, or the equivalent
\ref{it:51.F.eprime} of $\pi_0$ commuting to finite products, unless
explicitly specified.

We suppose now, moreover, $M$ endowed with a homotopy structure
$h$. We'll say that $h$ is \emph{$\pi_0$-admissible}, or simply
\emph{$0$-admissible}, if $h$ is compatible with the functor $\pi_0$,
which can be expressed by either one of the following two equivalent
conditions (cf.\ page \ref{p:130}):
\begin{enumerate}[label=(\roman*)]
\item\label{it:57.i}
  $f\hsim g$ implies $\pi_0(f)=\pi_0(g)$,
\end{enumerate}
for any two maps $f,g$ in $M$, or
\begin{enumerate}[label=(\roman*),resume]
\item\label{it:57.ii}
  $f\in W_h$ (i.e., $f$ a homotopism) implies $\pi_0(f)$ bijective.
\end{enumerate}

Another equivalent formulation is that the functor $\pi_0$ factors
through the quotient category $M_h$ of homotopy types
\[M\to M_h\to\Sets,\]
we'll still denote by $\pi_0$ the functor $M_h\to\Sets$ obtained.

If $h$ is $0$-admissible, it satisfies \ref{it:57.iprime} and
\ref{it:57.iiprime} below:
\begin{enumerate}[label=(\roman*')]
\item\label{it:57.iprime}
  For any weak homotopy interval $(I,\delta_0,\delta_1)\in\Sigma_h$, $\pi_0(\delta_0)=\pi_0(\delta_1)$.
\end{enumerate}
As $\pi_0(e_M)$ is a one-point set by the assumption \ref{it:51.F.d}
on $M$, for any section $\delta$ of an object $I$, $\pi_0(\delta)$ may
be described as just an element of $\pi_0(I)$, which is the unique
connected component of $I$ through which factors the given section;
thus \ref{it:57.iprime} can be expressed by saying that for any weak
homotopy interval, the two ``endpoints'' belong to the same connected
component of $I$, a natural condition indeed! It is automatically
satisfied\pspage{140} if $I$ is connected. In fact, \ref{it:57.iprime}
is satisfied if{f} $\Sigma_h$ admits a generating subset $\Sigma_h^0$
made up with \emph{connected} intervals.
\begin{enumerate}[label=(\roman*'),resume]
\item\label{it:57.iiprime}
  Any contractible object $X$ is connected (and hence $0$-connected).
\end{enumerate}

Conversely (cf.\ page \ref{p:130}), if condition \ref{it:51.F.e}
holds, i.e., $\pi_0$ commutes to products, and if moreover $h$ is
definable in terms of $\Sigma_h$, i.e., comes from a weak homotopy
interval structure, then \ref{it:57.iprime} implies
$0$-admissibility. If $M$ admits even a generating family of
contractible weak homotopy intervals, namely if $h$ comes from a
contractibility structure on $M$, then \ref{it:57.iiprime} equally
implies $0$-admissibility.
\begin{remarks}
  1)\enspace We can generalize these converse statements, by dropping
  the condition \ref{it:51.F.e} on $M$, but demanding instead that the
  connected component involved (namely $I$ itself in case
  \ref{it:57.iiprime}) is not only $0$-connected, but even
  ``\emph{$0$-connected over $e_M$}'', which just means that its
  products by any $0$-connected object of $M$ is again
  $0$-connected. In the case \ref{it:57.iiprime}, this condition es
  equally \emph{necessary} for admissibility. (The corresponding
  statements could have been made in the general context of page
  \ref{p:130} of course\ldots)

  2)\enspace The name of $0$-admissibility suggests there may exist
  correspondingly ``higher'' notions of $n$-admissibility for $h$,
  where $n$ is any natural integer. I \emph{do} see a natural
  candidate, namely whenever we have a functor $\pi_n$ from $M$ to
  $n$-truncated homotopy types (as is the case, say, when $M$ is
  either a topos -- we then rather get \emph{pro}homotopy types -- or
  a modelizer). But it would seem that in all cases of geometrical
  significance, and when moreover $h$ is defined by a weak homotopy
  interval structure, that $0$-admissibility implies already
  $n$-admissibility for any $n$.
\end{remarks}

Due to the existence of $\sup$ for an arbitrary subset, in each of the
ordered sets $\Hom_i(M)$ ($i\in\{0,3,4\}$) of homotopy structures in
$M$ (either unqualified, or weak homotopy interval structures, or
contractibility structures), it follows that for each of these three
types of homotopy structures, there is a widest one $h_i$ among all
those of this type which are $0$-admissible. We are interested here,
because of topological motivations, by the case of $\Hom_3$, namely
weak homotopy interval structures. We call the corresponding homotopy
structure $h=h_3$ the \emph{canonical homotopy structure} of $M$. In
case $M$ is totally $0$-connected (by which we mean condition
\ref{it:51.F.e}), the weak homotopy intervals for this structure are
just those intervals for which $\delta_0,\delta_1$ correspond to the
same connected component of $I$ -- and we get a generating set of weak
homotopy intervals, by just taking \emph{all connected intervals}.
\begin{remark}
  If $M$ is not\pspage{141} totally $0$-connected, we still get a
  description of $\Sigma_h$, as those intervals such that for any
  $0$-connected $X$, the corresponding sections of $X\times I$ over
  $X$ correspond to the same connected component of $X\times I$. It is
  enough for this that the common connected component $I_0$ of $I$ for
  $\delta_0,\delta_1$ should be $0$-connected over $e$ -- it is not
  clear to me whether this condition is equally necessary. Anyhow,
  presumably the case $M$ totally $0$-connected will be enough for all
  we'll have to do.
\end{remark}

In case $M$ is totally $0$-connected, the homotopy notions in $M$ for
the canonical homotopy structure are just those which can be described
in terms of ``homotopies'' using connected intervals -- which is
intuitively the first thing that comes to mind indeed, when trying to
mimic most naively, in an abstract categorical context, the familiar
homotopy notions for topological spaces.

Let let
\[ W\subset\Fl(M)\]
be any saturated set of arrows in $M$ (viewed as a notion of ``weak
equivalence'' in $M$). Consider the corresponding homotopy structure
$h_W$ on $M$, defined in terms of $W$-aspheric intervals as a
generating family of weak homotopy intervals. Let $h_M$ be the
canonical homotopy structure on $M$. Thus the condition
\[ h_W \le h_M\]
just means that $h_W$ is $0$-admissible, or equivalently, that
$W$-aspheric objects over $e$ which have a section, are
$0$-connected (for simplicity, I assume from now on $M$ totally
$0$-connected). This looks like a very reasonable condition indeed, if
$W$ should correspond at all to the intuitions associated to the
notion of ``weak equivalence''! As a matter of fact, this condition is
clearly implied by the condition that $W$ itself should be
``\emph{$0$-admissible}'', by which we mean that the functor $\pi_0$
is compatible with $W$, i.e., transforms weak equivalences into
bijections, or equivalently, factors through $M\to W^{-1}M$.

What we are looking for however is conditions on $W$ for the
\emph{equality}
\[ h_W=h_M\]
to hold. When the previous condition (expressing $h_W\le h_M$) is
satisfied, all that remain is to express the opposite inequality,
which is done in the standard way. We thus get:
\begin{proposition}
  Let\pspage{142} $M$ be a category, assume $M$ totally $0$-connected
  \textup(i.e., satisfying conditions
  \textup{\ref{it:51.F.a}\ref{it:51.F.b}\ref{it:51.F.c}\ref{it:51.F.e}}
  of page \ref{p:123}\textup). Let $W\subset\Fl(M)$ be a saturated set
  of arrows in $M$, consider the associated homotopy structure $h_W$
  on $M$, with $W$-aspheric intervals as a generating family of weak
  homotopy intervals. Consider also the canonical homotopy structure
  $h_M$ on $M$, with $0$-connected intervals as a generating family of
  weak homotopy intervals \textup(thus $h_M=h_{W_0}$, where
  $W_0\subset\Fl(M)$ consists of all arrows made invertible by the
  functor $\pi_0:M\to\Sets$\textup). In order for the equality
  $h_W=h_M$ to hold, it is necessary and sufficient that the following
  two conditions be satisfied:
  \begin{enumerate}[label=\alph*),font=\normalfont]
  \item\label{it:57.a}
    Any object $I$ of $M$ which is $W$-aspheric over $e$ and admits a
    section, is connected \textup(it is enough for this that $W$ by
    $0$-admissible, i.e., $f\in W$ imply $\pi_0(f)$ bijective\textup);
  \item\label{it:57.b}
    For any connected object $I$ of $M$ and any two sections
    $\delta_0,\delta_1$ of $I$, these can be ``joined'' by a finite
    chain of sections $s_i$ \textup($0\le i\le n$\textup),
    $s_0=\delta_0$, $s_n=\delta_1$, such that for any two consecutive
    ones, there exists an object $J$, $W$-aspheric over $e$, a map
    $J\to I$ and two sections of $J$ mapped into the sections
    $s_i,s_{i+1}$ of $I$.
  \end{enumerate}
\end{proposition}

The condition \ref{it:57.a} says there are not too many weak
equivalences, whereas \ref{it:57.b} says there are still enough for
``testing'' connectedness in terms of $W$-aspheric intervals. Both
conditions look plausible enough!

Next step one would think of, in this context, is to give conditions
on $W$ (independently of the previous ones) which will allow to
express $W$ in terms of $h_W$. But for this, I should develop first a
description of a notion of ``weak equivalence'' in terms of an
arbitrary homotopy structure $h$ on $M$, as contemplated in the
previous paragraph. I decided however to give precedence to the ``key
result'' still ahead.

Maybe the condition $h_W=h_M$, expressed in the previous proposition,
merits a name -- we'll say that the notion of weak equivalence $W$ is
``\emph{geometric}'', if it satisfies the two conditions \ref{it:57.a}
and \ref{it:57.b} above, or rather the slightly stronger
\namedlabel{it:57.aprime}{a')} in place of a) -- namely
$0$-admissibility of $W$ (plus \ref{it:57.b} of course). The
conditions \ref{it:57.aprime} and \ref{it:57.b} are the explicit ones
for checking -- but for using that $W$ is geometric, the more
conceptual statement $h_W=h_M$ (besides $0$-admissibility) is the
best. Thus, as any \emph{contractor} in $M$ which is connected (hence
a weak homotopy interval for $h_M$) is $h_M$-contractible, it is
$h_W$-contractible, and hence $W$-aspheric over $e$, and conversely of
course. Using this, we get a converse to part \ref{it:51.E.c}
and\pspage{143} \ref{it:51.E.d} of the proposition of page
\ref{p:121}, in the present context, which we may state in a more
complete form as follows:
\begin{proposition}
  Let $M$ be a totally $0$-connected category, $W$ a geometric
  saturated set of arrows in $M$. Assume moreover that for any two
  objects in $M$, the object $\bHom(X,Y)$ in $M$ exists. Let $X$ be an
  object of $M$, then the following conditions \textup{\ref{cond:57.a}} to
  \textup{\ref{cond:57.cdblprime}} are equivalent:
  \begin{description}
  \item[\namedlabel{cond:57.a}{a)}]
    For any object $Y$, $\bHom(Y,X)$ is $0$-connected.
  \item[\namedlabel{cond:57.aprime}{a')}]
    For any object $Y$, $\bHom(Y,X)$ is $W$-aspheric \textup(i.e., its
    map to $e$ is $\in W$\textup).
  \item[\namedlabel{cond:57.adblprime}{a'')}]
    For any object $Y$, $\bHom(Y,X)$ is $W$-aspheric over $e$
    \textup(i.e., its map to $e$ is ``universally in $W$''\textup).
  \item[\namedlabel{cond:57.atplprime}{a''')}]
    For any object $Y$, $\bHom(Y,X)$ is $h_W$-contractible.
  \item[\namedlabel{cond:57.b}{b)}\namedlabel{cond:57.bprime}{b')}%
    \namedlabel{cond:57.bdblprime}{b'')}\namedlabel{cond:57.btplprime}{b''')}]
    Same as above, with $Y$ replaced by $X$.
  \item[\namedlabel{cond:57.c}{c)}]
    For any object $Y$, $Y\to \bHom(X,Y)$ is in $W$.
  \item[\namedlabel{cond:57.cprime}{c')}]
    For any object $Y$, $Y\to \bHom(X,Y)$ is $W$-aspheric.
  \item[\namedlabel{cond:57.cdblprime}{c'')}]
    For any object $Y$, $Y$ as a subobject of $\bHom(X,Y)$ is a
    deformation retract with respect to $h_W$.
  \end{description}
  If moreover $X$ has a section, these conditions are equivalent to:
  \begin{description}
  \item[\namedlabel{cond:57.d}{d)}]
    $X$ is $h_M$-contractible.
  \item[\namedlabel{cond:57.dprime}{d')}]
    $X$ is $h_W$-contractible.
  \end{description}
\end{proposition}
\begin{example}
  Let $M=\Cat$, the assumptions on $M$ are clearly satisfied. If we
  take for $W$ the usual weak equivalences, it is clear too that $W$
  satisfies \ref{it:57.aprime} and \ref{it:57.b} above, i.e., $W$ is
  geometric. Thus the preceding proposition is just an elaboration of
  the result stated after the prop.\ of page \ref{p:97} -- which was
  the moment when it became clear that contractibility was an
  important notion in the context of test categories and test
  functors, and which was the main motivation too for the somewhat
  lengthy trip through generalities on homotopy notions, which is
  coming now (in the long last!) to a provisional end\ldots
\end{example}

To check if the notions developed in this section are handy indeed, I
would still like to try them out in the case $M$ is an elementary
modelizer \Ahat, corresponding to a test category $A$. But it's
getting late this night\ldots

\bigbreak
\noindent\hfill\ondate{5.4.}\pspage{144}\par

% 58
\hangsection[Case of totally $0$-connected category $M$. The category
\dots]{Case of totally \texorpdfstring{$0$}{0}-connected category
  \texorpdfstring{$M$}{M}. The category
  \texorpdfstring{$\overline\Cat$}{(Cat)} of
  \texorpdfstring{\textup(}{(}small\texorpdfstring{\textup)}{)}
  categories and homotopy classes of functors.}\label{sec:58}%
Some comments still on the canonical homotopy structure of a category
$M$, which we assume again totally $0$-connected. Two sections of an
object $K$ are homotopic if{f} they belong to the same connected
component of $X$, thus we get an \emph{injective} canonical map from
homotopy-classes of sections
\[\overline\Gamma(X)=\oHom(e,X) \to \pi_0(X).\]
We are interested in the case when this map is always a bijection, or
what amounts to the same, when $M$ satisfies the extra condition:
\begin{enumerate}[label=\alph*),start=6]
\item\label{it:58.f}
  Every $0$-connected objects of $M$ has a section.
\end{enumerate}
\begin{proposition}
  Let $M$ be a category satisfying conditions \textup{\ref{it:51.F.a}} to
  \textup{\ref{it:51.F.d}} of page \ref{p:123}. If $M$ satisfies
  moreover condition \textup{\ref{it:58.f}} above, it satisfies also
  condition \textup{ref{it:51.F.e}}, i.e., $M$ is totally $0$-connected.
\end{proposition}

Indeed, let $X,Y$ be $0$-connected objects, we must prove $X\times Y$
is $0$-connected. First, it is ``non-empty'', because it got a section
(as $X$ and $Y$ each have a section). All that remains to do is show
that two connected components of $X\times Y$ are equal. By assumption
\ref{it:58.f}, each has a section, say $(s_i,t_i)$ with
$i\in\{1,2\}$. These two sections can be joined by a two-step chain
\[ (s_1,t_1), \quad (s_2,t_1), \quad (s_2,t_2)\]
the first step contained in $X\times t_1$ which is connected, the
second in $s_2\times Y$ which is connected, hence the two sections
belong to the same connected component, qed.

The condition \ref{it:58.f} is \emph{strictly} stronger than
\ref{it:51.F.e} (when \ref{it:51.F.a} to \ref{it:51.F.d} are
satisfied), as we see by taking for $M$ the category of all sheaves on
an irreducible topological space $X$ -- thus $M$ is a totally aspheric
topos, a lot better it would seem than just totally $0$-connected, but
condition \ref{it:58.f} is satisfied if{f} the topos is equivalent to
the final topos or ``one-point topos'', i.e., if{f} the topology of
$X$ is the chaotic one ($X$ and $\emptyset$ are the only open
subsets). We'll say $M$ is \emph{strictly totally $0$-connected}, if
it satisfies the conditions \ref{it:51.F.a} to \ref{it:58.f} (where
\ref{it:51.F.e} is a consequence of the others). Thus, if $M$ is
strictly totally $0$-disconnected, we get for any object $X$ a
canonical bijection, functorial in $X$
\begin{equation}
  \label{eq:58.1}
  \overline\Gamma(X) \eqdef \;\begin{tabular}{@{}c@{}}homotopy classes\\of sections of
  $X$\end{tabular}\; \tosim \pi_0(X).
  \tag{1}
\end{equation}

Assume now, moreover, that $X,Y$ are two objects of $M$ such that the
object $\bHom(X,Y)$ exists. Then, using the observation of
\hyperref[cor:51.E.2]{cor.\ 2} (p.\ \ref{p:122}),\pspage{145} we get
the familiar relationship
\begin{equation}
  \label{eq:58.2}
  \oHom(X,Y) \tosim \pi_0(\bHom(X,Y)),
  \tag{2}
\end{equation}
where $\oHom$ denotes homotopy classes of maps, with respect to the
canonical homotopy structure of $M$.

The first example which I have in mind is the case $M=\Cat$. I was a
little short yesterday about $W_\Cat$ being ``geometric'' -- the
condition \ref{it:57.aprime} of page \ref{p:142} is evident indeed in
terms of the (geometric!) definition of weak equivalence in terms of
non-commutative cohomology (where we need to case only about zero
dimensional cohomology!). But condition \ref{it:57.b} is a consequence
of the fact that $0$-connectedness in \Cat{} can be checked using only
$\Simplex_1$ (i.e., $\Simplex_1$ is a generating contractor for the
canonical homotopy structure of \Cat), and that $\Simplex_1$ is
$W_\Cat$-aspheric over the final category $e$, i.e., that for any $C$
in \Cat, the projection
\[ C\times\Simplex_1\to C\]
is in $W_\Cat$. I want to start being attentive from now on about what
exactly are the formal properties of $W_\Cat$ we are using -- it
really seems they boil down to very few, which we have kept using
without ever having to refer to the ``meaning'' of weak equivalence in
terms of cohomology (except for proving the formal properties we
needed, in terms of the precise definition of weak equivalence we have
been starting with from the very beginning).

In terms of the canonical homotopy structure in \Cat, admitting
$\Simplex_1$ as a generating homotopy interval (a contractor, as a
matter of fact), we now get a notion of two functors from a category
$X$ to a category $Y$ being homotopic, and a corresponding notion of
homotopy classes of functors from $X$ to $Y$, which are in one-to-one
correspondence with the connected components of the category
$\bHom(X,Y)$ of all functors from $X$ to $Y$. An elementary homotopy
from a functor $f$ to another $g$ (with respect of course to the basic
generating interval $\Simplex_1$, which is always understood here) is
nothing else but a morphism from $f$ into $g$. Thus, homotopy classes
of functors are nothing but equivalence classes for the equivalence
relation in $\Hom(X,Y)$ generated by the relation ``there exists a
morphism from $f$ to $g$''. To consider at all this equivalence
relation, rather than the usual one of isomorphy between functors, is
rather far from the spirit in which categories are generally being
used -- as is also the very notion of a functor which is a weak
equivalence, which has been our starting point. Topological motivation
alone, it seems, would induce anyone to introduce such barbaric
looking notions into category theory!

From\pspage{146} our point of view, the main point for paying
attention to the homotopy relation for functors, is of course because
homotopic functors define the same map in the localized category \Hot,
which category for the time being is (together with the modelizers
designed for describing it) our main focus of attention. According to
the general scheme of homotopy theory as reviewed previously, homotopy
classes of functors give rise to a quotient category of \Cat, which is
at the same time a localization of \Cat{} with respect to homotopisms,
and which we'll denote by $\oCat$. Thus we get a factorization of the
canonical functor from \Cat{} to \Hot{}
\[ \Cat \to \oCat \to \Hot.\]
This is just the factorization of the localization functor with
respect to $W_\Cat=W$, through the ``\emph{partial localization}''
with respect to the smaller set $W_h\subset W$ consisting of
homotopisms only. All we have used about $W=W_\Cat$ (from which the
localization $\Hot=W^{-1}\Cat$ is deduced) for getting this
factorization, was that $f\in W$ implies $\pi_0(f)$
bijective,\footnote{\alsoondate{6.4.} As a matter of fact, the condition about
  $\pi_0$ visibly isn't needed!} and that the projections
$C\times\Simplex_1\to C$ are in $W$. (Of course, we assume tacitly that
$W$ is saturated too.) We may view \Hot{} as a localization of \oCat{}
with respect to the set of arrows $\overline W$ corresponding to $W$
-- as a matter of fact, $W$ may be viewed as the inverse image of
$\overline W$ by the canonical functor from \Cat{} to its partial
localization \oCat{} -- the category of ``homotopy types'' relative to
the canonical homotopy structure of \Cat.

Well-known analogies would suggest at this point that we may well be
able to describe \Hot{} in terms of $(\oCat, \overline W)$ by a
\emph{calculus of fractions} -- right fractions presumably, or maybe
either right or left. This may possibly lead to a direct proof of the
notion of weak equivalence we have been working with being strongly
saturated, without having to rely upon Quillen's closed model
theory. But it is not yet the moment to pursue this line of thought,
which would take us off the main focus at present.

% 59
\hangsection[Case of the ``next best'' modelizer \Spaces{} -- and
\dots]{Case of the ``next best'' modelizer
  \texorpdfstring{\Spaces}{(Spaces)} -- and need of introducing the
  \texorpdfstring{$\pi_0$}{pi-0}-functor as an extra structure on a
  would-be modelizer \texorpdfstring{$M$}{M}.}\label{sec:59}%
Before pushing ahead, I would like to make still another point about
the work done yesterday -- a point suggested by looking at the case of
the modelizer \Spaces, which after all is the next best ``naive''
modelizer, less close to algebra than \Cat, but still worth being
taken into account! This category satisfies the conditions
\ref{it:51.F.a} to \ref{it:58.f}, \emph{except} the condition
\ref{it:51.F.c} -- which would mean that the connected components of a
space (as defined in terms of usual topology) are open subsets, which
is true (for a space and its open subsets) \emph{only} for locally
connected spaces. The point is that this doesn't (or shouldn't)
really\pspage{147} matter -- the way topological spaces are used as
``homotopy models'' in standard homotopy theory, it is \emph{pathwise
  connected} components that count, and not the topological ones. In
terms of these, there is still a canonical functor
\[\pi_0:\Spaces\to\Sets,\]
this functor however is no longer left adjoint of the functor in
opposite direction, associating to every set $E$, the corresponding
discrete topological space. (To get an adjunction, we should have to
restrict to the category of pathwise locally connected topological
spaces.) It doesn't matter visibly -- all that's being used is that
$\pi_0$ commutes to arbitrary sums, and takes pathwise $0$-connected
spaces into one-point sets.

This suggests that we should generalize the notions around the
``canonical'' homotopy structure on a category $M$, to the case of a
category which need not satisfy the exacting conditions of total
$0$-connectedness, by introducing as an \emph{extra structure} upon
$M$ a given functor
\[\pi_0:M\to\Sets,\]
subject possibly to suitable restrictions. The first which comes to
mind here is commutation with sums -- it doesn't seem though we've had
to use this property so far. All we've used occasionally was existence
of finite products in $M$, and commutation of $\pi_0$ to these.

If we think of $M$ as a would-be modelizer, and therefore endowed with
a hoped-for functor
\[M\to \Hot,\]
there is a natural functor $\pi_0$ indeed on $M$, namely the
composition
\[ M \to \Hot \to \Sets,\]
where the canonical functor
\[\pi_0:\Hot\to\Sets\]
is deduced from the $\pi_0$-functor $\Cat\to\Sets$ considered
previously, by factorization through the localized category \Hot{} of
\Cat. Thus, ``the least we would expect'' from a category $M$ for
being eligible as a modelizer is that there should be a natural
functor $\pi_0$ around, corresponding to the intuition of connected
components. In case of a ``canonical'' modelizer $M$ (maybe we should
say rather: canonical with respect to a given $\pi_0$), there is the
feeling that the functor $M\to\Hot$ we are after could eventually be
squeezed out from just $\pi_0$, and that it could be viewed as
something like a ``total left derived functor'' of the functor
$\pi_0$. But this for the time being is still thin air\ldots

What\pspage{148} we \emph{can} do however at present, in terms of a
given functor $\pi_0$, is to introduce the corresponding notion of
$0$-connectedness (understood: with respect to $\pi_0$), namely
objects $X$ such that $\pi_0(X)$ is a one-point set, the notion of
compatibility of a homotopy structure $h$ on $M$ with $\pi_0$, and the
$\pi_0$-canonical (or simply, ``canonical'') homotopy structure on
$M$, which now should be denoted by $h_\piz$ rather than $h_M$ (unless
we write $h_\bM$ where \bM{} denotes the pair $(M,\piz)$), which is
the widest weak homotopy interval structure on $M$ which is
\piz-admissible, and can be described (assuming \piz{} commutes with
finite products) in terms of all $0$-connected intervals as a
generating family of weak homotopy intervals. The generalities of
par.\ \ref{sec:54} about the relationship of $h_\bM$ with $h_W$ (where
$W\subset\Fl(M)$) should carry over verbatim, as well as those of the
next, provided everywhere $0$-connectedness is understood relative to
the given functor \piz, and ``total $0$-connectedness'' is interpreted
as just meaning that \piz{} commutes to finite products. Thus, our
contact with ``geometry'' via true honest connected components of
objects was of short duration, and back we are to pure algebra with
just a functor given which we call \piz{}, God knows why -- the culprit
for this change of perspective being poor modelizer \Spaces, which was
supposed to represent the tie with so-called ``topology''\ldots

% 60
\hangsection[Case of strictly totally aspheric topos. A timid start on
\dots]{Case of strictly totally aspheric topos. A timid start on
  axiomatizing the set \texorpdfstring{\scrW}{W} of weak equivalences
  in \texorpdfstring{\Cat}{(Cat)}.}\label{sec:60}%
I almost forgot I still have to check ``handiness'' of the notions
developed yesterday, on the example of test categories or rather, the
corresponding elementary modelizers \Ahat. As usual, I can't resist
being a little more general, so let's start with an arbitrary topos
\scrA{} first. It always satisfies conditions \ref{it:51.F.a} and
\ref{it:51.F.b} of page \ref{p:123}. Condition \ref{it:51.F.c}, namely
that every object of \scrA{} could be decomposed into a sum of
$0$-connected ones, is equivalent with saying that \scrA{} admits a
generating subcategory $A$ made up with $0$-connected objects. In this
case, \scrA{} is called \emph{locally $0$-connected} or simply,
\emph{locally connected} -- which generalizes the notion known under
this name from topological spaces to topoi. On the other hand,
condition \ref{it:51.F.d} is expressed by saying that the topos
considered is \emph{$0$-connected} -- equally a generalization of the
corresponding notion for spaces. Condition \ref{it:51.F.e}, about the
product of two $0$-connected objects being $0$-connected, is a highly
unusual one in ordinary topology. For a topological space, it means
that the space is irreducible (hence reduced to a point if the space
is Hausdorff). In accordance with the terminology introduced
yesterday, we'll say that \scrA{} is \emph{totally $0$-connected} if
it is locally connected, and if the product of two $0$-connected
objects is again $0$-connected. The standard arguments show that for
this, it is enough\pspage{149} that the product of two elements in $A$
be $0$-connected. The topos is called \emph{strictly totally
  $0$-connected} if it is locally connected, and if moreover every
$0$-connected object admits a section -- which (as we saw earlier
today) implies \scrA{} is totally $0$-connected, as the wording
suggests. It amounts to the same to demand that every ``non-empty''
object have a section -- and for this it is enough that the elements
in $A$ have a section. This latter condition is trivially checked for
all standard test categories I've met so far (they all have a final
object, and there maps of the latter into any other object of $A$). A
noticeable counterexample here is $(\Simplexf)\uphat$ (semisimplicial
face complexes, \emph{without degeneracies}), where the \emph{weak}
test category \Simplexf{} hasn't got a final object ($\Simplex^0$
definitely isn't!) and no $\Simplex_n$ in \Simplexf{} except
$\Simplex_0$ only has got a section.
\begin{remark}
  I wonder, when \scrA{} is totally $0$-connected, and moreover
  modelizing, i.e., the Lawvere element is aspheric over $e$, if this
  implies \scrA{} is totally aspheric, and that every element which is
  ``non-empty'' has a section (i.e., \emph{strict} total $0$-connectedness).
\end{remark}

Next thing is to look at
\[W\subset\Fl(\scrA),\]
the set of weak equivalences (as defined by non-commutative cohomology
of topoi), and see if it is ``geometric'' (page \ref{p:142}). Condition
\ref{it:57.aprime} is clearly satisfied, there remains the condition
\ref{it:57.b}, namely whether $0$-connectedness of an object of
\scrA{} (\scrA{} supposed totally $0$-connected) can be tested, using
``intervals'' which are \emph{aspheric over $e$}. More specifically,
we want to test that two sections of $I$ belong to the same connected
component, using for ``joining'' them intervals that are aspheric over
$e$. The natural idea here is to assume the generating objects in $A$
to be aspheric over $e$ (which implies \scrA{} is \emph{totally
  aspheric}, not only totally $0$-connected), and to use these objects
(endowed with suitable sections) as testing intervals. This goes
through smoothly, indeed, if we assume moreover strict total
zero-connectedness. Thus:
\begin{proposition}
  Let \scrA{} be a topos which is strictly totally aspheric
  \textup(namely totally aspheric, and every ``non-empty'' object has
  a section\textup). Then the set $W\subset\Fl(\scrA)$ of weak
  equivalences in \scrA{} is ``geometric'', and accordingly, the
  homotopy structure $h_W$ defined in terms of aspheric homotopy
  intervals, is the same as the canonical homotopy structure $h_\scrA$
  defined in terms of merely $0$-connected homotopy
  intervals. Moreover, for any set $A\subset\Ob\scrA$ which is
  generating and whose objects are $0$-connected, the set of
  $0$-connected intervals $\bI=(I,\delta_0,\delta_1)$ with $I$ in $A$\kern1pt,
  generate the homotopy structure $h_W$.
\end{proposition}

A\pspage{150} topos \scrA{} as in the proposition (namely strictly
totally aspheric) need not be a modelizer, i.e., the Lawvere element
$L$ need not be aspheric, or what amounts to the same because of
$h_W=h_\scrA$ and $L$ being a contractor, $L$ need not be connected:
take $\scrA=\Sets$! I suspect though this to be the only
counterexample (up to equivalence). For \scrA{} to be a modelizer, we
need only find an object in $A$ which has got two \emph{distinct}
sections (because then they must be disjoint, i.e., $e_0\sand
e_1=\varnothing_\scrA$, because $e$ has only the full and the ``empty''
subobject, as a consequence of every ``non-empty'' object of \scrA{}
having a section), thus getting a ``homotopy interval'' (more
specifically, a separated and relatively aspheric one) as requested
for \scrA{} to be a modelizer. Now for any would-be test category met
with so far (except precisely \Simplexf{} and the like, which are
\emph{not} test categories but only weak ones), this condition that
there are objects in $A$ which have more than just one ``point'' ($=$
section), is trivially verified.

In case of a topos of the type \Ahat, the notion of weak equivalence
in \Ahat{} can be described (independently of cohomological notions)
in terms of the notion of weak equivalence in \Cat, more precisely
\[W_A \eqdef W_\Ahat = i_A^{-1}(\scrW),\]
where
\[i_A:\Ahat\to\Cat, \quad F\mapsto A_{/F}\]
is the canonical functor, and where
\[\scrW=W_\Cat\]
is the set of weak equivalences in \Cat. These of course, for the time
being, are defined in terms of cohomology (including a bit of
non-commutative one in dimension $1$\ldots). We may however start with
any $\scrW\subset\Fl(\Cat)$ and look at which formal properties on
\scrW{} (satisfied for usual weak equivalences) allow our arguments to
go through, in various circumstances. We may make a list of those
which have been used today, and go on this way a little longer, with
the expectation we'll finally wind up with an axiomatic
characterization of weak equivalences, i.e., of \scrW, in terms of the
category \Cat, say.
\begin{enumerate}[label=\alph*)]
\item\label{it:60.a}
  (Pour m\'emoire!) \scrW{} is saturated (cf.\ page \ref{p:101}).
\item\label{it:60.b}
  \scrW{} is $0$-admissible, i.e., if $f:C\to C'$ is in \scrW,
  $\piz(f)$ is bijective.
\item\label{it:60.c}
  $\Simplex_1$ is \scrW-aspheric over $e=\Simplex_0$, i.e., for every $C$
  in \Cat, the projection $C\times\Simplex_1\to C$ is in \scrW.
\item\label{it:60.d}
  Any $C$ in \Cat{} which has a final element is \scrW-aspheric, i.e,
  $C\to e$ is in \scrW.
\end{enumerate}

The\pspage{151} condition \ref{it:60.a} will be tacitly understood
throughout, when taking a \scrW{} to replace usual weak
equivalences. Conditions \ref{it:60.b} and \ref{it:60.c} then were
seen to be enough to imply that $h_\scrW=h_\Cat$. On the other hand,
one sees at once that for the proposition over for a topos \Ahat{}
which is strictly totally aspheric, when we define now
$\scrW_A\subset\Fl(\Ahat)$ as just $i_A^{-1}(\scrW)$, in order to
conclude $h_{\scrW_A}=h_\Ahat$, all we made use of was (besides
saturation of \scrW{} of course, i.e., \ref{it:60.a}) \ref{it:60.b}
and \ref{it:60.d}.

One may object that \ref{it:60.d} isn't expressed in terms of the
category structure of \Cat{} only, but we could express it in terms of
this structure, by the remark that $C$ has a final object if{f} there
exists a $\Simplex^-$-homotopy of $\id_C$ to a constant section of $C$
(this ``section'' will indeed be defined necessarily by a final object
of $C$). As was to be expected, in this formulation, as in
\ref{it:60.c} too, the object $\Simplex_1$ of \Cat{} is playing a
crucial role. But at this point it occurs to me that \ref{it:60.c}
implies \ref{it:60.d}, by the homotopy lemma -- thus for the time
being all we needed was \ref{it:60.a}\ref{it:60.b}\ref{it:60.c}.

% 61
\hangsection{Remembering about the promised ``key result'' at last!}\label{sec:61}%
We now in the long last get back to the ``key result'' promised time
ago, and which we kept pushing off. To pay off the trouble of the long
digression in between, maybe it'll come out more smoothly. It shall be
concerned with the functor
\begin{equation}
  \label{eq:61.1}
  i:A\to\Cat,
  \tag{1}
\end{equation}
where in the end $A$ will be (or turn out to be) a strict test
category, and we want to give characterizations for $i$ to be a (weak)
test functor, namely the corresponding functor
\begin{equation}
  \label{eq:61.2}
  i^*:\Cat\to\Ahat\tag{2}
\end{equation}
to induce an equivalence between the localizations, with respect to
``weak equivalences''.\footnote{Plus a little more, see below
  \eqref{eq:61.8}.} We'll now be a little more demanding, and instead
of just assuming it is the usual notion of weak equivalence either in
\Cat{} or in \Ahat, I'll assume that the set
$\scrW_A\subset\Fl(\Ahat)$ is defined in terms of a saturated set
$\scrW\subset\Fl(\Cat)$ of arrows in \Cat, by taking the inverse image
by the functor
\begin{equation}
  \label{eq:61.3}
  i_A:\Ahat\to\Cat\tag{3}
\end{equation}
as above, whereas in the target category \Cat{} of \eqref{eq:61.2},
we'll work with another saturated set $\scrW'$ -- thus besides
\eqref{eq:61.1}, the data are moreover
\begin{equation}
  \label{eq:61.4}
  \scrW,\scrW' \subset\Fl(\Cat),\tag{4}
\end{equation}
two\pspage{152} saturated sets of arrows in \Cat, with no special
assumption otherwise for the time being. We will introduce the
properties we need on these, as well as on $A$ and on $i$, stepwise as
the situation will tell us. We want to derive a set of conditions
ensuring that both \eqref{eq:61.2} and \eqref{eq:61.3} induce
equivalences for the respective localizations, namely
\begin{equation}
  \label{eq:61.5}
  \overline i^*:(\scrW')^{-1}\Cat \tosimeq \scrW_A^{-1}\Ahat
  \quad\text{and}\quad
  \overline i_A:\scrW_A^{-1}\Ahat \tosimeq \scrW^{-1}\Cat.
  \tag{5}
\end{equation}
We may assume beforehand that $A$ is a weak test category (``with
respect to \scrW'') and hence the second functor in \eqref{eq:61.5} is
already an equivalence, in which case the condition that the first
functor in \eqref{eq:61.5} exist and be an equivalence (existence just
meaning the condition
\begin{equation}
  \label{eq:61.star}
  \scrW'\subset (i^*)^{-1}(\scrW_A),\quad\text{i.e.,}\quad
  i^*(\scrW') \subset \scrW_A)\tag{*}
\end{equation}
is equivalent to the corresponding requirement for the composition
\begin{equation}
  \label{eq:61.6}
  f_i:\Cat \xrightarrow{i^*} \Ahat \xrightarrow{i_A} \Cat,\tag{6}
\end{equation}
namely that this induce an equivalence
\begin{equation}
  \label{eq:61.7}
  (\scrW')^{-1}\Cat \tosimeq \scrW^{-1}\Cat.\tag{7}
\end{equation}
As a matter of fact, we are going to be slightly more demanding (in
accordance with the notion of a weak test functor as developed
previously, cf.\ page \ref{p:85}), namely that the inclusion
\eqref{eq:61.star} be in fact an equality
\begin{equation}
  \label{eq:61.8}
  \scrW' = (i^*)^{-1}(\scrW_A),\tag{8}
\end{equation}
the similar requirement for the functor $i_A$ \eqref{eq:61.3} being
satisfied by the very definition of $\scrW_A$ in terms of \scrW{} as
\begin{equation}
  \label{eq:61.9}
  \scrW_A = i_A^{-1}(\scrW).\tag{9}
\end{equation}
In view of this, the extra requirement \eqref{eq:61.8} boils down to
the equivalent requirement in terms of the composition $f_i$
\eqref{eq:61.6}:
\begin{equation}
  \label{eq:61.10}
  \scrW' = f_i^{-1}(\scrW).\tag{10}
\end{equation}

To sum up, we want to at least develop sufficient conditions on the
data $(\scrW,\scrW',A,i)$ for \eqref{eq:61.8} to hold (which allows to
define the first functor in \eqref{eq:61.5}, whereas the second is
always defined), and the functors in \eqref{eq:61.5} to be
equivalences; or equivalently, for \eqref{eq:61.10} to hold, hence a
functor \eqref{eq:61.7}, and for the latter to be an equivalence, and
equally $i_A$ to induce an equivalence for the localizations (i.e.,
the second functor \eqref{eq:61.5} an equivalence). It should be noted
that the latter condition depends only on $(A,\scrW)$, not on $i$ nor
on $\scrW'$ -- it will be satisfied automatically if we assume $A$ to
be a \emph{weak test category} relative to \scrW{} (namely\pspage{153}
$i_A$ and the right adjoint functor
\[j_A:\Cat\to\Ahat\]
to induce quasi-inverse equivalences for the localizations
$\scrW_A^{-1}\Ahat$ and $\scrW^{-1}\Cat$). We have already developed
handy n.s.\ conditions for this in case $\scrW=W_\Cat$ -- and it would
be easy enough to look up which formal properties exactly on $W_\Cat$
have been used in the proof, if need be. At any rate, we know
beforehand that we can find $(A,\scrW)$ such that $A$ be a weak test
category (and even a strict test category!) relative to \scrW. When
$(A,\scrW)$ are chosen this way beforehand, the question just amounts
to finding conditions on $(\scrW',i)$ for \eqref{eq:61.10} to hold and
for \eqref{eq:61.7} to be an equivalence of categories. If we find
conditions which actually can be met, then we get as a byproduct the
formula \eqref{eq:61.10} precisely, which says that there is \emph{just one}
$\scrW'$ satisfying the conditions on $\scrW'$, namely
$f_i^{-1}(\scrW)$! Of course, taking \scrW{} to be just $W_\Cat$, it
will follow surely that $\scrW'$ is just $W_\Cat$ -- i.e., we should
get an axiomatic characterization of weak equivalences.

Let's now go to work, following the idea described in par.\
\ref{sec:57} (pages \ref{p:96}--\ref{p:98}), and expressed mainly in
the basic diagram of canonical maps in \Cat, associated to a given
object $C$ in \Cat:
\begin{equation}
  \label{eq:61.11}
  \begin{tabular}{@{}c@{}}
  \begin{tikzcd}[baseline=(O.base)]
    A_{/C} \ar[r] & A_{\sslash C} & \\
    & A\times C \ar[u]\ar[r] & |[alias=O]| A
  \end{tikzcd},
  \end{tabular}
  \tag{11}
\end{equation}
which will allow to compare $f_i(C) = A_{/C}$ with $C$. When
$\scrW=W_\Cat$, it was seen in loc.\ cit.\ that the two latter among
these three arrows are in \scrW, provided (for the middle one) we
assume that $i$ takes its values in the subcategory of \Cat{} of all
contractible categories. What remained to be done, for getting the
conditions for $f_i$ to be ``weakly equivalent'' to the identity
functor (and hence induce an equivalence for the localizations) was to
write down conditions for the first functor in \eqref{eq:61.11},
$A_{/C}\to A_{\sslash C}$, to be in \scrW. We have moreover to  be
explicit on the conditions to put on a general \scrW, in order for the
two latter maps in \eqref{eq:61.11} to be in \scrW. For the last
functor, this conditions as a matter of fact involves both $A$ and
\scrW, it is clearly equivalent to
\begin{description}
\item[\namedlabel{it:61.W1}{W~1)}]
  $A$ is \scrW-aspheric over $e$.
\end{description}
If we assume that $A$ has a final element, this condition is satisfied
provided \scrW{} satisfies the condition (where there is\pspage{154}
no $A$ anymore!) that any $X$ in \Cat{} with final element is
\scrW-aspheric over $e$ -- a condition which is similar to condition
\ref{it:60.d} above (page \ref{p:150}), but a littler stronger still
(as we want $X\times C\to C$ in \scrW{} for any $C$), it is a
consequence however of condition \ref{it:60.c}, as was seen on page
\ref{p:150} using the remark that $X$ is $\Simplex_1$-contractible. Thus
we get the handy condition
\begin{enumerate}[label=W~\arabic*')]
\item\label{it:61.W1prime}
  \scrW{} satisfies condition \ref{it:60.c} of page \ref{p:150}, i.e.,
  $\Simplex_1$ is \scrW-aspheric\\ over~$e$,
\end{enumerate}
which will allow even to handle the case of an $A$ which is
contractible (for the canonical homotopy structure of \Cat, namely
$\Simplex_1$-contractible), and not only when $A$ has a final object.

To insure that the canonical map
\begin{equation}
  \label{eq:61.star2}
  A\times C\to A_{\sslash C}\tag{*}
\end{equation}
is in \scrW, using the argument on page \ref{p:97}, we'll add one more
condition to the provisional list on page \ref{p:150}, namely:
\begin{enumerate}[label=\alph*),start=5]
\item\label{it:61.e}
  For any cartesian functor $u:F\to G$ of two fibered categories over
  a third one $B$ (everything in \Cat), such that the induced maps on
  the fibers are in \scrW, $u$ is in \scrW.
\end{enumerate}
We are now ready to state the condition we need (stronger than
\ref{it:61.W1prime}):
\begin{description}
\item[\namedlabel{it:61.W2}{W~2)}]
  \scrW{} satisfies conditions \ref{it:60.a} to \ref{it:60.c} (page
  \ref{p:150}) and \ref{it:61.e} above.
\end{description}

As a matter of fact, \ref{it:60.a} to \ref{it:60.c} ensure that
\scrW{} is ``geometric'', i.e., essentially $h_\scrW = h_\Cat$, hence
the proposition page \ref{p:143} applies, to imply that the maps
\[C \to \bHom(i(a), C)\]
are in \scrW{} (they are even $h_\scrW$-homotopisms) and by condition
\ref{it:61.e} this implies that \eqref{eq:61.star2} above is in
\scrW. We don't even need \ref{it:60.b} ($0$-admissibility for \scrW),
as all we care about is $h_\Cat\le h_\scrW$ (not the reverse
inequality), but surely we're going to need \ref{it:60.b} or something
stronger soon enough, as $\scrW=\Fl(\Cat))$ say surely wouldn't do!

Now to the last (namely first) map of our diagram \eqref{eq:61.11},
namely
\begin{equation}
  \label{eq:61.12}
  A_{/C} \to A_{\sslash C}.\tag{12}
\end{equation}
To give sufficient conditions for this to be in \scrW, we want to
mimic the standard asphericity criterion for a map in \Cat, which we
have used constantly before. This leads to the extra
condition
\begin{enumerate}[label=\alph*),resume]
\item\label{it:61.f}
  Let\pspage{155} $u:X'\to X$ be a map in \Cat{} such that for any $a$
  in $X$, the induced category $X'_{/a}$ be \scrW-aspheric, i.e.,
  $X'_{/a}\to e$ is in \scrW{} (or what amounts to the same if we
  assume \ref{it:60.d}, e.g., if we assume the stronger condition
  \ref{it:60.c}, the induced map $X'_{/a}\to X_{/a}$ is in
  \scrW). Then $u$ is in \scrW.
\end{enumerate}

If $u:X'\to X$ satisfies the condition stated above, namely that after
any base change $X_{/a}\to X$, the corresponding map $u_{/a}$ is in
\scrW, we'll say that $u$ is \emph{weakly} \scrW-aspheric (whereas
``\scrW-aspheric'' means that after \emph{any} base change $Y\to X$,
the corresponding $f_Y$ is in \scrW). Thus, condition \ref{it:61.f}
can be stated as saying that \emph{a weakly \scrW-aspheric map in
  \Cat{} is in \scrW}.

For making use of this latter assumption on \scrW, we have to look at
how the induced categories for the functor \eqref{eq:61.12} look like,
which functor (I recall) induced a bijection on objects. These can be
described as pairs $(a,p)$, with $a$ in $A$ and $p$ a map in \Cat
\[p:i(a)\to C.\]
An easy computation shows the
\begin{lemma}
  Let $(a,p)$ as above. The induced category $(A_{/C})_{/(a,p)}$
  \textup(for the functor \textup{\eqref{eq:61.12})} is canonically
  isomorphic to the induced category $A_{/G}$, where $G$ is the
  fibered product in \Ahat{} displayed in the diagram
  \begin{equation}
    \label{eq:61.13}
    \begin{tabular}{@{}c@{}}
    \begin{tikzcd}[baseline=(O.base)]
      G\ar[r]\ar[d] & a\ar[d] \\
      i^*(\bFl(C)) \ar[r] & |[alias=O]|i^*(C)
    \end{tikzcd},
    \end{tabular}
    \tag{13}
  \end{equation}
  where
  \[\bFl(C)\eqdef \bHom(\Simplex_1,C)\]
  and where the second horizontal arrow in \textup{\eqref{eq:61.13}}
  is the $i^*$-transform of the target map in \Cat
  \[\bFl(C) \xrightarrow t C.\]
\end{lemma}
\begin{corollary}
  In order for \textup{\eqref{eq:61.12}} to be weakly \scrW-aspheric,
  it is n.s.\ that the map
  \begin{equation}
    \label{eq:61.14}
    i^*(t) : i^*(\bFl(C)) \to i^*(C)\tag{14}
  \end{equation}
  in \Ahat{} be $\scrW_A$-aspheric \textup(i.e., be ``universally in $\scrW_A$''\textup).
\end{corollary}

To make the meaning of the latter condition clear, it should be noted
that the condition \ref{it:61.f} on \scrW{} guarantees precisely that
for a map $u:F'\to F$ in \Ahat{} ($A$ any category) to be
$\scrW_A$-aspheric, it is n.s.\ that the corresponding map $i_A(u)$ in
\Cat{} be weakly \scrW-aspheric -- the kind of thing we have been
constantly using before of course, when\pspage{156} assuming
$\scrW=W_\Cat$.

It is in the form of \eqref{eq:61.14} that weak \scrW-asphericity of
\eqref{eq:61.12} will actually be checked, whereas it will be
\emph{used} just by the fact that \eqref{eq:61.12} is in \scrW.

\bigbreak
\noindent\hfill\ondate{6.4.}\par

% 62
\hangsection[An embarrassing case of hasty
over-axiomatization. \dots]{An embarrassing case of hasty
  over-axiomatization. The unexpected riches\dots}\label{sec:62}%
I finally stopped with the notes last night, by the time when I
started feeling a little uncomfortable. A few minutes of reflection
then were enough to convince me that definitely I hadn't done quite
enough preliminary scratchwork yet on this ``key result'' business,
and embarked overoptimistically upon a ``mise en \'equation'' of the
situation, with the pressing expectation that a characterization of
weak equivalences should come out at the same time. First thing that
became clear, was that the introduction of two different localizing
sets of arrows $\scrW,\scrW'$ in \Cat{} was rather silly alas, nothing
at all would come out unless supposing from the very start
$\scrW=\scrW'$. Indeed, the crucial step for getting the ``key
result'' on test functors we are out for, goes as follows.

As the target map
\[t: \bFl(C)=\bHom(\Simplex_1,C) \to C\]
in \Cat{} is clearly a homotopy retraction, and $i^*:\Cat\to\Ahat$
commutes with products, we do have a good hold on the condition
\eqref{eq:61.14} of the last corollary, namely that $i^*(t)$ be
$\scrW_A$-aspheric -- e.g., it is enough that the contractor
$i^*(\Simplex_1)$ in \Ahat{} be $\scrW_A$-aspheric over $e_\Ahat$ (for
instance, it is often enough it be $0$-connected!). In view of the
corollary and the condition \ref{it:61.f} on \scrW{}
(\hyperref[p:155]{last page}), we thus get a very good hold upon the
map
\begin{equation}
  \label{eq:62.star}
  A_{/C} \to A_{\sslash C}\tag{*}
\end{equation}
in \Cat{} being in \scrW, and hence on all three maps in the diagram
\eqref{eq:61.11} (page \ref{p:153}) being in \scrW. With this in mind,
the key step can be stated as follows:
\begin{lemma}
  Assume that \scrW{} satisfies the conditions
  \textup{\ref{it:60.a}\ref{it:60.c}\ref{it:61.e}\ref{it:61.f}}
  \textup(pages \ref{p:150}, \ref{p:154}, \ref{p:155}\textup),
  that $A$ is \scrW-aspheric over $e$ \textup(i.e., $A\times C\to C$
  is in \scrW{} for any $C$, which will be satisfied if $A$ is
  $\Simplex_1$-contractible in \Cat, for instance if $A$ has a final or
  initial object\textup), and that the objects $i(a)$ in \Cat{}
  \textup(for any $a$ in $A$\textup) are contractible \textup(for the
  canonical homotopy structure of \Cat, i.e., $\Simplex_1$-contractible,
  or even only for the wider homotopy structure $h_\scrW$ based on
  \scrW-aspheric homotopy intervals). Under\pspage{157} these
  conditions, the following conclusions hold:
  \begin{enumerate}[label=\alph*),font=\normalfont]
  \item\label{it:62.a}
    $\scrW = f_i^{-1}(\scrW)\quad$ \textup(where $f_i=i_Ai^*$ with
    yesterday's notations\textup).
  \item\label{it:62.b}
    The functor $\overline f\!_i$ from $\scrW^{-1}\Cat$ to itself
    induced by $f_i$ \textup(which is defined because of
    \textup{\ref{it:62.a})} is isomorphic \textup(canonically\textup)
    to the identity functor, and hence is an equivalence.
  \end{enumerate}
\end{lemma}

The use we have for the three maps in the diagram \eqref{eq:61.12} is
completely expressed in this lemma. The pretty obvious proof below
would not work at all if in \ref{it:62.a} above, we replace \scrW{} in
the left hand side by a $\scrW'$! We have to prove that for a map
$C\to C'$ in \Cat, this is in \scrW{} if{f} $A_{/C}\to A_{/C'}$
is. Now this is seen from an obvious diagram chasing in the diagram
below, using saturation condition \ref{it:37.b} on \scrW:
\[\begin{tikzcd}[baseline=(O.base)]
  A_{/C} \ar[d]\ar[r] & A_{\sslash C} \ar[d] &
  A\times C \ar[r]\ar[d]\ar[l] & C \ar[d] \\
  A_{/C} \ar[r] & A_{\sslash C}  &
  A\times C \ar[r]\ar[l] & |[alias=O]| C
\end{tikzcd},\]
where all horizontal arrows are already known to be in \scrW{} (the
assumptions in the lemma were designed for just that end). At the same
time, we see that the corresponding statements are equally true for
the functors $C\mapsto A_{\sslash C}$ and $C\mapsto A\times C$, and
that two consecutive among the four functors we got from
$H_\scrW=\scrW^{-1}\Cat$ to itself, deduced by localization by \scrW,
are canonically isomorphic, which proves \ref{it:62.b} by taking the
composition of the three isomorphisms
\[\gamma(A_{/C}) \tosim \gamma(A_{\sslash C}) \tosim \gamma(A\times C)
\tosim \gamma(C),\]
where $\gamma:\Cat\to H_\scrW=\scrW^{-1}\Cat$ is the canonical
functor.

With this lemma, we have everything needed in order to write down the
full closed chain of implications, between various conditions on
$(\scrW, A, i)$, from which to read off the ``key result'' we're
after. Before doing so, I would like still to make some preliminary
comments on the role of \scrW, and on the nature of the conditions we
have been led so far to impose upon \scrW.

A first feature that is striking, is that all conditions needed are in
the nature either of stability conditions (if such and such maps are
in \scrW, so are others deduced from them), or conditions stating that
such and such unqualified maps (the projection $\Simplex_1\times C\to C$
for any $C$, say) are in \scrW. We did not have any use of the only
condition stated so far, namely \ref{it:60.b} (if $f\in\scrW$,
$\piz(f)$ is bijective) of a \emph{restrictive} type on the kind of
arrows allowed in \scrW{} -- which is quite contrary to my
expectations. Thus, all conditions are trivially met if we
take\pspage{158} $\scrW=\Fl(\Cat)$ -- \emph{all} arrows in \Cat! This
circumstance seems tied closely to the fact that, contrarily to quite
unreasonable expectations, \emph{we definitely do \emph{not} get an
  axiomatic characterization of weak equivalences}, in terms of the
type of properties of \scrW{} we have been working with so far. As
soon as one stops for considering the matter without prejudice, this
appears rather obvious. As a matter of fact, using still cohomological
invariants of topoi and categories, there are lots of variants of the
cohomological definition of ``weak equivalence'', which will share all
formal properties of the latter we have been using so far, and
presumably a few more we haven't met yet. For instance, starting with
any ring $k$ (interesting cases would be \bZ, $\bZ/n\bZ$, \bQ), we may
demand on a morphism of topoi
\[f:X\to Y\]
to induce as isomorphism for cohomology with coefficients in $k$, or
with coefficients in any $k$-module, or with any twisted coefficients
which are $k$-modules -- already three candidates for a \scrW,
depending on a given $k$! We may vary still more, by taking, instead
of just one $k$, a whole bunch $(k_i)$ of such, or a bunch of
(constant) commutative groups -- we are thinking of choices such as
all rings $\bZ/n\bZ$, with possibly $n$ being subjected to be prime to
a given set of primes, along the lines of the Artin-Mazur theory of
``localization'' of homotopy types. And we may combine this with an
isomorphism requirement on twisted non-commutative $1$-cohomology, as
for the usual notion of weak equivalence. Also, in all the isomorphism
requirements, we may restrict to cohomology up to a certain dimension
(which will give rise to ``truncated homotopy types''). The impression
that goes with the evocation of all these examples, is that the theory
we have been pursuing, to come to an understanding of ``models for
homotopy types'', while we started with just usual homotopy types in
mind and a corresponding tacit prejudice, is a great deal richer than
what we had in mind. Yesterday's (or rather last night's)
embarrassment of finding out finally I had been very silly, is a
typical illustration of the embarrassment we feel, whenever a
foreboding appears of our sticking to inadequate ideas; still more so
if it is not just mathematics but ideas about ourselves say or about
something in which we are strongly personally implies. This
embarrassment then comes as a rescue, to bar the way to an
unwished-for overwhelming richness dormant in ourselves, ready to wash
away forever those ideas so dear to us\ldots

I\pspage{159} am definitely going to keep from now on a general
\scrW{} and work with this and the corresponding localization, which
in case of ambiguity we better won't denote by \Hot{} any longer (as
we might be thinking of usual homotopy types in terms of usual weak
equivalences), but by $H_\scrW$ or $\Hot_\scrW$, including such
notions as rational homotopy types, etc.\ (for suitable choices of
\scrW). The idea that now comes to mind here is that possibly, the
usual $W_\Cat$ of usual weak equivalences could be characterized as
the \emph{smallest} of all \scrW's, satisfying the conditions we have
been working with so far (tacitly to some extent), and maybe a few
others which are going to turn up in due course -- i.e.\ that the
usual notion of weak equivalence is the \emph{strongest} of all
notions, giving rise to a modelizing theory as we are developing. This
would be rather satisfactory indeed, and would imply that other
categories $H_\scrW$ we are working with \emph{are all localizations
  of \Hot}, with respect to a saturated set of arrows in \Hot,
satisfying some extra conditions which it may be worth while writing
down explicitly, in terms of the internal structures of \Hot{}
directly (if at all possible). All the examples that have been
flashing through my mind a few minutes ago, do correspond indeed to
equivalence notions weaker than so-called ``weak'' equivalence, and
hence to suitable localizations of the usual homotopy category
\Hot. But it is quite conceivable that this is not so for all \scrW's,
namely that the characterization just suggested of $W_\Cat$ is not
valid. This would mean that there are refinements of the usual notion
of homotopy types, which would still however give rise to a homotopy
theory along the lines I have been pursuing lately. There is of course
an immediate association with Whitehead's \emph{simple homotopy types}
-- maybe after all they can be interpreted as elements in a suitable
localization $H_\scrW$ of \Cat{} (and correspondingly, of any one of
the standard modelizers, such as semi-simplicial complexes and the
like)? In any case, sooner or later one should understand what the
smallest of all ``reasonable'' \scrW's looks like, and to which
geometric reality it corresponds. But all these questions are not
quite in the present main line of thought, and it is unlikely I am
going to really enter into it some day\ldots

% 63
\hangsection{Review of terminology
  \texorpdfstring{\textup(}{(}provisional\texorpdfstring{\textup)}{)}.}%
\label{sec:63}%
What I should do though immediately, is to put a little order in the
list of conditions for a set \scrW, which came out somewhat
chaotically yesterday. After the notes I still did a little
scratchwork last night, which I want now to write down, before coming
to a formal statement of the ``key result'' -- as this will of course
make\pspage{160} use of some list of conditions on \scrW.

First of all, I feel a review is needed of the few basic notions which
have appeared in our work, relative to a set of arrows
$W\subset\Fl(M)$ in a general category $M$. We will not give to the
maps in $W$ a specific name, such as ``weak equivalences'', as this
may be definitely misleading, in the general axiomatic set-up we want
to develop; here $W_\Cat$ is just one among many possible \scrW's and
correspondingly for a small category $A$, $W_A=W_\Ahat$ is just one
among the many $\scrW_A$'s, associated to the previous \scrW's. When
$M=\Cat$, it will be understood we are working with a fixed set
$\scrW\subset\Fl(\Cat)$, consisting of the basic ``equivalences'', on
which the whole modelizer story hinges. We may call them
\scrW-equivalences -- for the time being there will be no question of
varying \scrW.

Coming back to a general pair $(M, W\subset\Fl(M))$ (not necessarily a
``modelizer''), we may call the maps in $W$ $W$-equivalences. If $M$
has a final object $e$, we get the corresponding notion of
\emph{$W$-aspheric object} of $M$, namely an object $X$ such that the
unique map
\[ X\to e\]
is in $W$, i.e., is a $W$-equivalence. We'll define a
\emph{$W$-aspheric map}
\[f : X\to Y\]
in $M$ as one which is ``universally in $W$'', by which I mean that
for any base-change
\[ Y' \to Y,\]
the fiber-product $X'=X\times_Y Y'$ exists (i.e., $f$ is
``squarable'') and the map
\[ f': X' \to Y'\]
deduced from $f$ by base change is in $W$. The thing to be quite
careful about is that for an object $X$ in $M$, to say that $X$ is
\emph{$W$-aspheric over $e$} (meaning that the \emph{map} $X\to e$ is
$W$-aspheric) implies $X$ is $W$-aspheric \&c., but the converse need
not hold true. This causes a slight psychological uneasiness, due to
the fact I guess that the notion of a $W$-aspheric object has been
defined after all in terms of the \emph{map} $X\to e$, and
consequently may be thought of as meaning is ``in $W$ over
$e$''. Maybe we shouldn't use at all the word ``$W$-aspheric object''
here, not even by qualifying it as ``weakly $W$-aspheric'' to cause a
feeling of caution, but rather refer to this notion as ``$X$ is a
$W$-object'' -- and denote by $M(W)$ the set of all these objects (or
the corresponding full subcategory of $M$, and call $X$ $W$-aspheric
(dropping ``over $e$'')\footnote{But we'll see immediately that this
  conflicts with the standard terminology in topoi -- so no good
  either!} when it is ``universally'' a $W$-object. The terminology we
have been using so far was of course\pspage{161} suggested by the case
when $M$ is a \emph{topos} and $W$ the usual notion of weak
equivalence, but then to call $X$ in $M$ ``$W$-aspheric'' or simply
``aspheric'' does correspond to the usual (absolute) notion of
asphericity for the induced topos $M_{/X}$, only in the case when the
topos $M$ itself is aspheric. This is so in the case I was most
interested in (e.g., $M$ a modelizing topos), but if we want to use it
systematically in the general setting, the term I used of
``$W$-aspheric object'' is definitely misleading. Thus we better
change it now than never, and use the word ``$W$-object'' instead, and
the notation $M(W)$. As for the notion of $W$-aspheric map, in the
present case of a topos with the notion of weak equivalence, it does
correspond to the usual notion of asphericity for the induced morphism
of topoi
\[M_{/X} \to M_{/Y},\]
which is quite satisfactory.

There is still need for caution with the notion of $W$-aspheric maps
($W$-aspheric objects have disappeared in the meanwhile!), when
working in $M=\Cat$ (and the same thing if $M=\Spaces$). Namely, when
we got a map $f:A\to B$ in \Cat, this is viewed for topological
intuition as corresponding to a morphism of topoi
\[\Ahat\to\Bhat.\]
Now, the requirement that $f:A\to B$ should be $W_\Cat$-aspheric is a
lot stronger than the asphericity of the corresponding morphism of
topoi. Indeed, the latter just means that for any base-change in
\Cat{} of the very particular ``localization'' or ``induction'' type,
namely
\[B_{/b} \to B,\]
the corresponding map deduced by base change
\[ f_{/b}: A_{/b} \to B_{/b} \]
is a weak equivalence (or equivalently, that $A_{/b}$ is aspheric),
whereas $W$-asphericity of $f$ means that the same should hold for
\emph{any} base-change $B'\to B$ in \Cat, or equivalently, that for
\emph{any} such base-change, with $B'$ having a final element
moreover, the corresponding category $A'=A\times_B B'$ is aspheric. To
keep this distinction in mind, and because the weaker notion is quite
important and deserves a name definitely, I will refer to this notion
by saying $f$ is \emph{weakly $W$-aspheric} (returning to the case of
a general $\scrW\subset\Fl(\Cat)$) if for any base change of the
particular type $B_{/b}\to B$ above, the corresponding\pspage{162} map
$A_{/b}\to B_{/b}$ is in \scrW. We could express this in terms of the
morphism of topoi $\Ahat\to\Bhat$ by saying that the latter is
$W$-aspheric -- being understood that the choice of an ``absolute''
\scrW{} in \Cat, implies as usual a corresponding notion of
\scrW-equivalence for arbitrary morphisms of arbitrary topoi, in terms
of the corresponding morphism between the corresponding homotopy
types. (This extension to topoi of notions in \Cat{} should be made
quite explicit sooner or later, but visibly we do not need it yet for
the time being.) One relationship we have been constantly using, and
which is nearly tautological, comes from the case of a map
\[u:F\to G\]
in a category \Ahat, hence applying $i_A$ a map in \Cat
\[A_F \to A_G.\]
For this map to be weakly \scrW-aspheric, it is necessary and
sufficient that for any base-change in \Ahat{} of the particular type
\[ G'=A \to G\quad\text{with $a$ in $A$,}\]
the corresponding map
\[ F\times_G a \to a\]
in \Ahat{} be in $\scrW_A$ -- a condition which is satisfied of course
if $u$ is $\scrW_A$-aspheric. This last condition is also necessary,
if we assume that \scrW{} satisfies the standard property we've kept
using all the time in case of \Cat, namely that a map in \Cat{} that
is weakly $W$-aspheric, is in \scrW{} (condition \ref{it:61.f} on page
\ref{p:155}). Thus we get the
\begin{proposition}
  Assume that any map in \Cat{} which is weakly \scrW-aspheric is in
  \scrW, and let $A$ be any small category, $u:F\to G$ a map in
  \Ahat. Then $u$ is $\scrW_A$-aspheric \textup(and hence in
  $\scrW_A$\textup) if{f} $i_A(u) : A_{/F} \to A_{/G}$ is weakly
  \scrW-aspheric.
\end{proposition}

The assumption we just made on \scrW{} is of such constant use, that
we are counting it among those we are making once and for all upon
\scrW{} (which I still have to pass in review).

As for \scrW-aspheric maps in \Cat, this is a very strong notion
indeed when compared with \scrW-equivalence or even with weak
\scrW-asphericity. We did not have any use for it yet, but presumably
this notion will be of importance when it comes to a systematic study
of the internal properties of \Cat{} with respect to \scrW{} (which is
still in our program!).

Coming\pspage{163} back to a general $(M,W)$, we have defined earlier
a canonical homotopy structure $h_W$ on $M$, which we may call
``associated to $W$'' -- this is also the weak homotopy interval
structure on $M$, generated by intervals in $M$ which are $W$-aspheric
over $e$. This makes sense at least, provided in $M$ finite products
exist. If moreover $M$ satisfies the conditions \ref{it:51.F.b} to
\ref{it:51.F.d} of page \ref{p:123} concerning sums and connected
components of objects of $M$, we have defined (independently of $W$)
the \emph{canonical} homotopy structure $h_M$ in $M$, which may be
viewed as the weak homotopy interval structure generated by all
$0$-connected intervals in $M$. It still seems that in all cases we
are going to be interested in, we have the equality
\[h_W = h_M.\]
When speaking of homotopy notions in $M$ (such as $f$ and $g$ being
homotopic maps, written
\[ f\sim g,\]
or a map being a homotopism, or an object being contractible) it will
be understood (unless otherwise stated) that this refers to the
homotopy structure $h_W=h_M$. In case we should not care to impose
otherwise unneeded assumptions which will imply $h_W=h_M$, we'll be
careful when referring to homotopy notions, to say which structure we
are working with.

% 64
\hangsection{Review of properties of the ``basic localizer''
  \texorpdfstring{$\scrW_\Cat$}{W(Cat)}.}\label{sec:64}%
We recall that a set of arrows $W\subset\Fl(M)$ is called
\emph{saturated} if it satisfies the conditions:
\begin{enumerate}[label=\alph*)]
\item\label{it:64.a}
  Identities belong to $W$.
\item\label{it:64.b}
  If $f,g$ are maps and $fg$ exists, then if two among $f,g,gf$ are in
  $W$, so is the third.
\item\label{it:64.c}
  If $f:X\to Y$ and $g:Y\to X$ are such that $gf$ and $fg$ are in $W$,
  so are $f$ and $g$.
\end{enumerate}

On the other hand, \emph{strong saturation} for $W$ means that $W$ is
the set of arrows made invertible by the localization functor
\[M\to M_W=W^{-1}M,\]
or equivalently, that $W$ can be described as the set of arrows made
invertible by some functor $M\to M'$. The trouble with strong
saturation is that it is a condition which often is not easy to check
in concrete situations. This is so for instance for the notion of weak
equivalence in \Cat, and the numerous variants defined in terms of
cohomology. Therefore, we surely won't impose the strong saturation
condition\pspage{164} on \scrW{} (which we may call the ``\emph{basic
  localizer}'' in our modelizing story), but rather be happy if we can
prove strong saturation as a consequence of other formal properties of
\scrW, which are of constant use and may be readily checked in the
examples we have in mind.

Let's give finally a provisional list of those properties for a
``localizer'' \scrW.
\begin{description}
\item[\namedlabel{it:64.L1}{L~1)}] \textbf{(Saturation)}
  \scrW{} is saturated, i.e., satisfies conditions
  \ref{it:64.a}\ref{it:64.b}\ref{it:64.c} above.
\item[\namedlabel{it:64.L2}{L~2)}] \textbf{(Homotopy axiom)}
  $\Simplex_1$ is \scrW-aspheric over $e$, i.e., for
  any $C$ in \Cat, the projection
  \[\Simplex_1\times C\to C\]
  is in \scrW.
\item[\namedlabel{it:64.L3}{L~3)}] \textbf{(Final object axiom)}
  Any $C$ in \Cat{} which has a final object is
  in $\Cat(\scrW)$, i.e., $C\to e$ is in \scrW{} (or, as we will still
  say when working in \Cat, $C$ is \scrW-aspheric\footnote{For a
    justification of this terminology,see proposition on p.\
    \ref{p:167} below.}).
\item[\namedlabel{it:64.L3prime}{L~3')}] \textbf{(Interval axiom)}
  $\Simplex_1$ is \scrW-aspheric, i.e., $\Simplex_1\to e$ is in \scrW.
\item[\namedlabel{it:64.L4}{L~4)}] \textbf{(Localization axiom)}
  Any map $u:A\to B$ in \Cat{} which is weakly \scrW-aspheric (i.e.,
  the induced maps $A_{/b} \to B_{/b}$ are in \scrW) is in \scrW.
\item[\namedlabel{it:64.L5}{L~5)}] \textbf{(Fibration axiom)}
  If $f:X\to Y$ is a map in \Cat{} over an object $B$ of \Cat,
  such that $X$ and $Y$ are fiber categories over $B$ and $f$ is
  cartesian, and if moreover for any $b\in B$, the induced map on
  the fibers $f_b:X_b\to Y_b$ is in \scrW, then so is $f$.
\end{description}

These properties are all I have used so far, it seems, in the case
$\scrW=W_\Cat$ we have been working with till now, in order to develop
the theory of test categories and test functors, including ``weak''
and ``strong'' variants, and including too the generalized version of
the ``key result'' which is still waiting for getting into the
typewriter. Let's list at once the implications
\[ \text{\ref{it:64.L2}} \Rightarrow \text{\ref{it:64.L3}} \Rightarrow
\text{\ref{it:64.L3prime},}\]
and
\[\parbox{0.8\textwidth}{if \ref{it:64.L1} and \ref{it:64.L4} hold
  (saturation and localization axioms), then the homotopy axiom
  \ref{it:64.L2} is already implied by the final object axiom
  \ref{it:64.L3}.}\]

Thus, the set of conditions \ref{it:64.L1} to \ref{it:64.L4} (not
including the last one \ref{it:64.L5}, i.e., the fibration axiom) is
equivalent to the conjunction of \ref{it:64.L1} \ref{it:64.L3}
\ref{it:64.L4}. This set of conditions is of such a constant use, that
we'll assume it throughout, whenever there is a \scrW{} around:
\begin{definition}
  A\pspage{165} subset \scrW{} of $\Fl(\Cat)$ is called a \emph{basic localizer},
  if it satisfies the conditions \ref{it:64.L1}, \ref{it:64.L3},
  \ref{it:64.L4} above (saturation, final object and localization
  axioms), and hence also the homotopy axiom \ref{it:64.L2}.
\end{definition}

These conditions are enough, I quickly checked this night, in order to
validify all results developed so far on test categories, weak test
categories, strict test categories, weak test functors and test
functors with values in \Cat{} (cf.\ notably the review in par.\
\ref{sec:44}, pages \ref{p:79}--\ref{p:88}), provided in the case of
test functors we restrict to the case of loc.\ cit.\ when each of the
categories $i(a)$ has a final object. All this I believe is
justification enough for the definition above.

As for the fibration axiom \ref{it:64.L5}, this we have seen to be
needed (at least in the approach we got so far) for handling test
functors $i:A\to\Cat$, while no longer assuming the categories $i(a)$
to have final objects (which was felt to be a significant
generalization to carry through, in view of being able subsequently to
replace \Cat{} by more general modelizers). While still in the nature
of a stability requirement, this fibration axiom looks to me a great
deal stronger than the other axioms. Clearly \ref{it:64.L5}, together
with the very weak ``interval axiom'' \ref{it:64.L3prime}
($\Simplex_1\to e$ is in \scrW) implies the homotopy axiom. It can be
seen too that when joined with \ref{it:64.L1}, it implies the
localization axiom \ref{it:64.L4} (using the standard device of a
mapping-cone for a functor\ldots). Thus, a basic localizer satisfying
the fibration axiom \ref{it:64.L5} can be viewed also as a \scrW{}
satisfying the conditions
\[\text{\ref{it:64.L1} (saturation),\enspace\ref{it:64.L3prime}
  (interval axiom),\enspace\ref{it:64.L5} (fibration axiom).}\]
In the next section, after we will have stated the two key facts about
weak test functors and test functors, which both make use of
\ref{it:64.L5}, we'll presumably, for the rest of the work ahead
towards canonical modelizers, assume the fibration condition on the
basic localizer \scrW.

There are some other properties of weak equivalence $W_\Cat$ and its
manifold variants in terms of cohomology, which have not been listed
yet, and which surely will turn up still sooner or later. Maybe it's
too soon to line them up in a definite order, as their significance is
still somewhat vague and needs closer scrutiny. I'll just list those
which come to my mind, in a provisional order.
\begin{description}
\item[\namedlabel{it:64.La}{L~a)}]
  $0$-admissibility of \scrW, namely $f\in\scrW$ implies $\piz(f)$ bijective.
\end{description}

This condition, together with the homotopy axiom \ref{it:64.L2}, will
imply\pspage{166}
\begin{equation}
  \label{eq:64.1}
  h_\scrW = h_\Cat\quad\parbox[t]{.55\textwidth}{the canonical
    homotopy structure on \Cat{} defined in terms of the generating
    contractor $\Simplex_1$ in \Cat,}
  \tag{1}
\end{equation}
whereas the homotopy axiom alone, I mean without
\hyperref[it:64.La]{a)}, will imply only the inequality
\begin{equation}
  \label{eq:61.1prime}
  h_\Cat\le h_\scrW,\tag{1'}
\end{equation}
which is all we care for at present. The latter implies that two maps
in \Cat{} which are $\Simplex_1$-homotopic (i.e., belong to the same
connected component of $\bHom(X,Y)$ have the same image in the
localization $\scrW^{-1}\Cat=\HotW$, and that any map in \Cat{} which
is a $\Simplex_1$-homotopism is in \scrW, and even is \scrW-aspheric if
it is a ``homotopy retraction'' with respect to the
$\Simplex_1$-structure $h_\Cat$. However, in practical terms, even
without assuming \hyperref[it:64.La]{a)} expressly, we may consider
\eqref{eq:64.1} to be always satisfied. This amounts indeed to the
still weaker condition than \hyperref[it:64.La]{a)}
\begin{description}
\item[\namedlabel{it:64.Laprime}{L~a')}]
  If $C$ in \Cat{} is \scrW-aspheric over $e$, it is $0$-connected,
  i.e., non-empty and connected.
\end{description}

But if it were empty, it would follow that for any $X$ in \Cat,
$\varnothing\to X$ is in \scrW{} and hence \HotW{} is equivalent to the
final category. If $C$ is non-empty and disconnected, choosing two
connected components and one point in each to make $C$ into a weak
homotopy interval for $h_\scrW$, one easily gets that any two maps
$f,g:X\rightrightarrows Y$ in \Cat{} are $h_\scrW$-homotopic, hence
have the same image in \HotW, which again must be the final
category. Thus we get:
\begin{proposition}
  If \scrW{} satisfies \textup{\ref{it:64.L1}},
  \textup{\ref{it:64.L2}} \textup(e.g., \scrW{} a basic
  modelizer\textup), then we have equality \eqref{eq:64.1}, except in
  the case when \HotW{} equivalent to the final category.
\end{proposition}

This latter case isn't too interesting one will agree. Thus, we would
easily assume \eqref{eq:64.1}, i.e.,
\hyperref[it:64.Laprime]{a')}. But the slightly stronger condition
\hyperref[it:64.La]{a)} seems hard to discard; even if we have not
made any use of it so far, one sees hardly of which use a category of
localized homotopy types \HotW{} could possibly be, if one is not even
able to define the \piz-functor on it! Thus, presumably we'll have to
add this condition, and maybe even more, in order to feel \scrW{}
deserves the name of a ``basic localizer''\ldots Among other formal
properties which still need clarification, even in the case of
$W_\Cat$, there is the question of exactness properties of the
canonical functor
\begin{equation}
  \label{eq:64.2}
  \Cat\to\scrW^{-1}\Cat=\HotW.\tag{2}
\end{equation}
I am thinking particularly of the following
\begin{description}
\item[\namedlabel{it:64.Lb}{L~b)}]
  The\pspage{167} functor \eqref{eq:64.2} commutes with finite sums,
\end{description}
possibly even with infinite ones, which should be closely related to
property \hyperref[it:64.La]{a)}, and
\begin{description}
\item[\namedlabel{it:64.Lc}{L~c)}]
  The functor \eqref{eq:64.2} commutes with finite products
\end{description}
(maybe even with infinite ones, under suitable assumptions).

The following property, due to Quillen for weak equivalences, is used
in order to prove for these (and the cohomological analogs) the
fibration axiom \ref{it:64.L5} (what we get directly is the case of
cofibrations, as a matter of fact -- cf.\ prop.\ page \ref{p:97}):
\begin{description}
\item[\namedlabel{it:64.Ld}{L~d)}]
  If $f:C\to C'$ is in \scrW, so is $f\op:C\op\to(C')\op$ for the dual
  categories.
\end{description}

This is about all I have in mind at present, as far as further
properties of a \scrW{} is concerned. The property
\hyperref[it:64.Lc]{c)} however brings to mind the natural (weaker)
condition that the cartesian product of two maps in \scrW{} should
equally be in \scrW. The argument in the beginning of par.\
\ref{sec:40} (p.\ \ref{p:69}) carries over here, and we get:
\begin{proposition}
  Let \scrW{} be a basic localizer, and $C$ in \Cat{} such that $C$ is
  \scrW-aspheric, i.e., $C\to e$ is in \scrW, then $C$ is even
  \scrW-aspheric over $e$, i.e., for any $A$ in $C$, $C\times A\to A$
  is in \scrW.
\end{proposition}

However, the proof given for the more general statement we have in
mind (of proposition on page \ref{p:69}) does not carry over using
only the localization axiom in the form \ref{it:64.L4} it was stated
above, as far as I can see. This suggests a stronger version
\namedlabel{it:64.L4prime}{L~4')} of \ref{it:64.L4} which we may have
to use eventually, relative to a commutative triangle in \Cat{} as on
page \ref{p:70}
\[\begin{tikzcd}[baseline=(O.base),column sep=tiny,row sep=small]
  P' \ar[dr]\ar[rr,"F"] & & P\ar[dl] \\ & |[alias=O]| C &
\end{tikzcd},\]
when assuming that the induced maps (for arbitrary $c$ in $C$)
\[F_{/c}: P'_{/c}\to P_{/c}\]
are in \scrW, to deduce that $F$ itself is in \scrW. However, this
``strong localization axiom'' is a consequence (as is the weaker one
\ref{it:64.L4}) of the fibration axiom \ref{it:64.L5}, which implies
also directly the property we have in mind, namely
\[\text{$f,g\in\scrW$ implies $f\times g\in\scrW$.}\]

To\pspage{168} come to an end of this long terminological and
notational digression, I'll have to say a word still about test
categories, modelizers, and test functors. We surely want to use
freely the terminology introduced so far, while we were working with
ordinary weak equivalences, in the more general setting when a basic
localizer \scrW{} is given beforehand. As long as there is only one
\scrW{} around, which will be used systematically in all our
constructions, we'll just use the previous terminology, being
understood that a ``modelizer'' say will mean a ``\scrW-modelizer'',
namely a category $M$ endowed with a saturated $W\subset\Fl(M)$, such
that $W^{-1}M$ is equivalent some way or other to
$\scrW^{-1}\Cat$. The latter category, however, I dare not just
designate as \Hot, as this notation has been associated to the very
specific situation of just ordinary homotopy types, therefore I'll
always write \HotW{} instead, as a reminder of \scrW{} after all! If
at a later moment it should turn out that we'll have to work with more
than one \scrW{} (for instance, to compare the \scrW-theory to the
ordinary $W_\Cat$-theory), we will of course have to be careful and
reintroduce \scrW{} in our wording, to qualify all notions dependent
on the choice of \scrW.

\bigbreak
\noindent\hfill\ondate{7.4.}\par

% 65
\hangsection[Still another review of the test notions (relative to
given \dots]{Still another review of the test notions
  \texorpdfstring{\textup(}{(}relative to given basic localizer
  \texorpdfstring{\scrW\textup)}{W)}.}\label{sec:65}%
It has been over a week now and about eighty pages typing, since I
realized the need for looking at more general test functors than
before and hit upon how to handle them, that I am grinding stubbornly
through generalities unending on homotopy notions. The grinding is a
way of mine to become familiar with a substance, and at the same time
getting aloof of it climbing up, sweatingly maybe, to earn a birds-eye
view of a landscape and maybe, who knows, in the end start a-flying in
it, wholly at ease\ldots I am not there yet! The least however one
should expect, is that the test story should now go through very
smoothly. As I have been losing contact lately with test categories
and test functors, I feel it'll be worth while to make still another
review of these notions, leading up to the key result I have been
after all that time. It will be a way both to gain perspective, and
check if the grinding has been efficient indeed\ldots

If
\[u:M\to M'\]
is a functor between categories endowed each with a saturated set of
arrows, $W$ and $W'$ say, we'll say $u$ is ``\emph{model preserving}''
(with respect to $(W,W')$ if it satisfies the conditions:
\begin{enumerate}[label=\alph*)]
\item\label{it:65.a}
  $W=u^{-1}(W')$\pspage{169} (hence the functor $\overline
  u:W^{-1}M\to (W')^{-1}M'$ exists),
\item\label{it:65.b}
  the functor $\overline u$ is an equivalence.
\end{enumerate}

We do not assume beforehand that $(M,W)$, $(M',W')$ are modelizers
(with respect to a given basic localizer $\scrW\subset\Fl(\Cat)$), but
in the cases I have in mind, we'll know beforehand at least one of the
pairs to be a modelizer, and it will follow the other is one too.

In all what follows, a basic localizer \scrW{} is given once and for
all (see definition on page \ref{p:165}). For the two main results
below on the mere general test functors, we'll have to assume \scrW{}
satisfies the fibration axiom \ref{it:64.L5} (page \ref{p:164}). We
are going to work with a fixed small category $A$, without any other
preliminary assumptions upon $A$, all assumptions that may be needed
later will be stated in due course. Recall that $A$ can be considered
as an object of \Cat, and we'll say $A$ is \scrW-aspheric if $A\to e$
is in \scrW, which implies that $A\to e$ is even \scrW-aspheric, i.e.,
$A\times C\to C$ is in \scrW{} for any $C$ in \Cat. More generally, if
$F$ is in \Ahat, we'll call $F$ \scrW-aspheric if the category
$A_{/F}$ is \scrW-aspheric. Thus, to say $A$ is \scrW-aspheric just
means that the final object $e_\Ahat$ of \Ahat{} is \scrW-aspheric.

We'll constantly be using the canonical functor
\begin{equation}
  \label{eq:65.1}
  i_A:\Ahat\to\Cat, \quad F\mapsto A_{/F},\tag{1}
\end{equation}
and its right adjoint
\begin{equation}
  \label{eq:65.2}
  j_A = i_A^*:\Cat\to\Ahat, \quad C\mapsto (a\mapsto
  \Hom(A_{/a},C)).\tag{2}
\end{equation}
The category \Ahat{} will always be viewed as endowed with the
saturated set of maps
\begin{equation}
  \label{eq:65.3}
  \scrW_A = i_A^{-1}(\scrW).\tag{3}
\end{equation}
This gives rise to the notions of $\scrW_A$-aspheric map in \Ahat, and
of an object $F$ of \Ahat{} being $\scrW_A$-aspheric over $e_\Ahat$,
namely $F\to e_\Ahat$ being $\scrW_A$-aspheric, which means that
$F\times G\to G$ is in $\scrW_A$ for any $G$ in \Ahat, which by
definition of \scrWA{} means that for any $G$, the map
\[A_{/F\times G} \to A_{/F}\]
in \Cat{} is in \scrW. Using the localization axiom on \scrW, one sees
that it is enough to check this for $G$ and object $a$ of $A$, in
which case $A_{/G}=A_{/a}$ has a final object and hence is
\scrW-aspheric by \ref{it:64.L3}, and the condition amounts to
$A_{/F\times a}$ being \scrW-aspheric, i.e. (with the terminology
introduced above), that $F\times a$ is \scrW-aspheric. Thus, an object
$F$ of \Ahat\pspage{170} is \scrWA-aspheric over $e_\Ahat$ if{f} for
any $a$ in $A$, $F\times a$ is \scrW-aspheric. We should beware that
for general $A$, this does not imply $F$ is \scrW-aspheric, nor is it
implies by it. We should remember that \scrW-asphericity of $F$ is an
``absolute notion'', namely is a property of the induced
\emph{category} $A_{/F}$ or equivalently, of the induced topos
$\Ahat_{/F}\simeq(A_{/F})\uphat$, whereas \scrWA-asphericity of $F$
over $e_\Ahat$ is a relative notion for the \emph{map} of categories
\[A_{/F} \to A\]
or equivalently, for the map of topoi $\Ahat_{/F}\to\Ahat$ (the
localization map with respect to the object $F$ of the topos
\Ahat). More generally, for a map $F\to G$ in \Ahat, the property for
this map of being \scrWA-aspheric is a property for the corresponding
map in \Cat
\[A_{/F} \to A_{/G},\]
namely the property we called \emph{weak \scrW-asphericity} yesterday
(page \ref{p:161}), as we stated then in the prop.\ page
\ref{p:162}. An equivalent way of expressing this is by saying that
for $F\to G$ to be \scrWA-aspheric, i.e., to be ``universally in
\scrWA'', it is enough to check this for base changes $G'\to G$ with
$G'$ an object $a$ in $A$, namely that the corresponding map
\[F\times_Ga\to a\]
in \Ahat{} should be in \scrWA (for any $a$ in $A$ and map $a\to G$),
which amounts to saying that $F\times_Ga$ is \scrW-aspheric for any
$a$ in $A$ and map $a\to G$.

With notations and terminology quite clear in mind, we may start
retelling once again the test category story!

% A
\subsection{Total asphericity.}\label{subsec:65.A}
Before starting, just one important pre-test notion to recall, namely
total asphericity, summed up in the
\begin{propositionnum}\label{prop:65.1}
  The following conditions on $A$ are equivalent:
  \begin{description}
  \item[\namedlabel{it:65.A.i}{(i)}]
    The product in \Ahat{} of any two objects of $A$ is
    \scrW-aspheric.
  \item[\namedlabel{it:65.A.iprime}{(i')}]
    Every object in $A$ is \scrWA-aspheric over the final element $e_\Ahat$.
  \item[\namedlabel{it:65.A.ii}{(ii)}]
    The product of two \scrW-aspheric objects is again \scrW-aspheric.
  \item[\namedlabel{it:65.A.iiprime}{(ii')}]
    Any \scrW-aspheric object of \Ahat{} is \scrWA-aspheric over $e_\Ahat$.
  \end{description}
\end{propositionnum}

This is just a tautology, in terms of what was just said. Condition
\ref{it:65.A.i} is just the old condition \ref{it:31.T2} on test
categories\ldots
\begin{definitionnum}\label{def:65.1}
  If $A$ satisfies these conditions and moreover $A$ is
  \scrW-aspheric, \Ahat is called \emph{totally} \scrW-aspheric.
\end{definitionnum}
\begin{remark}
  In all cases when we have met with totally aspheric \Ahat, this
  condition \ref{it:65.A.i} was checked easily, because we were in one
  of\pspage{171} the two following cases:
  \begin{enumerate}[label=\alph*)]
  \item\label{it:65.A.a} $A$ stable under binary products.
  \item\label{it:65.A.b} The objects of $A$ are \emph{contractible}
    for the homotopy structure $h_\scrWA$ of \Ahat{} associated to \scrWA.
  \end{enumerate}
\end{remark}

In case \ref{it:65.A.b}, in the cases we've met, for checking
contractibility we even could get away with a homotopy interval
$\bI=(I,\delta_0,\delta_1)$ which is in $A$, namely we got
\bI-contractibility for all elements of $A$, and hence for the
products. All we've got to check then, to imply
$h_\scrWA$-contractibility of the objects $a\times b$, and hence their
\scrW-asphericity, is that $I$ itself is \scrWA-aspheric over
$e_\Ahat$, namely the products $I\times a$ are \scrW-aspheric. This
now has to be checked indeed some way or other -- I don't see any
general homotopy trick to reduce the checking still more. In case when
$A=\Simplex$ (standard simplices) say, and while still working with
usual weak equivalences $W_\Cat$, we checked asphericity of the
products $\Simplex_1\times\Simplex_n$ by using a Mayer-Vietories argument,
each product being viewed as obtained by gluing together a bunch of
\emph{representable} subobjects, which are necessarily \scrW-aspheric
therefore. The argument will go through for general \scrW, if we
assume \scrW{} satisfies the following condition, which we add to the
provisional list made yesterday (pages \ref{p:166}--\ref{p:167}) of
extra conditions which we may have to introduce for a basic localizer:
\begin{description}
\item[\namedlabel{it:65.Le}{L~e)}] \textbf{(Mayer-Vietoris axiom)}
  Let $C$ be in \Cat, let $C',C''$ be two full subcategories which are
  cribles (if it contains $a$ in $C$ and if $b\to a$, it contains
  $b$), and such that $\Ob C=\Ob C' \sor \Ob C''$. Assume $C',C''$ and
  $C' \sand C''$ are \scrW-aspheric, then so is $C$.
\end{description}

This condition of course is satisfied whenever \scrW{} is described in
terms of cohomological conditions, as envisioned yesterday (page
\ref{p:158}). We could elaborate on it and develop in this direction a
lot more encompassing conditions (``of \v Cech type'' we could say),
which will be satisfied by all such cohomologically defined basic
localizers. It would be fun to work out a set of ``minimal''
conditions such as \ref{it:65.Le} above, which would be enough to
imply all \v Cech-type conditions on a basic localizer. At first sight,
it isn't even obvious that \ref{it:65.Le} say isn't a consequence of
just the general conditions \ref{it:64.L1} to \ref{it:64.L4} on \scrW,
plus perhaps the fibration axiom \ref{it:64.L5} which looks very
strong. As long as we don't have any other example of basic
localizers than in terms of cohomology, it will be hard to tell!

% B
\subsection{Weak \texorpdfstring{\scrW}{W}-test categories.}
\label{subsec:65.B}

\begin{definitionnum}\label{def:65.2}
  The\pspage{172} category $A$ is a weak \scrW-test category if it
  satisfies the conditions
  \begin{enumerate}[label=\alph*)]
  \item\label{it:65.B.a}
    $A$ is \scrW-aspheric.
  \item\label{it:65.B.b}
    The functor $i_A^*:\Cat\to\Ahat$ is model-preserving, i.e.,
    \begin{enumerate}[label=b\textsubscript{\arabic*})]
    \item\label{it:65.B.b1}
      $\scrW=(i_A^*)^{-1}(\scrWA)$ ($=f_A^{-1}(\scrW)$, where
      $f_A=i_Ai_A^*:\Cat\to\Cat$),
    \item\label{it:65.B.b2}
      the induced functor
      \[ \overline i_A^* : \HotW \eqdef \scrW^{-1}\Cat\to
      \scrWA^{-1}\Ahat\]
      is an equivalence.
    \end{enumerate}
  \end{enumerate}
\end{definitionnum}
\begin{propositionnum}\label{prop:65.2}
  The following conditions on $A$ are equivalent:
  \begin{description}
  \item[\namedlabel{it:65.B.i}{(i)}]
    $A$ is a weak \scrW-test category.
  \item[\namedlabel{it:65.B.ii}{(ii)}]
    The functors $i_A^*$ and $i_A$ are both model-preserving, the
    induced functors
    \[\begin{tikzcd}[cramped]
      \HotW\ar[r,bend left=10] & \scrWA^{-1}\Ahat \ar[l,bend left=10]
    \end{tikzcd}\]
    are equivalences quasi-inverse of each other, with adjunction
    morphisms in \HotW{} and in $\HotA\eqdef\scrWA^{-1}\Ahat$ deduced
    from the adjunction morphisms for the pair of adjoint functors $i_A,i_A^*$.
  \item[\namedlabel{it:65.B.iii}{(iii)}]
    The functor $i_A^*$ transforms maps in \scrW{} into maps in \scrWA
    \textup(i.e., $f_A=i_Ai_A^*$ transforms maps in \scrW{} into maps
    in \scrW\textup), and moreover $A$ is \scrW-aspheric.
  \item[\namedlabel{it:65.B.iiiprime}{(iii')}]
    Same as in \textup{\ref{it:65.B.iii}}, but restricting to maps
    $C\to e$, where $C$ in \Cat{} has a final object.
  \item[\namedlabel{it:65.B.iv}{(iv)}]
    The categories $f_A(C)=A_{/i_A^*(C)}$, where $C$ in \Cat{} has a
    final object, are \scrW-aspheric.
  \end{description}
\end{propositionnum}

The obvious implications are
\[ \text{\ref{it:65.B.ii}}
\Rightarrow \text{\ref{it:65.B.i}}
\Rightarrow \text{\ref{it:65.B.iii}}
\Rightarrow \text{\ref{it:65.B.iiiprime}}
\Rightarrow \text{\ref{it:65.B.iv}}\]
and the proof of \ref{it:65.B.iv} $\Rightarrow$ \ref{it:65.B.ii}
follows from an easy weak asphericity argument and general non-sense
on adjoint functors and localization (cf.\ page \ref{p:35} and prop.\
on page \ref{p:38}).
\begin{remark}
  In case \scrW{} is strongly saturated, and hence $A$ \scrW-aspheric
  just means that its image in \HotW{} is a final object, the
  condition of \scrW-asphericity of $A$ in \ref{it:65.B.iii} or in
  def.\ 2 can be restated, by saying that the endomorphism $\overline
  f_A$ of \HotW{} induced by $f_A$ transforms final object into final
  object -- which is a lot weaker than being an equivalence!
\end{remark}

% C
\subsection{\texorpdfstring{\scrW}{W}-test categories.}
\label{subsec:65.C}
\begin{definitionnum}\label{def:65.3}
  The\pspage{173} category $A$ is a \emph{\scrW-test category} if it
  is a weak \scrW-test category, and if the localized categories
  $A_{/a}$ for $a$ in $A$ are equally weak \scrW-test categories. We
  say $A$ is a \emph{local \scrW-test category} if the localized
  categories $A_{/a}$ are weak \scrW-test categories.
\end{definitionnum}

Clearly, $A$ is a \scrW-test category if{f} if is a local \scrW-test
category, and moreover $A$ is \scrW-aspheric (as the categories
$A_{/a}$ are \scrW-aspheric by \ref{it:64.L3}). Also, $A$ is a local
\scrW-test category if{f} the functors $i_{A_{/a}}^*$ (for $a$ in $A$)
are model preserving.
\begin{propositionnum}\label{prop:65.3}
  The following conditions on $A$ are equivalent:
  \begin{enumerate}[label=(\roman*),font=\normalfont]
  \item\label{it:65.C.i}
    $A$ is a local \scrW-test category.
  \item\label{it:65.C.ii}
    The Lawvere element
    \[L_\Ahat = i_A^*(\Simplex_1)\]
    in \Ahat{} is \scrWA-aspheric over $e_\Ahat$, i.e., the products
    $a\times L_\Ahat$ for $a$ in $A$ are all \scrW-aspheric.
  \item\label{it:65.C.iii}
    There exists a separated interval $\bI=(I,\delta_0,\delta_1)$ in
    \Ahat{} \textup(i.e., an object endowed with two sections such
    that $\Ker(\delta_0,\delta_1)=\varnothing_\Ahat$\textup), such that
    $I$ be \scrWA-aspheric over $e_\Ahat$, i.e., all products $a\times
    I$ are \scrW-aspheric.
  \end{enumerate}
\end{propositionnum}

The obvious implications here are
\[ \text{\ref{it:65.C.i}}
\Rightarrow \text{\ref{it:65.C.ii}}
\Rightarrow \text{\ref{it:65.C.iii}}.\]
on the other hand \ref{it:65.C.iii} $\Rightarrow$ \ref{it:65.C.ii} by
the homotopy interval comparison lemma (p.\ \ref{p:60}), and finally 
\ref{it:65.C.i} $\Rightarrow$ \ref{it:65.C.i} by the criterion for
weak \scrW-test categories of prop.\ \ref{prop:65.2} \ref{it:65.B.iv}, using an
immediate homotopy argument (cf.\ page \ref{p:62}).
\begin{corollarynum}\label{cor:65.1}
  $A$ is a \scrW-test category if{f} it is \scrW-aspheric and
  satisfies \textup{\ref{it:65.C.ii}} or \textup{\ref{it:65.C.iii}} of
  proposition \textup{\ref{prop:65.3}} above.
\end{corollarynum}
\begin{remark}
  In the important case when \Ahat{} is totally \scrW-aspheric (cf.\
  prop.\ \ref{prop:65.1}, the asphericity condition on $L_\Ahat$ or on $I$ in
  prop.\ \ref{prop:65.3} is equivalent to just \scrW-asphericity of $L_\Ahat$ resp.\
  of $I$. In case \Ahat{} is even ``strictly totally \scrW-aspheric'',
  i.e., if moreover every ``non-empty'' object in \Ahat{} admits a
  section, then we've seen that $h_\scrWA=h_\Ahat$ (prop.\ page
  \ref{p:149}, which carries over to a general \scrW{} satisfying
  \ref{it:64.Laprime} of page \ref{p:166}, i.e., provided \HotW{}
  isn't equivalent to the final category, which case we may discard!),
  then condition \ref{it:65.C.ii} just means that the contractor
  $L_\Ahat$ is $0$-connected -- a condition which does not depend upon
  the choice of \scrW.
\end{remark}

% D
\subsection{Strict \texorpdfstring{\scrW}{W}-test categories.}
\label{subsec:65.D}
\begin{propositionnum}\label{prop:65.4}
  The\pspage{174} following conditions on $A$ are equivalent:
  \begin{description}
  \item[\namedlabel{it:65.D.i}{(i)}]
    Both functors $i_A$ and $i_A^*$ are model preserving, moreover
    $i_A$ commutes to finite products ``modulo \scrW''.
  \item[\namedlabel{it:65.D.ii}{(ii)}]
    $A$ is a test category and \Ahat{} is totally \scrW-aspheric.
  \item[\namedlabel{it:65.D.iiprime}{(ii')}]
    $A$ is a weak test category and \Ahat{} is totally \scrW-aspheric.
  \item[\namedlabel{it:65.D.iii}{(iii)}]
    $A$ satisfies conditions \textup{\ref{it:31.T1}}
    \textup{\ref{it:31.T2}} \textup{\ref{it:31.T3}} of page
    \textup{\ref{p:39}}, with ``aspheric'' replaced by ``\scrW-aspheric''.
  \end{description}
\end{propositionnum}

This is not much more than a tautology in terms of what we have seen
before, as we'll get the obvious implications
\[ \text{\ref{it:65.D.i}}
\Rightarrow \text{\ref{it:65.D.iii}}
\Rightarrow \text{\ref{it:65.D.ii}}
\Rightarrow \text{\ref{it:65.D.iiprime}}
\Rightarrow \text{\ref{it:65.D.i}}.\]
\begin{definitionnum}\label{def:65.4}
  If $A$ satisfies the conditions above, it is called a \emph{strict
    \scrW-test category}.
\end{definitionnum}
\begin{remarks}
  1)\enspace When we know that the canonical functor from \Cat{} to the
  localization \HotW{} commutes with binary products, then the
  exactness property mod \scrW{} in \ref{it:65.D.i} implies that the
  same holds for the canonical functor from \Ahat{} to its
  localization \HotA, and conversely if \scrW{} is known to be
  saturated.

  2)\enspace In the case $\scrW=W_\Cat$ we've seen that condition
  \ref{it:31.T2} implies \ref{it:31.T1}, i.e., the conditions of
  prop.\ \ref{prop:65.1} imply $A$ is \scrW-aspheric, i.e.,
  \Ahat{} is totally \scrW-aspheric. The argument works for any
  \scrW{} defined by cohomological conditions of the type considered
  in yesterday's notes. To have it work for more general \scrW, we
  would have to introduce some \v Cech-type requirement on \scrW,
  compare page \ref{p:171}.

  3)\enspace In the statement of the theorem page \ref{p:46}, similar
  to the proposition above, in \ref{it:33.i} no assumption is made on
  $i_A^*=j_A$ -- which I believe was an omission by hastiness -- it is
  by no means clear to me that we could dispense with it, and get away
  with an assumption on $i_A$ alone.
\end{remarks}

% E
\subsection{Weak \texorpdfstring{\scrW}{W}-test functors and
  \texorpdfstring{\scrW}{W}-test functors.}\label{subsec:65.E}

Let\pspage{175}
\[(M,W) , \quad W\subset\Fl(M)\]
be a category endowed with a saturated set of arrows $W$, and
\begin{equation}
  \label{eq:65.4}
  i:A\to M\tag{4}
\end{equation}
a functor, hence a corresponding functor
\begin{equation}
  \label{eq:65.5}
  i^*: M \to \Ahat, \quad
  X\mapsto i^*(X)=(a\mapsto \Hom(i(a),X)).\tag{5}
\end{equation}
\begin{definitionnum}\label{def:65.5}
  The functor $i$ is called a \emph{weak \scrW-test functor}
  (with respect to the given $W\subset\Fl(M)$) if $A$ is
  a weak \scrW-test category and the functor $i^*$ is model-preserving
  (for $W$ and \scrWA), i.e., if $A$ satisfies the three conditions:
  \begin{enumerate}[label=\alph*)]
  \item\label{it:65.E.a}
    $i^*$ is model preserving,
  \item\label{it:65.E.b}
    $i_A^*: \Cat\to\Ahat$ is model preserving,
  \item\label{it:65.E.c}
    $A$ is \scrW-aspheric.
  \end{enumerate}
\end{definitionnum}

The conditions \ref{it:65.E.b} and \ref{it:65.E.c}, namely that $A$ be
a weak \scrW-test category, do not depend of course upon $M$, and it
may seem strange in the definition not to have simply asked beforehand
that $A$ satisfy this preliminary condition -- i.e., reduce to the
case when we start with a weak \scrW-test category $A$. The reason for
not doing so is that we'll find below handy criteria for all three
conditions to hold, without assuming beforehand $A$ to be a weak
\scrW-test category.

As \ref{it:65.E.b} and \ref{it:65.E.c} imply that
\[ i_A: \Ahat\to\Cat\]
is model-preserving too, condition \ref{it:65.E.a} above can be
replaced by the condition
\begin{enumerate}[label=\alph*')]
\item\label{it:65.E.aprime}
  The composition
  \[f_i=i_Ai^*: M\to\Cat\]
  is model-preserving (for the pair $W,\scrW$).
\end{enumerate}

Of course, as conditions \ref{it:65.E.b}, \ref{it:65.E.c} imply that
$(\Ahat,\scrWA)$ is a modelizer (with respect to \scrW), the condition
\ref{it:65.E.a} will imply $(M,W)$ is a modelizer too.

We recall the condition for $i^*$ to be model-preserving decomposes
into two:
\begin{enumerate}[label=a\textsubscript{\arabic*})]
\item\label{it:65.E.a1}
  $W=(i^*)^{-1}(\scrWA)\quad (=f_i^{-1}(\scrW))$,
\item\label{it:65.E.a2}
  The\pspage{176} functor $\overline{i^*}$ induced by $i^*$ on the
  localizations (which exists because of \ref{it:65.E.a1})
  \[W^{-1}M \to \HotA\eqdef \scrWA^{-1}\Ahat\]
  is an equivalence.
\end{enumerate}
\begin{definitionnum}\label{def:65.6}
  The functor $i$ is called a \emph{\scrW-test functor} if this
  functor \emph{and} the induced functors $i_{/a}:A_{/a}\to M$ (for
  $a$ in $A$) are weak \scrW-test functors.
\end{definitionnum}

In view of the definition \ref{def:65.3}, this amounts to the two
conditions:
\begin{enumerate}[label=\alph*)]
\item\label{cond:65.E.a}
  $A$ is a \scrW-test category, i.e., the functors $i_A^*$ and
  $i_{A_{/a}}^*$ are model-preserving and $A$ is \scrW-aspheric,
\item\label{cond:65.E.b}
  the functors $i^*$ and $(i_{/a})^*$ from $M$ into the categories
  \Ahat{} and $(A_{/a})\uphat\to\Ahat_{/a}$ are model-preserving (for
  $W$ and \scrWA{} resp.\ $\scrW_{A_{/a}}$).
\end{enumerate}
\begin{example}
  Consider the canonical functor induced by $i_A$
  \[i_A^0: A\to \Cat, \quad\text{\Cat{} endowed with \scrW,}\]
  this functor is a weak \scrW-test functor (resp.\ a \scrW-test
  functor) if{f} $A$ is a weak \scrW-test category (resp.\ a
  \scrW-test category).
\end{example}

These two definitions are pretty formal indeed. Their justification is
mainly in the two theorems below.

\emph{We assume from now on that the basic localizer \scrW{} satisfies
  the fibration axiom} \ref{it:64.L5} of page \ref{p:164}. Also, we
recall that an object $X$ in \Cat{} is \emph{contractible} (for the
canonical homotopy structure of \Cat) if{f} $X$ is non-empty and the
category $\bHom(X,X)$ is connected -- indeed it is enough even that
$\id_X$ belong to the same connected component as some \emph{constant}
map from $X$ into itself. This condition is satisfied for instance if
$X$ has a final or an initial object.
\begin{theoremnum}\label{thm:65.1}
  We assume that $M=\Cat$, $W=\scrW$, i.e., we've got a functor
  \[i:A\to\Cat, \quad\text{\Cat{} endowed with \scrW,}\]
  and we assume that for any $a$ in $A$\kern1pt, $i(a)$ is contractible
  \textup(cf.\ above\textup), i.e., that $i$ factors through the full
  subcategory $\Cat_{\mathrm{cont}}$ of contractible objects of
  \Cat. The following conditions are equivalent:
  \begin{description}
  \item[\namedlabel{it:65.E.i}{(i)}]
    $i$ is a \scrW-test functor \textup(def.\ \textup{\ref{def:65.6})}.
  \item[\namedlabel{it:65.E.iprime}{(i')}]
    For any $a$ in $A$\kern1pt, the induced functor $i_{/a}:A_{/a}\to\Cat$ is
    a weak \scrW-test functor, and moreover $A$ is \scrW-aspheric.
  \item[\namedlabel{it:65.E.ii}{(ii)}]
    $i^*(\Simplex_1)$ is \scrWA-aspheric over $e_\Ahat$, i.e., the
    products $a\times i^*(\Simplex_1)$ in \Ahat{} are \scrW-aspheric,
    for any $a$ in $A$\kern1pt, and $A$ is \scrW-aspheric.
  \end{description}
\end{theoremnum}

The\pspage{177} obvious implications here are
\[ \text{\ref{it:65.E.i}}
\Rightarrow \text{\ref{it:65.E.iprime}}
\Rightarrow \text{\ref{it:65.E.ii}},\]
for the last implication we only make use, besides $A$ being
\scrW-aspheric, that the functors $(i_{/a})^*$ transform the
projection $\Simplex_1\to e$ in \Cat, which is in \scrW{} by
\ref{it:64.L3prime}, into a map in $\scrW_{A_{/a}}$, i.e., that the
corresponding map in \Cat
\[A_{/a\times i^*(\Simplex_1)} \to A_{/a}\]
be in \scrW, which by the final object axiom implying that $A_{/a}\to
e$ is in \scrW, amounts to demanding that the left-hand side is
\scrW-aspheric, i.e., $a\times i^*(\Simplex_1)$ \scrWA-aspheric.

So we are left with proving \ref{it:65.E.ii} $\Rightarrow$
\ref{it:65.E.i}. By the criterion \ref{it:65.C.iii} of prop.\
\ref{prop:65.3} we know already (assuming \ref{it:65.E.ii}) that $A$
is a local \scrW-test category, hence a \scrW-test category as $A$ is
\scrW-aspheric (cor.\ \ref{cor:65.1}); indeed we can use
$I=i^*(\Simplex_1)$ as a \scrWA-aspheric interval, using the two
canonical sections deduced from the canonical sections of
$\Simplex_1$. The fact that these are disjoint follows from the fact
that $i(a)$ non-empty for any $a$ in $A$ -- we did not yet have to use
the contractibility assumption on the categories $i(a)$. Thus, we are
reduced to proving that $i^*$ is model-preserving -- the same will
then hold for the functors $i_{/a}$ (as required in part
\ref{cond:65.E.b} in def.\ \ref{def:65.6}), as the assumption
\ref{it:65.E.ii} is clearly stable under restriction to the categories
$A_{/a}$. As we know already that $i_A$ is model-preserving (prop.\
\ref{prop:65.2} \ref{it:65.B.i} $\Rightarrow$ \ref{it:65.B.ii}), all
we have to do is to prove the composition $f_i=i_Ai^*$ is
model-preserving. But this was proved yesterday in the key lemma of
page \ref{p:156}. We're through!

\begin{remark}
  The presentation will be maybe a little more elegant, if we
  complement the definition of a \scrW-test functor by the definition
  of a \emph{local \scrW-test functor}, by which we mean that the
  induced functors $i_{/a}:A_{/a}\to M$ are weak test functors, period
  -- which means also that the following conditions hold:
  \begin{enumerate}[label=\alph*)]
  \item\label{it:65.E.rem.a}
    $A$ is a local \scrW-test category (def.\ \ref{def:65.3}), i.e.,
    the functors $(i_{/a})^*: M\to(A_{/a})\uphat$ are all
    model-preserving;
  \item\label{it:65.E.rem.b}
    the functors $(i_{/a})^*$ (for $a$ in $A$) are model-preserving.
  \end{enumerate}
\end{remark}

Thus, it is clear that if $i$ is a \scrW-test functor, it is a local
\scrW-test functor such that moreover $A$ is \scrW-aspheric. The
converse isn't clear in general, because it isn't clear that if $A$ is
a \scrW-test category and moreover all functors $(i_{/a})^*$ are
modelizing, then $i^*$ is equally modelizing. The criterion
\ref{it:65.E.iprime} of theorem \ref{thm:65.1} shows however that this
is so in the case when $(M,W)=(\Cat,\scrW)$, and when we assume
moreover the objects $i(a)$ contractible. We could now reformulate the
theorem\pspage{178} as a twofold statement:
\begin{corollary}
  Under the assumptions of theorem \ref{thm:65.1}, $i$ is a
  \emph{local \scrW-test functor} \textup(i.e., all functors
  $i_{/a}: A_{/a}\to\Cat$ are weak \scrW-test functors, or
  equivalently the functors $(i_{/a})^*$ and $(i_{A_{/a}})^*$ from
  $\Cat \to (A_{/a})\uphat$ are all model-preserving\textup) if{f}
  $i^*(\Simplex_1)$ is \scrWA-aspheric over $e_\Ahat$, i.e., the
  products $a\times i^*(\Simplex_1)$ in \Ahat{} are \scrW-aspheric. When
  this condition is satisfied, in order for $i$ to be a \scrW-test
  functor, namely for $i^*$ to be equally model-preserving, it is
  n.s.\ that $A$ be \scrW-aspheric.
\end{corollary}

% F
\subsection{\texorpdfstring{\scrW}{W}-test functors
  \texorpdfstring{$A\to\Cat$}{A->(Cat)} of strict
  \texorpdfstring{\scrW}{W}-test categories.}
\label{subsec:65.F}
Let again
\[i:A\to\Cat,\quad\text{\Cat{} endowed with \scrW,}\]
be a functor such that the objects $i(a)$ be contractible, we assume
now moreover that \Ahat{} is \emph{totally \scrW-aspheric} (def.\
\ref{def:65.1}), which implies $A$ is \scrW-aspheric. Thus, by the
corollary above $i$ is a test functor if{f} it is a local \scrW-test
functor, and by the criterion \ref{it:65.B.iv} of prop.\
\ref{prop:65.2} (with $C=\Simplex_1$) we see it amounts to the same that
$i$ be a weak \scrW-test functor. (Here we use the assumption of total
\scrW-asphericity, which implies that if $i^*(\Simplex_1)$ is
\scrW-aspheric, it is even \scrWA-aspheric over $e_\Ahat$.) Thus, the
three variants of the test-functor notion coincide in the present
case. With this in mind, we can now state what seems to me the main
result of our reflections so far, at any rate the most suggestive
reformulation of theorem \ref{thm:65.1} in the present case:
\begin{theoremnum}\label{thm:65.2}
  With the assumptions above \textup(\Ahat{} totally \scrW-aspheric
  and the objects $i(a)$ in \Cat{} contractible), the following
  conditions on the functor $i:A\to\Cat$ are equivalent:
  \begin{enumerate}[label=(\roman*),font=\normalfont]
  \item\label{it:65.F.i}
    $i$ is a \scrW-test functor.
  \item\label{it:65.F.ii}
    $i^*: \Cat\to\Ahat$ is model-preserving, i.e., for any map $f$ in
    \Cat, $f$ is in \scrW{} if{f} $i^*(f)$ is in \scrWA{}
    \textup(i.e., if{f} $i_Ai^*(f)$ is in \scrW\textup), and moreover
    the induced functor
    \[\overline{i^*}:\HotW\to\HotA\eqdef\scrWA^{-1}\Ahat\]
    is an equivalence.
  \item\label{it:65.F.iii}
    The functor above exists, i.e., $f$ in \scrW{} implies $i^*(f)$ in
    \scrWA, i.e., $i_Ai^*(f)$ in \scrW.
  \item\label{it:65.F.iv}
    The functor $i^*$ transforms \scrW-aspheric objects into
    \scrWA-objects \textup(i.e., the condition in
    \textup{\ref{it:65.F.iii}} is satisfied for maps $C\to e$ in \scrW\textup).
  \item\label{it:65.F.v}
    The functor $i^*$ transforms contractible objects of \Cat{}
    into\pspage{179} objects of \Ahat, contractible for the homotopy
    structure $h_\scrWA$ associated to \scrWA{} -- or equivalently,
    $i^*$ is a morphism of homotopy structures \textup(cf.\ definition
    on page \textup{\ref{p:134})}.
  \item\label{it:65.F.vi}
    The functor $i^*$ transforms the projection $\Simplex_1\to e$ into a
    map in \scrW, or equivalently \textup(as $A$ is
    \scrW-aspheric\textup) $i^*(\Simplex_1)$ is \scrW-aspheric.
  \end{enumerate}
\end{theoremnum}

We have the trivial implications
\[ \text{\ref{it:65.F.i}}
\Rightarrow \text{\ref{it:65.F.ii}}
\Rightarrow \text{\ref{it:65.F.iii}}
\Rightarrow \text{\ref{it:65.F.iv}}
\Rightarrow \text{\ref{it:65.F.vi}}
\Leftarrow \text{\ref{it:65.F.v}},\]
where the implication \ref{it:65.F.v} $\Rightarrow$ \ref{it:65.F.vi}
is in fact an equivalence, due to the fact that the contractibility
structure on \Cat{} is defined in terms of $\Simplex_1$ as a generating
contractor, and that the assumption $i^*(\Simplex_1)$ \scrW-aspheric
implies already that it is \scrWA-aspheric over $e_\Ahat$ (because $A$
is totally \scrWA-aspheric), and hence contractible as it is a
contractor and $h_\scrWA$ is defined in terms of ``weak homotopy
intervals'' which are \scrWA-aspheric over $e_\Ahat$. Thus, the only
delicate implication is \ref{it:65.F.vi} $\Rightarrow$
\ref{it:65.F.i}, which however follows from theorem \ref{thm:65.1}
\ref{it:65.E.ii} $\Rightarrow$ \ref{it:65.E.i}.

We got the longed-for ``key result'' in the end!

\bigbreak

\noindent\hfill\ondate{9.4.}\pspage{180}\par

% 66
\hangsection{Revising \texorpdfstring{\textup(}{(}and
  fixing?\texorpdfstring{\textup)}{)} terminology again.}
\label{sec:66}%
After writing down nicely, in the end, that long promised key result,
I thought the next thing would be to pull myself up by my bootstraps
getting the similar result first for test functors $A\to\Bhat$ with
values in an elementary modelizer \Bhat, and then for general
``canonical'' modelizers $(M,W)$. So I did a little scratchwork
pondering along those lines, before resuming the typewriter-engined
work. What then turned out, it seems, is that there wasn't any need at
all to pass through the particular case $M=\Cat$ and the somewhat
painstaking analysis of our three-step diagram on page
\ref{p:96}. Finally, the most useful result of all the eighty pages
grinding, since that point, is by no means the so-called key result,
as I anticipated -- the day after I finally wrote it down, it was
already looking rather ``\'etriqu\'e''\scrcomment{``\'etriqu\'e'' can be
  translated as ``narrow-minded'' here, I think.} -- why
all this fuss about the special case of test functors with values in
\Cat! The main result has been finally more psychological than
technical -- namely drawing attention, in the long last, to the key
role of contractible objects and, more specifically, of the
contractibility structure associated to modelizers $(M,W)$, suggesting
that the localizer $W$ should, conversely, be describable in terms of
the homotopy structure $h_W$. This was point \ref{it:56.d} in the
``provisional plan of work'' contemplated earlier this week (page
\ref{p:138}) -- by then I had already the feeling this approach via
\ref{it:56.d} would turn out to be the most ``expedient'' one -- but
it was by then next to impossible for me to keep pushing off still
more the approach \ref{it:56.b} via test functors with values in \Cat,
which I finally carried through. One point which wasn't wholly clear
yet that day, as it is now, is the crucial role played by the
circumstance that for the really nice modelizers (surely for those I'm
going to call ``canonical'' in the end), the associated homotopy
structure is indeed a \emph{contractibility} structure. Here, as so
often in mathematics (and even outside of mathematics\ldots), the main
thing to dig out and discover is where the emphasis belongs -- which
are the really essential facts or notions or features within a given
context, and which are accessory, namely, which will follow suit by
themselves. It took a while before I would listen to what the things I
was in were insistently telling me. It finally got through I feel, and
I believe that from this point on the whole modelizing story is going
to go through extremely smoothly.

Before starting with the work, just some retrospective, somewhat more
technical comments, afterthoughts rather I should say. First of all, I
am not so happy after all with the terminological review a few days
ago\pspage{181} (pages \ref{p:159}--\ref{p:163}), and notably the use
of the word ``aspheric'' in the generalization ``$W$-aspheric'' map
(in a category $M$ endowed with a saturated set of arrows $W$) --
which then practically obliged me, when working in \Cat, to call
``weakly aspheric'' a functor $C'\to C$ which spontaneously I surely
would like to call simply ``aspheric'' -- and as a matter of fact, it
turned out I couldn't force myself to add a ``weakly'' before as I
decreed I should -- or if I did, it was against a very strong feeling
of inappropriateness. That decree precisely is an excellent
illustration of loosing view of where the main emphasis belongs, which
I would like now to make very clear to myself.

In all this work the underlying motivation or inspiration is
geometrico-topological, and expressed technically quite accurately by
the notion of a topos and of maps (or ``morphisms'') of topoi, and the
wealth of geometric and algebraic intuitions which have developed
around these. One main point here is that topoi may be viewed as
\emph{the} natural common generalization of both topological spaces
(the conventional support for so-called topological intuition), and of
(small) categories, where the latter may be viewed as the ideal purely
algebraic objects carrying topological information, including all the
conventional homology and homotopy invariants. This being so, in a
context where working with small categories as ``spaces'', the main
emphasis in choice of terminology should surely be in stressing
throughout, through the very wording, the essential identity between
situations involving categories, and corresponding situations
involving topological spaces or topoi. Thus, it has been about twenty
years now that the needs for developing \'etale cohomology have told
me a handful of basic asphericity and acyclicity properties for a map
of topoi (which apparently have not yet been assimilated by
topologists, in the context of maps of topological spaces\ldots),
including the condition for such a map to be aspheric. This was
recalled earlier (page \ref{p:37}), and the corresponding notion for a
functor $f:C'\to C$ was introduced. The name ``\emph{aspheric map}''
of topoi, or of topological spaces, or of categories, is here a
perfectly suggestive one. As the notion itself is visibly a basic one,
there should be no question whatever to change the name and replace it
say by ``weakly aspheric'', whereas the notion is surely quite a
strong one, and doesn't deserve such minimizing qualification! There
is indeed a stronger notion, which in the context of topological
spaces or \'etale cohomology of schemes reduces to the previous one in
the particular case of a map which is supposed \emph{proper}. This
condition could be expressed by saying that for \emph{any} base-change
$Y'\to Y$\pspage{182} for the map $f:X\to Y$ (at least any base-change
within the given context, namely either spaces or schemes with \'etale
topology), the corresponding map
\[ f': X'=X\times_YY'\to Y'\]
is a weak equivalence, or what amounts to the same, that for any $Y'$
the corresponding map is aspheric. This property, if a name is needed
indeed, would properly be called ``\emph{universally
  aspheric}''. Thus, in \Cat{} a map $f:X\to Y$ will be called either
aspheric, or universally aspheric, when for any base-change of the
special type $Y_{/y}\to Y$, namely ``localization'' in the first case,
or any base-change whatever $Y'\to Y$ in the second case, the
corresponding map $f'$ is a weak equivalence. On the other hand, if
$Y$ is just the final object $e$ of \Cat, it turns out the two notions for
$X$ (being ``aspheric over $e$'' and being ``universally aspheric over
$e$'') coincide, and just mean that $X\to e$ is a weak equivalence. In
accordance with the use which has been prevalent for a long time in
the context of spaces, such an object will be call simply an
\emph{aspheric object} -- which means that the corresponding topos is
aspheric (namely has ``trivial'' cohomology invariants, and hence
trivial homotopy invariants of any kind\ldots).

In case the notion of weak equivalence is replaced by a basic
localizer $\scrW\subset\Fl(\Cat)$, there is no reason whatever to
change anything in this terminology -- except that, if need, we will
add the qualifying \scrW, and speak of \emph{\scrW-aspheric} or
\emph{universally \scrW-aspheric maps} in \Cat, as well of
\emph{\scrW-aspheric objects} of \Cat.

What about terminology for maps and objects within a category \Ahat?
Here the emphasis should be of course perfect coherence with the
terminology just used in \Cat. An object $F$ of \Ahat{} should always
be sensed in terms of the induced topos $\Ahat_{/F}\simeq
(A_{/F})\uphat$, or what amounts to the same, in terms of the
corresponding object $A_{/F}$ in \Cat, which will imply that
``\emph{$F$ is aspheric}'' cannot possibly mean anything else but
$A_{/F}$ is aspheric as an object of \Cat; the same if qualifying by a
\scrW{} -- $F$ is called \emph{\scrW-aspheric} if $A_{/F}$ is a
\scrW-aspheric object of \Cat. Similarly for maps -- thus $f: F\to G$
will be called a weak equivalence, if the corresponding map for the
induced topoi is a weak equivalence, or equivalently, if the
corresponding map in \Cat
\[A_{/F} \to A_{/G}\]
is a weak equivalence. When a \scrW{} is given, we would say instead
(if confusion may arise) that $f$ is a \emph{\scrW-equivalence}. The
map will be called\pspage{183} \emph{aspheric}, or
\emph{\scrW-aspheric}, if the corresponding map in \Cat{} is. It turns
out that (because of the localization axiom on \scrW) this is
equivalent with $f$ being ``universally a \scrW-equivalence'', i.e.,
$f$ being ``universally in \scrWA'', namely that for any base-change
$G'\to G$ in \Ahat, the corresponding map in \Ahat
\[f' : F\times_GG'\to G'\]
be in \scrWA, i.e., be a \scrW-equivalence. Of course, when this
condition is satisfied, then for any base change, $f'$ will be, not
only a \scrW-equivalence, but even \scrW-aspheric -- thus we can say
that $f$ is ``universally \scrW-aspheric'' -- where ``universally''
refers to \emph{base change in} \Ahat. This of course does not mean
(and here one has to be slightly cautious) that the corresponding map
in \Cat{} is universally \scrW-aspheric (which refers to arbitrary
\emph{base change in} \Cat). But this apparent incoherence is of no
practical importance as far as terminology goes, as the work
``\scrW-aspheric map in \Ahat'' is wholly adequate and sufficient for
naming the notion, without any need to replace it by the more
complicated and ambiguous name ``universally \scrW-aspheric'', which
therefore will never be used. We even could rule out the formal
incoherence, by using the words \scrWA-equivalence, \scrWA-aspheric
maps (which are even universally \scrWA-aspheric maps, without any
ambiguity any longer), as well as \scrWA-aspheric objects -- replacing
throughout \scrW{} by \scrWA. In practical terms, I think that when
working consistently with a single given \scrW, we'll soon enough drop
anyhow both \scrW{} and \scrWA{} in the terminology and notations!

A last point which deserves some caution, is that for general $A$,
there is no implication between the two asphericity properties of an
object $F$ of \Ahat, namely of $F$ being \scrW-aspheric (i.e., the
object $A_{/F}$ of \Cat{} being \scrW-aspheric, i.e., the map
\begin{equation}
  \label{eq:66.*}
  A_{/F} \to e\tag{*}
\end{equation}
in \Cat{} being in \scrW), and the property that $F\to e$ be
\scrW-aspheric, namely that map
\begin{equation}
  \label{eq:66.starstar}
  A_{/F} \to A_{/e}=A\tag{**}
\end{equation}
in \Cat{} being aspheric (which also means that the products $F\times
a$ for $a$ in $A$ are \scrW-aspheric objects of \Ahat, i.e., the
categories $A_{/F\times a}$ are \scrW-aspheric, i.e., the maps
\[ A_{/F\times a} \to e\]
in \Cat{} are \scrW-aspheric. A third related notion, weaker than the
last one\pspage{184} is the property that $F\to e$ be a
\scrW-equivalence, which also means that the map
\eqref{eq:66.starstar} in \Cat{} is a \scrW-equivalence, i.e., is in
\scrW. If $A$ is \scrW-aspheric, this third notion however reduces to
the first one, namely $F$ to be \scrW-aspheric.

\emph{These terminological conventions, in the all-important cases of
  \Cat{} and categories of the type \Ahat, should be viewed as the
  basic ones} and there should be no question whatever to requestion
them, because say of the need we are in to devise a terminology,
applicable to the general case of a category $M$ endowed with any
saturated set of maps $W\subset\Fl(M)$ (which are being thought of as
still more general substitutes of ``weak equivalences''). This shows
at once that we will have to renounce to the name of ``$W$-aspheric''
which we have used so far, in order to designate maps which are
``universally in $W$''; indeed, this contradicts the use we are making
of this word, in the case of \Cat. The whole trouble came from this
inappropriate terminology, which slipped in while thinking of the
\Ahat{} analogy, and forgetting about the still more basic \Cat! The
mistake is a course one indeed, and quite easy to correct -- \emph{we
  better refrain altogether from using the word ``aspheric'' in the
  context of a general pair $(M,W)$}, and rather speak of maps which
are ``\emph{universally in $W$}'' or ``\emph{universal
  $W$-equivalences}'', which is indeed more suggestive, and does not
carry any ambiguity. The notion of ``$W$-aspheric map'' should be
reserved to the case when, among all possible base-change maps $Y'\to
Y$ in $M$, we can sort out some which we may view as ``localizing
maps'' -- all maps I'd think in cases $M$ is a topos, and pretty few
ones when $M=\Cat$. As for qualifying objects of $M$, we'll just be
specific in stating properties of the projection $X\to e_M$ -- such as
being a $W$-equivalence, or a universal $W$-equivalence, or a
homotopism for $h_W$ (in which case the name ``contractible object''
is adequate indeed). It may be convenient, when we got a $W$, to
denote by
\[\UW\subset W\subset \Fl(M)\]
the corresponding set of maps which are universally in $W$, a
property which then can be abbreviated into the simple notation
\[ f\in \UW\]
or ``$f$ is in \UW. It should be noted that \UW{} contains all
invertible maps and is stable by composition, but \emph{it need not be
  saturated}, thus $f$ and $fg$ may be in \UW{} without $g$ being so.

This\pspage{185} terminological digression was of a more essential
nature, as a matter of fact, than merely technical. There is still
another correction I want to make with terminology introduced earlier,
namely with the name of a ``contractor'' I used for intervals endowed
with a suitable composition law (page \ref{p:120}). The name in itself
seems to me quite appropriate, however I have now a notion in reserve
which seems to me a lot more important still, a reinforcement it turns
out of the notion of a strict test category -- and which I really
would like to call a \emph{contractor}. I couldn't think of any more
appropriate name -- thus I better change the previous terminology --
sorry! -- and call those nice intervals ``multiplicative intervals'',
thus referring to the composition law as a ``multiplication'' (with
left unit and left zero element). The name which first slipped into
the typewriter, when it occurred that a name was desirable, was not
``contractor'' by the way but ``intersector'', as I was thinking of
the examples I had met so far, when composition laws were defined in
terms of intersections and were idempotent. But this doesn't square
too well with the example of contractors $\bHom(X,X)$, when $X$ is an
object which has a section -- and this example turns out as equally
significant.

One last comment is about the ``\v Cech type'' condition
\ref{it:65.Le} on the basic localizer \scrW, introduced two days ago
(page \ref{p:171}). As giving a ``crible'' in a category amounts to
the same as giving a map
\[ C\to \Simplex_1\]
(by taking the inverse image of the source-object $\{0\}$ of
$\Simplex_1$), and therefore giving two such amounts to a map from $C$
into $\Simplex_1\times\Simplex_1$, we see that the situation when $C$ is
the union of two cribles is expressed equivalently by giving a functor
from $C$ into the subcategory
\[C_0 = \Biggl(
  \begin{tikzcd}[baseline=(O.base),row sep=0pt]
    & b \\ |[alias=O]| a \ar[ur]\ar[dr] & \\ & c
  \end{tikzcd}\Biggr)\]
of $\Simplex_1\times\Simplex_1$ (dual to the barycentric subdivision of
$\Simplex_1$). The asphericity conditions on $C'$, $C''$ and $C' \sand
C''$ then just mean that \emph{this functor is \scrW-aspheric}, which by
the localization axiom \ref{it:64.L4} implies that the functor itself
is a weak equivalence. Thus (by saturation), the conclusion that $C$
should be \scrW-aspheric, just amounts to the following condition,
which is in a way the ``universal'' special case when $C=C_0$, and
$C',C''$ are the two copies of $\Simplex_1$ contained in $C_0$:
\begin{description}
\item[\namedlabel{it:66.Leprime}{L~e')}]
  The category $C_0$ above is \scrW-aspheric.
\end{description}

If\pspage{186} we now look upon the projection map of $C_0$ upon one
factor $\Simplex_1$ (carrying $a$ and $b$ into $\{0\}$ and $c$ into
$\{1\}$), we get a functor which is fibering, and whose fibers are
$\Simplex_1$ and $\Simplex_0$, which are \scrW-aspheric. Hence \emph{the
  fibration axiom \textup{\ref{it:64.L5}} on \scrW{} implies the
  Mayer-Vietoris axiom} \ref{it:65.Le} (page \ref{p:171}). This
argument rather convinces me that the fibration axiom should be strong
enough to imply all \v Cech-type \scrW-asphericity criteria which one
may devise (provided of course they are reasonable, namely hold for
ordinary weak equivalences!). More and more, it seems that the basic 
requirements to make upon a basic localizer, which will imply maybe
all others, are \ref{it:64.L1} (saturation), \ref{it:64.L3prime} (the
``standard interval axiom'', namely $\Simplex_1$ is \scrW-aspheric), and
the powerful fibration axiom \ref{it:64.L5}. This brings to my mind
though the condition \ref{it:64.La} of page \ref{p:165}, namely that
$f\in\scrW$ implies $\piz(f)$ bijective, which wasn't needed really
for the famous ``key result'' I was then after, and which for this
reason I then was looking at almost as something accessory! I now do
feel though that it is quite an essential requirement, even though we
made no formal use of it (except very incidentally, in the slightly
weaker form \ref{it:64.Laprime}, which just means that \HotW{} isn't
equivalent to the final category). I would therefore add it now to the
list of really basic requirements on a ``basic localizer'', and
rebaptize it therefore as \ref{it:66.L6}, namely:
\begin{description}
\item[\namedlabel{it:66.L6}{L~6)}] \textbf{(Connectedness axiom)}
  $f\in\scrW$ implies $\piz(f)$ bijective, i.e., the functor
  $\piz:\Cat\to\Sets$ factors through $\scrW^{-1}\Cat=\HotW$ to give
  rise to a functor
  \[\piz:\HotW\to\Sets.\]
\end{description}
This, as was recalled on page \ref{p:166}, is more than needed to
imply
\[h_\scrW=h_\Cat,\]
namely the homotopy structure in \Cat{} associated to \scrW{} (in
terms of \scrW-aspheric intervals) is just the canonical homotopy
structure (defined in terms of $0$-connected intervals), which is also
the homotopy structure defined by the single ``basic''
(multiplicative) interval $\Simplex_1$.

\bigbreak
\noindent\hfill\ondate{11.4.}\par

I forget to clear up still another point of terminology -- namely
about ``weak homotopy intervals'' -- it turns out finally we never
quite came around defining what ``homotopy intervals'' which aren't
weak should be! The situation is very silly indeed - so henceforth
I'll just\pspage{187} drop the qualificative ``weak'' -- thus from now
on a ``\emph{homotopy interval}'' (with respect to a given homotopy
structure $h$ in a category $M$) is just an interval whose end-point
sections $\delta_0,\delta_1$ are homotopic. In case $h=h_W$, where $W$
is a given saturated set of arrows in $M$, the notion we get is a lot
wider than the notion of a homotopy interval (with respect to $W$)
introduced earlier (page \ref{p:132}, and which we scarcely ever used
it seems, so much so that I even forgot till this very minute it had
been introduced formally), where we were restricting to intervals for
which $I\to e$ is universally in $W$, i.e., is in \UW{} (we may call
such objects simply \UW-objects). Anyhow, it seems that so far, the
only property of such intervals we kept using from the beginning is
the one shared with all homotopy intervals in the wider sense I am now
advocating. There is just one noteworthy extra property which is
sometimes of importance, especially in the characterization of test
categories, namely the property $\Ker(\delta_0,\delta_1)=\varnothing_M$;
this was referred to earlier by the name ``separated interval'' --
which however may lead to confusion when for objects of $M$ we have
(independently of homotopy notions) a notion of separation. Therefore,
we better speak about \emph{separating intervals} as those for which
$\Ker(\delta_0,\delta_1)=\varnothing_M$ (initial object in $M$), hence a
notion of \emph{separating homotopy interval} (with respect to a given
homotopy structure $h$, or with respect to a given saturated $W$,
giving rise to $h_W$).

%%% Local Variables:
%%% mode: latex
%%% TeX-master: "main.tex"
%%% End:
