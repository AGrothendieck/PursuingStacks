%% Anti-Copyright 2015 - the scrivener

\chapter{Schematization}
\label{ch:VI}

\noindent\hfill\ondate{24.8.}\pspage{443}\par

% 110
\hangsection[More wishful thinking on ``schematization'' of homotopy
\dots]{More wishful thinking on ``schematization'' of homotopy
  types.}\label{sec:110}%
I pondered some more about homotopy types over a ground ring $k$, just
enough to become familiar again with the idea, and more or less
convinced that that there should exist such a thing, which should
amount, kind of, to putting a ``continuous'' structures (namely the
very rich structure of a scheme) upon something usually visualized as
something ``discrete'' -- namely a homotopy type. The basic analogy
here is free \bZ-modules $M$ of finite type -- a typical case of a
``\emph{discrete}'' structure. It gives rise, though, to a
\emph{vector bundle} $W(M)$ over the absolute base $S_0=\Spec(\bZ)$,
whose \bZ-module of sections is $M$, and the functor
\[M\mapsto W(M)\]
from free \bZ-modules of finite type to vector bundles over $S_0$ is
fully faithful. When $M$ is an arbitrary \bZ-module, i.e., an abelian
group, $W(M)$ still makes sense, namely as a functor
\[k\mapsto M\otimes_\bZ k\]
on the category of all commutative \bZ-algebra (i.e., just commutative
rings); it is no longer representable by a scheme over $S_0$ (except
precisely when $M$ is free of finite type), but it is very close
still, intuitively and technically too, to a usual vector bundle (the
``vector'' structure coming from the $k$-module structure on
$W(M)(k)=M\otimes_\bZ k$). Again, the functor $M\mapsto W(M)$ from
$\AbOf$ to the category of ``generalized vector bundles'' over $S_0$
is fully faithful. Working with semisimplicial \bZ-modules (say)
rather than just \bZ-modules, and more specifically with those
corresponding to $K(\pi,n)$ types, and using Postnikov
``dévissage''\scrcomment{``dévissage'' = ``decomposition''}
of a general homotopy type, one may hope to ``represent'' this type,
in a more or less canonical way in terms of the successive
semisimplicial Postnikov fibrations, by a semi-simplicial object in
the topos (say) of all functors from \bZ-algebras to sets which are
``sheaves'' for a suitable site structure on the dual category (namely
the category of affine schemes over \bZ) -- the so-called ``flat''
topology seems OK. (NB\enspace To eliminate logical difficulties, we
may have to restrict somewhat the rings $k$ used as arguments, for
instance take them to be of finite type over \bZ{} -- never mind such
technicalities!). This approach may possibly work, when restricting to
$1$-connected homotopy types, or at any rate to the case when the
fundamental groups of the connected components are abelian. If we wish
a ``schematization'' of arbitrary homotopy types, we may think of
going about it by keeping the fundamental group of groupoid
``discrete'', and ``schematize'' the Postnikov\pspage{444} truncation
involving only the higher homotopy groups ($\pi_i$ with $i\ge2$). This
suggests that for any integer $n\ge1$, there may be a ``schematization
above level $n$'' for a given homotopy type, leaving the Cartan-Serre
truncation of level $n$ discrete, and ``schematizing'' the
corresponding total Postnikov fiber (involving homotopy groups $\pi_i$
for $i\ge n$). Maybe such a schematization can be constructed equally
for level $0$, even without assuming the fundamental groupoid to be
abelian, only nilpotent -- but then we may have to change from ground
ring \bZ{} to the considerably coarser one \bQ{} (compare comments at
the end of section \ref{sec:94}). In any case, the key step in this
approach would consist in checking that, after schematization has been
carried through successfully up to a certain level in the successive
elementary Postnikov fibrations, the next elementary fibration
(described by a cohomology class in $\mathrm H^{n+2}(X_n,\pi_{n+1})$)
comes from a ``schematic'' one, and that the latter is essentially
unique; in other words, that the canonical map from the ``schematic''
$\mathrm H^{n+2}$ (with ``quasi-coherent'' coefficients) to the usual
``discrete'' one is bijective. Maybe this hope is wholly unrealistic
though. One fact which calls for some skepticism about this approach,
comes in when looking at the case of an ``abelian'' homotopy type,
described by a semisimplicial abelian group $X_\bullet$, in which case
we expect that base change $\bZ\to k$ should be just the usual base
change
\[X_\bullet \mapsto X_\bullet \otimes_\bZ k\]
(if $X_\bullet$ has torsion-free, i.e., flat components, at any
rate). But when $X_\bullet$ has homology torsion, the universal
coefficients formula shows us that the homology (= homotopy) groups of
$X_\bullet\otimes_\bZ k$ are \emph{not} just the groups $\mathrm
H_i(X_\bullet) \otimes_\bZ k=\pi_i(X_\bullet)\otimes_\bZ k$, as we
implicitly were assuming it seems in the approach sketched above, when
schematizing the homotopy groups $\pi_i$ one by one via
$W(\pi_i)$. Thus, maybe Postnikov dévissage isn't a possible approach
towards schematization of homotopy types, and one will have to work
out rather a comprehensive yoga of reconstructing a homotopy type from
one kind or other of ``abelianization'' or ``linearization'' of
homotopy types, endowed with suitable extra structure embodying
``multiplicative'' features of the homology and cohomology
structure. At any rate, I did not hit upon any ``simplistic way'' to
define homotopy types over any ground ring, and I have some doubts
there is any, in terms of the general non-sense we did so far.

Besides\pspage{445} this, I spent hours to try and put some order into
the mess of all $\Hom$ and tensor product type operations between
categories \Ahatk, \AhatM, \Bhatk, \BhatM{} (or their $\&$-style
generalizations), and the duality and Cartan-type isomorphisms between
these. There are a few more still than the fair bunch met with in
these notes so far -- I finally renounced to get really through and
work out a wholly satisfactory set of notations, taking into account
all symmetries in the situation. I realized this might well take days
of work, while at present there is no real need yet for it. I
sometimes find it difficult to find a proper balance in these notes
between the need of working on reasonably firm ground, and working out
suggestive terminology and notations for what is coming up, and on the
other hand my resolve not to get caught again by the ``Eléments de
Géométrie Algébrique'' style of work, when it was understood that
everything had to be worked out in complete detail and in greatest
generality, for the benefit of generations of
``usagers''\scrcomment{I'm leaving in ``usagers'' for color, though
  ``users'' would surely do\ldots} (besides my own till the end of my
life!\kern.35\fontdimen3\font.\kern.35\fontdimen3\font.). As a matter
of fact, this whole ``abelianization story'', going on now for well
over a hundred pages and nothing really startling coming out -- just
things I feel I should have known for ages, has been won (so to say)
over an inner reluctance against these ``digressions'' in the main
line of thought, the reluctance of one who is in a hurry to get
through. I know well this old reluctance, feeling silly whenever
working out ``trivial details'' with utmost care; as I know too that
through this work only would come to a thorough understanding of what
is going on, and new intuitions or relationships would flash up
sometimes and open up unexpected landscapes and provide fresh
impetus. The same has happened innumerable times too within the last
seven years, when ``meditating'' on personal matters -- constantly
``the-one-in-a-hurry'' has turned out to be just the servant of the
inner resistances against renewal, against a fresh, innocent look upon
things familiar, and consistently ignores as ``irrelevant''. It
doesn't seem the-one-in-a-hurry gets at all discouraged for not
getting his way many times -- he seems to be just as stubborn as the
one who likes to take his time and look up things thoroughly!

\bigbreak

\noindent\hfill\ondate{25.8.}\pspage{446}\par

% 111
\hangsection[Complexes of ``unipotent bundles'' as models, and
\dots]{Complexes of ``unipotent bundles'' as models, and ``schematic''
  linearization.}\label{sec:111}%
Still about ``schematization'' of homotopy types! Here is a tentative
approach, without any explicit use of Postnikov fibrations nor
abelianization, although both are involved implicitly. If $n$ is any
natural integer, I'll denote by
\[\HotOf_n\]
the full subcategory of the pointed homotopy category
$\HotOf^\bullet$, made up with $n$-connected homotopy types, with the
extra assumption for $n=0$ that the fundamental group be abelian. For
any (commutative) ring $k$, I want to define a category
\begin{equation}
  \label{eq:111.a}
  \HotOf_n(k),\tag{a}
\end{equation}
depending covariantly on $k$, in such a way that we have an
equivalence
\begin{equation}
  \label{eq:111.b}
  \HotOf_n(\bZ)\toequ\HotOf_n,\tag{b}
\end{equation}
which should come from a canonical functor
\begin{equation}
  \label{eq:111.c}
  \HotOf_n(k) \to \HotOf_n\tag{c}
\end{equation}
defined for any $k$, and which should be viewed as ground-ring
restriction from $k$ to \bZ{} -- more generally, for any ring
homomorphism
\[k'\to k\]
we expect a restriction functor, beside the ring extension functor
\begin{equation}
  \label{eq:111.d}
  \HotOf_n(k')\to\HotOf_n(k)\quad\text{and}\quad
  \HotOf_n(k)\to\HotOf_n(k').\tag{d}
\end{equation}
Among other important features to expect, is that for any object $X$
in $\HotOf_n(k)$, the homotopy groups $\pi_i(X)$ (defined via
\eqref{eq:111.c}) should be naturally endowed with structures of
$k$-modules, and the ring extension and restriction functors
\eqref{eq:111.d} should be compatible with these.

Here is an idea for getting such a theory. For given $k$, we first
define an auxiliary category
\begin{equation}
  \label{eq:111.e}
  U(k),\tag{e}
\end{equation}
whose objects may be called ``unipotent bundles over $k$''. These
``bundles'' will not be quite schemes over $k$, they will be defined
as functors
\begin{equation}
  \label{eq:111.f}
  \Alg_{/k}\to\Sets\tag{f}
\end{equation}
where $\Alg_{/k}$ is the category of (commutative) $k$-algebras (in
the basic universe \scrU). The opposite category may be identified
with the category
\[\Aff_{/k}\]
of\pspage{447} affine schemes over $k$, thus, we'll be working in the
category of functors (or presheaves over $\Aff_{/k}$)
\begin{equation}
  \label{eq:111.fprime}
  (\Aff_{/k})\op \to \Sets\tag{f'}
\end{equation}
more specifically, $U(k)$ will be a full subcategory of this category
of functors. We'll endow $\Aff_{/k}$ with one of the standard site
structures, the most convenient one here is the
fpqc\scrcomment{``fidèlement plat et quasi-compact''} topology
(faithfully flat quasi compact topology), and work in the category of
\emph{sheaves} in the latter. In terms of the interpretation
\eqref{eq:111.f} as covariant functors on $\Alg_{/k}$, this just means
that we are restricting to functors $X$ which 1)\enspace commute to
finite products and 2)\enspace are ``compatible with faithfully flat
descent'', i.e., for any map
\[k'\to k''\]
in $\Alg_{/k}$ such that $k''$ becomes a faithfully flat algebra over
$k'$, the following diagram in \Sets
\[X(k') \to X(k'') \rightrightarrows X(k'' \otimes_{k'} k'')\]
is exact. Thus, $U(k)$ will be defined as a full subcategory of the
category of such functors, or ``sheaves''.

One way for defining $U(k)$ is to present it as the union of a
sequence of subcategories $U_m(k)$ ($m$ a natural integer). We'll take
$U_0(k)$ to be just reduced to the final functors (i.e., $X(k')$ is a
one-point set for any $k'$ in $\Alg_{/k}$), and define inductively
$U_{m+1}(k)$ in terms of $U_m(k)$ as follows. For any $k$-module $M$,
let $W(M)$ be the corresponding ``vector bundle'', defined by
\begin{equation}
  \label{eq:111.g}
  W(M)(k')=M\otimes_k k',\tag{g}
\end{equation}
then an object $X$ is in $U_{m+1}(k)$ if{f} there exists an object $X_m$
in $U_m(k)$, and a $k$-module $M=M_{m+1}$, in such a way that
$X=X_{m+1}$ should be isomorphic to a ``torsor'' over $X_m$, with
group $W(M)$. I am not too sure, here, whether we should view the
objects of $U_m(k)$ as endowed with the extra structure consisting in
giving the modules $M_1,\dots,M_m$ used in the inductive construction,
and moreover the successive fibrations -- if so, then of course the
categories $U_m(k)$, and their union $U(k)$, will no longer be
interpreted as a mere subcategory of the category of\pspage{448}
sheaves (of sets) just described. Possibly, both approaches are of
interest and will yield non-equivalent notions of schematization. On
the other hand, although definitely $X$ is not representable by a
usual scheme over $k$ unless the $k$-modules $M_i$ are projective of
finite type (in which case $X$ will be an affine scheme, and even
isomorphic, at least locally over $\Spec(k)$, to standard affine space
$E_k^d$ for suitable $d$), it is felt that $X$, as far as cohomology
properties go, should be very close to being an affine scheme, and
that presumably its cohomology groups $\mathrm H^i$ with coefficients
in ``quasi-coherent'' sheaves such as $W(M)$ should vanish for $i>0$;
consequently, presumably the torsors used for the inductive
construction of $X$ are trivial, which means that $X$ is in fact
isomorphic to the product of all $W(M_i)$'s. In the case when we
disregard the successive fibration structure, this means that the
objects of $U(k)$ are just sheaves of sets which are isomorphic to
some $W(M)$ (where morally $M$ is the direct sum of the modules $M_i$
which have been used in our inductive definition). This gives then a
rather trivial description of the objects of $U(k)$ (and all
$U_m(k)$'s are already equal to $U(k)$, for $m\ge1$!), it should be
remembered, however, that maps in $U(k)$ from a $W(M)$ to a $W(M')$
are a lot more general than just $k$-linear maps $M\to M'$ (they may
be viewed as ``polynomial maps from $M$ to $M'$'').

The category $U(k)$ will be endowed with the \emph{sections functor}
\begin{equation}
  \label{eq:111.h}
  X\mapsto X(k) : U(k) \to \Sets.\tag{h}
\end{equation}
Now, let $A$ be any test category, for instance $A=\Simplex$, and
consider the functor
\begin{equation}
  \label{eq:111.hprime}
  \bHom(A\op,U(k)) \to \bHom(A\op,\Sets)\tag{h'}
\end{equation}
induced by \eqref{eq:111.h}. The second member modelizes homotopy
types, which therefore allows us to define homotopy invariants for
objects in the first member, and hence to define the property of
$n$-connectedness, and (if $n=0$) of $0$-connectedness with abelian
fundamental group. As a matter of fact, we would like to define a
subcategory $\scrM_n(k)$ of the first member, so that is should become
clear that for an object $X_*$ in it, its homotopy groups are endowed
with structures of $k$-modules, so that for $n=0$ the abelian
restriction on $\pi_1$ is superfluous, because automatic. To be
specific, we better restrict maybe to $A=\Simplex$, more familiar to
us, so that $X_*$ may be viewed as a ``complex'' ($n\mapsto X_n$) with
components in $U(k)$. The kind of restriction I am thinking of for
defining the\pspage{449} model category $\scrM_n(k)$ is:
\begin{equation}
  \label{eq:111.i}
  X_i=e \quad\text{for}\quad i\le n,\tag{i}
\end{equation}
where $e$ is the final object of $U(k)$, and also maybe, if we adopt
the more refined version of $U(k)$ as a strictly increasing union of
subcategories $U_m(k)$,
\begin{equation}
  \label{eq:111.iprime}
  \text{$X_i$ is in $U_{i-n}$, for any $i\ge n$.}\tag{i'}
\end{equation}
One may have to play around some more to get ``the correct''
description of the model category, which I tentatively propose to
define simply as a suitable full subcategory
\begin{equation}
  \label{eq:111.j}
  \scrM_n(k)\subset \bHom(A\op,U(k)).\tag{j}
\end{equation}
The functor \eqref{eq:111.hprime} allows to define a notion of weak
equivalence in $\scrM_n(k)$, hence a localized category $\HotOf_n(k)$,
and a functor \eqref{eq:111.c} from this category to $\HotOf_n$. The
ring extension and restriction functors \eqref{eq:111.d} are equally
defined in an evident way, via corresponding functors on the model
categories (with the task, however, to check that these are compatible
with weak equivalences). The key point here is to check that for
$k=\bZ$, the functor \eqref{eq:111.c} (namely \eqref{eq:111.b}) is
indeed an equivalence of categories. Thus, the main task seems to cut
out carefully a description of a model category $\scrM_n(k)$, in terms
of semisimplicial objects say, in a category such as $U(k)$, in such a
way as to give rise to an equivalence of categories \eqref{eq:111.b}.

One point which is still somewhat misty in this (admittedly overall
misty!) picture, is how to get, for an object $X$ in $\HotOf_n(k)$,
the promised operation of $k$ on the homotopy groups $\pi_i(X)$. I was
thinking about this when suggesting the conditions \eqref{eq:111.i}
and \eqref{eq:111.iprime} above on $k$-models for homotopy types --
but I really doubt these are enough. On the other hand, it seems hard
to imagine there be a good notion of homotopy types over $k$, without
the homotopy groups to be $k$-modules over $k$, not just abelian
groups. Even more still, there should be moreover a ``linearization
functor''
\begin{equation}
  \label{eq:111.k}
  \HotOf_n(k) \to \D_\bullet(\AbOf_k),\tag{k}
\end{equation}
with values in the derived category of the category of chain complexes
in $\AbOf_k= k\textup{-Mod}$ of $k$-modules, presumably coming by
localization from a functor
\begin{equation}
  \label{eq:111.kprime}
  \scrM_n(k)\to\Ch_\bullet(\AbOf_k),\tag{k'}
\end{equation}
and giving rise to a commutative diagram\scrcomment{unreadable
  footnote, maybe ``only for $n=\oo$''?}\pspage{450}
\begin{equation}
  \label{eq:111.l}
  \begin{tabular}{@{}c@{}}
    \begin{tikzcd}[baseline=(O.base)]
      \HotOf_n(k)\ar[r]\ar[d] & \D_\bullet(\AbOf_k)\ar[d] \\
      \HotOf_n\ar[r] & |[alias=O]| \D_\bullet\Ab      
    \end{tikzcd},
  \end{tabular}\tag{l}
\end{equation}
where the second horizontal arrow is the usual abelianization functor
for homotopy types, and the second vertical one comes from the ring
restriction functor $\AbOf_k\to\AbOf=\AbOf_\bZ$. In
\eqref{eq:111.kprime}, $\Ch_\bullet$ denotes the category of chain
complexes, and it looks rather mysterious again how to get such a
functor \eqref{eq:111.kprime}. We may of course think of the trivial
abelianization
\[X_\bullet \mapsto k^{(X_\bullet)} \eqdef (n\mapsto k^{(X_n)}),\]
where for an object $X$ in $U(k)$, or more generally any sheaf on
$\Aff_{/k}$, $k^{(X)}$ defines a trivial $k$-linearization of this
sheaf, in the sense of the topos of all such sheaves. Anyhow,
$k^{(X)}$ is a sheaf, not a $k$-module, so we should still take
sections to get what we want -- but this functor looks not only
prohibitively large and inaccessible, but just silly! A much better
choice for $k$-linearizing objects of $U(k)$ specifically seems the
following. Disregarding the fibration structures, such an object $X$
is isomorphic to an object $W(M)$, $M$ some $k$-module. We look for a
$k$-linearization
\begin{equation}
  \label{eq:111.m}
  X\;(=W(M)) \to W(L(X)),\tag{m}
\end{equation}
where $L(X)$ is a suitable $k$-module. Now, among all maps
\[X\to W(N)\]
of $X$ into sheaves \emph{of the type $W(N)$}, there is a universal
one, which in terms of $M$ can be described as
\[N=\Gamma_k(M),\]
where $\Gamma_k$ denotes the ``algebra with divided powers generated
by $M$'', the canonical map $M\to \Gamma_k(M)$ or rather
\begin{equation}
  \label{eq:111.n}
  W(M)\to W(\Gamma_k(M)), \quad
  x\mapsto \exp(x)=\sum_{i\ge0} x^{(i)}\tag{n}
\end{equation}
being \emph{the} ``\emph{universal} polynomial map'' of $M$ with
values in a module $N$ (or rather, of $W(M)$ into $W(N)$). Here,
$x^{(i)}$ denotes the $i$'th divided power of $x$, which is an element
of $\Gamma_k^i(M)$. It just occurs to me that this expression of
$\exp(x)$, the universal map, is infinite, thus, it doesn't take its
values in $W(N)$ actually, but in a suitable completion of it -- this
doesn't seem too serious a drawback, though! The point I wish to make
here, is that for given $X$ in $U(k)$, defining\pspage{451}
\begin{equation}
  \label{eq:111.o}
  L_k(X) = \Gamma\uphat_k(M) \eqdef \prod_{i\ge0} \Gamma_k^i(M),
  \quad\text{($k$-linearization of $X$),}\tag{o}
\end{equation}
where $M$ is any $k$-module endowed with an isomorphism
\[u: X\simeq W(M),\]
the $k$-module $L_k(X)$ \emph{does not depend, up to unique
  isomorphism, on the choice of a pair $(M,u)$}, because for two
$k$-modules $M,M'$, any morphism of \emph{sheaves of sets}
\[v:W(M)\to W(M')\]
induces a homomorphism of $k$-modules
\[\Gamma\uphat_k(v): \Gamma\uphat_k(M) \to \Gamma\uphat_k(M')\]
(compatible not with multiplications, but with diagonal maps\dots),
which will be an isomorphism if $v$ is.

Thus, we do have, it seems to me, a good candidate for
$k$-linearization. To check it is suitable indeed, the main point
seems to check that the corresponding diagram \eqref{eq:111.l}
commutes up to canonical isomorphism, the crucial case of course being
$k=\bZ$. This now looks like a rather down-to-earth question, which
seems to me a pretty good test, whether the intuition of
schematization of homotopy types is a sound one. Let's rephrase it
here. For this, let's first restate the description of the category
$U(k)$ (coarse version) in the more down-to-earth terms of linear
algebra. Objects may be viewed as just $k$-modules $M$, whereas
(non-additive) ``maps'' from $M$ to $M'$ (defined previously as maps
$W(M)\to W(M')$ of sheaves of sets) are described as just continuous
$k$-linear maps
\begin{equation}
  \label{eq:111.p}
  f:\Gamma\uphat_k(M)\to\Gamma\uphat_k(M'),\tag{p}
\end{equation}
which are moreover compatible with the natural augmentations to $k$,
and with the natural diagonal maps:
\[\varepsilon:\Gamma\uphat_k(M)\to k, \quad
\Delta:\Gamma\uphat_k(M)\to\Gamma\uphat_k(M)\hatotimes_k\Gamma\uphat_k(M)
\;(\simeq\Gamma\uphat_k(M\times M)),\]
(the latter deduced from the usual linear diagonal map $M\to M\times
M$). When $M$ is looked at as being embedded in $\Gamma\uphat_k(M)$ by
the exponential map \eqref{eq:111.n}, it is identified (if I remember
it right) to the set of elements $\xi$ in $\Gamma\uphat_k(M)$
satisfying the relations
\begin{equation}
  \label{eq:111.q}
  \varepsilon(\xi)=1, \quad \Delta(\xi)=\xi\otimes\xi,\tag{q}
\end{equation}
where $\varepsilon$ is the augmentation and $\Delta$ the diagonal map,
hence \eqref{eq:111.p} induces a map (in general not
additive)\pspage{452}
\begin{equation}
  \label{eq:111.pprime}
  \Gamma(f):M\to M'\tag{p'}
\end{equation}
(corresponding to the action of $f$, viewed as a map $W(M)\to W(M')$,
on \emph{sections} of $W(M)$) -- and likewise after any ring extension
$k\to k'$, defining a map
\[\Gamma(f)_{k'}: M\otimes_k k' \to M'\otimes_k k'\]
from $W(M)$ to $W(M')$ -- which is the description of the map of
sheaves $W(M)\to W(M')$ associated to a map \eqref{eq:111.p}. We have
thus a description, in terms of linear algebra, of a category $U(k)$,
and of a ``sections'' functor
\begin{equation}
  \label{eq:111.r}
  \Gamma: U(k)\to\Sets, \quad
  X\mapsto X(k),\tag{r}
\end{equation}
which is essentially the functor \eqref{eq:111.h} above, viewed in a
different light. Now to our

\begin{questionnum}\label{q:111.1}
  The category $U(k)$ and the functor \eqref{eq:111.r} being defined
  as above in terms of linear algebra over the ground ring $k$, let
  $X_*=(n\mapsto X_n)$ be a semisimplicial object in $U(k)$, consider
  the corresponding semi-simplicial set $X_*(k)=(n\mapsto X_n(k))$,
  and the semi-simplicial $k$-module $L(X_*)=(n\mapsto L(X_n))$,
  where, for an object $M$ of $U(k)$, $L(M)$ is defined as
  \begin{equation}
    \label{eq:111.s}
    L(M)=\Gamma\uphat_k(M),\tag{s}
  \end{equation}
  which depends functorially on $M$ in $U(k)$. Then in the derived
  category of \Ab, is there a canonical isomorphism between $L(X_*)$
  (with ground ring restricted from $k$ to \bZ) and the abelianization
  $\bZ^{(X_*(k))}$ of $X_*(k)$?\footnote{\scrcommentinline{\dots
      unreadable\dots} $k=\bZ$, and components $X_n$ projective.}
\end{questionnum}

We may have to throw in some extra assumption on $X_*$, at any rate
\begin{equation}
  \label{eq:111.t}
  X_0=e\quad\text{(final object of $U(k)$),}\tag{t}
\end{equation}
giving rise to $L(X_0)\simeq k$. Also, we may have to restrict to
$k=\bZ$, or otherwise correct the obvious drawback that the two chain
complexes don't have isomorphic $\mathrm H_0$ (one is $k$ I guess, a
$k$-module in any case, the other is \bZ), by truncating accordingly
the two chain complexes (``killing'' their $\mathrm H_0$). There is a
natural candidate for a map
\begin{equation}
  \label{eq:111.u}
  \bZ^{(X_*(k))} \to L(X_*),\tag{u}
\end{equation}
by using the functorial map
\begin{equation}
  \label{eq:111.uprime}
  \bZ^{(M)} \to L(M)=\Gamma\uphat_k(M),\tag{u'}
\end{equation}
deduced from the inclusion \eqref{eq:111.n}
\[M\hookrightarrow L(M),\]
and we may still specify the question above, by asking \emph{whether
  \eqref{eq:111.u} induces an isomorphism for homology groups} in
dimension $i>0$.

I\pspage{453} am not too sure whether all this isn't just complete
nonsense -- it is worth getting it clear whether it is or not, at any
rate! There is one case of special interest, the ``simplest'' one in a
way, namely when the simplicial maps between the $X_n$'s, each
represented by a $k$-module $M_n$, are in fact $k$-linear, in other
words, when $X_*$ comes from a semisimplicial $k$-module $M_*$ -- more
specifically still, when this is a $K(\pi,n)$ type, say the nicest
semisimplicial model of this, using the Kan-Dold-Puppe functor for the
chain complex of $k$-modules, having $\pi$ in degree $n$ and zero
elsewhere. Then the left-hand side of \eqref{eq:111.u} gives rise to
the Eilenberg-Mac~Lane homology groups
\begin{equation}
  \label{eq:111.v}
  \mathrm H_i(\pi,n;\bZ),\tag{v}
\end{equation}
which I guess should be $k$-modules a priori, because of the
operations of $k$ upon $\pi$, and the question then arises whether
these can be computed using the right-hand complex $L(X_*)$. Maybe
such a thing is even a familiar fact for people in the know? If it
turned out to be false even for $k=\bZ$, my faith in schematization of
homotopy types would be seriously shaken I confess\dots

\starsbreak

After a little break for dinner, just one more afterthought. Working
with the completions $\Gamma\uphat_k$ may seem a little forbidding,
and all the more so if it used for computing \emph{homology}
invariants, not cohomology (the latter more likely to involve infinite
products in the corresponding (cochain) complexes\dots). On the other
hand, as was explicitly stated from the beginning, the natural context
here seems to be \emph{pointed} homotopy types, and hence ``pointed''
algebraic paradigms for these -- an aspect we lost sight of, when
looking for a suitable description of some category $U(k)$ of
``unipotent bundles'' over $k$. It would seem that in the ``question''
above, we should therefore insist that $X_*$ should be a
semi-simplicial object of the category $U(k)^\bullet$ of ``pointed''
objects of $U(k)$, namely objects $X$ endowed with a section over the
final object $e$ (i.e., with an element in $X(k)$). This will be
automatic at any rate in terms of the condition \eqref{eq:111.t},
$X_0=e$. The point I wish to make is that the category of pointed
objects of $U(k)$ admits a somewhat simpler description (by choosing
the marked point as the ``origin'' for parametrization of the given
object $X$ of $U(k)$ by a $k$-module $M$), by model-objects which are
still arbitrary $k$-modules $M$, but the ``maps'' now being
\emph{$k$-linear continuous maps}
\begin{equation}
  \label{eq:111.pprimebis}
  f:\Gamma_k(M)\to\Gamma_k(M')\tag{p'}
\end{equation}
between\pspage{454} the $k$-modules $\Gamma_k$, without having to pass
to completions, satisfying compatibility with augmentations and
diagonal maps, and the extra condition (expressing that
$\Gamma(f)(0)=0$ in exponential notation):
\[f(1)=1,\]
i.e., $f$ reduced to component $\Gamma_k^0(M)\simeq k$ of degree zero
is just the identity of $k$ with $k\simeq\Gamma_k^0(M')$. Accordingly,
we have a less awkward $k$-linearization functor than $L$ in
\eqref{eq:111.s}, namely ``pointed linearization'' $L\subpt$:
\begin{equation}
  \label{eq:111.sprime}
  M\mapsto L\subpt(M)\eqdef \Gamma_k(M) : U(k)^\bullet\to\AbOf_k,\tag{s'}
\end{equation}
which seems to me the better candidate for describing
linearization. Thus, we better rephrase now the ``question'' above in
terms of \eqref{eq:111.sprime} rather than \eqref{eq:111.s}. One
trouble however is that the comparison map \eqref{eq:111.u} takes
values in $L(X_*)$, not $L\subpt(X_*)$, therefore, we may still have
to use the ``prohibitive'' $L(X_*)$ as an intermediary for comparing
the complexes $\bZ^{(X_*(k))}$ and $L\subpt(X_*)$. It may be noted now
that, while the first chain complex embodies Eilenberg-Mac~Lane
homology \eqref{eq:111.v} (in the special case considered above), the
second one $L\subpt(X_*)$ (in that same case) describes the value of
the total derived functor of the familiar $\Gamma_k$ functor, on the
``argument'' $\pi$ placed in degree $n$, and the statement that the
two are ``the same'' does sound like some standard Dold-Puppe type
result which everybody is supposed to know from the cradle -- sorry!

\starsbreak

After another break (visit, tentative nap), still another
afterthought. The final shape we arrived at for the ``question''
above, when working in the ``pointed'' category $U(k)^\bullet$ of
``pointed unipotent bundles over $k$'', was whether for any
semisimplicial object $X_*$ in $U(k)^\bullet$ satisfying
\eqref{eq:111.t} above, i.e., $X_0=e$ (final object of
$U(k)^\bullet$), the two canonical maps of chain complexes (in fact,
semisimplicial abelian groups)
\begin{equation}
  \label{eq:111.w}
  \bZ^{(X_*(k))} \to L(X_*) \hookleftarrow L\subpt(X_*)\tag{w}
\end{equation}
are quasi-isomorphisms (for $\mathrm H_i$ with $i>0$). I am not too
sure yet if some extra conditions on $X_*$ are not required for this
to be reasonable -- I want to review two that came into my mind.

As the maps \eqref{eq:111.w} are functorial for varying $X_*$, it
would follow\pspage{455} from a positive answer that whenever
\begin{equation}
  \label{eq:111.x}
  X_*\to X_*'\tag{x}
\end{equation}
is a map of semisimplicial objects in $U(k)^\bullet$ satisfying
condition \eqref{eq:111.t}, and such that the corresponding map
\begin{equation}
  \label{eq:111.xprime}
  X_*(k) \to X_*'(k)\tag{x'}
\end{equation}
is a weak equivalence, and hence the map between the
\bZ-abelianizations is a weak equivalence too, i.e., a
quasi-isomorphism, that the same holds for the corresponding map
\begin{equation}
  \label{eq:111.y}
  L\subpt(X_*)\to L\subpt(X_*')\tag{y}
\end{equation}
for the ``schematic'' $k$-linearizations. Now, this is far from being
an evident fact by itself, except of course in the case when the map
\eqref{eq:111.x} above is a homotopism. Take for instance the case
when we start with a map of chain complexes in $\AbOf_k$
\begin{equation}
  \label{eq:111.xdblprime}
  M_\bullet \to M_\bullet',\tag{x''}
\end{equation}
hence a map between the associated semisimplicial $k$-modules
\begin{equation}
  \label{eq:111.xtplprime}
  M_*\to M_*',\tag{x'''}
\end{equation}
which may be viewed as giving rise to a (componentwise linear) map
between the associated semisimplicial objects $X_*$, $X_*'$ in
$U(k)^\bullet$ via the canonical functor
\[\AbOf_k\to U(k)^\bullet;\]
the corresponding map \eqref{eq:111.y} is then just the componentwise
extension of \eqref{eq:111.xtplprime} to the enveloping algebras with
divided powers
\begin{equation}
  \label{eq:111.yprime}
  \Gamma_k(M_*) \to \Gamma_k(M_*').\tag{y'}
\end{equation}
The map \eqref{eq:111.xprime} can now be identified with
\eqref{eq:111.xtplprime}, hence it is a weak equivalence if{f}
\eqref{eq:111.xtplprime} is, i.e., if{f} \eqref{eq:111.xtplprime} is a
quasi-isomorphism. If we assume moreover the components of
$M_\bullet$, $M_\bullet'$ to be projective objects in $\AbOf_k$, then
from the assumption that \eqref{eq:111.xdblprime} is a
quasi-isomorphism it does follow that it is a chain homotopism, hence
by Kan-Dold-Puppe the map \eqref{eq:111.xtplprime} is a semisimplicial
homotopism, and hence the same holds for \eqref{eq:111.yprime}, and
\eqref{eq:111.yprime} therefore is indeed a quasi-equivalence. But
without the assumption that components are projective, it is surely
false that the mere fact that \eqref{eq:111.xdblprime} is a
quasi-isomorphism, implies that \eqref{eq:111.yprime} is -- otherwise
this would mean that in order to compute the left derived functor of
the non-additive functor\pspage{456}
\[\Gamma_k:\AbOf_k\to\AbOf_k,\]
it is enough, for getting its value on a chain complex $M_\bullet'$
say, to replace $M_\bullet'$ by $M_*'$ and apply the functor
$\Gamma_k$ componentwise, without first having to take a projective
resolution $M_\bullet$ of $M_\bullet'$ -- something rather absurd
indeed! Thus, the statement made on page \ref{p:454}, that when taking
for $M_\bullet'$ the $k$-module $\pi$ placed in degree $n$ and zero in
all other degrees, the corresponding $L\subpt(X_*')=L_k(M_*')$
embodies the value of the left derived functor $\mathrm L\Gamma_k$ on
$M_\bullet'$, is visibly incorrect if we don't assume moreover that
$\pi$ is projective (flat, presumably, would be
enough\dots). Otherwise, we should first replace $\pi$ by a projective
(or at any rate flat) resolution, which we shift by $n$ to get
$M_\bullet$, and then take $\Gamma_k(M_*)$ to get the correct value of
$\mathrm L\Gamma_k(M_\bullet')$.

This convinces me that in the question as to whether the maps in
\eqref{eq:111.w} are quasi-isomorphisms (the more crucial one of
course being the first of the two), \emph{we should assume moreover
  that the components of $X_*$ are described in terms of
  \emph{projective} $k$-modules $M_n$}, or at any rate $k$-modules
that are flat. Accordingly, we should make the same restriction on the
semisimplicial schematized model $X_*$, in order for the description
we gave of ``$k$-linearization'' as $L\subpt(X_*)$ (or $L(X_*)$, never
mind which) to be topologically meaningful. Very probably, in the
whole schematization set-up, namely in the very definition we gave of
$U(k)$ and $U(k)^\bullet$, we should stick to the same restriction. If
I insisted first (with some inner reluctance, I admit) on taking
$k$-modules $M$ unrestricted, this was because I was thinking of $M$,
more specifically of the $M_i$'s occurring in the inductive
``dévissage'' of an object of $U_m(k)$ (when thinking of the more
refined version of $U(k)$), as essentially the homotopy groups of the
homotopy type we want to modelize, or rather, as the components of the
corresponding semisimplicial $k$-modules (denoted $M_*'$ some minutes
ago). I was still thinking of course, be it implicitly, in terms of
Postnikov dévissage, despite yesterday's remark that to use such
dévissage literally may cause trouble (p.\ \ref{p:444}). Thus, the
feeling which gets into the fore now is that \emph{we should kind of
  forget Postnikov, and work with semisimplicial ``schematic'' models
  built up with $k$-modules which are \emph{projective}}, or at any
rate flat (namely torsion free, if $k=\bZ$).

It\pspage{457} may be remarked that if $M$ is any $k$-module, then the
property that $M$ be projective, or flat, can be described in terms of
the isomorphism class of the corresponding object $X$ in
$U(k)^\bullet$, or equivalently, of the functor $W(M)$ on $\Alg_{/k}$,
with values in the category of pointed sets. Indeed, the isomorphism
class of the $k$-module $\Gamma_k(M)$ depends only on the class of
$X$, and it is easily seen that $M$ is projective, resp.\ flat, if{f}
$\Gamma_k(M)$ is. (The ``only if'' is standard knowledge of
commutative algebra, the ``if'' comes from the fact that $M$ is a
direct factor of $\Gamma_k(M)$, hence projective resp.\ flat if the
latter is.) Using this, one even checks that it is enough to know the
isomorphism class of $X$ in $U(k)$ -- because two objects of
$U(k)^\bullet$ are isomorphic in $U(k)^\bullet$ if{f} they are in
$U(k)$.

\starsbreak

The second afterthought is that in the question on page \ref{p:452},
we should definitely assume $k=\bZ$. Already when asserting hastily
(p.\ \ref{p:453}) that en Eilenberg-Mac~Lane homology group $\mathrm
H_i(\pi,n;\bZ)$ should automatically inherit a structure of a
$k$-module, whenever $\pi$ had one, I was feeling uncomfortable,
because after all the dependence of this group upon variable $\pi$ is
not at all additive, thus the operation of $k$ upon this group
stemming from its operations on $\pi$ was no too likely to come out
additive! To take one example, take $n=1$, i.e., we just take ordinary
group homology for $\pi$, and assume that $k$ is free of finite type
over \bZ, and $\pi=M$ free of finite type over $k$ (for instance
$M=k$), hence free of finite type over \bZ{} too. Then it is
well-known that
\[\mathrm H_*(M; \bZ) \simeq \bigwedge_\bZ M,\]
the exterior algebra of $M$ over \bZ, which surely is not endowed with
a structure of a $k$-module in any natural way! This, if there was any
such structure (natural or not) on the highest non-zero term
(corresponding to the rank $d$ of $M$ over \bZ), it would follow that
we get a ring homomorphism from $k$ to $\bZ\simeq\End_\bZ(\mathrm
H_d\simeq\bZ)$, and we may choose $k$ in such a way that there is no
such homomorphism, for instance $k=\bZ[T]/(T^2+1)$. In this case,
there cannot be \emph{any} isomorphism between the $\mathrm H_d$'s of
the two members of \eqref{eq:111.u}! Another point, which I hit upon
first, is that when $X_*$ is defined in terms of a semisimplicial
object $M_*$ of $\AbOf_k$, then the functorial dependence of the first
term in \eqref{eq:111.w} with respect to\pspage{458} varying $M_*$, is
that to a direct sum corresponds the componentwise tensor product
\emph{over \bZ}, whereas for the last term of \eqref{eq:111.w} we have
to take tensor products \emph{over $k$} (in the middle term,
\emph{completed} tensor products over $k$) -- the two variances are
clearly at odds with each other.

Thus, when working with would-be ``$k$-homotopy types'' as defined
here via semisimplicial objects in $U(k)^\bullet$, \emph{we should
  altogether drop the idea that the homology groups of the
  corresponding semisimplicial set $X_*(k)$ are $k$-modules}. I
wouldn't really look at them as being ``the'' homology groups of the
$k$-homotopy type $X_*$, these should be rather given via
$k$-linearization $L\subpt$ and they are $k$-modules, indeed, they
come from a canonical object of $\D_\bullet(\AbOf_k)$, namely
$L\subpt(X_*)$. In the example just looked at, presumably we should
get the exterior algebra of $M$ \emph{over $k$} (not \bZ!). This makes
me suspect even that, except in the case $k=\bZ$, this semisimplicial
set $X_*(k)$ doesn't make much sense, namely its homology invariants
(and presumably its homotopy groups too) are not really relevant for
the $k$-homotopy type $X_*$, \emph{which has invariants of its own
  which are completely different. Thus, I am not at all convinced any
  more that the homotopy groups $\pi_i(X_*)$ carry $k$-module
  structures} (as expected at the beginning, p.\ \ref{p:446}) -- but
to clear our mind on that matter, we should take off from the
simplistic example when $X_*$ comes from a semisimplicial $k$-module
$M_*$, in which case of course we have by Whitehead's isomorphism
\[\pi_i(X_*(k)) \simeq \mathrm H_i(M_\bullet) \; (\simeq\pi_i(M_*)),\]
where $M_\bullet$ is the chain complex associated to $M_*$. Also, it
is now becoming obvious whether weak equivalences for objects $X_*$
should be defined (as we did) via $X_*(k)$, as suggested by the
``simplistic'' example above. If we define homology of $X_*$ by the
formula
\begin{equation}
  \label{eq:111.z}
  \LH_\bullet(X_*) \eqdef L\subpt(X_*)\quad
  \text{viewed as an object of $\D_\bullet(\AbOf_k)$,}\tag{z}
\end{equation}
and accordingly, the homology modules
\begin{equation}
  \label{eq:111.zprime}
  \mathrm H_i(X_*) \eqdef \pi_i(L\subpt(X_*)) = \mathrm
  H_i(\LH_\bullet(X_*)),\tag{z'} 
\end{equation}
maybe the better idea for defining weak equivalences $X_*\to X_*'$ is
by demanding that \emph{they should be transformed into
  quasi-isomorphisms by the total homology functor $\LH_\bullet$}, or
equivalently, induce isomorphisms for the homology modules
\eqref{eq:111.zprime}. If the answer to our crucial question is
affirmative (with the corrections made, including $k=\bZ$),
then\pspage{459} in case $k=\bZ$, the new definition just given for
weak equivalences is equivalent to the old one in terms of $X_*(\bZ)$
(taking \bZ-valued points), provided at any rate we admit or rather
assume that $X_*(\bZ)$ and $X_*'(\bZ)$ are simply connected, which
will be automatic if we work in the category of (schematized) models
$\scrM_1(k)$, the condition \eqref{eq:111.t} above ($X_0=e$) being
replaced by \eqref{eq:111.i} with $n=1$, i.e., by
\begin{equation}
  \label{eq:111.alpha}
  X_0=X_1=e,\tag{$\alpha$}
\end{equation}
to be on the safe side! Under this extra assumption at any rate, I
feel definitely more confident with the new definition of weak
equivalence, via the homology invariants \eqref{eq:111.zprime}, rather
than the old one. At any rate, the question of the two definitions
being equivalent or not should be cleared up, namely:
\begin{questionnum}\label{q:111.2}
  Let $k$ be any ring, define the category $U(k)^\bullet$ of ``pointed
  unipotent bundles over $k$'' in terms of projective $k$-modules,
  with maps defined as in \eqref{eq:111.pprimebis} page
  \ref{p:453}. (This is equally the correct set-up for
  question~\ref{q:111.1} on page \ref{p:452}, besides the extra
  condition $k=\bZ$, as we saw before.) Let
  \[u:X_*\to X_*'\]
  be a map of semisimplicial objects in $U(k)$, satisfying both the
  extra assumptions \eqref{eq:111.alpha} above. Then is it true that
  the corresponding map
  \begin{equation}
    \label{eq:111.beta}
    X_*(k)\to X_*'(k)\tag{$\beta$}
  \end{equation}
  of semisimplicial sets is a weak equivalence, if{f} the map
  \[L\subpt(X_*)\to L\subpt(X_*')\]
  is, i.e., if{f} $u$ induces an isomorphism for the homology
  invariants $\mathrm H_i$ defined in \eqref{eq:111.zprime} above (via
  the abelianization functor $L\subpt=\Gamma_k$).
\end{questionnum}

If instead of \eqref{eq:111.alpha} we only assume $X_0=X_0'=e$, it
seems that we may have to throw in some other extra condition on $X_*$
and $X_*'$, in order for the definition of weak equivalence in terms
of the mere homology invariants to be reasonable -- a condition which,
at the very least, and in case $k=\bZ$ say, should ensure that the
homotopy types defined by the two terms in \eqref{eq:111.beta} should
have \emph{abelian fundamental groups} (which doesn't look at first
sight to be automatic). At any rate, the conditions
\eqref{eq:111.alpha} above, which should be viewed as a
$1$-connectedness assumption, are natural enough, and it is natural
too to try first to push through a theory of schematization of
homotopy types, under this assumption.

\bigbreak

\noindent\hfill\ondate{26.8.}\pspage{460}\par

% 112
\hangsection[Postnikov dévissage and Kan condition for schematic
\dots]{Postnikov dévissage and Kan condition for schematic
  complexes.}\label{sec:112}%
I am continuing the ``wishful thinking'' about schematization of
homotopy types -- a welcome break in the ``overall review'' on
linearization (in the context of the modelizer \Cat), which had been
getting a little fastidious lately!

I'll admit, as one firm hold in all the wishfulness, that in the
``Eilenberg-Mac~Lane case'' of p.~\ref{p:453}, when moreover $k=\bZ$
and the components $X_i$ of the semisimplicial unipotent bundle $X_*$
are projective, the two maps \eqref{eq:111.w} of page \ref{p:454} are
indeed quasi-isomorphisms. From this should follow the similar
statement, when $X_*$ comes from a chain complex of $k$-modules
$M_\bullet$ with projective coefficients, by reducing to the case when
only a finite number of components of $M_\bullet$ are non-zero (by
suitable passage to the limit), and then by induction on the number of
these components, using the fact that the three terms in
\eqref{eq:111.w} depend on $X_*$ in a ``multiplicative'' way, namely
direct sums being transformed into tensor products. (NB\enspace Under
the assumptions of projectivity made, we may as well express the
quasi-isomorphisms \eqref{eq:111.w} we start with as being
semisimplicial \emph{homotopisms}, and remark that componentwise
tensor-product of such homotopisms is again one.) From this, using the
relevant spectral sequences in homology, should follow that the maps
\eqref{eq:111.w} are still quasi-isomorphisms, whenever $X_*$ can be
``unscrewed'' (``dévissé'') as a finite successive fibering with
fibers of the type $M_*$ as above. Another passage to the limit will
yield the same result for an infinite dévissage, provided the fibers
$M(i)_*$ ($i=1,2,\ldots$) are ``way-out'', i.e., for given $n$, only a
finite number of components $M(i)_n$ are non-zero (it amounts to the
same to demand that the sequence $M(i)_\bullet$ of corresponding chain
complexes be ``way out''). This will give already, it seems, a fair
number of cases when \eqref{eq:111.w} are
quasi-isomorphisms. (Admittedly, working this out will involve a fair
amount of work, especially for getting the relevant properties of
$L_\bullet$ and $L\subpt$, which should mimic very closely the known
ones for usual linearization, including spectral sequences or, more
neatly, transitivity isomorphisms in the relevant derived
categories\dots) The main point here is that those special types of
$X_*$'s (we may call them \emph{Kan-Postnikov complexes} in $U(k)$)
are enough in order to modelize, via the corresponding semisimplicial
sets $X_*(\bZ)$, arbitrary pointed homotopy types with abelian
$\pi_1$. This is seen of course using Postnikov\pspage{461} dévissage
of a given homotopy type, and replacing every homotopy group $\pi_i$
by a shifted projective resolution (a two-step resolution will do
here) $M(i)_\bullet$, as indicated on page \ref{p:456}, and using the
corresponding semisimplicial \bZ-modules $M(i)_*$ as fibers, in the
successive fiberings. To see that what is being done on the
``discrete'' level, working with semisimplicial \emph{sets}, can be
``followed'' in an essentially unique way on the ``schematic'' level,
we hit now of course upon the key difficulty, pointed out on page
\ref{p:444}, about the Postnikov cohomology group $\mathrm
H^{n+2}(X(n+1),\pi_{n+1})$ being isomorphic to the corresponding
``schematic'' group. Now, that this is so indeed should follow from
the homology isomorphisms \eqref{eq:111.w} I hope, just dualizing the
result to cohomology. All this seems to sound kind of reasonable, it
seems, even that for a given homotopy type in $\HotOf_0$, we should be
able to squeeze out this way a unique \emph{isomorphism type}, at any
rate, of semisimplicial unipotent bundles -- but to see whether it
does work, or if there is some major blunder which turns the whole
into nonsense, will come out only from careful, down-to-earth work,
which I am not prepared to dive into.

It occurred to me that the ``Kan-Postnikov'' complexes in $U(k)$ have
some special features among all possible complexes with $X_0=e$, and
also that some extra feature \emph{are} needed, if we want the maps in
\eqref{eq:111.w} to be quasi-isomorphisms. I want to dwell upon this a
little. First of all, the condition $X_0=e$ is indeed essential, as we
see by taking a constant complex with value $X_0$, then the homology
of the three chain complexes \eqref{eq:111.w} reduces to degree zero,
and the $\mathrm H_0$'s are respectively
\[\bZ^{(X_0(\bZ))}=\bZ^{(M_0)}, \quad
L_0(X_0)=\Gamma_bZ\uphat(M_0), \quad
L\subpt(X_0)=\Gamma_\bZ(M_0),\]
where $M_0$ is the \bZ-module giving rise to $X_0$ -- and none of the
two maps is an isomorphism, unless $M_0=0$.

Take now the next simplest case, when $X_*$ comes from a monoid object
$G$ in $U(k)$ in the usual way; then what we are after, in dual terms
of cohomology rather than homology (taking the dual complexes of those
in \eqref{eq:111.w}), amounts essentially to asking whether the usual
discrete cohomology of the discrete monoid $G(\bZ)$ can be computed,
using \emph{polynomial cochains} rather than arbitrary ones. Now, this
we did admit as ``well-known'' in the most evident case of all, when
$G$ is being represented as an object of $U(\bZ)$ by a projective
(hence free) \bZ-module $M$, the multiplication law is just usual
addition. It still looks reasonable enough when the monoid $G$ is a
\emph{group}, with $M$ of finite type say. In this case, the Borel
theory of algebraic\pspage{462} affine groups over a field (here, the
field of fractions \bQ{} of \bZ) tells us that $G_\bQ$ is a
\emph{nilpotent} algebraic group, and that therefore it admits a
composition series with factors isomorphic to the additive group
$\bG_{\mathrm a \bQ}$; presumably, the same dévissage then can
be obtained over the base \bZ, and using induction on the length of
the composition series, and the Hochschild-Serre type of relations
(traditionally expressed by a spectral sequence) between group
cohomology of a group, quotient group and corresponding subgroup, we
should get the wished for quasi-isomorphisms \eqref{eq:111.w}.

Take now, however, the simplest case of a monoid which isn't a group,
namely the multiplicative law on the affine line, given by the
polynomial law
\[W(\bZ) \times W(\bZ) \to W(\bZ) : (x,y) \mapsto xy.\]
the corresponding ``discrete'' monoid is just $\bZ^{(\times)}$, namely
the integers with multiplication, its $\mathrm H^1$ with coefficients
in \bZ{} is just the group of all homomorphisms
\begin{equation}
  \label{eq:112.star}
  \bZ^{(\times)} \to \bZ^{(+)} = \bZ,\tag{*}
\end{equation}
and denoting by \bP{} the set of all primes and using the prime
decomposition of integers, we find that
\[\Hom(\bZ^{(\times)}, \bZ^{(+)}) \simeq \bZ^\bP,\]
i.e., a family $(n_p)_{p\in\bP}$ of integers being associated the
homomorphism
\[\pm\prod_{p\in\bP} p^{\alpha_p} \mapsto \sum_{p\in\bP}\alpha_p
n_p.\]
On the other hand, the schematic $\mathrm H^1$ consists of all
homomorphisms \eqref{eq:112.star} that can be expressed by a
polynomial, hence induce a homomorphism of algebraic group schemes
$\bG\subm\to\bG\suba$, and it is well-known (and immediately checked)
that there is only the zero homomorphism!

Thus, it turns out that the assumptions made yesterday on $X_*$, in
order for the ``linearization theorem'' (!) to hold, namely the maps
\eqref{eq:111.w} to be quasi-isomorphisms, are definitely not strong
enough yet! One may think of throwing in the extra condition $X_1=e$,
so as to rule out monoids altogether (and even groups, too bad!), but
I don't think this helps at all (didn't try though to make a
counterexample). On the other hand, just restricting to Kan-Postnikov
complexes seems rather awkward, we definitely don't want to drag along
Postnikov fibrations as a compulsory ingredient of the complexes we
work with. The idea which comes up here is just to ``drop Postnikov
and keep Kan'' -- namely \emph{introduce a Kan type condition on
  semisimplicial complexes in $U(k)$}. If we mimic\pspage{463}
formally the usual ``discrete'' Kan condition, we get that (for given
pair of integers $k,n$ with $0\le k\le n$) a certain map from $X_n$,
to a certain finite projective limit defined in terms of the boundary
maps $X_{n-1}\to X_{n-2}$, should be epimorphic. Now, clearly $U(k)$
is by no means stable under fiber products, except under very special
assumptions (including differential transversality conditions, at any
rate), and on the other hand one feels that the notion of
``epimorphism'' one will have to work with in $U(k)$ will have to be a
lot more exacting than the map $X(k)\to Y(k)$ on sections being
surjective, or the usual categorical meaning within $U(k)$, which
looks kind of silly here. Even the most exacting surjectivity
condition on $X\to Y$, namely that it admit a section doesn't quite
satisfy me -- what I really want is that $X$ should be a trivial
bundle over $Y$, more specifically that $X$ is isomorphic to a product
$Y\times Z$, in such a way that the given map $X\to Y$ identifies with
the projection $Y\times Z\to Y$. Maybe this is too exacting a
condition, however, and hard to check in computational terms sometimes
(?), maybe we should be content with demanding only that $X\to Y$ has
a section, and moreover is ``smooth'', i.e., has everywhere a
surjective tangent map (which may be expressed on the corresponding
sheaf on $\Aff_{/k}$ by the familiar condition of ``formal
smoothness'', namely possibility of lifting sections over arbitrary
infinitesimal neighborhoods\dots). We'll have to choose at any rate
some such strong ``surjectivity'' notion in $U(k)$, which we'll call
``\emph{submersions}'' say. Thus, I feel a ``Kan complex'' in $U(k)$
should have boundary maps which are submersions. What we should do, is
to pin down some simple ``Kan condition'' on a complex $X_*$, in terms
of ``submersions'', in such a way as to ensure, at any rate
\begin{enumerate}[label=\alph*)]
\item\label{it:112.a}
  that for any pair $(n,k)$ with $0\le k\le n$, the object $X_*(n,k)$
  of ``horns of type $(n,k)$ of $X_*$'', expressed by the suitable
  finite inverse limits (in terms of boundary maps $X_{n-1}\to
  X_{n-2}$) is representable in $U(k)$, and
\item\label{it:112.b}
  the canonical map $X_n\to X_*(n,k)$ is a submersion,
\end{enumerate}
and such of course that all Kan-Postnikov complexes should satisfy
this Kan condition, at the very least.

The first non-trivial case, in view of $X_0=e$, is $n=2$, in which
case $X_*(2,k)$ is trivially representable by $X_1\times X_1$, and the
condition we get is that the three natural maps coming from boundary
maps
\begin{equation}
  \label{eq:112.starstar}
  \begin{tikzcd}[sep=small]
    X_2 \ar[r]\ar[r,shift left=1.5]\ar[r,shift right=1.5] & X_1\times X_1
  \end{tikzcd}\tag{**}
\end{equation}
should\pspage{464} be submersions for a ``\emph{Kan complex}''. In
case $X_*$ is defined by a monoid object $G$ as above, this clearly
implies that $G$ is a group -- which rules out the counterexample
above!

Of course, the very first thing we'll expect from a ``good notion'' of
Kan complexes in $U(k)$, is that for $k=\bZ$, it should make the
linearization theorem work, namely that maps in \eqref{eq:111.w}
p.~\ref{p:454} are quasi-isomorphisms. The next thing, very close to
this one but for arbitrary ground ring $k$ now, is that a map
\[X_* \to X_*'\quad \text{(with $X_0=X_0'=X_1=X_1'=e$)}\]
of Kan complexes in $U(k)$ is a homotopism if{f} it induces an
isomorphism on the homology modules \eqref{eq:111.zprime}
(p.~\ref{p:458}) -- which sounds reasonable precisely because we are
working with Kan complexes. If this is so, the homotopy category
$\HotOf_1(k)$ of $1$-connected homotopy types over $k$ may be
identified with a category of Kan complexes ``up to homotopy'', as
usual (but working now with complexes of unipotent bundles over
$k$). Third thing, still over arbitrary ground ring, would be a
development of the usual homotopy formalism in the unipotent context,
including (one hopes) homotopy fibers of maps, and Postnikov
dévissage. Again, it is hard to imagine how to get such dévissage,
without getting hold inductively of homotopy invariants $\pi_i$ which
are $k$-modules. This should come out if we are able to define
homotopy fibers as for an $(n-1)$-connected $k$-homotopy type (defined
here as one whose homology invariants $\mathrm H_i$ are zero for $i\le
n-1$), $\pi_n$ should be no more, no less than $\mathrm H_n$, which is
indeed a $k$-module. Coming back to $k=\bZ$ again, this should imply
that for $n\ge1$ at any rate, the canonical functor
\begin{equation}
  \label{eq:112.starstarstar}
  \HotOf_n(\bZ)\to\HotOf_n\tag{***}
\end{equation}
induces a bijection between isomorphism classes of schematic and
ordinary $n$-connected homotopy types -- and it will be hard to
believe this can be so, without this functor being actually an
equivalence of categories -- the expected apotheosis of the theory!
Maybe to this end, one may even be able to introduce reasonable
internal $\bHom$'s within the category of schematic Kan complexes, in
a way compatible with the familiar notion in the discrete set-up.

If\pspage{465} it is possible indeed to construct Postnikov dévissage
of a schematic Kan complex over any ground ring $k$, it is clear that
this is compatible with restriction of ground ring, hence it would
seem that formation of the homotopy invariants $\pi_i$ is compatible
with restriction of rings (whereas, as we noticed yesterday, the same
does definitely \emph{not} hold for the homology invariants $\mathrm
H_i$). Taking restriction to the ground ring \bZ, this shows that the
canonical functor \eqref{eq:112.starstarstar} from schematic to
discrete homotopy types is compatible with taking homotopy groups (but
not with homology) -- thus, the relation between $X_*$ and the complex
of sections $X_*(k)$ seems to be a rather close one, via the homotopy
groups, which are the same (and thus, the homotopy groups of $X_*(k)$
seem to turn out to be $k$-modules after all!). By the way, speaking
of ``restriction of ground ring'' for Kan complexes was a little
hasty, in view of the projectivity condition on the components, which
a priori seems to oblige us to assume $k$ to be a projective
$k_0$-module (for a given ring homomorphism
\[k_0\to k\quad\text{).}\]
Still, the remark about the sections functor $X_*\to X_*(k)$ makes
sense, without having to assume $k$ to be a projective \bZ-module!
Also, we feel that, by analogy of what can be done in the linear
set-up, when we define a total derived functor
\[\D_\bullet(\AbOf_k) \to \D_\bullet(\AbOf_{k_0})\]
without any assumption on the ring homomorphism $k_0\to k$, a notion
of ring restriction for schematic homotopy types should make sense
without any restriction, as was surmised yesterday. As in the linear
case, we should allow ourselves to work with schematic complexes which
are \emph{not} projective, but be prepared to take ``resolutions'' (in
some sense) of such general complexes by the more restricted ones
(with projective components).

There is no such difficulty in the case of the ring extension functor,
which transforms projective bundles over $k_0$ into projective bundles
over $k$. The reflections above suggest that, whereas ring extension
is compatible with taking total homology invariants $\LH_\bullet$, via
the corresponding functor
\[\D_\bullet(\AbOf_{k_0}) \to \D_\bullet(\AbOf_k),\]
it is compatible too with taking homotopy invariants $\pi_i$
separately.

\bigbreak

\noindent\hfill\ondate{27.8.}\pspage{466}\par

% 113
\hangsection[``Soft'' versus ``hard'' Postnikov dévissage, $\pi_1$ as
a group \dots]{``Soft'' versus ``hard'' Postnikov dévissage,
  \texorpdfstring{$\pi_1$}{pi-1} as a group scheme.}\label{sec:113}%
What I was thinking of last night (see last sentence) is that whereas
for total homology (not for the separate $\mathrm H_i$'s) we have the
comprehensive formula
\begin{equation}
  \label{eq:113.A}
  \LH_\bullet(X_* \otimes_k k') \simeq \LH_\bullet(X_*) \Lotimes_k k'
  \quad\text{(in $\D_\bullet(k')$),}\tag{A}
\end{equation}
(where $\Lotimes_k$ denotes the left derived functor of the ring
extension functor for $k\to k'$, and $X_*'=X_*\otimes_k k'$ denotes
ring extension for the semisimplicial unipotent bundle $X_*$), for
homotopy modules we should have the term-by-term isomorphisms
\begin{equation}
  \label{eq:113.B}
  \pi_i(X_*\otimes_k k')\leftarrow \pi_i(X_*) \otimes_k k'.\tag{B}
\end{equation}
This however was pretty rash indeed (it was time to go to sleep I
guess!). Whereas the map on sections
\[X_*(k)\to X_*'(k')\]
does induce a map
\begin{equation}
  \label{eq:113.Bprime}
  \pi_i(X_*) \simeq \pi_i(X_*(k)) \to \pi_i(X_*')\simeq\pi_i(X_*'(k'))
  \tag{B'} 
\end{equation}
which surely is $k$-linear, and hence induces a map \eqref{eq:113.B},
this map is certainly \emph{not} an isomorphism without some flatness
restriction either on $k'$ over $k$, or on the $k$-modules
$\pi_j(X_*)$ for $j<i$, as we had noted already three days ago when
looking at the case when $X_*$ comes from a chain complex $M_\bullet$
in $(\AbOf_k)$ (with projective coefficients say), and hence $X_*'$
comes from
\[M_\bullet' = M_\bullet \otimes_k k'.\]
If we look at the description of the \emph{$k$-module}
$\pi_i(X_*)=\pi_i$ in terms of a Postnikov dévissage of $X_*$, we
should recall that the semisimplicial group object $M(i)_*$ which
enters into the picture as the $i$'th step fiber is \emph{not} the one
defined \emph{directly} (via the Kan-Dold-Puppe functor) by $\pi_i$
placed in degree $i$, but rather by the chain complex $M(i)_\bullet$
\emph{with projective components}, obtained by taking first a shifted
projective resolution of $\pi_i$. Thus, by ring extension we get from
this dévissage of $X_*$ another one of $X_*'$, whose successive fibers
are
\[M(i)_*' = M(i)_* \otimes_k k',\]
corresponding to the chain complexes
\[M(i)_\bullet' = M(i)_\bullet \otimes_k k'.\]
The\pspage{467} latter has $\pi_i\otimes_k k'$ as homology module in
degree $i$ (and zero homology in degree $j<i$), but the homology
modules in degree $j>i$ need not be zero. In other words, there is a
canonical \emph{augmentation}
\begin{equation}
  \label{eq:113.Bdblprime}
  M(i)_*' \to K(\pi_i\otimes_k k', i),\tag{B''}
\end{equation}
(where the second member is the semisimplicial $k'$-module defined by
an $i$-shifted projective resolution of $\pi_i(X)\otimes_k k'$), but
this augmentation need not be a quasi-isomorphism, unless we make the
relevant flatness assumptions. To sum up, \emph{the dévissage of}
\[X_*' = X_*\otimes_k k'\]
\emph{deduced from a Postnikov dévissage of $X_*$ is \emph{not} a
  Postnikov dévissage of $X_*'$}, unless we assume either $k'$ flat
over $k$, or the $k$-modules $\pi_j(X_*)$ flat. This at the same time
solves the puzzle raised on page \ref{p:444}, and points towards a
\emph{serious shortcoming of the \textup(usual\textup) Postnikov
  dévissage -- namely that it is not compatible with ground ring
  extension}, or, as we would say in the language of algebraic
geometry, that this construction is not ``geometric'' -- a harsh thing
to say indeed!

At this point the idea comes up that we may define another dévissage,
a lot more natural in the spirit of a theory of ``abelianization'' of
homotopy types it would seem, and which is ``geometric'', namely
compatible with ring extension. Here, we'll have to work, though, with
the ``prohibitive'' abelianization functor
\[U(k)\to\AbOf_k, \quad
X\mapsto L(X)\; (=\Gamma\uphat_k(M))\]
(where $M$ is a $k$-module ``representing'' the object $X$ in $U(k)$),
as we'll need the functorial embedding\scrcomment{see also page
  \ref{p:474}}
\begin{equation}
  \label{eq:113.C}
  X\to W(L(X))\tag{C}
\end{equation}
(where
\[W : \AbOf_k\to U(k)\]
denotes the canonical functor from $k$-modules to unipotent
$k$-bundles); this is licit anyhow, if we admit that the canonical map
\[L\subpt(X_*) \to L(X_*)\]
(cf.\ page \ref{p:454}~\eqref{eq:111.w}) is a quasi-isomorphism, i.e.,
a weak equivalence, for the complexes $X_*$ we are working with. (This
of course should hold over an arbitrary ground ring, not just \bZ.)
Applying \eqref{eq:113.C} componentwise, we get for a complex of
bundles $X_*$ a canonical map into its\pspage{468} abelianization
\begin{equation}
  \label{eq:113.Cprime}
  X_*\to W(L(X_*)).\tag{C'}
\end{equation}
Postnikov's construction, for $(n-1)$-connected $X_*$, consists in
composing this map with the ``augmentation''
\[W(L(X_*)) \to K(\pi_n,n)\]
(where $\pi_n\simeq\mathrm H_n$ is the first possibly non-trivial
homology module of the chain complex corresponding to $L(X_*)$), and
after this only take the homotopy fiber, and iterate (the homotopy
fiber will be $n$-connected now). This process, in a way, breaks the
natural abelianization into pieces, a brutal thing to do one will
admit, all the more so as we start with a beautiful complex with
projective components, and kind of destroy its unmarred harmony by
tearing out of breaking off the most showy part, $\mathrm H_n$ to name
it, which now looks so lost and awkward we really can't just leave it
as it is, we first have to take a projective resolution of it,
choosing it as we may\dots But we won't do all this, will we, and
rather keep abelianization and \eqref{eq:113.Cprime} as God gave them
to us, and take the homotopy fiber (we hope God will give this
too\dots), and repeat the process, without even having to care at any
stage which $\pi_i$'s vanish and which not. Let's call this the
``\emph{soft Postnikov dévissage}'', in contrast to the ``brutal
one''. In describing the process (which of course makes sense in the
discrete context as well as in the schematic one), I implicitly
admitted that the $X_*$ we start with is $1$-connected, or for the
very least has abelian $\pi_1$ (a notion we'll have to come back to,
in the schematic set-up). But we may as well apply it to a (discrete)
$K(G,1)$ type, $G$ any discrete group, then it amounts to taking the
descending filtration of $G$ by iterated commutator groups, which is a
finite filtration if{f} $G$ is solvable. Maybe it would be more
natural still to take the similar descending filtration, suitable for
the study of \emph{nilpotent} groups rather than solvable ones,
with\scrcomment{this second superscript in this equation is rather
  hard to read in the typescript\dots}
\begin{equation}
  \label{eq:113.D}
  G^{(n+1)} = [G, G^{(n-1)}],\tag{D}
\end{equation}
where $[A,B]$ denotes the subgroup of $G$ generated by commutators
\[(a,b)=aba^{-1}b^{-1},\]
with $a$ in $A$, $b$ in $B$. It doesn't seem there is a similar
distinction to make in case we start with a $1$-connected $X_*$, more
specifically if $X_0=X_1=e$. There may be some extra caution needed,
however, when we assume only $X_0=e$ without assuming
$1$-connectedness, even when $\pi_1$ is abelian,\pspage{469} because
of the possibility of operation of $\pi_1$ upon the $\pi_i$'s. Maybe,
when trying to modelize usual homotopy types by complexes of unipotent
bundles over \bZ, we should restrict to homotopy types which are not
only $0$-connected and have abelian $\pi_1$, but moreover with $\pi_1$
operating trivially (or for the very least, in a unipotent way) upon
the higher $\pi_i$'s. At any rate, as soon as $\pi_1$ operates
non-trivially (on itself, or on the higher $\pi_i$'s) there will
presumably be two non-equivalent ways for defining soft Postnikov
dévissage, corresponding to the two standard descending commutator
group series in a discrete group $G$ The more relevant in view of
unipotent schematization would seem to be the ``nilpotent'' one.

Restricting for simplicity to the $1$-connected case $X_0=X_1=e$, I
would expect soft Postnikov dévissage to be the key for an
understanding as well of the behavior of the $\pi_i(X_*)$ modules with
respect to ring extension, as of the full relationship between these
invariants, and the homology invariants $\mathrm H_i(X_*)$.

\starsbreak

I still should have a look upon complexes $X_*$ in $U(k)^\bullet$
satisfying (as always in this game) $X_0=e$, but not necessarily
$X_1=e$. Even when we make the Kan assumption (plus
``\emph{smoothness}'' of the components, by which I mean that their
$k$-linearization is projective), I don't feel too sure yet if they
fit into a good formalism, for instance (when $k=\bZ$) if they satisfy
the ``linearization theorem'' (quasi-isomorphy for the two maps in
\eqref{eq:111.w} p.~\ref{p:454}). If we start for instance with a
group object $G$ of $U(k)$ and let $X_*$ be the corresponding
semisimplicial complex in $U(k)^\bullet$, then we get an isomorphism
\begin{equation}
  \label{eq:113.E}
  \pi_1(X_*) \eqdef \pi_1(X_*(k)) \simeq G(k),\tag{E}
\end{equation}
which shows us that $\pi_1(X_*)$ need not be abelian even when
$k=\bZ$. If we assume that $X_1=G$ is ``of finite presentation''
(namely the projective $k$-module which describes $X_1$ is of finite
presentation), or what amounts to the same, representable by an actual
(group) scheme, it is true, however, that $G$ and hence $\pi_1=G(k)$
is \emph{nilpotent} (this holds for any $k$). It looks an intriguing
question whether $\pi_1$ is nilpotent under the only assumption that
$X_*$ is a smooth Kan complex with $X_1$ a scheme (without assuming
anymore $X$ comes from a group object). At any rate, it follows
from\pspage{470} the Kan condition that $\pi_1$ may be interpreted as
a quotient set of $E \eqdef X_1(k)$ (without having to pass to the
full free group generated by this set), with a set of relations
\begin{equation}
  \label{eq:113.F}
  z_i = x_iy_i, \quad \text{$i$ in $X_2(k)=I$},\tag{F}
\end{equation}
indexed by $X_2(k)$, where
\[i\mapsto x_i, \quad i\mapsto y_i, \quad i\mapsto z_i\]
are the three boundary maps, remembering moreover the Kan condition
that the three maps
\[\begin{tikzcd}[cramped,sep=small]
  I\ar[r]\ar[r,shift left=1.5]\ar[r,shift right=1.5] & E\times E,
\end{tikzcd}\quad
i\mapsto(x_i,y_i), \quad i\mapsto(x_i,z_i),\quad i\mapsto(y_i,z_i)\]
are \emph{surjective}, which implies indeed that any element of the
group $\pi$ described by the set of generates $E$ and relations
\eqref{eq:113.F} comes from an element in $E$. Replacing $k$ by any
$k$-algebra $k'$, we see that we have a presheaf
\[k' \mapsto \pi_1(X_*(k')) = \pi_1(X_*\otimes_k k')\]
on the category $\Aff_{/k}$ of affine schemes over $k$, with values in
the category of groups, which may be viewed (as a presheaf of sets) as
a quotient presheaf of the presheaf on $\Aff_{/k}$ defined by
$X_1$. We feel that this presheaf will fit into a reasonable
``schematic'' set-up, only if it turns out to be a \emph{sheaf}, and
more exactingly still, if this sheaf is isomorphic (as a sheaf of
sets) to one stemming from an object of $U(k)$, i.e., if it is
isomorphic to a sheaf $W(M)$, for suitable $k$-module $M$ (not
necessarily a projective one). If we denote by $G$ this object of
$U(k)$, it will be endowed with a group structure, and \emph{it is
  this group object of $U(k)$, rather than just the set-theoretic
  group of its sections}, i.e., \emph{of $k$-valued ``points'', which
  merits to be viewed as the ``true'' $\pi_1(X_*)$}. To say it
differently, whereas the higher $\pi_i(X_*)\eqdef \pi_i(X_*(k))$ (for
$i\ge2$), in the cases considered so far, should be viewed as being
not mere abelian groups, but moreover endowed with a natural
$k$-module structure, in the case when $i=1$, i.e., for the
fundamental group $\pi_1(X_*)$, the natural structure to expect on
this (possibly non-commutative) group is a ``unipotent schematic''
structure, namely essentially a pointed ``parametrization'' of this
group by elements of a suitable $k$-module $M_1$, in such a way that
the composition law is expressed in terms of a polynomial law, making
sense therefore not only for $k$-valued points, i.e., for elements of
$M$, but for $k'$-valued points as well (for any $k$-algebra $k'$),
namely defining a group law on\pspage{471} $W(M)(k') = M\otimes_k
k'$. If henceforth we denote by $\pi_1(X_*)$ this group object of
$U(k)$, the relevant formula now is
\begin{equation}
  \label{eq:113.G}
  \pi_1(X_*(k')) \simeq \pi_1(X_*)(k'),\tag{G}
\end{equation}
a group isomorphism functorial with respect to variable $k$-algebra
$k'$, which will imply the corresponding isomorphism
\begin{equation}
  \label{eq:113.Gprime}
  \pi_1(X_*\otimes_k k') \simeq \pi_1(X_*)\otimes_k k'\tag{G'}
\end{equation}
of groups objects in $U(k')$, i.e., formation of the ``schematic''
$\pi_1$ is compatible with ground ring extension $k\to k'$ (provided
$\pi_1$ exists, for a given $X_*$).

We will expect the map of passage to quotient
\begin{equation}
  \label{eq:113.H}
  X_1\to \pi_1(X_*)=G\tag{H}
\end{equation}
to be ``epimorphic'' in a very strong sense, stronger even than just
in the sense of presheaves, the first thought that comes to mind here
is that it should be a ``\emph{submersion}'', in the sense suggested in
yesterday's reflections in connection with the description of the Kan
condition. If, however, we want to be able to get for
$\pi_1(X_*(k))=\pi_1(X_*)(k)$\scrcomment{in the typescript there is
  here a tautological equation, but I think this is what was
  meant\dots} \emph{any} abelian group beforehand, in the case $k=\bZ$
say, without demanding that it be a projective $k$-module, and still
get it via an $X_*$ with \emph{smooth} components, this shows that
when defining a notion of ``submersion'' for objects of $U(k)$ which
may not be smooth, we should not be quite as demanding as suggested
yesterday (cf.\ page~\ref{p:463}), but find a definition which will
include also any map $X\to Y$ coming from an epimorphism $M\to N$ of
$k$-modules (which will allow us to take $X_1$ as associated to a
projective $k$-module admitting the given $\pi_1$ as its
quotient). One idea that comes to mind here, is to take this property
as the \emph{definition} of a submersion, as an arrow in $U(k)$ which
is isomorphic to one obtained from an epimorphism in $\AbOf_k$. This,
of the three definitions that have come to my mind so far for this
notion, is the one which looks the most convincing to me. I wouldn't
expect too much from a complex $X_*$, even a smooth one and satisfying
the Kan condition, unless (in terms of the three boundary maps from
$X_2$ to $X_1$) it gives rise, as just explained, to a group object
$G=\pi_1(X_*)$ in $U(k)$, together with a \emph{submersion}
\eqref{eq:113.H}. Thus, definitely, when defining a schematic model
category $\scrM_n(k)$ of $n$-connected ss~complexes of unipotent
bundles over $k$, I feel like insisting in case $n=0$ at least upon
this extra condition (plus of course $X_0=e$). The still more
stringent condition one may think of, in order to have schematic
models as\pspage{472} close as one may wish to $k$-modules, is to
demand that moreover $G$ is isomorphic to the object of $U(k)$ defined
by a $k$-module $M_1$, the group law moreover coming from the addition
law in $M$. This condition is stronger still than merely demanding
that $G$ be commutative, even when $M_1$ is free of rank one, because
one knows that over a non-perfect field $k$ there may be ``forms'' of
the additive group $\bG\suba$ which are not isomorphic to $\bG\suba$;
presumably, there should be similar examples over \bZ{} too, with rank
larger than one, however.

We feel, however, that the case when $G=\pi_1(X_*)$ is a non-linear or
even a non-abelian group object of $U(k)$ is still worthy of
interest. The first test it would seem, to check if we do have a good
notion indeed, is to see if it does satisfy to the ``linearization
theorem'' in case $k=\bZ$, i.e., the maps \eqref{eq:111.w} on page
\ref{p:454} are quasi-isomorphisms. Another key test, which now makes
sense for arbitrary $k$, is whether for a smooth Kan complex in
$\scrM_0(k)$ (i.e., satisfying the extra assumption involving $G$),
$X_*$ is homotopic to a bundle over $K(G,1)$, with a $1$-connected
fiber, or more specifically, a fiber $Y_*$ satisfying
$Y_0=Y_1=e$. Among other features to expect is a natural operation of
the group object $G$ on the $k$-modules $\pi_i(X_*)$, as well as
$\mathrm H_i(X_*)$, If however we wish, for $k=\bZ$, to use models in
$\scrM_0(\bZ)$ for describing possibly homotopy types with nilpotent
$\pi_1$ say,\footnote{and moreover unipotent action on the $\pi_i$'s}
and devise a corresponding equivalence between suitable homotopy
categories, we should first investigate the question of the
relationship between nilpotent discrete groups, and group objects of
$U(k)$ -- a question already touched on earlier in our reflection on
linearization (see end of section \ref{sec:94}), and of separate
interest.

% 114
\hangsection{Outline of a program.}\label{sec:114}%
During these four days of reflection on schematization of homotopy
types, a relatively coherent picture has gradually been emerging from
darkness. How far this image reflects substantial reality, not just
daydreaming, I would be at a loss to tell now. Maybe some substantial
corrections will have to be made still, besides getting in other ideas
for a more complete picture -- I would be amazed at any rate if
everything should turn out as just nonsense! If it doesn't, there is
surely a lot of work ahead to get everything straightened out and
ready-to-use. I will leave it at that I suppose, for the time being --
maybe just finish this digression by a quick review of the set-up, and
of some main questions which have come out.

For\pspage{473} a given ground ring $k$, the basic category we'll use
of ``schematic'' objects over $k$ is the category of \emph{unipotent
  bundles} over $k$, which may be defined as the category of functors
from $\Alg_{/k}$ to \Sets{} isomorphic to functors of the type
\[W(M) = (k' \mapsto M\otimes_k k'),\]
where $M$ is any $k$-module. We do not restrict, here, $M$ to be
projective or flat, as we definitely want to have, for a ring
homomorphism $k\to k'$, a problemless functor ``\emph{restriction of
  scalars}''
\[U(k')\to U(k),\]
inserting in the commutative diagram
\begin{equation}
  \label{eq:114.I}
  \begin{tabular}{@{}c@{}}
    \begin{tikzcd}[baseline=(O.base)]
      \AbOf_{k'}\ar[r,"\text{restr.}"]\ar[d,"W_{k'}"'] &
      \AbOf_k\ar[d,"W_k"] \\
      U(k')\ar[r,"\text{restr.}"] &
      |[alias=O]| U(k)
    \end{tikzcd}.
  \end{tabular}\tag{I}
\end{equation}
Another reason is that we want that the $k$-modules of the type
$\pi_i(X_*)$ which will come out should be eligible for defining
objects in $U(k)$. We are more specifically interested though in
``\emph{smooth}'' objects of $U(k)$, namely those that correspond to
projective $k$-modules. (We prefer to call them ``smooth'' rather than
``projective'', in order to avoid confusion with the notion of a
projective object in the usual categorical sense for $U(k)$.) Another
relevant notion is the notion of a \emph{submersion}, namely a map in
$U(k)$ isomorphic to one coming from an epimorphism $M\to N$ in
$\AbOf_k$. (If the latter can be chosen to have a projective kernel,
we may speak of a smooth submersion.) The ring restriction functor
transforms submersions into submersions, and also smooth objects into
smooth ones provided $k'$ is projective as a module over $k$. We also
have a \emph{ring extension functor} from $U(k)$ to $U(k')$, giving
rise to a diagram (I') similar to \eqref{eq:114.I} above, it
transforms submersions into submersions, smooth objects into smooth
ones.

The smoothness condition is likely to come in in two ways, one is via
\emph{flatness} (we may call an object of $U(k)$ ``flat'' when it
isomorphic to some $W(M)$ with $M$ a flat $k$-module), whereas
projectivity is needed in order to ensure that in certain cases, weak
equivalences are homotopisms. Flatness is the kind of condition which
ensure the validity of ``naive'' universal coefficients formulæ for
homotopy or homology objects, whereas projectivity may be needed in
case of such formulæ for cohomology rather than homology.

The\pspage{474} description I just recalled of $U(k)$ is the one most
intuitive to my mind, other people may prefer the more computational
one on page \ref{p:451} in terms of $\Gamma\uphat_k(M)$ (endowed with
its augmentation to $k$ and its diagonal map), which is of importance
in its own right. It shows the existence of a canonical
\emph{$k$-linearization functor}
\begin{equation}
  \label{eq:114.J}
  L: U(k)\to\AbOf_k,\tag{J}
\end{equation}
giving rise to the commutative diagram (up to can.\ isom.)
\begin{equation}
  \label{eq:114.Jprime}
  \begin{tabular}{@{}c@{}}
    \begin{tikzcd}[baseline=(O.base),column sep=tiny]
      \AbOf_k\ar[rr,"W"]\ar[dr,"\Gamma\uphat_k"'] & &
      U(k)\ar[dl,"L"] \\
      & |[alias=O]| \AbOf_k
    \end{tikzcd},
  \end{tabular}\tag{J'}
\end{equation}
where
\[\Gamma\uphat_k(M) = \prod_{i\ge0} \Gamma^i_k(M).\]
This linearization is not quite compatible with ring extension, it
becomes so only when we view it as a functor with values, not just in
the category $\AbOf_k$ of $k$-modules, but of separated and complete
linearly topologized $k$-modules, the ring extension functor for these
being the \emph{completed} tensor product. This is a little (or big?)
technical drawback for this notion of linearization. We have a
canonical embedding
\begin{equation}
  \label{eq:114.K}
  x\mapsto\exp(x) : X\to W\uphat(L(X))\quad \text{(cf.\ p.\
    \ref{p:450}, \eqref{eq:111.n})}\tag{K}
\end{equation}
functorial in $X$, where for a topological $k$-module $M$ as above,
described as a filtering inverse limit of discrete ones $M_i$, we
define
\begin{equation}
  \label{eq:114.Kprime}
  W\uphat(M) = (k'\mapsto\varprojlim_i M_i\otimes_k k').\tag{K'}
\end{equation}
The map \eqref{eq:114.K} has a universal property with respect to all
possible maps $X\to W\uphat(M)$ with $M$ a linearly topologized
separated and complete $k$-module, which accounts for its role as
``linearization''. It should be noted here that the map
\eqref{eq:113.C} on page \ref{p:467} doesn't quite exist, we have
corrected this point here -- definitely we cannot in \eqref{eq:114.K}
replace $W\uphat$ by $W$. Of course, linearization $L$ (or its variant
$L\subpt$) doesn't commute in any sense whatever to restriction of
ground ring.

The image of $X$ in $W\uphat(L(X))$ is characterized by the simple
formulæ \eqref{eq:111.q} p.\ \ref{p:451}. Maps from $X$ to $Y$ may be
described as just continuous $k$-linear maps from $L(X)$ to $L(Y)$,
compatible with augmentations and diagonal maps.

We'll\pspage{475} more specifically work in the category
$U(k)^\bullet$ of \emph{pointed} objects of $U(k)$, namely objects
endowed with a section over the final object $e$, the so-called
\emph{pointed unipotent bundles}. We now have a functor
\begin{equation}
  \label{eq:114.L}
  W^\bullet : \AbOf_k \to U(k)^\bullet\tag{L}
\end{equation}
deduced from $W$ using the fact that $W(0)=e$, and a ``\emph{pointed
  linearization functor}''
\begin{equation}
  \label{eq:114.M}
  L^\bullet\text{ or }L\subpt : U(k)^\bullet \to \AbOf_k,\tag{M}
\end{equation}
giving rise to a commutative diagram similar to \eqref{eq:114.Jprime}
\begin{equation}
  \label{eq:114.Mprime}
  \begin{tabular}{@{}c@{}}
    \begin{tikzcd}[baseline=(O.base),column sep=tiny]
      \AbOf_k\ar[rr,"W^\bullet"]\ar[dr,"\Gamma_k"'] & &
      U(k)^\bullet\ar[dl,"L^\bullet"] \\
      & |[alias=O]| \AbOf_k
    \end{tikzcd};
  \end{tabular}\tag{M'}
\end{equation}
the notation $L^\bullet$ seems here the most coherent one, but may
bring about confusion with the similar notation for some cochain
complex say, therefore we had first used the alternative notation
$L\subpt$, to which one may still come back if needed. This time the
functor $L^\bullet$ commutes to ring extension without any grain of
salt. We have of course a canonical embedding
\begin{equation}
  \label{eq:114.N}
  L^\bullet(X)\hookrightarrow L(X)\tag{N}
\end{equation}
defined via the corresponding embedding for an object $M$ of $\AbOf_k$
\begin{equation}
  \label{eq:114.Nprime}
  \Gamma_k(M)\hookrightarrow \Gamma\uphat_k(M),\tag{N'}
\end{equation}
by which $L(X)$ may be viewed as the completion of $L^\bullet(X)$ with
respect to the topology it induces on it, which is a \emph{canonical
  topology} on $L^\bullet(X)$. Maps in $U(k)^\bullet$ correspond to
\emph{$k$-linear maps}
\[L^\bullet(X)\to L^\bullet(Y)\]
which are moreover \emph{continuous, and compatible with
  coaugmentation} (i.e., transforms $1$ into $1$) \emph{as well as
  with augmentations and diagonal maps}. I wonder if there is any
simple characterization of submersions in $U(k)^\bullet$ in terms of
the corresponding map between the linearizations. At any rate, an
object $X$ of $U(k)^\bullet$ is smooth resp.\ flat if{f}
$L^\bullet(X)$ is a projective resp.\ a flat $k$-module.

For any natural integer, we want now to define a model category
\begin{equation}
  \label{eq:114.O}
  \scrM_n(k)\subset\bHom(\Simplexop, U(k)^\bullet),\tag{O}
\end{equation}
which should be a full subcategory of the category of
semisimplicial\pspage{476} objects in $U(k)^\bullet$. We'll get a
functor
\begin{equation}
  \label{eq:114.P}
  X_* \mapsto X_*(k) :
  \scrM_n(k)\to\bHom(\Simplexop,\mathrm{Sets}^\bullet) \tag{P} 
\end{equation}
from this category to the category of semisimplicial pointed sets. For
$n\ge1$, the only condition, it seems, to impose upon $X_*$ in the
second member of \eqref{eq:114.O}, i.e., upon a semisimplicial pointed
unipotent bundle over $k$, in order to belong to $\scrM_n(k)$, is
\begin{equation}
  \label{eq:114.Qn}
  X_i = e\quad\text{for}\quad i\le n.\tag{Q\textsubscript{n}}
\end{equation}
This, for $n=0$, reduces to the common condition
\begin{equation}
  \label{eq:114.Qzero}
  X_0=e,\tag{Q\textsubscript{0}}
\end{equation}
which definitely is not enough, though, to get a category of
``models'' $\scrM_0(k)$ whose objects should have the kind of
properties we are after. There are various kinds of extra restrictions
one may want to impose, according to the type of situations one wants
to describe, some hints along these lines are given on pages
\ref{p:469}--\ref{p:472}. For a preliminary study, the case $n\ge1$,
and more specifically, the case $n=1$, is quite enough, the latter
corresponding to the restrictions
\begin{equation}
  \label{eq:114.Qone}
  X_0=X_1=e.\tag{Q\textsubscript{1}}
\end{equation}

From $P$ we get a functor
\begin{equation}
  \label{eq:114.Pprime}
  \scrM_n(k)\to\HotOf_n^\bullet =
  \begin{tabular}[t]{@{}l@{}}
    category of pointed $n$-connected\\
    homotopy types,
  \end{tabular}\tag{P'}
\end{equation}
we define a map in $\scrM_n(k)$ to be a \emph{weak equivalence} if its
image by \eqref{eq:114.P} is, i.e., its image by \eqref{eq:114.Pprime}
is an isomorphism, and localizing by weak equivalences we get the
category
\begin{equation}
  \label{eq:114.R}
  \HotOf_n(k)\tag{R}
\end{equation}
of ``$n$-connected schematic homotopy types over $k$'', together with
a ``sections functor'' induced by \eqref{eq:114.P}
\begin{equation}
  \label{eq:114.Rprime}
  \HotOf_n(k)\to\HotOf_n.\tag{R'}
\end{equation}
One main point in our definitions is that \emph{we hope this functor
  to be an equivalence of categories, in the case when $k=\bZ$}, and
of course $n\ge1$.

The description just given of categories $\HotOf_n(k)$ is suitable for
defining functors of restriction of ground ring for $k\to k'$
\begin{equation}
  \label{eq:114.S}
  \HotOf_n(k')\to\HotOf_n(k),\tag{S}
\end{equation}
compatible with the sections functor \eqref{eq:114.Rprime} for $k$ and
$k'$. It isn't directly suited, though, for describing ring extension
-- as a matter of\pspage{477} fact, ring extension for homotopy types
(an operation of greater interest than ring restriction surely) is
\emph{not} expressed, in general, by just performing the trivial ring
extension operation
\[X_*\mapsto X_*\otimes_k k'\]
on models in $\scrM_n(k)$, unless we assume $k'$ to be flat over $k$
say -- but even in this case it is by no means clear a priori that the
operation above transforms weak equivalences into weak
equivalences. This is very clearly shown by the linear analogon, the
categories $\scrM_n(k)$ being replaced by the categories of chain
complexes in $\AbOf_k$ say, or by $\Comp^-(\AbOf_k)$ or the like. In
order to correctly describe ground ring extension on homotopy types,
we'll have first to take a suitable ``resolution'' of $X_*$, namely
replace $X_*$ by some $K_*$ say, endowed with a weak equivalence
\[K_*\to X_*,\]
and $K_*$ satisfying some extra assumptions. Maybe flatness of the
components would be enough here. For other purposes, we may have to
use resolutions which are even smooth (componentwise), or which
satisfy a suitable \emph{Kan condition} (or a type outlined on page
\ref{p:463}), or both. Our expectation is that, when we restrict to
the subcategory
\[\mathrm{sK}\scrM_n(k)\]
of the model category $\scrM_n(k)$ made up with smooth Kan complexes,
\emph{that the category $\HotOf_n(k)$ may be described simply in terms
  of such sK-complexes ``up to homotopy''}, as usual. If this is so,
the ground ring extension functor follows trivially from a
corresponding functor on the sK-model categories
\begin{equation}
  \label{eq:114.T}
  \sKM_n(k)\to\sKM_n(k'), \quad X_*\mapsto X_*\otimes_k k',\tag{T}
\end{equation}
hence
\begin{equation}
  \label{eq:114.Tprime}
  \HotOf_n(k)\to\HotOf_n(k').\tag{T'}
\end{equation}

From the sections functor \eqref{eq:114.Rprime} we get homotopy
invariants $\pi_i$ for an object in $\HotOf_n(k)$, but the relevant
$k$-module structure on these is not apparent on this definition. We
have a better hold, via linear algebra over $k$, upon homology
invariants of $X_*$, which are $k$-linear objects, and are definitely
distinct (unless $k=\bZ$) from the homology invariants of $X_*(k)$,
which for general $k$ are definitely of little interest it seems. The
definition of homology goes via the pointed linearization functor
\eqref{eq:114.M}
\begin{equation}
  \label{eq:114.U}
  \left\{\begin{tabular}{@{}l@{}}
    $\LH_\bullet(X_*) \eqdef L\subpt(X_*)$, viewed as an object
      in $\D_\bullet(\AbOf_k)$, \\
    $\mathrm H_i(X_*) \eqdef \mathrm H_i(\LH_\bullet(X_*)) =
      \pi_i(L\subpt(X_*))$ in $\AbOf_k$,
  \end{tabular}\right.\tag{U}
\end{equation}
where\pspage{478} the $\mathrm L$ in $\LH_\bullet$ suggests that we
are taking something similar to a total left derived functor, and
where definitely in the right-hand member we had to write $L\subpt$
and not $L^\bullet$, in order not to get sunk into a morass of
confusion! In the formulæ \eqref{eq:114.U} we should assume however
that $X_*$ is a smooth Kan complex, which will imply (if indeed
$\HotOf_n(k)$ may be described in terms of $\sKM_n(k)$ as said above)
that $\LH_\bullet$ may be viewed as a functor
\begin{equation}
  \label{eq:114.Uprime}
  \HotOf_n(k)\to\D_\bullet(\AbOf_k),\tag{U'}
\end{equation}
and likewise the $\mathrm H_i$'s are functors from $\HotOf_n(k)$ to
$\AbOf_k$. In order to compute these homology invariants for an
arbitrary complex in $\scrM_n(k)$, we'll first have to resolve it by a
$\mathrm{sK}$ complex, and then apply \eqref{eq:114.U}.

We expect \emph{that a map $X_*\to Y_*$ in $\scrM_n(k)$ is a weak
  equivalence if{f} the corresponding map for $\LH_\bullet$ is a
  quasi-isomorphism}, in other words we expect the functor
\eqref{eq:114.Uprime} to be ``conservative'': a map in the first
category is an isomorphism if{f} its image in the second one is. A
second main feature we expect from linearization, is that in the case
$k=\bZ$ it corresponds to the usual abelianization of homotopy
types. This statement, when made more specific as in \eqref{eq:111.w}
page \ref{p:454}, decomposes into two distinct ones. One is of
significance over an arbitrary ring $k$, and states that \emph{for a
  $\mathrm{sK}$ complex $X_*$, the inclusion} (coming from
\eqref{eq:114.N})
\begin{equation}
  \label{eq:114.V}
  L\subpt(X_*)\to L(X_*)\tag{V}
\end{equation}
\emph{from $L\subpt$ into its completion, when viewed as a map of
  chain complexes in $\AbOf_k$} (using the simplicial differential
operator, or passing to the corresponding ``normalized'' chain
complexes first) \emph{is a quasi-isomorphism}. Whether this is always
so or not, or whether noetherian conditions on $k$ or some finiteness
conditions for the components of $X_*$ are needed, looks like a rather
standard question of linear homological algebra! On the other hand,
using the exponential embedding \eqref{eq:114.K} for sections, we get
another map of semisimplicial $k$-modules
\begin{equation}
  \label{eq:114.Vprime}
  k^{(X_*(k))} \to L(X_*),\tag{V'}
\end{equation}
and here \emph{the question again} (the expectation I might say?)
\emph{is whether this is a quasi-isomorphism}. This would just mean
(if coupled with quasi-isomorphy of \eqref{eq:114.V}) that the
homology invariants \eqref{eq:114.U} are just the usual homology
invariants of the discrete homotopy type modeled by $X_*(k)$, but with
coefficients not in \bZ, but in $k$. We certainly do expect this to be
true for $k=\bZ$ -- which was the content of ``question
\ref{q:111.1}''\pspage{479} on page \ref{p:452} (taken up again on
page \ref{p:454} and following). Of course, in case we don't assume
$k=\bZ$, writing $\bZ^{(X_*(k))}$ instead of $k^{(X_*(k))}$ as we did
in \eqref{eq:111.w} p.~\ref{p:454} now looks kind of silly, and the
idea in this $k$-linear context to take $k$-valued homology of
$X_*(k)$ rather than \bZ-valued one is evident enough! However, I was
confused by the misconception that the \emph{internal} homology of
$X_*(k)$ should carry $k$-linear structure, as this was what I
expected too for the invariants $\pi_i(X_*(k))$ (which seems to turn
out to be correct). This misconception was corrected a few pages later
(p.~\ref{p:457}) but still I kept dragging along the silly first
member of \eqref{eq:111.w}. Anyhow, it now just occurs to me that
except in case $k=\bZ$, it is definitely \emph{false} that
\eqref{eq:114.Vprime} is a quasi-isomorphism, except in some wholly
trivial cases. Indeed, let $\mathrm H_i$ be the first non-vanishing
homology invariant \eqref{eq:114.U} of $X_*$ (or more safely still,
take $i=n+1$), then we definitely expect to have a canonical
isomorphism of $k$-modules
\begin{equation}
  \label{eq:114.Vbis}
  \pi_i(X_*(k)) \simeq \mathrm H_i(X_*) = \mathrm H_i\tag{V}
\end{equation}
but we equally have by Hopf's theorem, as the lower $\pi_j$'s of
$X_*(k)$ are zero
\[\pi_i(X_*(k)) = \mathrm H_i(X_*(k),\bZ),\]
hence
\begin{equation}
  \label{eq:114.Vprimebis}
  \mathrm H_i \simeq \mathrm H_i(S,\bZ), \quad\text{where
    $S=X_*(k)$,}\tag{V'}
\end{equation}
which is not compatible with the guess that $\mathrm H_i\simeq\mathrm
H_i(S,k)$ ($\simeq\mathrm H_i\otimes_\bZ k$). Presumably, the
isomorphism \eqref{eq:114.Vprimebis} above is induced by the first map
in \eqref{eq:111.w} above, but (except for $k=\bZ$) we should expect
in \eqref{eq:111.w} to have an isomorphism only for the lowest
dimensional homology groups which are occurring in the two first
members. Anyhow, it appears after all that this map in
\eqref{eq:111.w} is the more reasonable one compared to
\eqref{eq:114.Vprime} above, as it yields an isomorphism on homology
in the key dimension $n+1$, whereas \eqref{eq:114.Vprime} apparently
will practically \emph{never} give an isomorphism.

What is mainly lacking still in this review of the expected main
features of schematization of homotopy types, is description of the
$k$-module structure on the homotopy groups
\begin{equation}
  \label{eq:114.W}
  \pi_i(X_*) \eqdef \pi_i(X_*(k)),\tag{W}
\end{equation}
or preferably still, a direct description of those invariants as
$k$-modules, working within the model category $\scrM_n(k)$. This, as
suggested in yesterday's notes (p.~\ref{p:464}), may be achieved
\emph{by developing a theory of Postnikov dévissage within
  $\scrM_n(k)$ and using \eqref{eq:114.V}} in order to pull ourselves
by the bootstraps, defining homotopy finally in terms of
homology.\pspage{480} At this point it should be noted that the
dévissage we'll have to use here is the ``brutal'' one, which we
frowned upon earlier today! To develop such a formalism, it seems
essential to work with smooth Kan complexes and projective resolutions
of the $k$-modules $\pi_i$ as they appears one by one. Whether we want
to describe ``hard'' or ``soft'' Postnikov dévissage (see
p.~\ref{p:468} for the latter), one common key step is the
\emph{linearization map} coming from the exponential map
\eqref{eq:114.K} applied componentwise
\begin{equation}
  \label{eq:114.Wbis}
  X_* \to W\uphat(L(X_*)),\tag{W}
\end{equation}
which we would like to look upon as defining a homotopy class of maps
in $\scrM_n(k)$
\begin{equation}
  \label{eq:114.Wq}
  X_* \to W(L\subpt(X_*)),\tag{W?}
\end{equation}
where the second member moreover is endowed with its natural abelian
group structure (its components are abelian group objects of $U(k)$
and the simplicial maps are additive). To pass from
\eqref{eq:114.Wbis} to \eqref{eq:114.Wq}, it is felt that the
essential step is that \eqref{eq:114.V} above be a quasi-isomorphism,
hence, applying the functor $W$, we should get a weak equivalence,
hence an isomorphism in the derived category $\HotOf_n(k)$, hence
\eqref{eq:114.Wbis} implies \eqref{eq:114.Wq}. The main flaw in this
``argument'' comes from the $W\uphat$ in the second member of
\eqref{eq:114.Wbis}, which isn't quite the same as $W$
definitely. Thus, some further amount of work will be needed,
presumably, to get \eqref{eq:114.Wq} from \eqref{eq:114.Wbis}. Of
course, we can't possibly just keep \eqref{eq:114.Wbis} as it is, as
for getting dévissage we need a map in $\scrM_n(k)$, whereas the map
\eqref{eq:114.Wbis} is just a map of semisimplicial sheaves on
$\Aff_{/k}$, where the second member is \emph{not} in $\scrM_n(k)$,
i.e., its components are not in $U(k)$. Once we got \eqref{eq:114.Wq}
factoring \eqref{eq:114.Wbis} up to homotopy (NB\enspace of course we
assume $X_*$ to be an $\mathrm{sK}$ complex in all this), we still
need a reasonable notion of homotopy fibers of maps in $\scrM_n(k)$,
in order to push through the inductive step.

Thus, a large part of the weight of the work ahead may well lie upon
\emph{developing the standard homotopy constructions within the model
  category $\scrM_n(k)$}, as contemplated on page \ref{p:464}. This
should be fun, if it can be done indeed! One difficulty here seems to
be that Quillen's standard machines won't work, not ``telles quelles''
at any rate, because of the category $U(k)$ failing to be stable under
finite limits -- it doesn't even have fiber products. But I think I'll
stop my ponderings on schematization here\dots

\bigbreak

\noindent\hfill\ondate{28.8.}\pspage{481}\par

% 115
\hangsection{\texorpdfstring{$L(X)$}{L(X)} as the pro-quasicoherent
  substitute for
  \texorpdfstring{$\scrO_k(X)$}{Ok(X)}.}\label{sec:115}%
For the last four days, while reflecting on ``schematization'', each
time I think I am going to be through with that unforeseen green apple
within an hour or two, and get back to ``l'ordre du jour'' -- and
overnight something else still appears I feel I should still look into
just a little; and there I am again, sure enough, with some extra
reflection on ``schematic linearization'' which I hadn't quite
understood yet, it appears to me now. These last days I had given up
numbering formulas as usual by Arab ciphers (1), (2), etc., as I
didn't want to ``cut'' the numbering of that unending ``review'' of
section \ref{sec:104} to \ref{sec:109} which wasn't quite finished
yet, got it only till formula \eqref{eq:109.136}. But now I will stop
this nonsense with numberings (a), (b) and (A), (B), after all even if
there are in-between ``Arab'' formulas now, this doesn't prevent me,
when it comes to it, to start a ``review'' section with formula (137)
and go on till (1000) if I like\dots  And now to the schematic
linearization functor again, for unipotent bundles!

When writing up that schematization program yesterday, some technical
difficulties appeared at the end (see page before) for a proper
understanding of the relationship between the two linearizations
$L\subpt$ and $L$, in order to define, in a suitable derived category,
a map
\[X_*\to W(L\subpt(X_*))\]
using the canonical term-by-term exponential map
\[X_*\to W\uphat L(X_*).\]
It seems to me that the exact significance of the objects $L(X_*)$ or
$W\uphat L(X_*)$ isn't quite understood yet, and that the confusion
which occurred between which usual kind of linearization we should
compare this with, whether $X\mapsto \bZ^{(X)}$ as I did first, or
$X\mapsto k^{(X)}$ as it occurred to me yesterday (pages
\ref{p:478}--\ref{p:479}) , is quite typical of this lack of
understanding. It now occurred to me that neither term, for a general
ground ring $k$ (namely, not assuming $k=\bZ$), is reasonable, whereas
the reasonable ``usual'' kind of linearization comparing with $L(X)$
(when $X$ is a unipotent bundle or a ss~complex of such) is
\begin{equation}
  \label{eq:115.1}
  X\mapsto \scrO_k^{(X)},\tag{1}
\end{equation}
where $\scrO_k$ is the basic \emph{quasicoherent} sheaf of rings over
$k$, i.e., over $\Aff_{/k}$, given by the tautological functor
\begin{equation}
  \label{eq:115.2}
  \scrO_k: (\Aff_{/k})\op \equeq \Alg_{/k} \to \Rings, \quad
  k'\mapsto k',\tag{2}
\end{equation}
associating\pspage{482} to any affine scheme $S=\Spec(k')$ over $k$
the ring of sections of its usual Zariski structure sheaf, which ring
is canonically isomorphic to $k'$ itself! The operation
\eqref{eq:115.1} is the usual linearization operation with respect to
this sheaf of rings, working in the topos of fpqc sheaves of sets over
$k$ which we described at some length in section \ref{sec:111}
(p.~\ref{p:447}). As I was fearing that working in such a thing would
cause anguish to a number of prospective readers, I took pains to
translate unipotent bundles from the geometric language which is the
suggestive one, to the language of commutative algebra which is more
liable to hide than to disclose geometrical meaning; so much so that
in the process I myself lost contact somewhat with the geometric
flavor, and more specifically still with this basic fact, that in our
context of unipotent bundles and complexes of such, the ``natural''
coefficients for cohomology (such as the $\mathrm
H^{n+2}(X(n)_*,\pi_{n+1})$ groups occurring in Postnikov dévissage)
are by no means ``\emph{discrete}'' ones such as \bZ{} or $k$, but
\emph{quasi-coherent} ones, namely provided by quasi-coherent sheaves
of $\scrO_k$-modules or complexes of such. Thus, in the above
Postnikov obstruction group, $\pi_{n+1}$ does \emph{not} stand as a
constant group of coefficients (if it was, this would drag us into the
niceties and difficulties of étale cohomology for the components
$X(n)_i$ of the semisimplicial unipotent bundle $X(n)_*$); but using
the $k$-module structure of $\pi_{n+1}$ for defining a
\emph{quasi-coherent} sheaf of modules $W(\pi_{n+1})$ ``over $k$'',
i.e., over $\Aff_{/k}$, it is this ``continuous'' sheaf (or ``vector
bundle'') over $k$, lifted of course to the various components
$X(n)_i$, which yields the correct answer. This was kind of clear in
my mind the very first day when I started reflecting on
schematization, even before introducing formally unipotent bundles
(pages \ref{p:443}--\ref{p:444}), but this instinctive understanding
later became dulled somewhat, largely due, it seems to me, to the
concession I had made to algebra, giving up to some extent the
language of geometry.

Let's recall that the operation \eqref{eq:115.1} may be defined as the
solution of a universal problem, namely sending the non-linear object
$X$ into a ``linear'' one, namely into a sheaf of $\scrO_k$-modules
(or a $\scrO_k$-module, as we'll simply say). This is expressed by a
canonical map of sheaves of sets
\begin{equation}
  \label{eq:115.3}
  X \to \scrO_k^{(X)}\tag{3}
\end{equation}
(which I am tempted to call the \emph{``exponential'' map} for $X$,
and denote by a corresponding symbol such as $\exp_X$), giving rise,
for every module (over $\scrO_k$) to a corresponding map which is
\emph{bijective}\pspage{483}
\begin{equation}
  \label{eq:115.4}
  \Hom_{\scrO_k}(\scrO_k^{(X)},F) \simeq
  \Hom(X,F)\quad(\simeq\Gamma(X,F_X)),\tag{4} 
\end{equation}
where in the last member (included as a more geometric interpretation
of the second) $F_X$ denotes the restriction of $F$ to the object $X$,
more accurately to the topos (or site) induced on $X$ by the ambient
topos (or site) we are working in. Thus, we may indeed view
\eqref{eq:115.1} as the most perfect notion of linearization, as far
as generality goes -- it makes sense of course in any ringed topos
(without even a commutativity assumption!). The only trouble is that,
even for such a down-to-earth $X$ as a unipotent bundle, the standard
affine line $E_k^1$ say, the sheaf $\scrO_k^{(X)}$ in \eqref{eq:115.1}
is not quasi-coherent and therefore not too amenable it seems to
computations -- thus, we get easily from \eqref{eq:115.4} a canonical
map (for general $X$)
\[k^{(X(k))} \to \Gamma(k,\scrO_k^{(X)}) \quad(=\scrO_k^{(X)}(k))\]
(where the $\Gamma$ in the second member denotes sections over $k$,
i.e., value of a functor on $\Alg_{/k}$ on the initial object $k$, and
remembering in the first member that the ring of sections of $\scrO_k$
is $k$), but I would be at a loss to make a guess as for reasonable
conditions for this map to be an isomorphism! This may seem a
prohibitive ``contra'' against using at all such huge sheaves as
$\scrO_k^{(X)}$, the point though is that in most questions where such
linearizations are introduced (mainly questions where interest lies in
computing cohomology invariants), one is practically never interesting
in taking the groups of sections of these, but rather in looking at
their maps into sheaves of modules $F$ precisely, which is achieved by
\eqref{eq:115.4}, or taking more generally their $\Ext^i$ with such an
$F$, which is achieved by the similar familiar formula
\begin{equation}
  \label{eq:115.5}
  \Ext^i_{\scrO_k}(\scrO_k^{(X)},F) \simeq \mathrm H^i(X,F_X),\tag{5}
\end{equation}
more neatly
\begin{equation}
  \label{eq:115.5prime}
  \RHom_{\scrO_k}(\scrO_k^{(X)},F) \simeq \RGamma_X(F_X),\tag{5'}
\end{equation}
valid of course again for any ringed topos. In the present context
however, the ``coefficients'' $F$ we are interested in, as was just
pointed out, are not arbitrary $\scrO_k$-modules, but rather
\emph{quasi-coherent} ones. Thus, if we get a variant of
\eqref{eq:115.1}
\begin{equation}
  \label{eq:115.6}
  X\to L(X)\tag{6}
\end{equation}
with $L(X)$ some \emph{quasi-coherent} module, giving rise to
\eqref{eq:115.4}, or even to \eqref{eq:115.5} and
\eqref{eq:115.5prime}, this would be for us a perfectly good
substitute for \eqref{eq:115.3}, which would deserve the name of a
``\emph{quasicoherent envelope}'' of $X$. Of\pspage{484} course, this
module $L(X)$ would be unique up to unique isomorphism, as the
solution of a universal problem embodied by \eqref{eq:115.4}, namely
as the quasi-coherent module representing the functor
\begin{equation}
  \label{eq:115.7}
  F\mapsto\Hom(X,F)\simeq\Gamma(X,F_X)\tag{7}
\end{equation}
on the category this time of all \emph{quasicoherent}
$\scrO_k$-modules.

For the unipotent schematization story, we are more specifically
interested in the case when $X$ comes from a quasicoherent module
itself, by forgetting its module structure. Now, as well-known, the
functor
\begin{equation}
  \label{eq:115.8}
  M\mapsto W(M):\AbOf_k=\kMod \to
  \begin{tabular}[t]{@{}l@{}}
    category of quasicoherent\\
    $\scrO_k$-modules, $\Qucoh(k)$ say
  \end{tabular}\tag{8}
\end{equation}
is an equivalence of categories. Thus, for $X$ defined by such an $M$,
the question of representability of \eqref{eq:115.7} within the
category of quasicoherent modules, amounts to the similar question in
$\AbOf_k$ for the functor
\begin{equation}
  \label{eq:115.7prime}
  N\mapsto\Hom(W(M),W(N)),\tag{7'}
\end{equation}
where the $\Hom$ denotes homomorphisms of sheaves \emph{of sets} of
course. Now, as suggested first, somewhat vaguely still, in section
\ref{sec:111} (page \ref{p:450}), we have an alternative expression of
this functor, via
\begin{multline}
  \label{eq:115.9}
  \Hom(W(M),W(N))\simeq\Homcont_k(\Gamma\uphat_k(M),N)\\
  \simeq \varinjlim_i \Hom_k(\Gamma_k(M)(i),N),\tag{9}
\end{multline}
where in the second member, $\Homcont_k$ denotes the set of
$k$-ho\-mo\-mor\-phisms which are continuous on
\begin{equation}
  \label{eq:115.10}
  \Gamma\uphat_k(M) = \prod_{i\ge0}\Gamma^i_k(M)\tag{10}
\end{equation}
(endowed with the product of discrete topologies), and in the third we
have written
\begin{equation}
  \label{eq:115.10prime}
  \Gamma_k(M)(i) = \prod_{j\le i}\Gamma_k^j(M)\tag{10'}
\end{equation}
for the product of the $i$ first factors occurring in
\eqref{eq:115.10}. The map \eqref{eq:115.9} is deduced in the evident
way from the exponential map
\begin{equation}
  \label{eq:115.11}
  M\to W\uphat\Gamma\uphat_k(M) \eqdef \varprojlim_i W(\Gamma_k(M)(i)).\tag{11}
\end{equation}
(NB\enspace The relation between the description \eqref{eq:115.9} of
maps $W(M)\to W(N)$ with the description given p.~\ref{p:451} in terms
of maps \eqref{eq:111.p} from $\Gamma\uphat_k(M)$ to
$\Gamma\uphat_k(N)$, is by associating to such a map $f$ its
composition with the projection of the target upon its factor
$N$\dots). An incorrect way of expressing \eqref{eq:115.9}, which I
slipped into in section \ref{sec:111} and kind of remained in till
now, is by pretending that the $k$-module $\Gamma\uphat_k(M)$
represents the\pspage{485} functor \eqref{eq:115.7prime}, this is
clearly false, as we do not have any canonical map from $W(M)$ into
$W(\Gamma\uphat_k(M))$, only into $W\uphat\Gamma\uphat_k(M)$ -- we
have an embedding
\begin{equation}
  \label{eq:115.12}
  W(\Gamma\uphat_k(M))\hookrightarrow W\uphat\Gamma\uphat_k(M),\tag{12}
\end{equation}
but it is clear that in general, the exponential map \eqref{eq:115.11}
does not factor through the first term in \eqref{eq:115.12}. (It does
of course when we look at sections over $k$ only, but when we go over
to a general $k'$, we hit into the trouble that formation of inverse
limits does not commute with ring extension $\otimes_k k'$!) We may
however express \eqref{eq:115.9} by stating that the functor
\eqref{eq:115.7prime} is ``\emph{prorepresentable}'' by the
\emph{pro-object}
\begin{equation}
  \label{eq:115.13}
  \Pro \Gamma_k(M) \eqdef (\Gamma_k(M)(i))_{i\ge0} \quad
  \text{in $\Pro(\AbOf_k)$,}\tag{13}
\end{equation}
this is even a \emph{strict} pro-object (the transition morphisms are
epimorphisms), which implies that the functor it prorepresents is
representable if{f} this projective system is ``essentially constant''
in the most trivial sense, which means here
\[\Gamma^i_k(M)=0\quad\text{for large $i$,}\]
a condition which presumably is satisfied only for $M=0$! Thus, the
``correct'' interpretation of non-pointed quasi-coherent linearization
seems to me to be the corresponding functor, which I would like now to
call $L_k$ or simply $L$ as before but with slightly different
meaning:
\begin{equation}
  \label{eq:115.14}
  L\text{ or }L_k : U(k)\to \Pro(\AbOf_k) \equeq \Pro(\Qucoh(k)),\tag{14}
\end{equation}
where $\Qucoh(k)$ is defined in \eqref{eq:115.8}. In computational
terms, I would like to view $L(X)$ (for a unipotent bundle $X$) to be
a pro-$k$-module, but in terms of geometric intuition, I would see it
rather as a pro-$\scrO_k$-module, i.e., essentially as an inverse
system of quasicoherent modules. It is in these latter terms that the
construction we just gave generalizes to unipotent bundles over
arbitrary ground schemes, not necessarily affine ones. As for the
``pointed'' quasicoherent linearization functor
\begin{equation}
  \label{eq:115.15}
  L\subpt \text{ or } {L_k}\subpt: U(k)^\bullet \to (\AbOf_k) \equeq \Qucoh(k),\tag{15}
\end{equation}
which I like best to view as taking values $L\subpt(X)$ which are
quasicoherent sheaves, it maps into $L$ by
\begin{equation}
  \label{eq:115.16}
  L\subpt(X)\hookrightarrow L(X),\tag{16}
\end{equation}
interpreting objects of a category as special cases of
pro-objects. We'll denote by
\begin{equation}
  \label{eq:115.17}
  WL(X) \in \Pro(U(k))\tag{17}
\end{equation}
the\pspage{486} pro-unipotent bundle defined in terms of $L(X)$ via
the canonical extension
\[\Pro(W)\text{ or simply } W:\Pro(\AbOf_k)\to\Pro(U(k))\]
of $W$ (cf.~\eqref{eq:115.8}) to pro-objects. Thus, instead of the map
\eqref{eq:115.6} which doesn't quite exist, we get a canonical
``exponential'' map
\begin{equation}
  \label{eq:115.18}
  X\to WL(X)\tag{18}
\end{equation}
in $\Pro U(k)$, which has of course little chance to factor through
\begin{equation}
  \label{eq:115.16prime}
  WL\subpt(X)\to WL(X)\tag{16'}
\end{equation}
deduced from \eqref{eq:115.16} by applying $W$. It is via this map
\eqref{eq:115.18} that we may declare that $L(X)$ prorepresents the
functor \eqref{eq:115.7prime} -- it may be viewed as the universal map
of the type
\[X\to W(N),\]
where now $N$ is (not just a $k$-module, but) a variable object in
$\Pro \AbOf_k$. Whereas the pro-object $L(X)$ is of a ``$k$-linear''
nature and may be viewed as the (\emph{quasi-coherent})
\emph{$k$-linearization of the unipotent bundle $X$}, the pro-object
$WL(X)$ of $U(k)$ has lost its $k$-linear nature, we would rather view
it as the canonical ``abelianization'' of $X$, retaining mainly its
additive structure (plus maybe operation of $k$ on it, which is a lot
weaker, though, than structure of an $\scrO_k$-module\dots).

I would like now to examine if the quasi-coherent pro-object $L(X)$,
which has been obtained as the suitable quasi-coherent substitute for
$\scrO_k^{(X)}$ in order to get the basic isomorphism \eqref{eq:115.4}
for quasicoherent $F$, may serve the same purpose for
$\RHom_{\scrO_k}$, in analogy to \eqref{eq:115.5},
\eqref{eq:115.5prime}. Quite generally, if
\[\Gamma=(\Gamma_\alpha)\]
is any pro-$\scrO_k$-module, let's define for any module $F$
\begin{equation}
  \label{eq:115.19}
  \RHom_{\scrO_k}(\Gamma,F) \eqdef \Hom_{\scrO_k}(\Gamma,C^\bullet(F)),\tag{19}
\end{equation}
where $C^\bullet(F)$ is an injective resolution of $F$ -- thus, the
definition extends the usual one when $\Gamma$ is just an
$\scrO_k$-module. Our expectation now would be
\begin{equation}
  \label{eq:115.20}
  \RHom_\scrOk(L(X),F) \tosim \RGamma_X(F_X),\tag{20}
\end{equation}
giving rise to
\begin{equation}
  \label{eq:115.20prime}
  \Ext_\scrOk^n(L(X),F) \tosim \mathrm H^n(X,F_X),\tag{20'}
\end{equation}
for\pspage{487} any unipotent bundle $X$ over $k$, and any
quasicoherent sheaf
\[F=W(N),\]
where $N$ is any $k$-module. Of course, \eqref{eq:115.19} yields (for
general $\Gamma$)
\begin{equation}
  \label{eq:115.19prime}
  \Ext_\scrOk^i(\Gamma,F)\simeq \varinjlim_\alpha\Ext_\scrOk^i(\Gamma_\alpha,F),\tag{19'}
\end{equation}
so that \eqref{eq:115.20prime} may be rewritten more explicitly, if
$X\simeq W(M)$, as
\begin{equation}
  \label{eq:115.21}
  \mathrm H^n(X,W(N)_X) \simeq \bigoplus_{i\ge0}\Ext_k^n(\Gamma_k^i(M),N).\tag{21}
\end{equation}
At any rate, we have a canonical map \eqref{eq:115.20} in
$\D^+(\AbOf_k)$, hence maps \eqref{eq:115.20prime}, in view of the
isomorphism \eqref{eq:115.5} and the canonical map
\begin{equation}
  \label{eq:115.star}
  \scrO_k^{(X)} \to L(X)\tag{*}
\end{equation}
deduced from \eqref{eq:115.18}, and the question now is whether these
are isomorphisms. We may of course assume in \eqref{eq:115.21}
\[X=W(M),\quad n\ge1.\]
If $M$ is projective, so are the modules $\Gamma_k^i(M)$, and hence
the second member in \eqref{eq:115.21} is zero, so we should check the
first member is too. This is clear when $M$ is of finite type, hence
$X$ representable by an affine scheme, whose quasicoherent cohomology
is well-known therefore to vanish in dim.~$n>0$. The general case
should be a consequence of this, representing $X$ as the filtering
direct limit of its submodules which are projective of finite type --
this should work at any rate when $M$ is free with a basis which is at
most countable, using the standard so-called ``Mittag-Leffler''
argument for passage to limit. Thus, in case $M$ projective,
\eqref{eq:115.21} and hence \eqref{eq:115.20} seems OK indeed. When
$M$ is \emph{not} projective, however, there must be some $k$-module
$N$ such that $\Ext_k^1(M,N)\ne0$, and hence the second member of
\eqref{eq:115.19} is non-zero for $n=1$, which should imply rather
unexpectedly
\[\mathrm H^1(X,F)=\mathrm H^1(W(M),W(N)_{W(M)})\ne0,\]
whereas till this very moment I had been under the impression that
quasi-coherent cohomology of unipotent bundles should be zero, just as
for affine schemes! Maybe it has been familiar to Larry Breen for a
long time that this is \emph{not} so? Maybe also for what we want to
do it isn't really basic to find out whether \eqref{eq:115.21} is true
in full generality, as for the purpose of studying Postnikov type
dévissage, the\pspage{488} unipotent bundles $X$ we are going to work
with will be smooth, i.e., $M$ projective (and we may even get away
with free $M$'s, if we need so). The natural idea here for getting
\eqref{eq:115.20} via \eqref{eq:115.21} in full generality, is to use
a projective resolution of $M$ (even a free one), but I'll not try to
work this out now. The main impression which remains is that for the
more relevant cases (involving cohomology groups of a \emph{smooth}
unipotent bundle $X$ at any rate, with quasicoherent coefficients),
the quasicoherent ``pro''-linearization $L(X)$ is just as good for
computing cohomology invariants, as the forbidding $\scrO_k^{(X)}$
modules we were shrinking from.

If now we take an $X_*$ instead of just $X$, namely a ss~complex in
$U(k)$, and assuming the components $X_n$ to be smooth (to be safe),
the isomorphisms \eqref{eq:115.20prime} should give rise to
isomorphisms (for $F=W(N)$)
\begin{equation}
  \label{eq:115.22}
  \mathrm H^n(X_*,F) \simeq \Ext_\scrOk^n(L(X_*),F) \simeq
  \Ext_k^n(L(X_*),N),\tag{22} 
\end{equation}
where the $\Ext^n$ should be viewed as hyperext functors (not
term-by-term), and where in the last member $L(X_*)$ may be
interpreted as a chain complex in $\Pro(\AbOf_k)$. The first member of
\eqref{eq:115.22} is the kind of group occurring as obstruction group
in the Postnikov-type dévissage of $X_*$ into linear structures
$W(M(i)_*)$. The chain-pro-complex $L(X_*)$ may still look a little
forbidding, our hope, though, now is that in the ``pointed'' case we
are really interested in, we may replace $L(X_*)$ by $L\subpt(X_*)$,
which is just a true honest chain complex in $\AbOf_k$. Now, from
\eqref{eq:115.16} we get indeed a canonical map
\begin{equation}
  \label{eq:115.23}
  \Ext_k^n(L(X_*),N) \to \Ext_k^n(L\subpt(X_*),N)\tag{23}
\end{equation}
and we hope that this is an isomorphisms, under suitable assumptions
on $X_*$, the most basic one I can think of now being
\[X_0=e.\]
The map \eqref{eq:115.23} was defined as the transposed of a map of
chain complexes in $\Pro \AbOf_k$
\begin{equation}
  \label{eq:115.24}
  L\subpt(X_*)\to L(X_*),\tag{24}
\end{equation}
deduced from \eqref{eq:115.16} by applying it componentwise. We
recognize here, but with a different interpretation (which seems to me
``the correct'' one), the second map in the often referred-to diagram
\eqref{eq:111.w} of page \ref{p:454}, or \eqref{eq:114.V} in
yesterday's reflections (p.~\ref{p:478}). To say that it gives rise to
isomorphisms \eqref{eq:115.23}, for any $n$ and any module $N$, should
be equivalent to saying that \eqref{eq:115.24} is a quasi-isomorphism
-- but to make sure I should demand a little work on foundations
matters on pro-complexes I guess; also, to see\pspage{489} if the
assumption that \eqref{eq:115.24} is a quasi-isomorphism should imply
the same statement with $L(X_*)$ replaced by its componentwise
projective limit -- namely that \eqref{eq:114.V} on p.~\ref{p:478} is,
which we'll need of course in case $k=\bZ$ for the so-called
``linearization theorem''. Thus, we get three isomorphism or
quasi-isomorphism statements, concerning \eqref{eq:115.23},
\eqref{eq:115.24} and \eqref{eq:114.V} in yesterday's notes, which are
at any rate closely related, and which one hopes to be true, because
this seems needed for a schematization theory of homotopy types to
work. But I should confess I have not tried even to get any clue as to
why this should be true, under the only assumptions, say, that the
components of $X_*$ should be smooth (and possibly the Kan
assumption?), plus $X_0=e$ say.

Now to the second ingredient of the looked-for ``linearization
theorem'', which previously was the first map in \eqref{eq:111.w}
p.~\ref{p:454}, or \eqref{eq:114.Vprime} on page \ref{p:478},
involving maps of
\[\text{either $\bZ^{(X_*(k))}$ or $k^{(X_*(k))}$}\]
into what was previously called $L(X_*)$, and which we would now
rather denote by
\[\varprojlim L(X_*)\quad (\simeq\Gamma\uphat_k(M_*)\text{ if
  $X_*=W(M_*)$}).\]
We made sure that, unless $k=\bZ$, none of the two had any chance to
be a quasi-isomorphism. The only positive thing that came out in this
direction was that the first one of these maps would induce an
isomorphism on $\mathrm H_i$ in the critical dimension (namely $i=n+1$
in the $n$-connected case). We now understand why, for $k\ne\bZ$, we
would not get any actual quasi-isomorphism -- namely, the ``correct''
naive linearization which compares reasonably with $L(X_*)$ should not
be relative to a \emph{constant} ring such as \bZ{} or $k$, but
relative to $\scrO_k$, via the map \eqref{eq:115.star}
(p.~\ref{p:487}) giving rise now to
\begin{equation}
  \label{eq:115.25}
  \scrO_k^{(X_*)}\to L(X_*).\tag{25}
\end{equation}
The more reasonable question now, making good sense really for any
ground ring $k$, is whether this map (under the usual assumptions say
on $X_*$) is a quasi-isomorphism. I wouldn't really but it is, as I
have some doubts as to whether the homology sheaves of the first
member (both members of course being viewed as chain complexes of
\scrOk-modules or ``pro'' such) are quasi-coherent -- but for the time
being I am not sure either if those of the second member are
essentially constant pro-objects! But even if \eqref{eq:115.25} isn't
a quasi-isomorphism, it does behave like one for all practical
purposes of computing quasi-coherent cohomology it would seem, as this
boils down indeed to the isomorphisms \eqref{eq:115.20} or
\eqref{eq:115.20prime}.

\bigbreak

\noindent\hfill\ondate{7.9.}\pspage{490}

% 116
\hangsection{The need and the drive.}\label{sec:116}%
For ten days I haven't written any notes, and the time when I stopped
looks a lot more remote still. For two days still after I last wrote
on the notes, I kept pondering about schematization of homotopy types
-- it were rather lively days, first day I found the amazingly simple
description of the homotopy groups in the schematic set-up, which got
me quite excited; next day, from a phone call to Illusie, it turned
out that the key assumption in all my ponderings on schematization,
namely the ``abelianization theorem'' (sic) asserting isomorphism
between discrete and schematic (namely quasi-coherent) homology (or
equivalently, cohomology) invariants, was definitely false:
consequently, the canonical functor from schematic to discrete
homotopy types turns out definitely not to be an equivalence. This
completely overthrows the idyllic picture in my mind about the
relationship between schematic and discrete homotopy types -- but the
reflection on schematic homotopy types ``in their own right'' had by
then proceeded far enough, the very day before, so that my faith in
the relevance of schematic homotopy types wasn't seriously shaken --
rather, I got excited at drawing a systematic
``bilan''\scrcomment{``bilan'' translates as ``assessment'',
  ``results'', ``balance sheet'', or even ``death toll''\dots} from
the evidence now at hand, about the prospects of developing a theory
of ``schematic homotopy types'' satisfying some basic formal
properties, whether or not such theory be based on semisimplicial
unipotent bundles as models, or on any other kind of models making
sense over arbitrary ground rings. Before reverting to a review of the
more formal properties of abelianization in the context of the basic
modelizer \Cat, I would like still to write down with some care what
had thus come up Monday and Tuesday last week\dots

The next days I felt a great fatigue in all my body, and I then
stopped (till yesterday) any involvement in mathematical reflection. I
am glad I followed this time the hint that had come to me through my
body, rather than brush it aside and go on rushing ahead with the work
I was so intensely involved in, as had been a rule in my life for many
years. This time, I understood that the reluctance of the body to
follow that forward rush, even though I was taking good care of myself
with sleep and food, had strong reasons, which had nothing to do with
neither sleep or food nor with my general way of life. Rather, during
the weeks before and also during those very days, a number of things
had occurred in my life, not all visibly related and of differing
weight and magnitude, to none of which I had really devoted serious
reflection, nor even a minimum of time and attention needed for giving
me a\pspage{491} chance to let these things and their meaning
``enter''. In lack of this, there was little chance my response to
current events would be any better than purely mechanical, and my
interaction with some of the people I love would be in any way
creative. There was this need for being attentive, an urgent need
springing from life itself and which I was about to ignore -- and
there was this drive, this impatience driving me recklessly ahead,
with no look left nor right. Of course, I did know about the need,
``somewhere'' -- and in my head too I kind of knew, but the head was
prejudiced as usual and would take no notice, not of the need and not
of the conflict between an urgent need and a powerful, ego-invested
passion. The head was prejudiced and foolish -- so it was my body
finally which told me: now you stop this nonsense and you take care of
what you well know you better take care \emph{first}, and \emph{now}!
And its language was strong and simple enough and cause me to listen.

Thus, the main work I was involved in for these last seven days was to
let a number of things ``enter'' -- mainly things that were being
revealed through the death of my granddaughter Ella. It surely
\emph{was} ``work'', taking up the largest part of my nights and my
days, -- so much so that I can't really say there is any less fatigue
now than seven days ago. It seems to me, though, it isn't quite the
same fatigue -- this time it is the fatigue coming from work done, not
from work shunned. The ``work done'' wasn't really done by me, I feel,
rather work taking place within me, and ``my'' main contribution has
been to allow it to take place, by providing the necessary time and
quietness; and of course, also, to allow the outcome of this work to
become conscious knowledge, rather than burying it away in some dark
corner of the mind. The rough material, as well as the outcome of this
work, have not been this time new facts or new insights; rather,
things which I had come to perceive and notice, for some time already,
over the last two years, without granting them the proper weight and
perspective -- somehow as if I didn't quite believe what I was
unmistakingly perceiving, or didn't take it quite seriously. Such a
thing, I noticed, happens quite often, not only with me, and takes
care of making even the most lively perceptions innocuous, by
disconnecting them at all price from the image of reality and of
ourselves we are carrying around with us, that the image remains
static, unaffected by any kind of ``information'' flowing around or
through us.

\bigbreak

\noindent\hfill\ondate{10.9.}\pspage{492}

% 117
\hangsection[``Schematic'' versus ``formal'' homology and cohomology
\dots]{``Schematic'' versus ``formal'' homology and cohomology
  invariants.}\label{sec:117}%
Maybe the best will be to write up (and possibly develop some) my
reflections (of the two days after I stopped with the notes) roughly
in the order as they occurred.

There were some somewhat technical afterthoughts. One was about the
logical difficulty coming from the site $\Aff_{/k}$ on page
\ref{p:446} not being a \scrU-site (where \scrU{} is the universe we
are working in), hence strictly speaking, the category of all sheaves
on this site is possibly not even a \scrU-category (i.e., the $\Hom$
objects need not be small, i.e., with cardinal in \scrU), still less a
topos, and hence the standard panoply of notions and constructions in
a topos does not apply. This doesn't look really serious, though, one
way out is to limit beforehand the ``size'' of the unipotent bundles
we are allowing, i.e., of the $k$-modules describing them, in terms
say of cardinality of a family of generators of the latter -- and then
restrict accordingly the size of the $k$-algebras $k'$ taken as
``arguments'' for our sheaves, i.e., as objects of the basic site we
are working on. For instance, when working with unipotent bundles of
finite type only (i.e., corresponding to $k$-modules of finite type --
a rather interesting and natural finiteness condition anyhow on the
components of a schematic model $X_*$), it is appropriate to work on
the ``fppf site'',\scrcomment{``fidèlement plat de présentation
  finie''} where the arguments $k'$ are $k$-algebras of \emph{finite
  presentation}. If we should be unwilling to be limited by a fixed
size restriction on the unipotent bundles we are working with (and
hence also on the corresponding homotopy types), we may have to work
with a hierarchy of size restrictions and passage from one to any
other less stringent one -- a technical nuisance to be sure, if we
don't find a more elegant way out, but surely not a substantial
difficulty. At the present heuristic stage of reflections, it doesn't
seem worth while really to dwell on such questions any longer.

Another afterthought is about the functor
\[M\mapsto W(M): \AbOf_k = \kMod\to(\scrOk\textup{-Mod})\]
from $k$-modules to \scrOk-modules -- a fully faithful functor we
know, whose essential image by definition consists of the so-called
\emph{quasi-coherent} sheaves. A little caution is needed, as this
functor is right exact, but \emph{not exact}, i.e., it does not
commute with formation of kernels, because for a $k$-algebra $k'$
which isn't flat, the functor
\[M\mapsto M\otimes_k k'\]
doesn't.\pspage{493} When we wrote down a formula such as
\eqref{eq:115.21} on page \ref{p:487}, we were implicitly making use
of the assumption that for $k$-modules $M,N$ we have a canonical
isomorphism
\begin{equation}
  \label{eq:117.1}
  \RHom_k(M,N) \fromsim \RHom_\scrOk(W(M),W(N)),\tag{1}
\end{equation}
at any rate such a formula is needed if we want to view
\eqref{eq:115.21} as a more explicit way of writing \eqref{eq:115.20}
or \eqref{eq:115.20prime}. Using a free resolution $L_\bullet$ of $M$,
and the fact that
\[\Ext_\scrOk^i(\scrOk, W(N)) \simeq \mathrm H^i(\Spec(k),W(N)) = 0
\quad\text{if $i>0$,}\]
stemming from the cohomology properties of flat descent, we easily get
a map \eqref{eq:117.1}, but I did not check that this map is an
isomorphism, the difficulty coming from the fact that $W(L_\bullet)$
need not be a resolution of $W(M)$, unless $M$ is flat. This
perplexity already arises in the fppf context -- surely Larry Breen
should know the answer. For what we are after here it doesn't seem to
matter too much, as the computations of loc.\ cit.\ were of interest
mainly (maybe exclusively) in the case when $M$ is projective or at
any rate flat, i.e., when working with flat unipotent bundles -- in
which case \eqref{eq:117.1} is indeed an isomorphism.

The interpretation of polynomial maps between $k$-modules $M$, $N$ in
terms of the topological augmented coalgebras $\Gamma\uphat_k$
associated to these may seem a little forbidding to some readers. In
the all-important and typical case when $M$ and $N$ are projective and
of finite type, things become quite evident, though, by just dualizing
the more familiar concepts around polynomial functions and
homomorphisms between rings of such. Thus, $W(M)$ is just a usual
vector-bundle, hence also a true honest affine scheme over $k$, whose
affine ring is the ring of polynomial functions on $M$, which can be
identified with $\Sym_k^*(M\op)$, the symmetric algebra on the dual
module:\scrcomment{To make sense of these formulæ I have added some
  $\op$s to denote dual modules. I hope this matches AGs
  intentions\dots}
\begin{equation}
  \label{eq:117.2}
  W(M) \simeq \Spec(\Sym_k^*(M\op)),\tag{2}
\end{equation}
and similarly for $W(N)$. Polynomial maps from $M$ to $N$, i.e., maps
from the $k$-scheme $W(M)$ to the $k$-scheme $W(N)$, just correspond
to $k$-algebra homomorphisms
\begin{equation}
  \label{eq:117.3}
  \Sym_k^*(N\op) \to \Sym_k^*(M\op)\tag{3}
\end{equation}
(irrespective of the graded structures). As each of these algebras, as
a $k$-module, is a filtering direct limit of its projective submodules
of finite type such as
\begin{equation}
  \label{eq:117.4}
  \Sym_k(M\op)(i) = \bigoplus_{j\le i} \Sym_k^j(M),\tag{4}
\end{equation}
the\pspage{494} dual module
\[\Gamma_k\uphat(M)\simeq (\Sym_k^*(M))\upvee \simeq
\prod_j(\Sym_k^j(M) \simeq \Gamma_k^j(M))\]
may be viewed as the inverse limit of the duals of those submodules
(compare p.~\ref{p:484} \eqref{eq:115.11}), and topologized
accordingly; linear maps \eqref{eq:117.3} may be interpreted in terms
of continuous maps between the dual structures (or, equivalently,
between the corresponding pro-objects)
\begin{equation}
  \label{eq:117.3prime}
  \Gamma\uphat_k(M) \to \Gamma\uphat_k(N),\tag{3'}
\end{equation}
and compatibility of \eqref{eq:117.3} with multiplication and units is
expressed by compatibility of \eqref{eq:117.3prime} with
``comultiplication'', i.e., diagonal maps, and with augmentations. On
the other hand, maps $W(M)\to W(N)$ respecting the ``pointed
structures'' coming from zero sections, correspond to map
\eqref{eq:117.3prime} transforming $1$ into $1$, besides the other
requirements. Such maps, it turns out, automatically induce a map
between the submodules
\begin{equation}
  \label{eq:117.4bis}
  \Gamma_k(M)\to\Gamma_k(N)\tag{4}
\end{equation}
which is rather evident indeed, if we remind ourselves of the fact
that the submodule $\Gamma_k(M)$ may be viewed as the (topological)
dual of $\Sym_k^*(M)$, topologized by the powers of the augmentation
ideal
\[\Sym_k^+(M)=\bigoplus_{i>0} \Sym_k^i(M),\]
or equivalently of the corresponding adic completion
\begin{equation}
  \label{eq:117.5}
  {\Sym_k^*}\uphat(M) = \varprojlim_i \Sym_k^*(M)(i) \simeq
  \prod_{j\ge0}\Sym_k^j(M),\tag{5} 
\end{equation}
and correspondingly for $N$. The ``pointed'' assumption on a map
$W(M)\to W(N)$ in terms of the corresponding homomorphism of
$k$-algebras \eqref{eq:117.3}, just translates into compatibility with
the augmentations, or equivalently, with the corresponding ideals
$\Sym_k^+$, which implies that it induces a homomorphism of the
corresponding adic rings, and hence by duality a homomorphism
\eqref{eq:117.4} on their duals.

These reminders bring near that working with the (discrete)
$k$-algebras $\Sym_k^*(M)$, or equivalently, with the
\emph{topological} coalgebras $\Gamma_k\uphat(M)$, amounts to working
with ``unipotent bundles'' (projective and of finite type), which are
just usual schemes (of a rather particular structure of course),
whereas working with the \emph{topological} $k$-algebras
${\Sym_k^*}\uphat(M)$, or equivalently, with the discrete coalgebras
$\Gamma_k(M)$, amounts to working with \emph{formal schemes}, namely
essentially, with the formal completions of the former along the zero
sections. (Of course, the topological objects just
considered\pspage{495} may be equally viewed as being
\emph{pro-modules} endowed with suitable extra structure.)
Correspondingly, we will expect the cohomology invariants constructed
in terms of (the apparently more forbidding) $\Gamma_k\uphat$ to
express quasi-coherent cohomology of the corresponding \emph{schemes},
or semisimplicial systems of such; whereas, working with the
(apparently more anodyne!) $\Gamma_k$ will lead to cohomology
invariants of \emph{formal schemes} and semisimplicial systems of
such. Both types of invariants are of interest it would seem, the one
however which looks the more relevant in connection with studying
ordinary homotopy types in terms of schematic ones, is surely the
first. On the other hand, there isn't any reason whatever to believe
that under fairly general conditions, these two types of invariants
are going to be isomorphic, by the evident map from ``schematic'' to
``formal'' quasicoherent cohomology. To say it differently, I do no
longer expect that under reasonably wide assumptions, the map
\eqref{eq:115.24} of p.~\ref{p:488}
\begin{equation}
  \label{eq:117.6}
  L\subpt(X_*)\to L(X_*),\tag{6}  
\end{equation}
is a quasi-isomorphism, nor behaves like one with respect to taking
$\Ext^i$'s with values in a quasicoherent module, as I was hastily
surmising for about one week, while loosing track of the geometric
meaning of the algebraic objects I was playing around with. To give
just one example, take $X_*$ to be the standard semisimplicial
unipotent bundle associated to the group-object $\bG\suba$ in
$\scrU(k)$, namely just the usual affine line with addition law. The
$\Ext^i$'s of the two members of \eqref{eq:117.6} with values in the
$k$-module $k$ may be interpreted as either schematic or formal
Hochschild cohomology of the additive group, with coefficients in
$\scrO_k=\bG\suba$. The map of the former into the latter is not
always an isomorphism, already in dimension $1$, where the two groups
to be compared are just the groups of endomorphisms of $\bG\suba$, and
of the corresponding formal group. If $k$ is of char.\ $p>0$, a prime,
then the latter group can be described as the group of all formal
power series of the type
\[F(t)=\sum_{i\ge0} c_i t^{(p^i)},\]
whereas for the first group we must restrict to those $F$ which are
polynomials, i.e., only a finite number of the coefficients $c_i$ are
non-zero.

One may of course object to this example, because the $X_*$ we are
working with is not simply connected, and because the example does not
apply over a ring such as \bZ, which is the one we are interested in
most of all. I am convinced now, however, that even when assuming
$X_1=X_0=e$ and $k=\bZ$, \eqref{eq:117.6} is very far from being a
quasi-isomorphism, even for such\pspage{496} basic structures as
$K(\bZ,n)$ with $n\ge2$. At any rate, the ``way-cut'' argument I have
finally been thinking of, in order to check \eqref{eq:117.6} is a
quasi-isomorphism, rests on vanishing assumptions which (as I was
informed by Illusie the next day) are wholly unrealistic. This finally
clears up, it would seem, a tenacious misconception which has been
sticking to my first heuristic ponderings about the homology and
cohomology formalism for schematic homotopy types: \emph{one should be
  very careful not to substitute the ``pointed'' linearization
  $L\subpt(W(M)) \simeq \Gamma_k(M)$ for the non-pointed one
  $L(W(M))\simeq\Gamma_k\uphat(M)$, in computing homology and
  cohomology invariants of schematic homotopy types}. To say it
differently, in order to be able to compute (or just define) the
``schematic'' homology and cohomology invariants, we do need as a
model a full-fledged semisimplicial unipotent bundle, not just the
corresponding formal one, giving rise to invariants of it's own,
namely ``formal'' homology and cohomology, which are definitely
distinct from the former.

% 118
\hangsection[The homotopy groups $\pi_i$ as derived functors of the
``Lie \dots]{The homotopy groups \texorpdfstring{$\pi_i$}{pi-i} as
  derived functors of the ``Lie functors''. Lack of satisfactory
  models for \texorpdfstring{$S^2$ and $S^3$}{S2 and S3}.}\label{sec:118}%
It is all the more remarkable, in view of the preceding findings, that
the homotopy invariants
\[\pi_i(X_*), \quad i\ge0,\]
of a pointed semisimplicial unipotent bundle $X_*$ (still assuming the
components $X_n$ to be smooth, i.e., to correspond to projective
modules) turn out to be invariants of the corresponding ``formal''
object, and, more startling still, of the corresponding
``infinitesimal'' object of order $1$. More specifically, consider the
``Lie functor'' or ``tangent space at the origin''
\begin{equation}
  \label{eq:118.7}
  \Lie: U(k)^\bullet \to \AbOf_k, \quad X=W(M)\mapsto M,\tag{7}
\end{equation}
which we'll need only for the time being for smooth $X$, when the
geometric meaning of it is clear. This functor transforms
semisimplicial pointed bundles into semisimplicial $k$-modules
$\Lie(X_*)$, thus we should get, besides abelianization, another
remarkable functor, from $\scrM_1(k)$ say to $\D_\bullet(\AbOf_k)$:
\begin{equation}
  \label{eq:118.8}
  \Lie: \scrM_1(k)\to\D_\bullet(\AbOf_k), \quad \text{via $X_*\mapsto
    \Lie(X_*)$,}\tag{8} 
\end{equation}
granting that $X_*\mapsto\Lie(X_*)$ transforms quasi-isomorphisms into
quasi-isomorphisms. Now, that this must be so follows from the really
startling formula
\begin{equation}
  \label{eq:118.9}
  \pi_i(X_*)\simeq\pi_i(\Lie(X_*)) \quad
  \begin{tabular}[t]{@{}c@{}}
    ($\simeq\mathrm{H}_i$ of
    the associated chain\\
    complex in $\AbOf_k$),
  \end{tabular}\tag{9}
\end{equation}
where the left-hand side, I recall, is defined as
\begin{equation}
  \label{eq:118.10}
  \pi_i(X_*)=\pi_i(X_*(k)).\tag{10}
\end{equation}
I\pspage{497} don't have, I must confess, any direct description of
such an isomorphism \eqref{eq:118.9}, valid for any semisimplicial
bundle $X_*$, say, satisfying the assumptions
\[X_0=X_1=e, \quad \text{$X_n$ smooth for any $n$ (maybe flat is
    enough),}\]
plus possibly (if needed) a Kan type condition. However, we have such
isomorphisms \eqref{eq:118.9} in a tautological way, when $X_*$ comes
in the usual way from a chain complex in $\AbOf_k$ with projective
components, hence also when $X_*$ admits a Postnikov-type dévissage
into ``abelian'' pieces as above. If we admit that any $X_*$
satisfying the assumptions is homotopic to one admitting such a
dévissage, the isomorphisms \eqref{eq:118.9} should follow, except of
course that extra work would be still needed to get naturality of
\eqref{eq:118.9}.

I have the feeling however that, besides the specific abelianization
functor in the schematic context, \emph{formula \eqref{eq:118.9}
  should be made a cornerstone of a theory of schematic homotopy
  types, and serve as ``the'' natural definition of the homotopy
  invariants} of a model $X_*$, within the context of schematic models
and without any need a priori to tie them up with, let one subordinate
their study to, invariants of the corresponding discrete homotopy type
$X_*(k)$. Accordingly, weak equivalences should be defined (for
semisimplicial bundles satisfying the suitable assumptions at any
rate) as maps inducing isomorphisms for the $\pi_i$ invariants, namely
inducing quasi-isomorphisms for the corresponding Lie chain
complexes. It is immediately checked that this implies that the
corresponding map for ``formal homology'', namely
\[L\subpt(X_*)\to L\subpt(X_*)\]
is then a quasi-isomorphism too, when viewed as a map of chain
complexes and hopefully the same should hold for ``schematic
homology''
\[L(X_*)\to L(X_*),\]
and of course one would expect converse statements to hold too.

I would like to comment a little on the significance of formula
\eqref{eq:118.9}. As far as I know, this is the only fairly general
formula, not reducing to an ``abelian'' case, where the homotopy
groups $\pi_i$ appear as just the $\mathrm H_i$ invariants of a
suitable chain complex, defined up to unique isomorphism in the
relevant derived category. This chain complex comes here, moreover,
with an amazingly simple description, of immediate geometrical
significance, and suggestive of relationships with\pspage{498} the
homology invariants a lot more precise, presumably, than those
currently used so far. Of course, the significance of \eqref{eq:118.9}
for the study of the usual, ``discrete'' homotopy types, will be
subordinated to how difficult it will turn out for such a homotopy
type to be a)\enspace realizable by a schematic over \bZ, and
b)\enspace to get hold of a more or less explicit description of such
a schematic homotopy type, via say a semisimplicial bundle as a
model. One is of course thinking more specifically of the case of the
spheres $S^n$, the first case (besides the trivial $S^1$ case) being
$S^2$, the sequence of homotopy groups of which (as far as I know) it
not understood yet. Viewing the spheres $S^n$ as successive
suspensions of $S^1$, where $S^1$ is fitting nicely into the formalism
of schematic homotopy types as a $K(\bZ,1)$ (except that the
$1$-connectedness condition $X_1=e$ is not satisfied), this brings
near the question of defining the suspension operation in a relevant
derived category $\scrM_0(k)$ or $\scrM_1(k)$ (whereas before we had
met with the ``dual'' question of constructing homotopy fibers of
maps, such as loop spaces). Thus it would seem that \emph{a
  breakthrough in getting hold of the standard homotopy constructions
  within the schematic context}, assuming that these constructions do
still make a sense, \emph{may well mean a significant advance for the
  understanding of the homotopy groups of spheres}. This looks like a
very strong motivation for trying to carry through those constructions
(possibly even construct a corresponding ``derivator'' embodying any
kinds of finite ``integration'' and ``cointegration'' operations on
schematic homotopy types) and at the same time warns us that such
work, if at all feasible, will most probably be a highly non-trivial
one.

The question of ``schematizing'' the homotopy types of $S^2$ and $S^3$
reminds me of a fact which struck me a long time ago (maybe
J.~P.~Serre or someone else pointed it out to me first). Namely, in
some respects there doesn't seem to be really satisfactory algebraic
models for these homotopy types, taking into account the basic
relationship between the two, namely: the $2$-sphere (or,
equivalently, the projective complex line) is a \emph{homogeneous
  space} under the quaternionic \emph{group} $S^3$ (or, equivalently,
under the complex linear group $\mathrm{SL}(2,\bJ)$).\scrcomment{I'm
  guessing that $\mathrm{SL}(2,\bJ)$ is another notation for
  $\mathrm{SU}(2)$\dots} This relationship, and its manifold
``avatars'' in the realms of discrete groups, Lie groups, algebraic
groups or group schemes etc., is one of the few key situations met
with, and of equal basic significance, in the most diverse quarters in
mathematics, from topology to arithmetic. Thus, $\mathrm{SL}(2)$ as a
simple Lie (or algebraic) group of minimum rank $1$, plays the role
of\pspage{499} \emph{the} basic building block for building up the
most general semisimple groups, whereas $P^1$ may be viewed as being
the most significant homogeneous space under this group, namely the
first and most elementary case of flag manifolds. In view of this
significance of $S^2$ and $S^3$, it is all the more reasonable that no
simple, non-plethoric semisimplicial model say in
\scrcommentinline{unreadable}, in terms say of a semisimplicial group
having the homotopy type of
\[S^3 \sim \mathrm{SL}(2,\bC)=G,\]
and a subgroup having the homotopy type of
\[S^1 \sim \bC^\times = \mathrm{GL}(1,\bC) \sim K(\bZ,1)\]
and playing the part of a Borel subgroup or a maximal torus, in such a
way that the quotient will have the homotopy type of
\[S^2= P^1_\bC \simeq S^3/S^1 \simeq G/B.\]
It would be tempting now to try and construct such a model of the
situation in terms of semisimplicial unipotent bundles over $\bZ$ --
which would at the same time display the homotopy groups of $S^2$ and
$S^3$ (not much of a difference!) via formula \eqref{eq:118.9}. All
the more tempting of course, as it is felt that the geometric objects
and their relationship, the homotopy shadow of which we want to
modelize schematically, are themselves already, basically, most
beautiful schemes over $\Spec(\bZ)$!

Another way of getting a display of the homotopy groups of $S^2$ and
$S^3$ would be in terms of a (discrete) model of the situation above,
in terms of ``hemispherical complexes'' rather than semisimplicial
ones. On the other hand, there is no reason why a theory of schematic
homotopy types could not be carried through as well, using
hemispherical complexes rather than semisimplicial ones. The latter
kind of complexes have the advantage that they have
\scrcommentinline{unreadable} thoroughly familiar through constant use
by topologists and homotopy people for thirty years or so -- the
former however are newcomers, have the advantage of still greater
formal simplicity (just two boundary operations, and just one
degeneracy), and more importantly still, of allowing for a direct
computational description of the homotopy invariants, in the discrete
set-up. When working with hemispherical complexes of unipotent bundles
as models for schematic homotopy types, we'll get then \emph{two}
highly different descriptions of the homotopy invariants $\pi_i$, one
by the ``infinitesimal'' formula \eqref{eq:118.9} interpreting them as
derived functors of the Lie functor, the other one via the\pspage{500}
hemispherical set $X_*(k)$, handled as if it came from an actual
\oo-groupoid (by taking its source and target operations etc.) even if
it does not. (I confess I did not check that this process does
correctly describe the homotopy groups of a hemispherical set, even
without assuming it comes from an \oo-groupoid or an \oo-stack, but
never mind for the time being\dots)\enspace Among the interesting
things still ahead (once we get a little accustomed to working with
hemispherical complexes) is to try and understand how these two
descriptions relate to each other, which may be one means for a better
understanding of the basic formula \eqref{eq:118.9}, in the context of
hemispheric schematic models.

\bigbreak

\noindent\hfill\ondate{26.9.}\par

% 119
\hangsection[Breakdown of an idyllic picture -- and a tentative next
\dots]{Breakdown of an idyllic picture -- and a tentative next best
  ``binomial'' version of the ``comparison theorem'' for schematic
  versus discrete linearization.}\label{sec:119}%
After the last notes (of September 10) I was a little sick for a few
days, then I was taken by current tasks from professional and family
life, which left little leisure for mathematical reflection, except
once or twice for a couple of hours, by way of recreation. It would
seem now that in the days and weeks ahead, there will be more time to
go on with the notes, and I feel eager indeed to push ahead. Also, I
more or less promised the publisher, Pierre Bérès, that a first volume
would be ready for the printer by the end of this calendar year, and I
would like to keep this promise.

I still have to tie in with the reflections and happenings of the end
of last month, as I started upon with the last notes (of
Sept.~10). Next thing then to report upon is the ``coup de
théâtre''\scrcomment{a sudden or unexpected event in a play\dots}
occurring through the phone call to Luc Illusie. When I told him about
what by then still looked to me as the key assumption for a theory of
schematization of homotopy types, namely that the homology of
$K(\pi,n)$ should be computable in terms of derived functors of the
``divided power algebra''-functor $\Gamma_\bZ$, he at once felt rather
skeptical, and later he called me back to tell me it was definitely
false. He could not give me an explicit counterexample for $\mathrm
H_i(n,\bZ;\bZ)$, say, with given $n$ and $i$, rather he said that when
suitably ``stabilizing'' the assumption I had in mind, it went against
results of Larry Breen on $\Ext^i$ functors of $\bG\suba$ with itself
over prime fields $\bF_p$. I don't know if I am going some day to give
into Illusie's argument and into Larry Breen's results -- however,
even before I got Illusie's confirmation that definitely my assumption
was wrong, I convinced myself that at any rate it was false for
$n=1$. This is a non-simply connected case and hence not\pspage{501}
entirely conclusive maybe, but still it was enough to shake my
confidence that the assumption was OK. The counterexample is in terms
of cohomology
\[\mathrm H^2(1,\bZ;\bZ)=\mathrm H^2(\bZ,\bZ) = \mathrm
  H^2(B_\bZ,\bZ)\]
rather than homology, as usual. As the classifying space $B_\bZ=S^1$
of \bZ{} is one dimensional, its cohomology is zero in dimensions
$i\ge2$. On the other hand, $\mathrm H^2$ classifies central
extensions of \bZ{} by \bZ, and an immediate direct argument shows
indeed that such extensions (indeed, any extension of \bZ{} by any
group) split. If we take the schematic $\mathrm H^2$, defined by
Hochschild cochains which are polynomial functors, we get the
classification of central extensions of $\bG\suba$ by $\bG\suba$, as
group schemes over the ring of integers. Now, it is easy to find such
an extension which does \emph{not} split, the first one one may think
of being the group scheme representing the functor
\[k\mapsto W_2(k) =
  \begin{tabular}[t]{@{}l@{}}
    group of truncated power series $1+at+bt^2$ \\
    in $k[t]/(t^3)$,
  \end{tabular}\]
where $a,b$ are parameters in the commutative ring $k$, the group
structure being multiplication. These parameters define in an evident
way a structure of an extension of $\bG\suba$ by $\bG\suba$ upon
$W_2$, a splitting of which would correspond to a group homomorphism
\[\bG\suba \to W_2, \quad a\mapsto 1+at+P(a)t^2,\]
with
\[P\in \bZ[t].\]
Expressing compatibility with the group laws gives the condition
\[P(a+a') = P(a)+P(a')+aa',\]
which has, as unique solutions in $\bQ[t]$, expressions
\[P(t)=t(t-1)/2 + ct\]
with $c$ in \bQ, none of which has coefficients in \bZ. This argument
shows in fact that for given ring $k$, $(W_2)_k$ is a split extension
if{f} $2$ is invertible in $k$, in which case a splitting is given by
\[a\mapsto 1+at+(a(a-1)/2)t^2.\]

This example brings near one plausible ``reason'' why the expected
comparison statement about discrete and schematic linearization could
not reasonably hold true, and in particular why we shouldn't expect
discrete and schematic Hochschild cohomology (for group schemes over
\bZ{} such as $\bG\suba$\pspage{502} or successive extensions of such)
to give the same result. Namely, the latter is computed in terms of
cochains which are polynomial functions \emph{with coefficients in
  \bZ}, whereas there exist polynomial functions \emph{with
  coefficients in \bQ} (not in \bZ) which, however, give rise to
integer-valued functions on the group of integer-valued points. Such
are the binomial expressions
\[P_n(t) = t(t-1)\dots(t-n+1)/n!\quad\text{(for $n\in\bN$).}\]
These (in the case of just one variable $t$) are known to form a basis
of the \bZ-module of all integer-valued functions on \bZ, and these is
a corresponding basis for integer-valued functions on $\bZ^r$, for any
natural integer $r$. Thus, the hope still remains that a sweeping
comparison theorem for discrete versus ``schematic'' linearization
might hold true, provided it is expressed in such a way that the
``schematic models'' we are working with should be built up with
``schemes'' (of sorts) described in terms of spectra not of polynomial
algebras $\bZ[t]$ and tensor powers of these, but rather of ``binomial
algebras'' $\bZ{\angled t}$ built up with the binomial expressions
above, and tensor powers of such. If we want to develop a
corresponding notion of homotopy types over a general ground ring $k$,
we should then require upon $k$ an extra structure of a ``binomial
ring'' (as introduced in the Riemann-Roch Seminar SGA~6 in some talks
of Berthelot),\scrcomment{\textcite{SGA6}} namely a ring
endowed with operations
\[x\mapsto \binom xn : k\to k \quad (n\in\bN),\]
satisfying the formal properties of the binomial functions $x\mapsto
P_n(x)$ in the case $k=\bZ$ or $\bQ$. Whereas linearization of
homotopy types via De~Rham complexes with divided powers relies on a
``commutative algebra with divided powers'' (which was developed
extensively by Berthelot and others for the needs of crystalline
cohomology), linearization via unipotent bundles (assuming it can be
done in such a way as to ensure that any discrete homotopy type can be
``schematized'' in an essentially unique way) might well rely on the
development of a ``binomial commutative algebra'' and a corresponding
notion of ``binomial schemes''. There should be a lot of fun ahead
developing the necessary algebraic machinery, which may prove of
interest in its own right.\footnote{See comments next section
  p.~\ref{p:506}--\ref{p:507}.} It should be realized, however, that
for a ring $k$ to admit a binomial structure is a rather strong
restriction -- thus, for a given prime $p$, no field of char.~$p$
(except possibly the prime field?) admits such a structure. This
remark may temper somewhat the enthusiasm for pushing in this
direction, even granting that a ``binomial comparison theorem'' for
discrete versus ``binomial'' linearization holds true.

Maybe\pspage{503} it is worthwhile to give a down-to-earth formulation
of such a comparison statement. For any free \bZ-module $M$ of finite
type, let
\[\Symbin_\bZ(M) \subset \Sym_\bQ^*(M_\bQ)\quad\text{(where
    $M_\bQ=M\otimes_\bZ \bQ$)}\]
be the subalgebra of the algebra of polynomial functions on
$M_\bQ\upvee \simeq (M\upvee)_\bQ$ which are integral-valued on
$M\upvee$ = dual module of $M$. Now let $L_*$ be any semisimplicial
\bZ-module whose components are free of finite type, and consider the
canonical map of cosemisimplicial \bZ-modules
\[\Symbin_\bZ(L_*\upvee) \to \Maps(L_*,\bZ),\]
described componentwise in an obvious way. The question is whether
this is a weak equivalence, i.e., induces a quasi-isomorphism for the
associated cochain complexes, under the extra assumption that $L_*$ is
$0$-connected, i.e., the associated chain complex has zero $\mathrm
H_0$ (and possibly, if needed, assuming even $1$-connectedness, i.e.,
$\mathrm H_0$ \emph{and} $\mathrm H_1$ of the associated chain complex
are both zero). Presumably, by easy dévissage arguments one should be
able to reduce to the case when $L_*$ is a $K(\bZ,n)$ type, and more
specifically still, that it is the semisimplicial abelian complex
associated to the chain complex reduced to \bZ{} placed in degree $n$
(where $n\ge1$). Thus, the question is whether Eilenberg-Mac~Lane
cohomology (with coefficients in \bZ) for $K(\pi,n)$ types (or more
specifically, $K(\bZ,n)$ types) can be expressed in terms of derived
functors of the $\Symbin(M\upvee)$ functor. At any rate, whether
$\Symbin$ is just the right functor to fit in or not, it looks like an
interesting question whether Eilenberg-Mac~Lane cohomology (or, more
relevantly still, homology) can be expressed in terms of the derived
functor of a suitable non-additive contravariant (resp.\ covariant)
functor $B$ from \Ab{} (or from the subcategory of free \bZ-modules)
to itself. If so, there should be a way of defining a (possibly
somewhat sophisticated) notion of ``$B$-schematic homotopy types''
(over a ground ring $k$ endowed with suitable extra structure, such as
a binomial structure), in terms of ``unipotent $B$-bundles'', in such
a way that any ``discrete'', namely usual homotopy type, satisfying a
suitable $1$-connectedness restriction, admits an essentially unique
``$B$-schematization''.

I don't feel like pursuing these questions here, which would take me
too far off the main line of investigation I've been out for. At any
rate, whether or not Eilenberg-Mac~Lane homology may be expressed in
terms of the total left derived functor of a suitable functor from
\Ab{} to itself, it would seem that the somewhat naive approach
towards schematic homotopy types we have been following, valid over an
arbitrary (commutative) ground ring $k$ without any extra structure
needed on it, is worthwhile\pspage{504} pursuing even for the mere
sake of studying ordinary homotopy types. The main reason for feeling
this way is the amazingly simple description of the homotopy modules
$\pi_i$ of a homotopy type defined in terms of a semisimplicial (or
hemispherical) unipotent bundle, as derived functors (so to say) of
the Lie functor (cf.\ previous section \ref{sec:118}). The main test
for deciding whether there is indeed a rewarding new tool to be dug
out, is to see whether or not in the model categories $\scrM_n(k)$ we
have been working with so far, the standard homotopy constructions
(around loop spaces and suspensions) make sense, and in such a way of
course that the canonical functor from schematic to discrete homotopy
types should commute to these operations. It may well turn out that to
get a handy formalism, one will have still to modify more or less the
conceptual set-up of unipotent bundles I've been tentatively working
with so far. I already lately hit upon suggestions of such
modifications, and presumably I'm going to discuss this still, before
leaving the topic of schematizations.

Another reason which makes me feel that there should exist a notion of
homotopy types over more general ground rings $k$ than \bZ, is that
for a number of rings, such a notion has been known for quite a
while. If I got it right, already in the late sixties (even before I
withdrew from the mathematical milieu) I heard about such things as
homotopy types over residue class rings $\bZ/n\bZ$, or over rings
(such as \bQ) which are localizations of \bZ, or over rings such as
$\bZ\uphat$ or $\bZ_p$ ($p$-adic integers) which are completions of
\bZ{} with respect to a suitable linear topology. Last week, which was
the first time I was at a university after the Summer vacations, I
took from the library the
Bousfield-Kan\scrcomment{\textcite{BousfieldKan1972}} Lecture Notes
book on homotopy limits (which had been pointed out to me by Tim
Porter in June, when he had taken the trouble to tell me about ``shape
theory'' and its relations to (filtering) homotopy limits), and while
glancing through it, I noticed there is a systematic treatment of such
homotopy types.

At this very minute I had a closer look upon the introduction of part
I, it turns out that Bousfield and Kan are working with an arbitrary
(commutative) ground ring $k$, and they are defining corresponding
$k$-completion $k_\oo X$ of a homotopy type $X$, rather than a notion
of ``homotopy type over $k$''. But the two kinds of notions are surely
closely related, the $k$-completion of BK presumably should have more
or less the meaning of ground ring extension $\bZ\to
k$.\footnote{definitely not, in general!} At any rate, for
$1$-connected spaces and $k=\bZ$ the completion operation seems to be
no more no less than just the identity, thus it would seem that the
implicit notion of ``homotopy type over \bZ'' should be just the
ordinary ``discrete'' notion of homotopy types -- unlike the notion
of\pspage{505} schematic homotopy types (over $k=\bZ$) defined via
semisimplicial unipotent bundles. Definitely, an understanding of
schematic homotopy types will have to include the (by now classical)
Bousfield-Kan ideas, and these are also relevant for my reflections on
``integration'' and ``cointegration'' operations (in connection with
the notion of a derivator (section \ref{sec:69})), called in their
book ``homotopy direct limits'' and ``homotopy inverse limits'' (in
the special case of the derivator associated to ordinary homotopy
types, if I got it right). It came as a surprise that in their book,
these operations are developed mainly as technical tools for
developing their theory of $k$-completions, whereas in my own
reflection they appeared from the start as ``the'' main operations in
homotopical as well as homological algebra. There had been quite a
similar surprise when Tim Porter had sent me a reprint (in July, just
before I stopped with my notes for a month or so) of Don Anderson's
beautiful paper ``Fibrations and Geometric
Realizations''\scrcomment{\textcite{Anderson1978}} (Bulletin of
the Amer.\ Math.\ Soc.\ September 1978), where a very general and (as
I feel) quite basic existence theorem for integration and
cointegration operations in the set-up of closed model categories of
Quillen of the type precisely I was after, is barely alluded to at the
end of the introduction, and comes more or less as just a by-product
of work done in view of a result on geometric realizations which (to
an outsider like me at any rate) looks highly technical and not
inspiring in the least!

It is becoming clear that I cannot put off much longer getting
acquainted with the main ideas and results of Bousfield-Kan's book,
which definitely looks like one of the few basic texts on foundational
matters in homotopy theory. Still, before doing some basic reading, I
would like to write down the sporadic reflections on schematic
homotopy types I went into during the last weeks, while they are still
fresh in my mind!

\bigbreak

\noindent\hfill\ondate{28.9.}\pspage{506}\par

% 120
\hangsection[Digression on the Lazare ``analyzers'' for ``binomial''
\dots]{Digression on the Lazare ``analyzers'' for ``binomial''
  commutative algebra, \texorpdfstring{$\lambda$}{lambda}-commutative
  algebra, etc.}\label{sec:120}%
Yesterday (prompted by the reflections from the day before, cf.\
section \ref{sec:119}), I pondered a little on the common features of
the various set-ups for ``commutative algebra'' (possibly, too, for
corresponding notions of ``schemes'') one gets when introducing extra
operations on commutative rings or algebras, such as divided power
structure (on a suitable ideal) or a $\lambda$-structure with
operations $\lambda^i$ paraphrasing exterior powers, or a binomial
structure with operations $x\mapsto\binom xn$ paraphrasing binomial
coefficients, or an $S$-structure with operations $S^i$ of Adams' type
paraphrasing sums of $i$\textsuperscript{th} powers of roots of a
polynomial (a weakened version of a $\lambda$-structure). It seems
that the unifying notion here is the notion of an ``analyzer''
(analyseur) of Lazare, ``containing'' the Lazare analyzer for
commutative rings (not necessarily with unit), so that the components
$\Omega_n$ ($n\ge-1$) are commutative rings (not necessarily with
unit). In case the extra operations we want to introduce on
commutative rings are to be defined on rings \emph{with units} (not
just on a suitable ideal of such a ring, as is the case for the
divided power structure), and they all can be defined in terms of
operations involving just one argument, the reasonable extra axiom on
the corresponding analyzer (as suggested by the examples at hand) is
that for $n\ge1$, $\Omega_n$ can be recovered in terms of $\Omega_0$
and $\Omega_{-1}=k_0$ (the latter acting as a ground ring for the
theory) as the $(n+1)$-fold tensor power of $\Omega_0$ over
$k_0$. Thus, the whole structure of the analyzer may be thought of as
embodied in the system $\Omega=(k_0,\Omega_0)$, where $k_0$ is a
commutative ring (with unit, now), $\Omega_0$ a commutative
$k_0$-algebra with unit, endowed moreover with a composition operation
$(F,G)\mapsto F\circ G$, satisfying a bunch of simple axioms I don't
feel like writing down here. The simplest case of all of course
(corresponding to usual commutative algebra ``over $k_0$'' as a ground
ring, with no extra structure on commutative algebras with unit over
$k_0$) is $\Omega_0=k[T]$, with the usual composition of polynomials,
$T$ acting as the two-sided unit for composition. In the general case,
$\Omega_0$ and its tensor powers $\Omega_n$ over $k_0$ are going to
play the part played by polynomial rings in ordinary commutative
algebra. There should be a ready generalization, in this spirit, of
taking the symmetric algebra of a $k_0$-module (which, for a module
free and of finite type will yield an $\Omega$-structure isomorphic to
one of the $\Omega_n$'s). An $\Omega$-structure on a set $k$ amounts
to giving a structure of a commutative $k_0$-algebra with unit on $k$,
plus a map
\[\Omega_0\to\Maps(k,k), \quad F\mapsto (x\mapsto F(x))\]
compatible\pspage{507} with the structures of $k_0$-algebras as well
as composition operations, and satisfying moreover two conditions for
\[F(x+y), \quad\text{resp.~$F(xy)$}\]
in terms of two diagonal maps
\[
  \begin{tikzcd}[cramped,sep=small]
    \Omega_0 \ar[r,shift left,"\Delta\suba"]
    \ar[r,shift right,"\Delta\subm"'] &
    \Omega_1 = \Omega_0 \otimes_{k_0} \Omega_0
  \end{tikzcd}
\]
(which may be described in terms of the composition structure on
$\Omega_0$, as expressing the compositions $F\circ (G'+G'')$ resp.\
$F\circ(G'G'')$), and moreover one trivial condition for $F(\lambda)$
when $\lambda$ in $k$ comes from $k_0$, namely compatibility of the
map $k_0\to k$ with the operations of $\Omega_0$ on both $k_0$ and
$k$. (NB\enspace the operation of $\Omega_0$ on $k_0$ is defined by
\[F(\lambda)=F\circ\lambda,\]
where $k_0$ is identified to a subring of $\Omega_0$.)

I didn't pursue much further these ponderings, just one digression
among many in the main line of investigation! I also read through the
preprint of David W.\ Jones on
Poly-$T$-complexes,\scrcomment{\textcite{Jones1983}} which Ronnie
Brown (acting as David Jones' supervisor) had sent me a while
ago. There he develops a notion of polyhedral cells, with a view of
using these instead of simplices or cubes for doing combinatorial
homotopy theory. As I had pondered a little along this direction (cf.\
sections \ref{sec:91}, topic \ref{q:91.8}, and section \ref{sec:93}),
I was hoping that some of the perplexities I had been meeting would be
solved in David Jones' notes -- for instance that there would be handy
criteria for a category $M$ made up with such polyhedral cells to be a
weak test category, namely that objects of $M\upvee$ may be used as
models for homotopy types; also, that the ``standard'' chain complex
constructed in section \ref{sec:93} is indeed an ``abelianizator'' for
$M$, i.e., may be used for computing homology of objects of
$M\upvee$. David Jones' emphasis, however, is a rather different one
-- he seems mainly interested in generalizing the theory of ``thin''
structures of M.~K.~Dakin\scrcomment{\textcite{Dakin1977}} from the
simplicial to the more general polyhedral set-up and prove a
corresponding equivalence of categories. Thus, my perplexities remain
-- they are admittedly rather marginal in the main line of thought,
and I doubt I'll stop to try and solve them.

% 121
\hangsection{The basic pair of adjoint functors
  \texorpdfstring{$\smash{\widetilde K:\Hotabz \leftrightarrows \Hotz:
    \LtH_*}$}{Hotab0<->Hot0}.}\label{sec:121}%
The\pspage{508} tentative approach towards defining and studying
``schematic'' homotopy types I have been following lately relies
heavily on a suitable notion of ``linearization'' of such homotopy
types. One can imagine that many different approaches (for instance
via De~Rham complexes with divided powers, or additive small
categories with diagonal maps) may be devised for ``schematic''
homotopy types, but in any case it seems likely that a suitable notion
of linearization will play an important role. It may be worthwhile
therefore to try and pin down the wished-for main features of such a
theory, with the hope maybe of getting an axiomatic description for
it, with a corresponding unicity statement. Before doing so, the first
thing to do seems to review some main formal features of linearization
for ordinary (``discrete'') homotopy types.

Recall the definition
\begin{equation}
  \label{eq:121.1}
  \HotabOf \eqdef \D_\bullet\Ab =
  \begin{tabular}[t]{@{}l@{}}
    derived category of the category of\\
    abelian chain complexes, with\\
    respect to quasi-isomorphisms,
  \end{tabular}
  \tag{1}
\end{equation}
and the two canonical functors
\begin{equation}
  \label{eq:121.2}
  \begin{tikzcd}[baseline=(O.base),cramped]
    \HotOf \ar[r,shift left,"\LH_\bullet"] &
    |[alias=O]| \HotabOf \ar[l,shift left,"K_\pi"]
  \end{tikzcd},\tag{2}
\end{equation}
Where $\LH_\bullet$ is the ``abelianization functor'', and $K_\pi$ is
defined via the Kan-Dold-Puppe functor, associating to a chain complex
the corresponding semisimplicial abelian group. The diagram
\eqref{eq:121.2} may be viewed (up to equivalence) as deduced from the
corresponding diagram
\begin{equation}
  \label{eq:121.2prime}
  \begin{tikzcd}[baseline=(O.base),cramped]
    \Simplexhat \ar[r,shift left,"W"] &
    |[alias=O]| \Simplexhatab \ar[l,shift left,"\DP"]
  \end{tikzcd}\quad(\equeq\Ch_\bullet\Ab)
  \tag{2'}
\end{equation}
by passing to the suitable localized categories $\HotOf$ and
$\HotabOf$. In the diagram \eqref{eq:121.2prime} $W$ is left adjoint
to $\DP$, which is now just the forgetful functor. One main fact about
\eqref{eq:121.2} is
\begin{equation}
  \label{eq:121.3}
  \text{$K_\pi$ is right adjoint to $\LH_\bullet$,}\tag{3}
\end{equation}
which is just a neater way for expressing the familiar fact that for
given abelian group $\pi$ and natural integer $n$, the object
$K(\pi,n)$ in $\HotOf$ (namely the image by $K_\pi$ of the chain
complex $\pi[n]$ reduced to $\pi$ in degree $n$) represents the
cohomology functor
\[X\mapsto \mathrm H^n(X,\pi) \simeq
  \Hom_{\HotabOf}(\LH_\bullet(X),\pi[n]).\]
The functor $K_\pi$ may be called the ``\emph{Eilenberg-Mac~Lane
  functor}'', as its values are immediate generalizations of the
Eilenberg-Mac~Lane objects\pspage{509} $K(\pi,n)$. As any object in
$\HotabOf$ is isomorphic to a product of objects $\pi[n]$, it follows
that in order to check the adjunction formula between $\LH_\bullet$ and
$K_\pi$ it is enough to do so for objects in $\HotabOf$ of the type
$\pi[n]$, which is the ``familiar fact'' just recalled. The notation
$K_\pi$ in \eqref{eq:121.2} is meant to suggest the Eilenberg-Mac~Lane
$K(\pi,n)$ object generalized by the objects $K_\pi(L_\bullet)$, and
also to recall that we recover the homology invariants of $L_\bullet$
from $K_\pi(L_\bullet)$ via the $\pi_i$ invariants, by the formula
\begin{equation}
  \label{eq:121.4}
  \pi_i(K_\pi(L_\bullet))\simeq\mathrm H_i(L_\bullet),\tag{4}
\end{equation}
which implies by the way that the functor $K_\pi$ is ``conservative'',
i.e., a map in $\HotabOf$ which is transformed into an isomorphism is
an isomorphism. This should not be confused with the stronger property
of being fully faithful, or equivalently of the left adjoint
$\LH_\bullet$ being a localization functor, or equivalently still, the
adjunction morphism
\begin{equation}
  \label{eq:121.5}
  \LH_\bullet(K_\pi(L_\bullet))\to L_\bullet\tag{5}
\end{equation}
being an isomorphism in $\HotabOf$, which is definitely false!

The other adjunction morphism
\begin{equation}
  \label{eq:121.6}
  X\to K_\pi(\LH_\bullet(X)) \eqdef X\subab\tag{6}
\end{equation}
is still more interesting, its effect on the homotopy invariants
$\pi_i$ are the Hurewicz homomorphisms
\begin{equation}
  \label{eq:121.7}
  \pi_i(X)\to \mathrm H_i(X) \eqdef \pi_i(X\subab) \simeq \mathrm
  H_i(\LH_\bullet(X));\tag{7} 
\end{equation}
introducing the homotopy fiber of \eqref{eq:121.6} (in the case of
pointed homotopy types) and denoting by $\gamma_i(X)$ its homotopy
invariants, we get the exact sequences of J.~H.~C.~Whitehead (as
recalled in a letter from R.~Brown I just got)
\begin{equation}
  \label{eq:121.7prime}
  \cdots \to\gamma_i(X) \to \pi_i(X) \to \tH_i(X) \to
  \gamma_{i-1}(X) \to \cdots,\tag{7'}
\end{equation}
where $\tH_i$ denotes the ``reduced'' homology group of a
\emph{pointed} homotopy type, equal to $\mathrm H_i$ for $i\ne0$ and
to $\Coker(\mathrm H_0(\mathrm{pt}) \to \mathrm H_0(X))\simeq \mathrm
H_0(X)/\bZ$ for $i=0$.

The case of pointed homotopy types seems of importance for schematic
homotopy types, and deserves some extra mention and care. We may
factor diagram \eqref{eq:121.2} into
\begin{equation}
  \label{eq:121.8}
  \begin{tikzcd}[baseline=(O.base)]
    \Simplexhat \ar[r,shift left,"\alpha"] &
    {\Simplexhat}^\bullet \ar[l, shift left,"\beta"]
    \ar[r,shift left,"\widetilde W"] &
    |[alias=O]| \Simplexhatab \ar[l, shift left, "\widetilde\DP"]
  \end{tikzcd},\tag{8}
\end{equation}
where $\Simplexhat^\bullet$ is the category of \emph{pointed}
semisimplicial complexes, $\beta$ the forgetful functor from these to
non-pointed complexes, and $\alpha$ its left adjoint, which may be
interpreted as\pspage{510}
\begin{equation}
  \label{eq:121.9}
  \alpha(X_*) = X_* \amalg e_*,\tag{9}
\end{equation}
where $e_*$ is the final object of \Simplexhat, and the second member
is pointed by its summand $e_*$. The functor $\widetilde\DP$ comes
from applying componentwise the obvious functor from $\Ab$ to
$\pSets$ (pointed sets), $\widetilde W$ is its left adjoint. Passing to
the suitable localized categories, we get from \eqref{eq:121.8}
\begin{equation}
  \label{eq:121.10}
  \begin{tikzcd}[baseline=(O.base)]
    \HotOf \ar[r,shift left,"\alpha"] \ar[rr,bend left,"\LH_\bullet"] &
    \HotOf^\bullet \ar[l,shift left,"\beta"]
    \ar[r,shift left,"\LtH_\bullet"] &
    |[alias=O]| \HotabOf\ar[l,shift left,"\widetilde K_\pi"]
    \ar[ll,bend left,"K_\pi"]
  \end{tikzcd},\tag{10}
\end{equation}
factoring \eqref{eq:121.2}, where $\alpha$ is now defined by the
formula similar to \eqref{eq:121.9}
\begin{equation}
  \label{eq:121.9prime}
  \alpha(X) = X\amalg e,\tag{9'}
\end{equation}
where $e$ denotes the final object of $\HotOf$, and is used for
defining the pointed structure of the second member. The functor
$\widetilde W$ in \eqref{eq:121.8} can be described (as is seen
componentwise) as
\begin{equation}
  \label{eq:121.11}
  \widetilde W(X_*) = W(\beta(X_*)) / \Imm W(e_*),\tag{11}
\end{equation}
where
\[W(e_*) \to W(X_*), \quad\text{i.e.,}\quad
  \bZ^{(e)} \to \bZ^{(X)}\]
is deduced from the pointing map $e_*\to X_*$. Accordingly, we get an
expression
\[\LtH_\bullet(X) \simeq \LH_\bullet(X)/\LH_\bullet(e),\]
more accurately, an exact triangle
\begin{equation}
  \label{eq:121.12}
  \begin{tabular}{@{}c@{}}
    \begin{tikzcd}[baseline=(O.base),column sep=small]
      & \LtH_\bullet(X) \ar[dl] & \\
      \LH_\bullet(e)=\bZ[0] \ar[rr] & &
      |[alias=O]| \LH_\bullet(X)\ar[ul]
    \end{tikzcd},
  \end{tabular}\tag{12}
\end{equation}
where $X$ is any object in $\HotOf^\bullet$ and $\LH_\bullet(X)$ is
short for $\LH_\bullet(\beta(X))$. Of course, the functors $\LtH$ and
$\widetilde K_\pi$ are still adjoint (one hopes!), hence for any pointed
homotopy type $X$ an adjunction map in $\HotOf^\bullet$
\begin{equation}
  \label{eq:121.6prime}
  X\to\widetilde K_\pi(\LH_\bullet(X)),\tag{6'}
\end{equation}
and \eqref{eq:121.7} and \eqref{eq:121.7prime} are deduced from
\eqref{eq:121.6prime} and its homotopy fiber, rather than from
\eqref{eq:121.6} where it doesn't really make sense because of lack of
canonical base points for taking $\gamma_i$'s and homotopy fibers.

\bigbreak

\noindent\hfill\ondate{2.10.}\pspage{511}\par

% 122
\hangsection[Rambling reflections on $\LtH_*$, Postnikov
invariants, \dots]{Rambling reflections on
  \texorpdfstring{$\LtH_*$}{LH}, Postnikov invariants,
  \texorpdfstring{$S(H,n)$}{S(H,b)}'s -- and on the non-existence of a
  ``total homotopy''-object \texorpdfstring{$\mathrm L\pi_*(X)$}{Lpi(X)} for
  ordinary homotopy types.}\label{sec:122}%
Since the last notes, I have been doing three days' scratchwork
(including today's) on various questions around abelianization in
general (for discrete homotopy types) and on Postnikov dévissage, in
connection with the review on some main formal properties of
abelianization, started upon in the previous section~\ref{sec:121}. I
didn't get anything really new for me, rather it was just part of the
necessary rubbing against the things, in order to get a better feeling
of what they are like, or what they are likely to be like -- what is
likely to be true, and what not. The most interesting, maybe, is that
I got an inkling of a fairly general version of a Kan-Dold-Puppe kind
of relationship, in terms of derived categories, valid presumably for
any local test category, and in particular for categories like
$\Simplex_{/X}$, with $X$ in \Simplexhat. It would be untimely,
though, to build up still more the (already pretty high) tower of
digressions, and for the time being I'll stick to what is relevant
strictly to my immediate purpose -- namely getting through with the
wishful thinking about schematization! Thus, I'll be content to work
with the category \Simplex{} in order to construct models for homotopy
types and perform constructions with them (such as abelianization),
without getting involved at present in looking up how much is going
over (and how) to the case of more general small categories $A$\dots

It occurred to me that the variant $\widetilde W$ or $\LtH_\bullet$
for the abelianization functors $W$ and $\LH_\bullet$, introduced in
the previous section using a pointed structure for the argument $X$ in
\Simplexhat{} or $\HotOf^\bullet$, could be advantageously defined
without this extra structure. The construction for $\widetilde W$
which follows goes through indeed in any topos whatever (not only
\Simplexhat). Let $X$ be an object in \Simplexhat, then there is a
canonical augmentation map
\[\varepsilon : W(X) =\bZ^{(X)} \to \bZ_\Simplex\]
towards the constant semisimplicial group $\bZ_\Simplex$,
corresponding to the constant map
\[X\to \bZ_\Simplex\]
with value $1$. We thus get a functorial exact sequence
\begin{equation}
  \label{eq:122.1}
  0 \to \widetilde W(X) \to W(X) \xrightarrow\varepsilon{} \bZ \to
  0,\tag{1} 
\end{equation}
where $\widetilde W(X)$ denotes the kernel of the augmentation above,
hence an extension of \bZ{} by $\widetilde W(X)$, or (what amounts
essentially to the same) a torsor under the group object $\widetilde
W(X)$, which we'll denote by
\begin{equation}
  \label{eq:122.2}
  W(X)(1) = \varepsilon^{-1}(1).\tag{2}
\end{equation}
Moreover,\pspage{512} the canonical map
\[X\to W(X)=\bZ^{(X)}\]
factors (by construction) through
\begin{equation}
  \label{eq:122.3}
  X\to W(X)(1)\quad(\hookrightarrow W(X)),\tag{3}
\end{equation}
and this map is universal for all maps of $X$ into torsors under
abelian group objects of \Simplexhat. It is an immediate consequence
of Whitehead's theorem that a weak equivalence $X\to Y$ in
\Simplexhat{} induces weak equivalences in \Simplexhat
\begin{equation}
  \label{eq:122.4}
  \widetilde W(X)\to\widetilde W(Y), \quad
  W(X)(1) \to W(Y)(1)\tag{4}
\end{equation}
(and also $W(X)(n)\to W(Y)(n)$ for any $n\in\bZ$), and accordingly
that the map between the normalized chain complexes corresponding to
$\widetilde W(X)$ and $\widetilde W(Y)$, namely by definition
\begin{equation}
  \label{eq:122.4prime}
  \LtH_\bullet(X) \to \LtH_\bullet(Y)\tag{4'}
\end{equation}
is a quasi-isomorphism. Thus, we get a functor
\[\LtH_\bullet: \HotOf\to\HotabOf,\]
whose composition $\LtH_\bullet\circ \beta$ with the forgetful functor
\[\beta:\HotOf^\bullet\to\HotOf\]
is canonically isomorphic to the functor $\LtH_\bullet$ of
p.~\ref{p:510} (formula \ref{eq:121.9}). More specifically, when $X$
is in $\Simplexhat^\bullet$, then the map
\[\bZ=\bZ^{(e)} \to \bZ^{(X)}\quad\text{deduced from $e\to X$}\]
defines a splitting of the extension \eqref{eq:122.1}, i.e., an
isomorphism
\begin{equation}
  \label{eq:122.5}
  \bZ^{(X)} = W(X) \simeq \widetilde W(X)\oplus \bZ,\tag{5}
\end{equation}
and accordingly, $\widetilde W(X)$ may be interpreted, at will, as a
quotient group of $W(X)$ (which we did on p.~\ref{p:510}), or as a
subobject of $W(X)$, which is the better choice, because it makes good
sense without using any pointed structure. Again, without using the
pointed structure, we get a canonical exact triangle in $\D\Ab$
interpreting \eqref{eq:122.1}
\begin{equation}
  \label{eq:122.6}
  \begin{tikzcd}[column sep=small]
    & \bZ\ar[dl] & \\
    \LtH_\bullet(X) \ar[rr] & &
    \LH_\bullet(X) \ar[ul]
  \end{tikzcd}\tag{6}
\end{equation}
replacing (\eqref{eq:121.12} p.~\ref{p:510}), which (in the case
considered there, namely $X$ pointed) should be replaced by the more
precise relationship
\begin{equation}
  \label{eq:122.5prime}
  \LH_\bullet(X)\simeq \LtH_\bullet(X) \oplus\bZ,\tag{5'}
\end{equation}
an isomorphism functorial for $X$ in $\HotOf^\bullet$.

Truth\pspage{513} to tell, these generalities are more interesting
still for an argument $X$ in a category like $\Simplexhat_{/Y}\equeq
(\Simplex_{/Y})\uphat$, where $Y$ is an arbitrary object in
\Simplexhat, rather than in \Simplexhat{} itself. In the latter case,
$X$ always admits a pointed structure, i.e., a section over the final
object (provided $X$ is non-empty), hence there always exists a
splitting for \eqref{eq:122.1}, and hence one for \eqref{eq:122.6};
whereas for an object $X$ over $Y$, there does not always exist a
section over $Y$, and accordingly it may well happen that the
extension similar to \eqref{eq:122.1} (but taken ``over $Y$'') does
not split, i.e., that the torsor $W_{/Y}(X)(1)$ is not trivial. The
class of this torsor is an element
\begin{equation}
  \label{eq:122.7}
  c(X/Y) \in \mathrm H^1(Y, \widetilde W_{/Y}(X)),\tag{7}
\end{equation}
a very interesting invariant indeed, giving rise to the Postnikov
invariants when $Y$ is one of the $X_n$'s occurring in the Postnikov
dévissage of $X$. It was while trying to understand the precise
relationship between a cohomology group as in \eqref{eq:122.7}, with
``continuous'' coefficients (by which I mean to suggest that
$\widetilde W_{/Y}(X)$ is viewed intuitively as a fiberspace over the
``space'' $X$, whose fibers are topological abelian groups which are
by no means discrete, but got a bunch of non-vanishing $\pi_i$'s!),
and a cohomology group with ``discrete'' coefficients, more accurately
with coefficients in a complex of chains in the category of abelian
sheaves over $X$, i.e., over $\Simplex_{/X}$ whose homology sheaves
are definitely of a ``discrete'' nature (for instance, they are
locally constant if $X\to Y$ is a Kan fibration, and their fibers are
\bZ-modules of finite type provided we make a mild finite-type
assumption on the fibers of $X\to Y$), that I got involved in a more
general reflection on a Kan-Dold-Puppe type of relationship. I hope to
come back to this when getting back to the general review of
linearization begun in part \ref{ch:V} of these notes, and abruptly
interrupted after section~\ref{sec:109}, when getting caught
unwittingly by the enticing mystery of schematization!

Another afterthought to the previous notes is that I am going to
denote by $K, \widetilde K$, the functors
\begin{equation}
  \label{eq:122.7bis}
  K:\HotabOf\to\HotOf, \quad
  \widetilde K:\HotabOf\to\HotOf^\bullet, \quad
  K=\beta\circ\widetilde K,\tag{7}
\end{equation}
denoted by $K_\pi$, $\widetilde K_\pi$ in section~\ref{sec:121}. This
gives a formula such as
\begin{equation}
  \label{eq:122.8}
  K(\pi[n]) = K(\pi,n),\tag{8}
\end{equation}
where $\pi[n]$ denotes the chain complex in \Ab{} which is $\pi$
placed in degree $n$. I was intending to complement the former
notation $K_\pi$ by a similar notation $K_{\mathrm H}$ (where $\mathrm
H$ stands for ``homology'', as $\pi$ was standing\pspage{514} for
``homotopy'', and $K$ means ``complex''), but I finally found the
notation $S$ (suggestive of ``spheres'') more congenial, giving rise
to
\begin{equation}
  \label{eq:122.8prime}
  S(\mathrm H[n]) = S(\mathrm H,n) =
  \begin{tabular}[t]{@{}l@{}}
    sphere-like homotopy type whose $\LtH_\bullet$ \\
    is isomorphic to $\mathrm H[n]$.
  \end{tabular}\tag{8'}
\end{equation}
But I am anticipating somewhat on some of the wishful thinking still
ahead, involving the would-be description (preferably in the schematic
set-up, but more elementarily maybe in the discrete one of ordinary
homotopy types) of a functor
\[S: \HotabOf = \D_\bullet\Ab \to \HotOf,\]
whose most important formal property should be
\[\LH_\bullet(S(L_\bullet)) \simeq L_\bullet,\]
compare with formula \eqref{eq:122.8prime} for
$L_\bullet=\pi[n]$.\scrcomment{in \eqref{eq:122.8prime}, the $\mathrm
  H[n]$ was originally written $\pi[n]$. I'm not sure what to make of
  this\dots} However, such a formula cannot hold for any $L_\bullet$,
i.e., an arbitrary chain complex in \Ab{} is not quasi-isomorphic to
an object $\LH_\bullet(X)$ with $X$ in \Hot, as a necessary condition
for this is that $\mathrm H_0(L_\bullet)$ should be free. Thus, we
better restrict to the $0$-connected case, and take $L_\bullet$
subject to
\[\mathrm H_0(L_\bullet) = 0,\quad\text{i.e., $L_\bullet$ in
    $\D_{\ge1}\Ab \eqdef \HotabOfzc$,}\]
and describe $S$, or better still $\widetilde S$, as a functor
\begin{equation}
  \label{eq:122.9}
  \widetilde S: \HotabOfzc \to \HotOfzc \hookrightarrow \HotOf^\bullet,\tag{9}
\end{equation}
subject to the condition
\begin{equation}
  \label{eq:122.10}
  \LtH_\bullet(\widetilde S(L_\bullet)) \simeq L_\bullet,\tag{10}
\end{equation}
giving rise to
\begin{equation}
  \label{eq:122.10prime}
  \tH_i(\widetilde S(L_\bullet)) \simeq \mathrm H_i(L_\bullet),\tag{10'}
\end{equation}
which should be compared to the familiar formula
\begin{equation}
  \label{eq:122.11}
  \pi_i(\widetilde K(L_\bullet)) \simeq \mathrm H_i(L_\bullet).\tag{11}
\end{equation}

Maybe formula \eqref{eq:122.10} is not quite enough for characterizing
the functor $\widetilde S$ up to unique isomorphism, and I confess I
didn't try to construct in some canonical way a functor
\eqref{eq:122.9} satisfying \eqref{eq:122.10} -- and I am not quite
sure even whether such a functor exists. One main evidence for
existence of such a functor would be the existence and unicity (up to
unique isomorphism) of the homotopy types $S(H,n)$ in
\eqref{eq:122.8prime} -- and I have a vague remembrance of having
looked through a paper, fifteen or twenty years ago, where such spaces
were indeed introduced and studied. Thus, a functor $\widetilde S$ is
perhaps nowadays more or less standard knowledge in\pspage{515}
homotopy theory. I was led to postulate the existence of a functor
$\widetilde S$ in the context of \emph{schematic} homotopy types, by
reasons of symmetry in the $\LH_\bullet$ and $\mathrm L\pi_\bullet$
formalism, which I'll try to get through a little later. It should be
clear from the outset, though, that in the schematic set-up, even the
mere existence of (schematic) homotopy types $S(\bZ,n)$ (corresponding
to ordinary spheres), and even for $n=2$ or $3$ only, is very far from
trivial, and as a matter of fact isn't even known (in the set-up of
semisimplicial unipotent bundles). This shows at the same time that if
such a functor $\widetilde S$ can actually be constructed in the
schematic case, it is likely to be a lot more interesting still than
in the discrete case, as it will presumably give information on
homotopy groups of spheres (via the description of the $\pi_i$'s of a
semisimplicial unipotent bundle via the Lie functor (cf.\
section~\ref{sec:118})).

There is a fourth tentative functor in between the categories
$\HotabOf$ and $\HotOf$, or more accurately, from a suitable
subcategory $\HotOf^\bullet(0)$ of $\HotOf^\bullet$ (corresponding to
$0$-connected pointed homotopy types whose $\pi_1$ is abelian and acts
trivially on the higher $\pi_i$'s) to $\HotabOfzc=\HotabOf(0)$,
strongly suggested by the schematic case of section~\ref{sec:118},
namely a functor
\begin{equation}
  \label{eq:122.star}
  \Lpi_\bullet: \HotOf^\bullet(0) \to \HotabOf^\bullet(0)\text{???,}\tag{*}
\end{equation}
whose main functorial property should be
\begin{equation}
  \label{eq:122.12}
  \mathrm H_i(\Lpi_\bullet(X)) \simeq \pi_i(X).\tag{12}
\end{equation}
To play really safe, one might hope such a functor to be defined at
any rate for $1$-connected pointed homotopy types. Another equally
important property, suggested by the schematic set-up as well as by
\eqref{eq:122.11} and \eqref{eq:122.12}, is the formula
\begin{equation}
  \label{eq:122.13}
  \Lpi_\bullet(\widetilde K(L_\bullet)) \simeq L_\bullet,\tag{13}
\end{equation}
similar to formula \eqref{eq:122.10} (with the pair $(\widetilde
S,\LtH_\bullet)$ replaced by the pair $(\widetilde
K,\Lpi_\bullet)$). Of course, for $X$ of the type $\widetilde
K(L_\bullet)$, \eqref{eq:122.12} follows from \eqref{eq:122.13}

\bigbreak

\noindent\hfill\ondate{3.10.}\par

We'll see, though, that a functor $\Lpi_\bullet$ as in
\eqref{eq:122.star}, satisfying the properties above, \emph{does not
  exist}. Indeed, applying $\Lpi_\bullet$ to the adjunction morphism
\[X\to \widetilde K(\LtH_\bullet(X))\]
(cf.\ p.~\ref{p:509}, \eqref{eq:121.6}), and using \eqref{eq:122.13},
we should get a Hurewicz homomorphism\pspage{516}
\begin{equation}
  \label{eq:122.14}
  \Lpi_\bullet(X)\to\LtH_\bullet(X)\tag{14}
\end{equation}
more precise than the separate homomorphisms
\[\pi_i(X)\to\tH_i(X),\]
and applying this to an object $\widetilde K(L_\bullet)$ and applying
\eqref{eq:122.13} again, we should get a functorial homomorphism
\begin{equation}
  \label{eq:122.15}
  L_\bullet \to \LtH_\bullet(\widetilde K(L_\bullet))\tag{15}
\end{equation}
in opposite direction from the adjunction morphism
\begin{equation}
  \label{eq:122.16}
  \LtH_\bullet(\widetilde K(L_\bullet)) \to L_\bullet\tag{16}
\end{equation}
(p.~\ref{p:509}, \eqref{eq:121.5}), the composition of the two being
the identity in $L_\bullet$. In other words, \eqref{eq:122.15} should
be a canonical splitting of the natural adjunction morphism
\eqref{eq:122.16}. Take for instance $L_\bullet=\pi[n]$, then the
first member of \eqref{eq:122.16} is the Eilenberg-Mac~Lane homology
$\LtH_\bullet(\pi,n)$, whose first non-vanishing homology group is
$\pi$, and \eqref{eq:122.16} is just the canonical augmentation
\[\LtH_\bullet(\pi,n)\to\pi[n],\]
and taking the $\Ext_\bZ^i$ of both members with $M[n]$ ($M$ any
object in \Ab) yields the transposed canonical homomorphism
\begin{equation}
  \label{eq:122.17}
  \Ext_\bZ^i(\pi,M)\to\mathrm H^{n+i}(\pi,n;M),\tag{17}
\end{equation}
which is an isomorphism for $i=0$ (whereas for $i<0$ both members are
zero). For $i=1$, we get an exact sequence (``universal
coefficients'')
\begin{equation}
  \label{eq:122.18}
  0 \to \Ext_\bZ^i(\pi,M) \to \mathrm H^{i+1}(\pi,n;M) \to
  \Hom_\bZ(\mathrm H_{i+1}(\pi,n), M) \to 0.\tag{18}
\end{equation}
A canonical splitting of \eqref{eq:122.15} would yield a canonical
splitting of this exact sequence, which definitely looks somehow as
``against nature''! Surely, it shouldn't be hard to find, for any
given integer $n\ge1$, suitable abelian groups $\pi$ and $M$, such
that there does not exist a splitting of \eqref{eq:122.18} stable
under the action of the group
\[G = \Aut(\pi) \times \Aut(M)\]
(acting by ``transport de structure'' on the three terms of
\eqref{eq:122.18}); presumably even, looking at $\Aut(\pi)$ should be
enough. As I am not at all familiar with Eilenberg-Mac~Lane
cohomology, except a little in the case $n=1$, i.e., for usual group
cohomology, I didn't try to construct an example for any $n$, only one
for $n=1$, i.e., in a non-simply connected case, which again is a
little less convincing as if it was one for $n=2$\dots

Thus, let's take $n=1$, a vector space of finite dimension over the
prime field $\bF_2$, and $M=\bF_2$, in this case standard
calculations\pspage{517} show that the exact sequence
\eqref{eq:122.18} can be identified with the familiar exact sequence
\begin{equation}
  \label{eq:122.18prime}
  0\to V'\to \Sym^2(V')\to \bigwedge^2 V'\to 0,\tag{18'}
\end{equation}
where the first arrow associates to every linear form on $V$ its
square, i.e., the same form (!) but viewed as being a quadratic form
on $V$; and the second arrow associates to every quadratic form on $V$
the associated bilinear form, which is alternate because of
char.~$2$. It is well-known I guess that for $\dim V\ge2$, there does
not exist a splitting of \eqref{eq:122.18prime} which is stable under
$\Aut(V)\simeq\Aut(V')$.

\begin{remark}
  If we admit that a similar example can be found for non-splitting of
  \eqref{eq:122.18}, with $n\ge2$ (which shouldn't be hard I guess for
  someone familiar with Eilenberg-Mac~Lane homology), this shows that
  there does \emph{not} exist (as was contemplated at the beginning,
  cf.\ section~\ref{sec:111}) an equivalence of categories between the
  schematic $1$-connected homotopy types (defined via the model
  category $\scrM_1(\bZ)$ of semisimplicial unipotent bundles over
  \bZ{} satisfying $X_0=X_1=e$) and $1$-connected pointed usual
  homotopy types, as this would imply the existence of a functor
  $\Lpi_\bullet$, hence of a canonical splitting of
  \eqref{eq:122.18}. The same argument will show that the once
  hoped-for comparison theorem between usual Eilenberg-Mac~Lane
  homology (or cohomology), and the schematic one, is false, because
  in the schematic set-up $\Lpi_\bullet$ does exist, and the direct
  construction of \eqref{eq:122.15} (a functorial splitting of
  \eqref{eq:122.16}) is then anyhow a tautology. Thus after all, we
  don't have to rely on delicate results of Breen's on
  $\Ext^i(\bG\suba,\bG\suba)$ over the prime fields $\bF_p$ (as
  suggested by Illusie), in order to get this ``negative'' result,
  causing unreasonable expectations to crash\dots
\end{remark}

\bigbreak

\noindent\hfill\ondate{4.10.}\pspage{518}\par

% 123
\hangsection[The hypothetical complexes ${}^n\Pi_*=\Lpi_*(S^n)$, and
\dots]{The hypothetical complexes
  \texorpdfstring{${}^n\Pi_*=\Lpi_*(S^n)$}{nPi=Lpi(Sn)}, and comments
  on homotopy groups of spheres.}\label{sec:123}%
For the last few days -- since I resumed the daily mathematical
ponderings (interrupted more or less for nearly one month), the wind
in my sails has been rather slack I feel. It has been nearly six weeks
from now that I started pondering on schematization and on schematic
homotopy types -- in the process I got rid of some misplaced
expectations, very well. Still, the unpleasant feeling remains of not
having really any hold yet on those would-be schematic homotopy types,
due mainly to my incapacity so far of performing (in the model
category of semisimplicial unipotent bundles say) the basic homotopy
operations of taking homotopy fibers and cofibers of maps. In lack of
this, I am not (``morally'') sure yet that there does exist indeed a
substantial reality of the kind I have been trying to foreshadow. I am
not even wholly clear yet of how to define the notion of ``weak
equivalence'' in t he model category $\scrM_0(k)$ (or in the smaller
one $\scrM_1(k)$, to play safe, as even the definition of $\scrM_0(k)$
isn't too clear yet) -- there are three ways to define weak
equivalences, using $\LH_\bullet$, or $\Lpi_\bullet$, or the sections
functor from $\scrM_1(k)$ to ordinary semisimplicial complexes, and it
isn't clear yet whether these are indeed equivalent. We may of course
call ``weak equivalence'' in $\scrM_1(k)$ a map which becomes an
isomorphism by any of these three functors. But I wouldn't say I am
wholly confident yet that there exists a localization $\HotOf_1(k)$ of
$\scrM_1(k)$, with respect to this notion of weak equivalence or some
finer one, in such a way that in $\HotOf_1(k)$ one may introduce the
two types of ``fibration'' and ``cofibration'' sequences with the
usual properties, and the sections functor
\[\HotOf_1(k)\to\HotOf_1\]
should respect this structure, and moreover give rise to functorial
isomorphisms
\[\mathrm H_i(\Lpi_\bullet(X_*)) \simeq \pi_i(X_*(k)),\]
nor even do I feel wholly confident that a theory of this kind can be
developed, possibly with different kinds of models for $k$-homotopy
types from the ones I have been using. At any rate, it seems to me
that the two kinds of conditions or features I have just been stating,
are indeed the crucial ones, plus (I should add) existence in
$\HotOf_1(k)$ of an object
\[S^2_k = S(2,k),\]
standing for the $2$-sphere, with
\[\LtH_\bullet(S^2_k) \simeq \bZ[2],\]
whose\pspage{519} image in $\HotOf_1$ (in case $k=\bZ$ say) should be
the (homotopy type of the) ordinary $2$-sphere. Taking suspensions,
we'll get from this the $n$-spheres $S^n_k=S(n,k)$ over $k$, for any
$n\ge2$. (NB\enspace In case the relevant homotopy theoretic
structures can be introduced in a suitable larger $\HotOf_0(k)$,
containing the objects $\widetilde K(L_\bullet)$ for $L_\bullet$ a
$0$-connected chain complex of $k$-modules, $\HotOf_0(k)$ contains an
object $S(1,k)=\widetilde K(1,k)$, and $S^2_k$ may be simply described
as its suspension.)\enspace One main motivation for trying to push
through a theory of schematic homotopy types may be the hope that this
may provide new insights into the homotopy groups of spheres. A first
interesting consequence would be that for any given sphere $S^n$ (in
the set-up now of ordinary homotopy theory), the series of all its
homotopy groups $\pi_i(S^n)=\Hom_{\HotOf}(S^i,S^n)$ may be viewed as
being the homology modules of an object of $\D_\bullet\Ab$, namely
$\Lpi_\bullet(S^n_\bZ)$, canonically associated to $S^n$. As was seen
by an example in the previous section, the existence of such a
\emph{canonical} representation is by no means a trivial or innocuous
fact, it is indeed definitely false for arbitrary homotopy types --
here then it would turn out as a rather special feature of the full
subcategory of $\HotOf^\bullet$ made up with the spheres $S^n$
($n\ge2$). Thus, for any $\alpha\in\pi_i(S^n)$, i.e., for any map
\[\alpha:S^i\to S^n\]
in $\HotOf^\bullet$, there should be an induced map
\[\Lpi_\bullet(\alpha):\Lpi_\bullet(S^i)\to\Lpi_\bullet(S^n)\]
with associativity condition for a composition
\[S^i \xrightarrow\alpha{} S^n \xrightarrow\beta{} S^m.\]
If we write
\[{}^n\Pi_\bullet \eqdef \Lpi_\bullet(S^n),\]
this defines on the set of chain complexes (more accurately, abelian
homotopy types) ${}^n\Pi_\bullet$ ($n\ge2$) quite a specific
structure, which merits to be investigated a priori (under the
assumption, of course, of the existence of a reasonable theory of
schematic homotopy types, satisfying the criteria above). Maybe after
all, it will turn out that the homotopy groups of spheres
\emph{cannot} be fitted into such an encompassing structure -- so much
the better, as this will show that the kind of theory I started trying
to dig out doesn't exist, which will clear up the situation a great
deal! But if there does exist such a canonical structure, it surely
shouldn't be ignored, and it should be pleasant work to try and pin
down exactly what extra information on homotopy groups of spheres
is\pspage{520} involved in such extra structure. It may be noted for
instance that, when taking $i=n$ hence $\pi_i(S^n)=\bZ$, we get an
operation of the multiplicative monoid $\bZ^{\mathrm{mult}}$ on
${}^n\Pi_\bullet$, whose action on the homology groups of
${}^n\Pi_\bullet$, i.e., on the homotopy groups $\pi_j(S^n)$, is
surely an important extra structure on these, which I hope has been
studied by the homotopy people extensively. Surely it has been known,
too, for a long time (at any rate since Artin-Mazur's foundational
work on profinite homotopy types) that when $\pi_j(S^n)$ is finite
(i.e., practically in all cases except $j=n$) this action comes from a
continuous action of the multiplicative monoid
\[ M={\bZ\uphat}^{\mathrm{mult}},\]
where $\bZ\uphat$ is the completion of \bZ{} with respect to the
topology defined by subgroups of finite index:
\[\bZ\uphat = \lim \bZ/n\bZ \simeq \prod_p \bZ_p,\]
where in the last member the product is taken over all primes, and
$\bZ_p$ denotes $p$-adic integers. Taking homotopy types over the ring
$k=\bZ\uphat$, we should get that this monoid $M$ operates on the
object
\[{}^n\Pi_\bullet \biggl(\bigoplus^l{}_\bZ \bZ\uphat\biggr) \quad\text{in
    $\D_\bullet(\AbOf_{\bZ\uphat})$,}\]
as this may be interpreted as $\Lpi_\bullet(S^n_k)$, and
$k^{\mathrm{mult}}$ operates on $S^n_k=S(n,k)$ for any ring $k$ (if
indeed $S(n,\pi)$ depends functorially on the variable $k$-module
$\pi$). At any rate, it is surely well-known that
$M={\bZ\uphat}^{\mathrm{mult}}$ operates on the profinite completions
of all spheres. For the odd-dimensional spheres, and taking the
restrictions of this operation to the largest subgroup
\[M^\times = {\bZ\uphat}^\times \simeq \prod_p \bZ_p^\times\]
of $M$, this group may be viewed as the profinite Galois group of the
maximal cyclotomic extension of the prime field \bQ{} (deduced by
adjoining all roots of unity), and its action on the profinite
completion ${S^n}\uphat$ of $S^n$ may be viewed as the canonical
Galois action, when $S^{n=2m-1}$ is interpreted as the homotopy type
of affine (complex) $m$-space minus the origin (which makes sense as a
scheme over the prime field \bQ). This was the interpretation in my
mind, when stating that the action of $\bZ^{\mathrm{mult}}$ and of its
completion ${\bZ\uphat}^{\mathrm{mult}}$ on the homotopy groups of
spheres was in important extra structure on these.

% 124
\hangsection[Outline of a program (second version): an autodual
\dots]{Outline of a program \texorpdfstring{\textup(}{(}second
version\texorpdfstring{\textup)}{)}: an autodual formulaire for the
basic ``four functors'' \texorpdfstring{$\LtH_*$, $\Lpi_*$,
  $\widetilde K$, $\widetilde S$}{LH, Lpi, K, S}.}\label{sec:124}%
I\pspage{521} guess I'll skip giving a more or (rather!) less complete
axiomatic description of $\HotOf_0^\bullet$ or of
$\HotOf_0^\bullet(k)$, in terms of the pair of adjoint functors
$\LtH_\bullet$ and $\widetilde K$ and Postnikov dévissage, although I
did go through this exercise lately -- it doesn't really shed new
light on the approach we are following here towards ``schematic''
homotopy types over arbitrary ground rings. In contrast with the two
other approaches I have been thinking of before (namely De~Rham
complexes with divided powers, and small categories with diagonal
maps, cf.\ sections \ref{sec:94} and \ref{sec:109} respectively), the
characteristic feature of this approach seems to be that it takes into
account the existence of the canonical functor
$\widetilde K: \D_\bullet\Ab\to\Hot$, paraphrased by a functor
\begin{equation}
  \label{eq:124.1}
  \widetilde K:\D_\bullet(\AbOf_k)\zc\to\HotOf_0(k) \tag{1}
\end{equation}
compatible with ring extension $k\to k'$; while all three approaches
have in common that they yield a paraphrase
\begin{equation}
  \label{eq:124.2}
  \LtH_\bullet:\HotOf_0(k)\to\D_\bullet(\AbOf_k)\zc\tag{2}
\end{equation}
of the (total) homology functor, in a way again compatible with
extension of ground rings. In the two earlier approaches this latter
functor comes out in a wholly tautological way, whereas in the present
approach via semisimplicial schemes, it isn't quite so tautological
indeed. If we want to define a functor $\widetilde K$ in the context
say of De~Rham algebras with divided powers, one may think of
associating first to a chain complex in $\AbOf_k$ the corresponding
semisimplicial $k$-module, view this as a semisimplicial set, and take
its De~Rham complex with divided powers and coefficients in $k$ (as we
want an object over $k$). But this is visibly silly, except possibly
for $k=\bZ$, as the result doesn't depend on the $k$-module structure
of the chain complex we start with. It is doubtful, anyhow, that in
this context, the tautological functor \eqref{eq:124.2} admits a right
adjoint (which we then would call $\widetilde K$ of course). --\enspace
Another noteworthy difference between the present approach towards
defining ``schematic'' homotopy types, and the two earlier ones
(``earlier'' in these notes, at any rate!) is that for $k=\bZ$ say,
the category of ``homotopy types over \bZ'' we get here maps into the
category of ordinary homotopy types, whereas it was the opposite with
the two other approaches.

What, however, I still would like to do, is a little more daydreaming
about the expected formal properties of the basic functors in between
the two categories appearing in \eqref{eq:124.1} and \eqref{eq:124.2}
-- namely, essentially, between ``schematic homotopy types'' and
``abelian'' ones. It will be\pspage{522} convenient to denote the
category of the latter by
\begin{equation}
  \label{eq:124.3}
  \HotabOf_0(k)\tag{3}
\end{equation}
rather than $\D_\bullet\Ab\zc$, the subscript $0$ denoting the
$0$-connectedness condition. The reason for this change in notation is
that, according to the choices made in the basic definitions, the
``$k$-linear'' algebraic interpretation of this category may vary
still. Strictly speaking, in the approach we have been following in
terms of semisimplicial unipotent bundles, the category
\eqref{eq:124.3} cannot really be described as the subcategory of
$0$-connected objects of $\D_\bullet(\AbOf_k)$, we have seen, rather,
that in order to define a functor $\LtH_\bullet$, we have to work with
chain complexes in $\Pro(\AbOf_k)$ rather than in $(\AbOf_k)$; cf.\
section \ref{sec:115}, where this point is made rather
forcefully. (Recall that taking projective limits instead wouldn't
help, because then the linearization functor $\LtH_\bullet$ would no
longer commute to ground ring extension!)\enspace The trouble then is
that in order to define
\[\widetilde K:\HotabOf_0(k) \to \HotOf_0(k),\]
we are obliged to enlarge accordingly the category of models used for
describing $\HotOf_0(k)$, namely take semisimplicial
\emph{pro}-unipotent bundles, rather than just semisimplicial
unipotent bundles. This sudden invasion of the picture by pro-objects
may appear forbidding -- but maybe it will appear less so, or at any
rate kind of natural and inescapable, if we recall that when it will
come to working with homotopy types of general topoi, these are anyhow
``prohomotopy types'' rather, and they can be described only by
working systematically in terms of pro-objects of various
categories. We'll see, however, that there may be a way out of the
``pro''-mess, by using a slightly different, more or less dual
approach towards the idea of ``unipotent bundles''. Of course, one may
also think of using the ``pointed linearization'' $L\subpt$ for
$\LH_\bullet$, instead of $L$, but as was pointed out in section
\ref{sec:117}, this will lead to ``formal'' homology and cohomology
invariants rather than to ``schematic'' ones, and the relation between
these and corresponding invariants for ordinary homotopy types will be
looser still; at any rate, it is suited for describing Postnikov
dévissage in the set-up of ``formal schematic'' homotopy types only,
not for schematic homotopy types. Still, the theory of the former may
have an interest in its own right, even though its relation to
ordinary homotopy types isn't so clear at present. Thus, we may state
that there are around at present three or four different candidates
for possibly fitting the daydreaming I want to write down.\pspage{523}
With this in mind, we shouldn't be too specific (for the time being)
about the exact meaning to be given to the symbols $\HotabOf_0(k)$ and
$\HotOf_0(k)$. It may be safer to replace these by $\HotabOf_1(k)$ and
$\HotOf_1(k)$ (where the subscript $1$ means ``$1$-connected''), all
the more so as we haven't been able yet, in the context of unipotent
bundles, to give even a tentative precise definition of $\HotOf_0(k)$,
i.e., of $\scrM_0(k)$ (mimicking the condition of abelian $\pi_1$ with
trivial action on the higher $\pi_i$'s), except by expressing things
via the associated ordinary homotopy type (passing over to $X_*(k)$),
which looks kind of stupid indeed! But nothing is ``safe'' here
anyhow, and the subscript $0$ looks definitely more natural here than
subscript $1$, so we may as well keep it!

\bigbreak

\noindent\hfill\ondate{5.10.}\par

What we are mainly interested in, for expressing the relationships
between the non-additive category $\HotOf_0(k)$ and its additive
counterpart $\HotabOf_0(k)$, is an array of four functors in between
these, two of which being $\LtH_\bullet$ ((total) ``homology'', or
``linearization'') and $\widetilde K$ (Eilenberg-Mac~Lane functor) in
\eqref{eq:124.1} and \eqref{eq:124.2}, the two remaining ones being
$\Lpi_\bullet$ (``total homotopy'') and $\widetilde S$ (the
``spherical functor'', compare p.~\ref{p:514} \eqref{eq:122.9}),
fitting into the following diagram
\begin{equation}
  \label{eq:124.4}
  \begin{tabular}{@{}c@{}}
    \begin{tikzcd}[baseline=(O.base),sep=large]
      \HotabOf_0(k) \ar[r,shift left,"\widetilde K"]
      \ar[r,shift right,"\widetilde S"'] &
      \HotOf_0(k) \ar[l,bend right,"\LtH_\bullet"{swap}]
      \ar[l,bend left,"\Lpi_\bullet"{name=O}] \\
    \end{tikzcd}.
  \end{tabular}\tag{4}
\end{equation}
I would now like to discuss the main formal properties to be expected
from this set of functors.

% A
\subsection[Adjunction properties]{Adjunction properties:}
\label{subsec:124.A}
\begin{equation}
  \label{eq:124.5}
  \left\{
    \begin{tabular}{@{}l@{}}
      a)\enspace $(\LH_\bullet,\widetilde K)$ is a pair of adjoint functors,
    \\
      b)\enspace $(\widetilde S, \Lpi_\bullet)$ is a pair of adjoint functors.
    \end{tabular}
  \right.\tag{5}
\end{equation}

Thus, we get adjunction homomorphisms (for $L_\bullet$ in
$\HotabOf_0(k)$ and $X$ in $\HotOf_0(k)$)
\begin{equation}
  \label{eq:124.6a}
  \begin{cases}
    X\to \widetilde K(\LtH_\bullet(X)) &
    \text{($\eqdef \widetilde W(X)$ (``pointed linearization''))} \\
    \LtH_\bullet(\widetilde K(L_\bullet)) \to L_\bullet &
    \text{,}
  \end{cases}\tag{6a}
\end{equation}
and\pspage{524}
\begin{equation}
  \label{eq:124.6b}
  \begin{cases}
    \widetilde S(\Lpi_\bullet(X)) \to X & \text{} \\
    L_\bullet \to \Lpi_\bullet(\widetilde S(L_\bullet)) & \text{.}
  \end{cases}\tag{6b}
\end{equation}

The adjunction formulas for the two pairs in \eqref{eq:124.5} are
\begin{equation}
  \label{eq:124.7}
  \left\{
    \begin{tabular}{@{}lll@{}}
      a) & $\Hom(\LtH_\bullet(X),L_\bullet) \simeq \Hom(X,\widetilde
           K(L_\bullet))$ &\\
      b) & $\Hom(L_\bullet, \Lpi_\bullet(X)) \simeq \Hom(\widetilde
           S(L_\bullet), X)$ & .
    \end{tabular}
  \right.\tag{7}
\end{equation}

The first-hand side of \hyperref[eq:124.7]{a)} should be viewed as
\emph{cohomology} of $X$ with coefficients in $L_\bullet$. More
specifically, as $\widetilde{\mathrm H}^0(X,L_\bullet)$, and in case
$L_\bullet = \pi[n]$, posing
\begin{equation}
  \label{eq:124.8a}
  K(\pi[n]) \eqdef K(\pi,n),\tag{8a}
\end{equation}
we get that the Eilenberg-Mac~Lane objects $K(\pi,n)$ represents the
cohomology functor
\[ X\mapsto \widetilde{\mathrm H}^0(X,\pi[n]) = \mathrm
H^n(X,\pi).\]

Symmetrically, the first-hand side of \hyperref[eq:124.7]{(7b)} should
be viewed as a ``mixed homotopy module'' of $X$ (relative to the
``co-coefficient'' $L_\bullet$), I am tempted to denote it as
\[\pi_0(X,L_\bullet),\quad\text{resp.\ $\pi_n(X,H)$ if
  $L_\bullet=H[n]$;}\]
thus, posing
\begin{equation}
  \label{eq:124.8b}
  \widetilde S(H[n]) \eqdef S(H,n) \quad\text{(sphere-like object for
    $H,n$),}\tag{8b}
\end{equation}
we get that the $H$-sphere $S(H,n)$ over $k$ represents the
(covariant) functor on $\HotOf_0(k)$
\[X\mapsto \pi_n(X,H).\]
In case $H=k$, we get
\[\pi_n(X,k) = \Hom(k[n],\Lpi_\bullet(X)) = \mathrm
H_n(\Lpi_\bullet(X)) \eqdef \pi_n(X),\]
and we get that the ``usual'' homotopy module functor
\[X\mapsto \pi_n(X)\]
is represented by the ``usual'' $n$-sphere (over $k$ however!)
$S(n,k)$, as it should. (This is of course the main justification why
we expect $(\widetilde S,\Lpi_\bullet)$ to be a pair of adjoint
functors, whereas adjunction for the pair $(\LtH_\bullet,\widetilde
K)$ is already fairly familiar from the set-up of ordinary homotopy
types.)

The adjunction formulæ \eqref{eq:124.7} show that the objects
$K(L_\bullet)$ in $\HotOf_0(k)$ are group objects, whereas the objects
$\widetilde S(L_\bullet)$ are co-group objects. This gives some inner
justification for calling $\widetilde W(X)=\widetilde
K(\LtH_\bullet(X))$ the\pspage{525} (pointed) ``linearization'' of the
schematic homotopy type $X$, which maps into the former by the first
adjunction morphism in \eqref{eq:124.6a}. Dually, we may call
$\widetilde S(\Lpi_\bullet(X))$ the (pointed) ``co-linearization'' of
$X$, it maps into $X$ by the first adjunction morphism in
\eqref{eq:124.6b}.

The source $\LtH_\bullet(\widetilde K(L_\bullet))$ in the second
adjunction map \eqref{eq:124.6a} may be viewed as Eilenberg-Mac~Lane
type (total) homology, corresponding to the chain complex
$L_\bullet=\pi[n]$ (but in the schematic sense of course!), whereas
the target $\Lpi_\bullet(\widetilde S(L_\bullet))$ in the second
adjunction map \eqref{eq:124.6b} may be viewed as ``total homotopy''
of a sphere-type space, reducing in case $L_\bullet=\pi[n]$ to the
total homotopy of the standard $n$-sphere over $k$, $S(n,\pi)$. In
case $k=\bZ$, we expect of course the homology groups of this chain
complex to be the ordinary homotopy groups of the usual sphere $S^n$
-- cf.~\ref{subsec:124.G} below.

% B
\subsection[Inversion formulæ]{Inversion formulæ:}
\label{subsec:124.B}%
two functorial isomorphisms
\begin{equation}
  \label{eq:124.9}
  \left\{
    \begin{tabular}{@{}lll@{}}
      a) & $\Lpi_\bullet(\widetilde K(L_\bullet)) \tosim L_\bullet$ & , \\
      b) & $\LtH_\bullet(\widetilde S(L_\bullet)) \tosim L_\bullet$ & .
    \end{tabular}
  \right.\tag{9}
\end{equation}

Maybe we should have begun with these, as they kind of fix the meaning
of the two functors $\widetilde K, \widetilde S$ in terms of
$\LtH_\bullet, \Lpi_\bullet$, which may be viewed as embodying
respectively the two main sets of invariants of a homotopy type,
namely homology and homotopy.

% C
\subsection[Hurewicz map]{Hurewicz map:}
\label{subsec:124.C}%
\begin{equation}
  \label{eq:124.10}
  \Lpi_\bullet(X) \to \LtH_\bullet(X).\tag{10}
\end{equation}

We get such a map by applying $\Lpi_\bullet$ to the linearization map
in \eqref{eq:124.6a}, and using \hyperref[eq:124.9]{(9a)};
symmetrically, we may apply $\LtH_\bullet$ to the colinearization map
\eqref{eq:124.6b}, and use \hyperref[eq:124.9]{(9b)}. We get a priori
two maps \eqref{eq:124.10}, and the statement is that these two maps
are the same. Moreover, we have the all-important \emph{Hurewicz
  theorem}: The first non-vanishing homology objects for
$\Lpi_\bullet(X)$ and $\LtH_\bullet(X)$ occur in the same dimension,
$n$ say, and \eqref{eq:124.10} induces an isomorphism for $\mathrm
H_n$, an epimorphism for $\mathrm H_{n+1}$.

% D
\subsection[Exactness properties]{Exactness properties:}
\label{subsec:124.D}%
they can be stated shortly by saying that in each one of the two pairs
of adjoint functors \eqref{eq:124.5}, the left adjoint one respects
cofibration sequences, whereas the right adjoint respects fibration
sequences. This may be detailed as four exactness statements, one for
each one of the four basic functors.

Thus,\pspage{526} $\LtH_\bullet$ respects cofibration sequences, which
means essentially that for such a sequence in $\HotOf_0(k)$
\[Y\to X\to Z,\quad\text{$Z$ the homotopy cofiber of $Y\to X$,}\]
we get a corresponding exact triangle in $\HotabOf_0(k)$
\begin{equation}
  \label{eq:124.11b}
  \begin{tabular}{@{}c@{}}
    \begin{tikzcd}[baseline=(O.base),column sep=small]
      & \LtH_\bullet(Z) \ar[dl,"i"'] & \\
      \LtH_\bullet(Y) \ar[rr] & &
      |[alias=O]| \LtH_\bullet(X) \ar[ul]
    \end{tikzcd},
  \end{tabular}\tag{11b}
\end{equation}
which is the most complete and elegant way, I guess, for expressing
behaviour of homology with respect to homotopy cofibers and
suspensions. Dually, $\Lpi_\bullet$ respects fibration sequences,
which means that for such a sequence in $\HotOf_0(k)$
\[Z\to X\to Y,\quad\text{$Z$ the (connected) homotopy fiber of $X\to Y$,}\]
we get a corresponding exact triangle in $\HotabOf_0(k)$
\begin{equation}
  \label{eq:124.11a}
  \begin{tabular}{@{}c@{}}
    \begin{tikzcd}[baseline=(O.base),column sep=small]
      & \Lpi_\bullet(Z) \ar[dl,"i"'] & \\
      \Lpi_\bullet(X) \ar[rr] & &
      |[alias=O]| \Lpi_\bullet(Y) \ar[ul]
    \end{tikzcd},
  \end{tabular}\tag{11a}
\end{equation}
which is a more complete and elegant way of stating the exact homotopy
sequence of a fibration, in terms of total homotopy.

Formula \eqref{eq:124.11b} implies that $\LtH_\bullet$ commutes with
suspension functors $\Sigma$, the latter in $\HotabOf_0(k)$
(visualized as a derived chain complex category) is just the shift
functor
\[\Sigma\subab: L_\bullet\mapsto L_\bullet[1]\quad
(L[1]_n = L_{n-1},\]
which is a \emph{fully faithful} functor. Dually, formula
\eqref{eq:124.11a} implies that $\Lpi_\bullet$ commutes to the
loop-space functors $\Omega^0$; the latter in $\HotabOf_0(k)$
cannot be described as just a shift in opposite direction
\[L_\bullet \mapsto L_\bullet[-1],\]
as this will get us out of $0$-connected chain complexes, we have to
truncate, moreover, at dimension $1$ (afterwards, or at dimension $2$
beforehand). This functor is not fully faithful therefore -- we loose
something when passing from $L_\bullet$ to
$\Omega\subab^0(L_\bullet)$.

As for the two functors $\widetilde K,\widetilde S$ in the opposite
direction, from $\HotabOf_0(k)$ (i.e., essentially chain complexes) to
$\HotOf_0(k)$, that the first one respects fibration sequences is
surely quite familiar a fact (and kind of tautological) in the set-up
of ordinary homotopy types. That the second, less familiar functor
$\widetilde S$ should respect cofibration sequences\pspage{527} should
be useful in order to give a more or less explicit construction of
$\widetilde S(L_\bullet)$, for given $L_\bullet$, in terms of the
``spheres over $k$'' $S(H,n)$.

\begin{remarks}
  1)\enspace The superscript $0$ for the loop functor $\Omega^0$ or
  $\Omega\subab^0$ should not be confused with the subscript $n$ in
  the iterated loop-space functor $\Omega_n$, it is added here to
  suggest that we are taking the \emph{neutral} connected component of
  the ``true'' full loop-space, this being imposed by the restriction
  of working throughout with $0$-connected objects; likewise, the
  ``homotopy fiber'' operation in the present context should be viewed
  as meaning ``connected component at the marked point of the full
  homotopy fiber''. These necessary readjustments of the usual notions
  is being felt as an unwelcome feature (of which I have become aware
  only at this very moment, I confess, through the writing of the
  notes). Intuitively, the restriction to $0$-connected (pointed)
  homotopy types doesn't seem to imperative, technically, however,
  when working with semisimplicial schematic models, we had felt like
  introducing the condition $X_0=e$, which may be viewed as a strong
  form of a $0$-connectedness condition. Truth to tell, it isn't so
  clear that this condition is going to be of great utility -- anyhow,
  it isn't enough to ensure what we're really after at present (namely
  abelian $\pi_1$ and trivial action on the higher $\pi_i$'s), a
  condition which is anyhow independent of any $0$-connectedness type
  assumption, and is moreover (it would seem) stable under the basic
  fiber and cofiber operations. To sum up, it may well turn out that
  we better replace the categories of $0$-connected homotopy types
  $\HotabOf_0(k)$ and $\HotOf_0(k)$, by slightly larger ones, so as to
  get rid of the $0$-connectedness restriction. This, however, is at
  present a relatively minor point, and therefore we'll leave the
  notations as they are.

  2)\enspace Behaviour of $\LtH_\bullet$ with respect to
  \emph{fibration} sequences (instead of cofibration sequences) is a
  relatively delicate matter, it is governed by the Leray spectral
  sequence, whose initial term is the homology of the base $Y$ with
  ``coefficients'' in the homology of the fiber. I wonder if there is
  anything similar for the behavior of $\Lpi_\bullet$ with respect to
  cofibration sequences?

  3)\enspace In the display of the main expected properties of the
  four basic functors, there is a striking \emph{symmetry}, which we
  tried to stress by the way of presented the main formulæ. One way to
  express this symmetry is to say that things look as if there was an
  auto-duality in the pair of categories $(\HotabOf_0(k),
  \HotOf_0(k))$, namely a pair of contravariant involutive functors
  $(D\subab,D)$, each of which interchanges\pspage{528} fibration and
  cofibration sequences, i.e., transforms fibers into cofibers and
  vice versa, and the pair of functors interchanging $\LtH_\bullet$
  and $\Lpi_\bullet$ on the one hand, $\widetilde K$ and $\widetilde
  S$ on the other. This heuristic formulation is compatible with all
  we have stated so far, except one thing -- namely, the suspension
  functor $\Sigma\subab$ in $\HotabOf_0(k)$ is fully faithful, whereas
  the (supposedly ``dual'') loop-space functor $\Omega\subab^0$ is
  not. Thus, it will be more accurate to say that, if we view the
  formulaire developed so far as being the description of a certain
  structure type, whose basic ingredients are two categories
  $\scrH\subab$ and \scrH, endowed with fiber and cofiber operations,
  tied by four functors as above satisfying a bunch of properties,
  then the axioms are autodual in an obvious sense; namely if they are
  satisfied for a system $(\scrH\subab,\scrH,\LtH_\bullet,\widetilde
  K,\widetilde S,\Lpi_\bullet)$, they are equally satisfied by the
  system $(\scrH\subab\op,\scrH\op,\Lpi_\bullet,\widetilde
  S,\widetilde K,\LtH_\bullet)$. (Of course, among the properties of
  the structure type, we are \emph{not} going to include that the
  suspension functor in $\scrH\subab$ is fully faithful!)\enspace As
  already stated before, the main reason for introducing the fourth
  functor $\widetilde S$ in the picture, was because it was felt that
  this was lacking in order to round it up. Thus for any kind of
  notion or statement in this set-up, suggested by some kind of
  geometric insight, it becomes automatic to look at its dual and see
  whether it makes sense.

  As for formal autoduality, we already noticed there is none, even in
  $\HotabOf_0(k)$ just by itself. Thinking of this category as
  $\D_\bullet(\AbOf_k)$, namely as a full subcategory of the derived
  category $\D(\AbOf_k)$, we cannot help, though, but thinking of the
  standard ``dualizing'' functor
  \[L_\bullet \mapsto \RHom(L_\bullet,k): \D^-(\AbOf_k)\to
  \D^+(\AbOf_k),\]
  inducing a perfect duality within the category of ``perfect''
  objects in $\D(\AbOf_k)$, namely objects which are isomorphic to
  those which may be described by complexes in $\AbOf_k$ having a
  bounded span of degrees, and all of whose components are projective
  of finite type. But this autoduality of course transforms
  \emph{chain} complexes into \emph{cochain} complexes -- we have to
  shift these in order to get chain complexes again. This suggests
  that maybe we'll hit upon an actual autoduality, provided we go over
  from $\HotOf_0(k)$ to a suitable ``stabilized'' category, the
  suspension category say (deduced from the initial one by introducing
  formally a quasi-inverse for the suspension functor). The homology
  functor $\LtH_\bullet$ extends to a functor from the suspension
  category to the corresponding category for $\HotabOf_0(k)$, say
  $\D^-(\AbOf_k)$. The $\Lpi_\bullet$\pspage{529} functor, though, is
  lost on our way -- I am afraid I am confusing the kind of duality I
  am after here, with a rather different type of duality, kin to
  Poincaré duality, and discovered I believe by J.~H.~C.~Whitehead, in
  the context of spaces having the homotopy type of a finite
  complex. For such a complex, the idea (as far as I remember) is to
  embed $X$ into a large-dimensional sphere, and to take the
  complement $X\upvee$ of an open tubular neighborhood of $X$. Up to
  suspension, the homotopy type of $X\upvee$ does not depend on the
  choices made and (shifting back by $n$) we thus get a canonical
  object in the suspension category, depending contravariantly on
  $X$. The functor $X\mapsto X\upvee$ (if I got it right) is an
  autoduality of the relevant full subcategory of the suspension
  category, and the (reduced) homology functor $\LtH_\bullet$ maps
  this subcategory into perfect complexes, and commutes to
  autodualities. The whole story seems tailored towards a study of
  duality relations for the functor $\LtH_\bullet$ exclusively --
  without any reference to the homotopy invariants $\pi_i$. I don't
  know if one can devise a similar story for $\Lpi_\bullet$, by
  stabilizing with respect to the loop space functor (or is this just
  nonsense?). At any rate, there doesn't seem any autoduality in view,
  exchanging homology and homotopy invariants\dots
\end{remarks}

% E
\subsection[Conservativity properties]{Conservativity properties:}
\label{subsec:124.E}%
The functors $\LtH_\bullet$ and $\Lpi_\bullet$ are both conservative,
i.e., a map in $\HotOf_0(k)$ which by either of these functors becomes
an isomorphism, is an isomorphism. (Here of course the
$0$-connectedness assumption for the homotopy types we are working
with is essential, as far as the functor $\Lpi_\bullet$ is concerned
at any rate.)

% F
\subsection[Base change properties]{Base change properties:}
\label{subsec:124.F}%
All four functors, and the adjunction and inversion maps
\eqref{eq:124.7} and \eqref{eq:124.9}, are compatible with ring
extension $k\to k'$, it being understood that such a ring homomorphism
defines functors
\begin{equation}
  \label{eq:124.12}
  \HotabOf_0(k) \to \HotabOf_0(k'), \quad
  \HotOf_0(k) \to \HotOf_0(k')\tag{12}
\end{equation}
compatible with the fibration and cofibration structures.

In opposite direction, there should be, too, a ``restriction'' functor
(less important, though, I feel), I didn't try to find out what its
formal properties should be with respect to the formalism developed
here.

% G
\subsection[Comparison with ordinary homotopy types]{Comparison with
  ordinary homotopy types:}
\label{subsec:124.G}%
We got functors
\begin{equation}
  \label{eq:124.13}
  \HotabOf_0(k) \to \HotabOf = \D_\bullet\Ab, \quad
  \HotOf_0(k) \to \HotOf_0,\tag{13}
\end{equation}
(the first being interpreted as ``forget $k$'' functor, the second as
a sections functor on semisimplicial schematic models). This pair
of\pspage{530} functors is compatible with the functors $\widetilde
K$, but (even if $k=\bZ$) definitely \emph{not} with their left
adjoints $\LtH_\bullet$, namely with homology. Despite this fact, we
hope if $k=\bZ$ that the functors \eqref{eq:124.13} are compatible
with the functors $\widetilde S$ (assuming that $\widetilde S$ can
actually be constructed in the discrete set-up too), so that spheres
are transformed into spheres. Compatibility with $\Lpi_\bullet$
doesn't have a meaning strictly speaking, as this functor is not
defined on ordinary homotopy types, however, the functors $\pi_i$ are
defined in both contexts, and the sections functor should commute to
these (for arbitrary $k$). Thus, the only serious incompatibility
trouble concerns the homology functor $\LH_\bullet$ and should not
arise, however, for sphere-like objects $S(H,n)$. I forgot to state
from the outset that the two functors \eqref{eq:124.13} are expected
to respect fibration and cofibration sequence structures, of course.

\bigbreak

\noindent\hfill\ondate{13.10.}\par

% 125
\hangsection[Digression on Baues' cofibration and fibration
\dots]{Digression on Baues' cofibration and fibration categories, and
  on weak equivalences alone as the basic data for a model
  category.}\label{sec:125}%
During the last eight days, I have been busy with a number of things,
which left little leisure for mathematical ponderings. I got a letter
from H.~J.~Baues, who had seen my notes, which induced him to send me
his preprint ``On\scrcomment{I'm not sure what this refers to; perhaps
  \textcite{Baues1985}\dots} the homotopy classification problem'' (Chapters~I
to~V + chapter~Ext). I spent two evenings looking through parts of
these, where Baues carries as far as possible (namely quite far
indeed) the homotopy formalism in the context of his so-called
``cofibration (or fibration) categories'', using as his leading
thread his ideas on ``obstruction theory''. He makes the point that he
tried by his basic notion to pinpoint the weakest axiomatic set-up,
sufficient however for developing all the major familiar (and even
some not at all familiar!) notions, operations and statements of usual
homotopy theory. In his letter, he suggested that maybe in any
``universe'' where homotopy constructions make sense, one or the other
of his two mutually dual set-ups should be around. Such suggestion was
of course quite interesting for my present reflections, as I do have
the hope indeed that there exists a ``universe'' of schematic homotopy
types, which may be described in terms of the models (namely
semisimplicial unipotent bundles) I have been using so far, or at any
rate by closely related kinds of models. More specifically, I do hope
for the two kinds of operations to make sense, namely ``integration''
(including homotopy cofibers and suspensions) and ``cointegration''
(including homotopy fibers and loop objects), which in Baues' set-up
should correspond on the category of models to a structure of a
cofibration category and a fibration category, respectively. As the
kind of models I have been working with\pspage{531} don't allow for an
amalgamated sum (or ``push-out'') construction, except in the trivial
case when one of the two arrows to be amalgamated is an isomorphism,
it is clear that we cannot hope to get with these models a cofibration
category. There may be, however, a fibration category structure, and I
started playing around a little with a possible notion of Kan
fibration -- without coming to a definite conclusion yet, however. I
feel I shouldn't dwell much longer still, for the time being, on
getting off the ground a theory of schematic homotopy types, maybe
I'll come back upon it later, after next chapter, when (hopefully)
I'll be a little more ``in the game'' of Kan type conditions, closed
model category structures and Baues' ``halved'' variants of these.

At any rate, whether or not a homotopy cofiber construction can be
carried through for schematic homotopy types, it seems rather clear to
me that Baues' suggestion or expectation, about the set-up of
cofibration and fibration categories he ended up with, is not quite
justified. Already by the time (in the late sixties) when I first
heard from Quillen about his approach (and the same applied to Baues'
ones), when applied say to semisimplicial objects in some category $A$
as ``models'', is pretty strongly relying on the existence of
``enough'' projectives (or dually, of enough injectives, if working
with cosemisimplicial objects) in $A$. When $A$ is a topos say, then
the category $\bHom(\Simplexop,A)$ of semisimplicial objects of $A$
doesn't carry, it seems (except in very special cases, say of a
totally disconnected topos), a reasonable structure of a Quillen model
category, nor even (I would think) of a cofibration or a fibration
category in the sense of Baues; however, I am pretty sure that the
derived category (with respect to the notion of weak equivalence
introduced by Illusie\scrcomment{AG wanted to insert a reference, but
  didn't, probably to \textcite{Illusie1971,Illusie1972}\dots}) has
important geometric meaning and is indeed a ``universe for homotopy
types'' -- and the first steps in developing such theory have been
taken by Illusie already in Chapter~I of his thesis. Quite similarly,
if $A$ is an abelian category, Verdier's derived category $\D(A)$,
deduced from the category $\mathrm C(A)$ of complexes in $A$ by
localizing with respect to the set $W$ of quasi-isomorphisms, does
allow for a homotopy formalism (as does any ``triangulated category''
in the sense of Verdier), however, it doesn't seem that this formalism
may be deduced from a structure of a cofibration category or a
fibration category on $\mathrm C(A)$. The same I guess holds for the
subcategories $\mathrm C^-(A)$ and $\mathrm C^+(A)$, giving rise to
$\D^-(A)$ and $\D^+(A)$, which are more important still than $\D(A)$
in the everyday cohomology formalism of ``spaces'' of all kinds
(namely, essentially, of ringed topoi). When\pspage{532} $A$ admits
enough projectives or enough injectives, respectively, these
categories (as pointed out by Quillen) are associated to closed model
structures on $\mathrm C^-(A)$ and $\mathrm C^+(A)$, respectively, but
(it seems) not otherwise.

It strikes me that nobody apparently so far has tried to develop
homotopy theory, starting as basic data with a category of models $M$
together with a set $W$ of weak equivalences in $M$, satisfying
suitable assumptions, without giving moreover a notion of
``fibration'' or of ``cofibrations'' in $M$. Various examples,
including the all-important case (from my point of view) of \Cat,
suggest that the choice of the notion of fibration or cofibration
isn't really so imperative, that it is to a certain extent arbitrary,
different choices (compatible with the basic data $W$) leading to
essentially ``the same'' homotopy theory. This is due to the fact that
they lead to the same notions of ``integration'' and ``cointegration''
of homotopy types, which depends indeed only on $W$ and which, in my
eyes, are \emph{the} two main operations of homotopy theory (compare
section~\ref{sec:69}). They seem to me the key for defining such a
thing as a specific ``homotopy theory'', independently of any
particular choice of a category of models (+ extra structure on it,
and notably, weak equivalences) used for describing it. The precise
technical notion achieving this is of course the notion of a
``derivator'' -- and I do hope that it shouldn't be too awfully hard
to construct, for instance, the ``derivator of schematic homotopy
types'', and maybe even characterize it axiomatically (as well as the
derivator of usual homotopy types), without having to make any mention
of models for such characterization. One may say that, after the one
major step in the foundations of homological algebra, consisting in
introducing the derived category of an abelian category (and
systematically working with derived categories for stating the main
facts about cohomology of all kind of ``spaces'', namely topoi, such
as usual topological spaces, schemes and the like\dots), the work on
foundations more or less stopped short, while the next step to take
was to come to a grasp of the full structure involved in derived
categories, namely the structure of a derivator. And it turns out that
as well the step of taking derived categories (namely localizations of
suitable model categories), as the next, namely introducing the
corresponding structure of a derivator, make good sense also in
non-abelian set-ups, namely for doing ``homotopy theory'', thus viewed
as a non-commutative version of homology theory.\pspage{533}

% 126
\hangsection[Dissymmetry of $\LtH_*$ versus $\Lpi_*$ (no
filtration dual to \dots]{Dissymmetry of
  \texorpdfstring{$\LtH_*$}{LH} versus
  \texorpdfstring{$\Lpi_*$}{Lpi} (no filtration dual to Postnikov
  dévissage).}\label{sec:126}%
Let's come back, though, to schematic homotopy types. Last thing we
looked at was the four basic functors between ``schematic homotopy
types'' and ``linear schematic homotopy types'' over a given ground
ring $k$, making up the two basic categories
\[\HotabOf_0(k)\quad\text{and}\quad
\HotOf_0(k).\]
We have been dwelling somewhat on the remarkable formal symmetry to be
expected for these relations. It is tempting, then, to try and dualize
any kind of basic notion or construction which makes sense in terms of
the basic data, namely the four functors and the adjunction and
inversion maps relating them. Maybe the very first thing which forces
attention is the Postnikov dévissage of an object of $\HotOf_0(k)$,
which had been (together with abelianization) the very starting point
for our approach towards schematic homotopy types. One main ingredient
of this dévissage is the ``tower'' of the Cartan-Serre type functors
\begin{equation}
  \label{eq:126.1}
  X\mapsto X_n\tag{1}
\end{equation}
and maps
\begin{equation}
  \label{eq:126.2}
  X\to X_n\tag{2}
\end{equation}
(with $n$ a natural integer), where $X_n$ is deduced from $X$ by
``killing its $\pi_i$'s for $i>n$''. We may call an object $Y$ of
$\HotOf_0(k)$ \emph{$n$-co-connected} (a notion in a way symmetric to
$n$-connectedness) if $\pi_i(X)=0$ for $i>n$, and denote by
\begin{equation}
  \label{eq:126.3}
  \HotOf_0(k)_{/n}\tag{3}
\end{equation}
the full subcategory of $\HotOf_0(k)$ made up with these objects,
which is therefore the inverse image by the functor
\[\Lpi_i : \HotOf_0(k)\to\HotabOf_0(k)\]
of the corresponding full subcategory
\begin{equation}
  \label{eq:126.4}
  \HotabOf_0(k)_{\le n}\tag{4}
\end{equation}
of $\HotabOf_0(k)$. The most natural way for defining $X_n$ in terms
of $X$ (in an axiomatic set-up with the four functors as basic data),
together with the canonical ``fibration'' \eqref{eq:126.2}, is by
describing \eqref{eq:126.2} as the ``universal'' map (in
$\HotOf_0(k)$) of $X$ into an $n$-co-connected object. In other words,
we are surmising that the inclusion functor
\[\HotOf_0(k)_{\le n} \hookrightarrow \HotOf_0(k)\]
admits a left adjoint, and the latter is denoted by $X\mapsto
X_n$. This\pspage{534} description gives rise at once to the familiar
``tower'' structure for variable $n$
\begin{equation}
  \label{eq:126.5}
  X_{n+1} \to X_n \to \dots \to X_1 \to X_0 (=e).\tag{5}
\end{equation}
Intuitively, the maps in \eqref{eq:126.5} are viewed as being
(surjective) fibrations between (connected) ``spaces'', $X$ being
viewed as a kind of inverse limit of the $X_n$'s. Dually, we would
expect to get a sequence of inclusions
\begin{equation}
  \label{eq:126.star}
  {}_0X \hookrightarrow {}_1X \hookrightarrow \cdots \hookrightarrow
  {}_nX \hookrightarrow {}_{n+1}X \hookrightarrow \cdots,\tag{*}
\end{equation}
with $X$ appearing as a kind of direct limit. In the set-up of
ordinary homotopy types, modelized by semisimplicial sets, one will
think at once of the filtration by skeleta -- which however isn't
quite the right thing surely, because if we dualize the familiar
Cartan-Serre requirement on \eqref{eq:126.2} (namely that it induces
an isomorphism for $\pi_i$ for $i\le n$), we see that the
``inclusion''
\[{}_nX \hookrightarrow X\]
should induce an isomorphism for $\mathrm H_i$ for $i\le n$, which
isn't quite true for the skeletal filtration (it is OK for $i\le n-1$
only); the condition
\[\mathrm H_i({}_nX) = 0\quad\text{for $i>n$}\]
is OK, though, for this filtration. So the next idea would be to
modify a little the $n$-skeleton ${}_nX$, to straighten things out. I
played around some along these lines, and after some initial optimism,
came to the feeling that there does not exist (in the discrete nor in
the schematic set-up) such an increasing canonical filtration of a
homotopy type. I didn't make any formal statement and proof for this
(in the set-up of ordinary homotopy types, say), however, in the
process of playing around in became soon clear that in various ways,
\emph{there is some essential dissymmetry} between the seemingly
``dual'' situations, when trying to get the two types of
``filtrations'' of the object $X$. One dissymmetry occurs already in
the very definition of the subcategory \eqref{eq:126.3}, and of the
corresponding ``dual'' subcategory
\begin{equation}
  \label{eq:126.6}
  {}_{(\le n)}\HotOf_0(k)\tag{6}
\end{equation}
of objects satisfying
\[\mathrm H_i(Y) = 0\quad\text{for $i>n$.}\]
Namely, both subcategories \eqref{eq:126.3} and \eqref{eq:126.6} are
defined in terms of the \emph{same} subcategory of $\HotabOf_0(k)$,
namely \eqref{eq:126.4}, as the inverse image of the\pspage{535}
latter by either $\Lpi_\bullet$, or $\LH_\bullet$. Now, the point is
that the properties of this subcategory, with respect to the inner
structure of $\HotabOf_0(k)$ (involving the ``left'' and the ``right''
homotopy operations, notably the suspension and the loop functors,
respectively) are by no means autosymmetric. The main dissymmetry, it
would seem, turns out in this, that \eqref{eq:126.4} is stable under
the loop functor, and by no means under the suspension functor. This
is the reason why, even if we assume that (in analogy to what happens
for the categories \eqref{eq:126.3}) the inclusion functors from the
categories \eqref{eq:126.5} into $\HotOf_0(k)$ do admit the relevant
(namely right) adjoints (which I greatly doubt anyhow\dots), and thus
give rise for any object $X$ to an increasing filtration
\eqref{eq:126.star}, the Postnikov-type relationships between $X_n$
and $X_{n+1}$ cannot be quite dualized to a similar relationship
between ${}_nX$ and ${}_{n+1}X$. We may think next, of course, of
defining an increasing sequence of subcategories \eqref{eq:126.5} of
$\HotOf_0(k)$ in terms of $\LH_\bullet$ and a corresponding sequence
of subcategories of $\HotabOf_0(k)$, different from the categories
\eqref{eq:126.4}, and stable under suspensions. But there doesn't seem
to be anything reasonable around along these lines.

To sum up, it doesn't seem one should overemphasize the somewhat
startling symmetry which appeared in section~\ref{sec:124} between
``homotopy'' embodied in $\Lpi_\bullet$, and ``homology'' embodied in
$\LH_\bullet$ -- in some essential respects, it would seem that the
corresponding two functors do have non-mutually symmetry properties. I
guess I have to apologize for having taken that long for coming to a
conclusion which, presumably, must be felt as a kind of self-evidence
by all homotopy people!

% 127
\hangsection[Schematic homotopy types and Illusie's derived category
\dots]{Schematic homotopy types and Illusie's derived category for a
  topos.}\label{sec:127}%
Before leaving (for the time being) the topic of schematic homotopy
types and schematization, I would like still to add a few comments,
about various possibilities for working with different kinds of models
for defining schematic homotopy types. My point here is not in
replacing the basic test-category we are working with, here
$\Simplex$, by some other (say the category $\Globe$ of standard
hemispheres) -- this choice I feel should be more or less irrelevant,
so we may as well keep $\Simplex$. Thus, we are going to work with
semisimplicial objects, and the main question then is (for a given
ground ring $k$) to say precisely what kind of objects we are allowing
(or imposing!) as components for our complexes. At any rate, they
should be ``objects over $k$'', and the most\pspage{536} encompassing
choice for such objects seems to be sheaves on the category of all
schemes over $k$ (or equivalently, of all affine schemes over $k$),
for a suitable topology such as the fpqc topology (compare
section~\ref{sec:111}, pages \ref{p:446}--\ref{p:447}). Apart from the
technico-logical nuisance of this not being a \scrU-site, where
\scrU{} is our basic universe (cf.\ p.~\ref{p:492}), which we'll
ignore here (as it isn't really too serious a difficulty), when
working the semisimplicial sheaves, the embryo of foundational work of
Illusie's on the derived category of a topos via semisimplicial
sheaves\scrcomment{\textcite{Illusie1971}, chapter 1} becomes
available. Thus, for a map of semisimplicial sheaves over $k$,
\begin{equation}
  \label{eq:127.1}
  X_*\to Y_*,\tag{1}
\end{equation}
we know already what it means to be a quasi-isomorphism (= weak
equivalence). It seems likely that, even when drastically restricting
the sheaves allowed as components for our semisimplicial models, the
notion of quasi-isomorphism relevant for these models should be the
same as Illusie's for the encompassing topos. This is one point we
have been neglecting so far. I am not going here to recall Illusie's
definition (in terms of the homotopy sheaves
$\boldsymbol\pi_i(X_*,s)$\scrcomment{in the typescript, the ``$\pi$''
  is underlined\dots} where $s$ is a section of $X_*$), or translate
it into cohomological terms. One point I want to make here, is that it
is by no means automatic that if \eqref{eq:127.1} is a weak
equivalence, the same holds for the corresponding map of
semisimplicial sets
\begin{equation}
  \label{eq:127.2}
  X_*(k) \to Y_*(k).\tag{2}
\end{equation}
Thus, there is no more-or-less tautological ``sections'' functor from
Illusie's derived category to the category \Hot{} of usual homotopy
types. (There surely \emph{is} a canonical functor, though, in the
case of a general topos $T$ and Illusie's corresponding derived
category, namely what we would like to call the \emph{``cointegration
  over $T$'' functor}, which should come out of the formalism of
stacks we haven't begun to develop yet.)\enspace
Presumably, however, when working with semisimplicial models whose
components are restricted to be unipotent or something pretty close to
these (see below for examples), whenever \eqref{eq:127.1} is a weak
equivalence, the same will hold for \eqref{eq:127.2}. As the choice of
objects (let's call them simply the ``bundles'') we are allowing
should clearly be stable under ring extension, it will then follow
that more generally, for any algebra $k'$ over $k$, the corresponding
map
\begin{equation}
  \label{eq:127.2prime}
  X_*(k')\to Y_*(k')\tag{2'}
\end{equation}
is\pspage{537} a weak equivalence, too. Thus, a ``schematic homotopy
type over $k$'' should define a functor $\Alg_{/k} \to \HotOf$.

For any choice of a ``section''
\begin{equation}
  \label{eq:127.3}
  s\in X_0(k)\tag{3}
\end{equation}
of $X_*$, Illusie's constructions yield sheaves
\begin{equation}
  \label{eq:127.4}
  \bpi_i(X_*,s)\quad
  (i\ge0),\tag{4}
\end{equation}
where \bpiz{} is a sheaf of sets, $\bpi_1$ a sheaf of groups, acting
on the higher $\bpi_i$'s which are abelian sheaves. We will be mainly
interested of course in the case when \bpiz{} is the final sheaf
(we'll say that $X_*$ is \emph{relatively $0$-connected over $k$}),
and moreover $\bpi_1$ is abelian and its action on the higher $\bpi_i$
is trivial (let's say in this case that the relative homotopy type
defined by the semisimplicial sheaf $X_*$ is
\emph{pseudo-abelian}). In this case, up to canonical isomorphisms,
the abelian sheaves $\bpi_i(X_*,s)$ do not depend on the choice of
$s$, and (provided the pseudoabelian notion is defined locally) they
make sense, independently even of the existence of a section $s$. Our
hope is, by suitably restricting our notion of ``bundle'', to get on
the abelian sheaves $\bpi_i$ for $i\ge2$, or even on all $\bpi_i$
($i\ge1$), a natural structure of an $\scrO_k$-module, and one
moreover which in many ``good'' cases turns it into a quasi-coherent
$\scrO_k$-module. Also, the natural maps
\begin{equation}
  \label{eq:127.5}
  \pi_i(X_*(k)) \to \bpi_i(X_*)(k)\tag{5}
\end{equation}
should be isomorphisms, i.e., taking $\pi_i$ and $\bpi_i$ should
commute to taking sections (which will imply indeed that if
\eqref{eq:127.1} is a quasi-isomorphism, then so is
\eqref{eq:127.2}). Thus, a first basic question here is to find a
suitable notion of a ``bundle'' over $k$, in such a way that for
semisimplicial bundles satisfying some mild extra assumption (such as
$X_0=X_1=e$) implying that $X_*$ is pseudoabelian, the maps
\eqref{eq:127.5} should be isomorphisms. I expect this to be true for
the notion we have been working with so far, namely for ``unipotent
bundles'' (cf.\ section~\ref{sec:111}), but I haven't made any attempt
yet to prove this. But even granting this property for a given notion
of ``bundles'', the module structure on the $\bpi_i$'s at this point
remains still a mystery, as long as we don't tie them in with the Lie
functor (compare section~\ref{sec:118})\dots

\begin{remark}
  Maybe, in the set-up of a general topos $T$ and taking sections on
  the latter, the maps \eqref{eq:127.5} are isomorphisms, whenever
  the\pspage{538} homotopy sheaves $\bpi_i$ satisfy the relations
  \begin{equation}
    \label{eq:127.6}
    \mathrm H^i(T, \bpi_j(X_*))=0 \quad
    \text{for $i>0$,}\tag{6}
  \end{equation}
  as a consequence, maybe, of some spectral sequence whose abutment is
  the graded homotopy of the cointegration of $X_*$ over $T$. If so,
  then \eqref{eq:127.5} are quasi-isomorphisms whenever the sheaves
  $\bpi_i$ can be endowed with the structure of a quasi-coherent
  $\scrO_k$-module.
\end{remark}

% 128
\hangsection[Looking for the right notion of ``bundles''; $V$-bundles
\dots]{Looking for the right notion of ``bundles'';
  \texorpdfstring{$V$}{V}-bundles versus
  \texorpdfstring{$W$}{W}-bundles.}\label{sec:128}%
Next requirement about the notion of ``bundles'' we are going to work
with, is about existence of a ``linearization functor''; associating
to every bundle $X$ another bundle, namely its ``linearization''
$L(X)$, endowed moreover with the structure of an
$\scrO_k$-module. \emph{We insist that for a given bundle $X$, $L(X)$
  should be at any rate a bundle too}. More specifically, if $U(k)$
denotes the category of all bundles over $k$, we should also introduce
a corresponding ``$k$-linear'' category, or more accurately an
$\scrO_k$-linear category, $U\subab(k)$, whose objects should be
objects of $U(k)$ endowed with the extra structure of an
$\scrO_k$-module, possibly subject to some restrictions -- thus, we'll
get a forgetful functor
\begin{equation}
  \label{eq:128.7}
  K: U\subab(k)\to U(k).\tag{7}
\end{equation}
When we take $U(k)$ to be unipotent bundles in the sense of
section~\ref{sec:111}, the evident choice for $U\subab(k)$ is to take
the category of quasicoherent $\scrO_k$-modules, equivalent to the
category of $k$-modules, $\AbOf_k$. Dually, we may take $U(k)$ to be
the category of sheaves over $k$ isomorphic to the underlying sheaf of
sets of a vector bundle $V(M)$ over $k$ associated to a $k$-module
$M$, by the requirement
\begin{equation}
  \label{eq:128.8}
  V(M)(k') = \Hom_\kMod(M,k').\tag{8}
\end{equation}
The evident corresponding choice for $U\subab(k)$ is to take the
category of all vector bundles over $k$, which is equivalent to the
category $\AbOf_k\op$ \emph{opposite} to the category of $k$-modules,
as the functor $M\mapsto V(M)$ is contravariant in $M$. Maybe we
should distinguish between these two choices of bundles by different
notations, namely
\begin{equation}
  \label{eq:128.9}
  \text{$U_W(k)$ and $U_V(k)$,}\tag{9}
\end{equation}
where the subscripts $W$ and $V$ are meant to suggest the standard
descriptions of objects via (the underlying sheaves of sets of the
$\scrO_k$-modules)
\[\text{$W(M)$ and $V(M)$}\]
respectively, where (we recall)
\begin{equation}
  \label{eq:128.10}
  W(M)(k')=M\otimes_k k'.\tag{10}
\end{equation}

Reverting\pspage{539} to a general notion of ``bundles'' $U(k)$, we
define now a linearization functor
\begin{equation}
  \label{eq:128.11}
  L: U(k)\to U\subab(k)\tag{11}
\end{equation}
to be a functor left adjoint to the forgetful functor
\eqref{eq:128.7}, i.e., giving rise to an adjunction isomorphism
\begin{equation}
  \label{eq:128.12}
  \Hom_{U(k)}(X,K(L)) \simeq \Hom_{U\subab(k)}(L(X),L),\tag{12}
\end{equation}
where $X$ is in $U(k)$, $L$ in $U\subab(k)$. Passing over to
semisimplicial objects and the corresponding derived categories, the
functors $K$ and $L$ in \eqref{eq:128.7} and \eqref{eq:128.11} should
give rise to the functors $\widetilde K$ and $\LH_\bullet$ of
section~\ref{sec:124}.

When we take $U(k)=U_W(k)$, there is a drawback, though, because (as
we saw in section~\ref{sec:115}) the functor $K$ \emph{does not} admit
a left adjoint, only a \emph{proadjoint}, associating to an object $X$
a \emph{proobject} $L(X)$ of $U\subab(k)$. As pointed out in
section~\ref{sec:124} (p.~\ref{p:522}), if we want a nice pair $(K,L)$
of adjoint functors, this forces us to work with pro-unipotent bundles
instead of just unipotent ones, thus getting out of the haven of
sheaves over $k$ and into the somewhat dubious sea of prosheaves and
semisimplicial prosheaves, which have not been provided for in
Illusie's foundational ponderings! If we do stick to the $W$-approach,
this promises us a fair amount of extra sweat, putting in proobjects
everywhere, not too enticing a prospect, is it?

It would seem that we are better off with the $V$-approach, in which
case the bundles (more accurately, $V$-bundles) we are working with
are actual schemes, indeed affine schemes over $k$, as we get
\begin{equation}
  \label{eq:128.13}
  V(M)\simeq \Spec(\Sym_k(M)).\tag{13}
\end{equation}
Now, let more generally $X$ be any scheme, and let's look at maps from
$X$ into any vector bundle $V(M)$, we get
\begin{equation}
  \label{eq:128.star}
  \begin{split}
    \Hom(X,V(M)) &\eqdef
    \Hom_{\scrO_X}(p^*(\widetilde M), \scrO_X) \\
    &\qquad\qquad\overset{\textup{adjunction}}{\simeq}
    \Hom_{\scrO_S}(\widetilde M, p_*(\scrO_X)),     
  \end{split}\tag{*} 
\end{equation}
where
\[p:X\to S=\Spec k\]
is the structural map of $X$, and $\widetilde M$ the restriction of
$W(M)$ to the usual Zariski site of $S=\Spec(k)$. It is well-known
that under a rather mild restriction on $X$ (namely $X$ quasi-compact
and quasi-separated), always satisfied when $X$ is affine, $p_*$ takes
quasi-coherent sheaves on $X$ (for the usual small Zariski site) into
quasi-coherent sheaves on $S=\Spec k$; thus, if\pspage{540} $A$ is the
$k$-module (a $k$-algebra as a matter of fact) such that
\[\widetilde A\simeq p_*(\scrO_X),\quad
\text{i.e., $A=\Gamma(S,p_*(\scrO_X))\simeq\Gamma(X,\scrO_X)$,}\]
the last member of \eqref{eq:128.star} is
\[\Hom_{\scrO_X}(\widetilde M,\widetilde A) \simeq \Hom_k(M,A) \simeq
\Hom_{\scrO_k}(V(A),V(M)),\]
and finally we get
\begin{equation}
  \label{eq:128.14}
  \Hom(X,V(M)) \simeq \Hom_{\scrO_k}(V(A),V(M)).\tag{14}
\end{equation}
This shows that the forgetful functor from the category ${U\subab}_V$
of all vector bundles over $k$ to the category of all $k$-schemes
which are quasi-compact and quasi-separated, admits a left adjoint
$L_V$, where
\begin{equation}
  \label{eq:128.15}
  L_V(X)=V(A), \quad
  \text{where $A=\Gamma(S,{p_X}_*(\scrO_X)) \simeq
    \Gamma(X,\scrO_X)$.}\tag{15} 
\end{equation}
In case $X$ is affine, $A$ is just the affine ring of $X$, which is a
$k$-algebra, from which we retain only (in formula \eqref{eq:128.15})
the $k$-module structure.

Thus, as far as the notion of $k$-linearization goes, $V$-bundles
behave in a considerably nicer way than $W$-bundles, without any need
to go over to proobjects. Thus, it may be preferable to work with
$V$-bundles rather than $W$-bundles. We may wonder at this point why
not admit, then, as ``bundles'' any $k$-scheme $X$ which is
quasi-compact and quasi-separated, or at any rate any affine
$k$-scheme, as these can be quite conveniently described in terms of
$k$-algebras. Thus, semisimplicial affine schemes over $k$ just
correspond to \emph{co}-semisimplicial commutative algebras over $k$,
and likewise for maps -- and linearization just corresponds to
forgetting the algebra structure in this co-semisimplicial object, and
retain the structure of a co-semisimplicial $k$-module, which
corresponds dually to a semisimplicial vector bundle. This is a
perfectly simple relationship -- why bother about restricting the
``bundles'' from arbitrary affine $k$-schemes to those which are
isomorphic to a vector bundle?

We should remember here our initial motivation, which was, in case
when $k=\bZ$, to get a category of ``schematic'' homotopy types as
close as possible to the usual one. One plausible way of achieving
this is by restricting the notion of a bundle the more we can, so as
to get still the possibility of ``schematization'' for a very sizable
bunch of ordinary homotopy types. Postnikov dévissage then suggested
to work with so-called ``unipotent bundles'', and it was
almost\pspage{541} a matter of chance 50/50 that we took first the
choice of using $W$-bundles, rather than $V$-bundles which are the
dual choice, giving rise in some respects to a simpler algebraic
formalism (and to a less satisfactory one in some others\dots). In
both cases, an instinct of ``economy'' is leading us. It isn't always
clear that instinct isn't misleading at times -- after all, it would
be nice too to have a so-called ``schematic'' homotopy type (and hence
a usual one) associated to rather general types of semisimplicial
schemes, say. But here already, if we want to get a usual homotopy
type just by taking sections, we've seen that this isn't so automatic,
that this is tied up with the expectation that the maps
\eqref{eq:127.5} should be isomorphisms, which presumably will not be
true unless we make rather drastic extra assumptions on the
``bundles'' we are working with.

Thus, a first main test whether the choice of $W$-bundles or of
$V$-bundles is a workable one, is to see whether this condition on
\eqref{eq:127.5} is satisfied, possibly under a suitable extra
assumption, such as $X_0=e$ or $X_0=X_1=e$. The next test, presumably
a deeper one, is whether these choices allow for a description of the
homotopy sheaves $\bpi_i(X_*)$ in terms of the Lie functor, under the
natural flatness assumption on the components of $X_*$ (compare
section~\ref{sec:118}). The only clue so far for such a relationship
comes from Postnikov dévissage, and this relationship isn't proved
either, except in the more or less tautological ``Postnikov case'',
and without naturality. In a way, it appears as a rather strange kind
of relationship, one which implies that the homotopy type of a
semisimplicial bundle is very strongly dominated (almost determined,
one might say) by its formal completion along a section -- and even by
the first-order infinitesimal neighborhood already. Instinct again
tells us that reducing, as it were, a scheme to a tangent space at one
of its points might make sens when the scheme is isomorphic to affine
$n$-space, and that it is surely nonsense for general affine schemes
(such as an elliptic curve minus a point say!). To say it differently,
we feel that such a thing may be reasonable only when the given
schemes $X_n$ may be thought of as ``homotopically trivial'' in some
sense or other, which for an algebraic curve, say, over an
algebraically closed field $k$ is surely \emph{not} the case, unless
precisely the curve is isomorphic to the affine line.

\bigbreak

\noindent\hfill\ondate{14.10.}\pspage{542}\par

% 129
\hangsection[Quasi-coherent homological quasi-isomorphisms, versus
\dots]{Quasi-coherent homological quasi-isomorphisms, versus weak
  equivalences.}\label{sec:129}%
I have not been clear enough in yesterday's notes, when introducing
the linearization functor \eqref{eq:128.11}
\[L: U(k)\to U\subab(k)\]
for a suitable notion of ``bundles'' and ``linear bundles'', that it
is by no means automatic that such a functor (even when its existence
is granted) will induce a linearization functor
\begin{equation}
  \label{eq:129.16}
  \LH_\bullet:\HotOf_0(k)\to \HotabOf_0(k)\tag{16}
\end{equation}
for the corresponding derived categories. In other words, it is by no
means clear (and we haven't even tried yet to prove in the $U_W$
set-up of ``unipotent bundles'') that if
\begin{equation}
  \label{eq:129.17}
  X_*\to Y_*\tag{17}
\end{equation}
is a weak equivalence of semisimplicial objects of $U(k)$, that the
corresponding map
\begin{equation}
  \label{eq:129.17prime}
  L(X_*)\to L(Y_*)\tag{17'}
\end{equation}
of semisimplicial objects in $U\subab(k)$ will again be a weak
equivalence. At any rate, for this statement even to make sense, we'll
have to make clear what notion of weak equivalence we are taking for
maps between semisimplicial objects of $U\subab(k)$, so as to get a
localized category $\HotabOf_0(k)$. The first obvious choice, of
course, that comes to mind is to take the notion of weak equivalence
for the corresponding semisimplicial sheaves of sets, which presumably
is going to be the right one. Thus, the existence of a ``total
homology'' or ``linearization'' functor \eqref{eq:129.16} may be
viewed as the ``schematic'' analog of W.~H.~C.~Whitehead's theorem in
the discrete set-up, which we had been puzzling about already in
section~\ref{sec:92} (in connection with replacing the test category
$\Simplex$ by a more or less arbitrary small category). In the case of
semisimplicial $W$-bundles or $V$-bundles, under some extra
assumption, presumably, such as $X_0=e$ or $X_0=X_1=e$ on the
semisimplicial objects we are working with, I do expect that the
linearization functor transforms weak equivalences into weak
equivalences, and that the converse holds true, too. This seems to me
to be the third main test (besides the two considered on previous
page~\ref{p:541}) about a notion of ``bundle'' being suited for
developing a theory of schematic homotopy types.

This suggests that when working with more general semisimplicial
sheaves over $k$, such as semisimplicial schemes, say, it may be
useful\pspage{543} to introduce a notion of ``quasi-coherent
homological quasi-isomorphism'' between such objects, as a map
\eqref{eq:129.17} such that the corresponding map
\eqref{eq:129.17prime} should be a quasi-isomorphism, which presumably
may be interpreted also in cohomological terms, as usual. As just
noticed, it is doubtful that this is implied by \eqref{eq:129.17}
being a weak equivalence, and even if it should be implied, it looks
considerably weaker in a way -- just as in the set-up of usual
homotopy types, a homology equivalence is a considerably weaker notion
than homotopy equivalence, unless we make a $1$-connectedness
assumption. We could reinforce, of course, this notion of
(quasi-coherent) ``homological'' or ``cohomological''
quasi-isomorphism by taking to the cohomological version of it, and
instead of quasi-coherent coefficients \emph{coming from the base}
$S=\Spec(k)$, admit equally the analog of ``\emph{twisted
  coefficients}'', which would amount here to taking as coefficients
quasi-coherent sheaves $F_*$ on $Y_*$, such that for any structural
morphism
\[\varphi:Y_m\to Y_n\]
associated to a map in $\Simplex$, the corresponding map of
quasi-coherent sheaves on $Y_m$
\[\varphi^*(F_n)\to F_m\]
should be an isomorphism. (These ``twisted coefficients'' should
correspond to quasi-coherent sheaves on $S=\Spec(k)$, \emph{on which
  the sheaf of groups $\bpi_1(Y_*)$ operates}, in case at least when
$Y_*$ is relatively $0$-connected and endowed with a section, so that
$\bpi_1(Y_*)$ makes sense.)\enspace My point here is that it may be
interesting to take the derived category of a suitable category of
semisimplicial schemes (submitted to some very mild conditions, such
as quasi-compactness and quasi-separation, say) with respect to this
notion of q.c.h.~quasi-isomorphism -- with the hope that the set of
maps with respect to which we are now localizing is wide enough, so as
to get \emph{the same} derived category as when working only with
``bundles'', namely getting just schematic homotopy types. This would
give a very strong link between more or less arbitrary semisimplicial
schemes over $k$ (and stronger still when $k=\bZ$), and ordinary
homotopy types, of a much subtler nature than the known link with
ordinary pro-homotopy types via étale cohomology with discrete
coefficients. But maybe the daydreaming is getting here so much out of
reach or maybe simply crazy, that I better stop along this
line!\pspage{544} 

% 130
\hangsection{A case for non-connected bundles.}\label{sec:130}%
In a more down-to-earth line, and reverting to the ``unipotent''
approach still (in either $W$- or $V$-version for ``unipotent
bundles''), I would still like to point out one rather mild extension
of the set-up as contemplated initially. The suggestion here is to
admit as components for our semisimplicial models not merely sheaves
of sets which are ``unipotent bundles'', but equally direct sums of
such. After making such extension, the linearization functor $L$
(\eqref{eq:128.11} p.~\ref{p:539}) still makes sense, provided
$U\subab(k)$ is stable under infinite sums (otherwise, we'd have to
restrict to a finite number of connected components for our
``bundles''). Presumably, once we get into this, we will have to admit
``twisted'' finite direct sums as well, in order to have basic notions
compatible with descent -- never mind such technicalities at the
present stage of reflections! Thus, applying componentwise (i.e., to
each component $X_n$) the \piz-functor (``connected components''), a
semisimplicial ``bundle'' $X_*$ in the wider sense gives rise to an
associated usual semisimplicial set, let's call it $\pi_{0/k}(X_*)$,
together with a map
\begin{equation}
  \label{eq:130.18}
  X_*\to \pi_{0/k}(X_*),\tag{18}
\end{equation}
where the second-hand side is interpreted as a semisimplicial constant
object over $k$. (For simplicity, we have assumed here $\Spec(k)$ to
be $0$-connected, and that the direct sums involved in the $X_n$'s are
not twisted\dots)\enspace Intuitively, we may interpret
\eqref{eq:130.18} as defining $X_*$ as a (``strict'', namely
componentwise connected) \emph{schematic} homotopy type, lying
``over'' the \emph{discrete} (or ``constant'') homotopy type
$\pi_{0/k}(X_*)=E_*$. The latter introduces homotopy invariants of its
own, which strictly speaking shouldn't be viewed as being of a
``schematic'' nature. Thus, when our exclusive emphasis is on studying
the ``strict'' schematic homotopy types, we'll restrict our models
$X_*$ by demanding that the associated discrete homotopy type
$\pi_{0/k}(X_*)$ should be \emph{aspheric}, i.e., isomorphic in
$\HotOf$ to a one-point space. Under this restriction, presumably,
working with those slightly more general models should give (up to
equivalence) the same derived category $\HotOf_0(k)$, as when working
with (connected) unipotent bundles. Allowing connected components may
prove useful for giving a little more ``elbow freedom'' in working
with models, as it allows for instance anodyne operations such as
taking direct sums. Thus, all constant semisimplicial objects will be
allowed, or at any rate those which correspond to aspheric
semisimplicial sets -- which includes notably the objects of
\Simplexhat{} represented\pspage{545} by the standard simplices
$\Simplex_n$; and it is useful indeed to be able to have these among
our models.

My motivation for suggesting to allow for connected components for our
bundles, comes from an attempt to perform the ``integration''
operation for an indexes family of schematic homotopy types, under
suitable assumptions -- one among these being that the indexing small
category $I$ should be aspheric (in order to meet the above
asphericity requirement on $E_*$, for the semisimplicial bundle
obtained). When just paraphrasing the construction of integration for
ordinary homotopy types (as indicated in section~\ref{sec:69} -- we're
going to come back upon this in a later chapter), we just cannot help
but run into a bunch of connected components, for the components of
the semisimplicial sheaf expressing the integrated type. The general
idea here, of course, is to perform the basic homotopy operations
(essentially, integration and cointegration operations) \emph{in the
  context of semisimplicial sheaves}, and then look and see whether
(or when) this doesn't take us out of the realm semisimplicial
``bundles''. I am sorry I didn't work out any clear-cut result along
these lines yet, and I am going to leave it at that for the time
being.

One serious drawback, however, when allowing for connected components
of the components $X_n$ or our semisimplicial bundles, is that we can
hardly expect anymore that the homotopy sheaves of $X_*$ may be
expressed in terms of tangent sheaves, as contemplated in
section~\ref{sec:118}. At any rate, this extension of our category of
``models'' would seem a reasonable one only if we are able to show
that we get the same derived category (up to equivalence) as when
working with the more restricted models, using as components $X_n$
only (connected) unipotent bundles. I didn't make any attempt either
to try and prove such a thing.

\noindent\hfill\ondate{16.9.}\par

% 131
\hangsection[Tentative description of the spherical functor
$\smash{\widetilde S}$, and \dots]{Tentative description of the spherical
  functor \texorpdfstring{$\widetilde S$}{S}, and ``infinitesimal''
  extension of the basic notion of ``bundles''.}\label{sec:131}%
Yesterday again, I didn't do any mathematics -- instead, I have been
writing a ten pages typed report on the preparation and use of kimchi,
the traditional Korean basic food of fermented vegetables, which I
have been practicing now for over six years. Very often friends ask me
for instructions for preparing kimchi, and a few times already I
promised to put it down in writing, which is done now. Besides this, I
wrote to Larry Breen to tell him a few words about my present
ponderings as he is the one person I would think of for whom my
rambling\pspage{546} reflections on schematization and on schematic
homotopy types may make sense.

Definitely, my suggestion in the last notes, to work with
non-connected ``bundles'', isn't much more than the reflection of my
inability so far to make a breakthrough and get the ``left'' homotopy
constructions in terms of semisimplicial (connected) unipotent bundles
alone. Besides the serious drawback already pointed out at the end of
section~\ref{sec:130}, another one occurred to me -- namely that with
the suggested extension, one is losing track of the extra condition
$X_0=e$ (or even $X_0=X_1=e$) on our models, which is often remaining
implicit in the notes, and which, however, is an important
restriction, actually needed for the kind of things we want to do. For
one thing, when we drop it, and even when we otherwise restrict to
components $X_n$ which are standard affine spaces $E_k^{d_n}$, say,
there is no hope of getting a Lie-type description for the homotopy
sheaves $\bpi_i(X_*)$ -- for instance, the Lie-type invariants we get
by using different sections of $X_0$ over $S=\Spec(k)$ are by no means
related (as the $\bpi_i$ should) by a transitive system of canonical
isomorphisms; this is easily seen already when taking the ``trivial''
semisimplicial bundle described by $X_n=X_0$ for all $n$ (the constant
functor with value $X_0$ from $\Simplexop$ to $U(k)$)! Of course, even
when allowing connected components for the ``bundles'' $X_n$, we may
formally still throw in the condition $X_0=e$ -- but this is cheating
and no use at all, if we remember that the main motivation for
allowing connected components was in order to be able (in suitable
cases) to perform an integration operation of schematic homotopy
types. But when following the standard construction, for the resulting
$X_*$ the component $X_0$ as well as all others will have lots of
connected components, and hence the condition $X_0=e$ will not hold
true. The same remark applies also to the ``constant'' semisimplicial
bundles (namely with components which are ``constant'' schemes over
$S$) defined by the standard simplices $\Simplex_n$. To sum up, while
it \emph{is} certainly quite useful to view semisimplicial
``bundles'', used for describing the ``schematic homotopy types'' we
are after, as particular cases of more general semisimplicial sheaves
on the fpqc site of $S=\Spec(k)$, we will probably have to be quite
careful in keeping the ``bundles'' we are working with restricted
enough, and not confuse our models for schematic homotopy types
(allowing for a nice description of the basic $\Lpi_\bullet$ and
$\LH_\bullet$ invariants) with more general semisimplicial sheaves
which may enter the picture in various ways.

I\pspage{547} would like, however, to suggest still another extension
of the notion of a ``bundle'', which maybe will prove something better
than just a random way out of embarrassment! It has to do with an
attempt to come to a description of one among the four ``basic
functors'' of section~\ref{sec:124}, namely the ``spherical'' functor
\begin{equation}
  \label{eq:131.1}
  \widetilde S: \HotabOf_0(k)\to\HotOf_0(k),\tag{1}
\end{equation}
which for the time being is remaining hypothetical, due notably to my
inability so far to carry out the suspension operation for schematic
homotopy types. There are two basic formal properties giving us some
clues about this functor, namely, it should be left adjoint to the
``total homotopy'' or ``Lie'' functor $\Lpi_\bullet$, and it should be
right inverse to the (reduced) total homology functor $\LtH_\bullet$
(cf.\ pages~\ref{p:523}--\ref{p:525}). Let's work for the sake of
definiteness, for describing $\HotOf_0(k)$, with semisimplicial
$V$-bundles $X_*$ satisfying the extra assumption $X_0=e$ (and
presumably a little more), corresponding therefore to
co-semisimplicial algebra $A^*$ satisfying $A^0=k$, such that the
components $A^n$ be isomorphic to symmetric algebras
\begin{equation}
  \label{eq:131.2}
  A^n\simeq \Sym_k(M^n),\tag{2}
\end{equation}
where we'll assume the modules $M^n$ (or equivalently, the algebras
$A^n$) \emph{flat} over $k$. The condition $A^0=k$ implies that the
algebras $A^n$ are $k$-augmented, and if $\mathfrak m^{(n)}$ is the
augmentation ideal, we get
\begin{equation}
  \label{eq:131.3}
  \mathfrak m^{(n)} / (\mathfrak m^{(n)})^2 \simeq M^n,\tag{3}
\end{equation}
thus, the modules $M^n$ may be viewed as the components of a
co-semisimplicial $k$-module $M^*$, giving rise to a semisimplicial
vector bundle $V(M^*)$ over $k$, and we may write
\begin{equation}
  \label{eq:131.4}
  \Lpi_\bullet(X_*) \simeq V(M^*),\tag{4}
\end{equation}
where the second member is viewed as a chain complex (rather than as a
semisimplicial module) in the usual way. Consider now an object $L_*$
in $\HotabOf_0(k)$, it is described in terms of a co-semisimplicial
$k$-module $L^*$ by the formula
\begin{equation}
  \label{eq:131.5}
  L_*=V(L^*),\tag{5}
\end{equation}
and we get now
\begin{equation}
  \label{eq:131.star}
  \Hom_{\scrO_k}(L_*,\Lpi_\bullet(X_*)) \simeq
  \Hom_{\scrO_k}(V(L^*),V(M^*)) \simeq
  \Hom_{\scrO_k}(M^*,L^*).\tag{*}
\end{equation}
Now, if $M$ is any $k$-module, let
\begin{equation}
  \label{eq:131.6}
  D(M) = \Sym_k(M)(1) \simeq k\oplus M\tag{6}
\end{equation}
be\pspage{548} the corresponding augmented $k$-algebra admitting $M$
as a square-zero augmentation ideal, and let
\begin{equation}
  \label{eq:131.7}
  I(M) = \Spec(D(M))\tag{7}
\end{equation}
its spectrum, which is a first-order infinitesimal neighborhood of
$S=\Spec(k)$. For variable $M$, $D(M)$ depends covariantly, $I(M)$
contravariantly on $M$. Thus,
\begin{equation}
  \label{eq:131.8}
  D(L^*)\tag{8}
\end{equation}
is a co-semisimplicial algebra, depending covariantly on $L^*$, hence
contravariantly on $L_*=V(L^*)$, and accordingly
\begin{equation}
  \label{eq:131.9}
  I(L^*) = \Spec(D(L^*))\tag{9}
\end{equation}
(where in both members $I$ and $\Spec$ are applied again
componentwise) is a semisimplicial affine scheme, indeed a first-order
infinitesimal one, depending contravariantly on $L^*$, hence
covariantly on $L_*$. Now, the last member of \eqref{eq:131.star} may
be interpreted non-linearly as
\begin{equation}
  \label{eq:131.starstar}
  \Hom_{\textup{co-ss $k$-alg}}(A^*,D(L^*))
  \simeq
  \Hom(I(L^*), X_*),\tag{**}
\end{equation}
where in the second term the $\Hom$ means homomorphisms of
semisimplicial schemes. Let's now write (with an obvious
afterthought!)
\begin{equation}
  \label{eq:131.10}
  \widetilde S(L_*) = I(L^*)\quad
  \text{whenever $L_*=V(L^*)$,}\tag{10}
\end{equation}
the sequence of isomorphisms \eqref{eq:131.star} and
\eqref{eq:131.starstar} may be summed up by
\begin{equation}
  \label{eq:131.11}
  \Hom(L_*,\Lpi_\bullet(X_*)) \simeq
  \Hom(\widetilde S(L_*),X_*),\tag{11} 
\end{equation}
where the first $\Hom$ means maps of semisimplicial $\scrO_k$-modules,
whereas the second is a $\Hom$ of semisimplicial schemes. It looks
very much like the adjunction formula (p.~\ref{p:524},
\hyperref[eq:124.7]{(7b)}) we are after, with however two big grains
of salt. The smaller one is that this formula doesn't take place in
derived categories, but rather in the categories of would-be
models. The considerably bigger grain of salt is that $\widetilde
S(L_*)$ isn't at all in our model category, its components are purely
infinitesimal, first-order schemes, and far from being vector bundles!
Essentially, what we have been using for getting the (admittedly quite
tautological) adjunction formula \eqref{eq:131.11} is that the Lie
functor on schemes-with-section over $S$ is representable in an
obvious way, namely by the scheme-with-section
\[I(k)=\Spec(k[T]/(T^2)),\]
which is a first-order infinitesimal scheme-with-section over $k$.

Our\pspage{549} tentative $\widetilde S$ functor in \eqref{eq:131.10}
has been constructed in the most evident way, in order to satisfy an
adjunction formula \eqref{eq:131.11}, valid on the level of
semisimplicial objects (and carrying over, hopefully, to the similar
adjunction formula for suitable derived categories). Next question is
then, what about the inversion formula
\begin{equation}
  \label{eq:131.12}
  \LtH_\bullet( \widetilde S(L_*)) \simeq L_*\text{?}\tag{12}
\end{equation}
The question makes sense, as $\LtH_\bullet$ is defined for any
semisimplicial affine scheme-with-section over $k$, or equivalently
for any co-semisimplicial augmented $k$-algebra $A^*$ over $k$, by
taking the augmentation ideal $\mathfrak m^*$ in the latter and
retaining only its linear (co-semisimplicial) structure. Keeping this
in mind, formula \eqref{eq:131.12} comes out indeed a tautology again!

The tentative description we just got is indeed of a most seducing
simplicity, as seducing indeed as the description of homotopy in terms
of the Lie functor, and closely related to the latter. It gives as a
particular case an exceedingly simple description of the sought-for
``spheres over $k$'' $S(k,n)$. But it is clear that this description
is liable to makes sense only at the price of suitably extending the
notion of a ``bundle'' we are working with, in a rather different
direction, I would say, from adding (or allowing) connected
components, as suggested in the previous section. Maybe we might view
it, though, as a kindred, but somewhat subtler extension of our
initial bundles, namely that we are now allowing, not a discrete
non-trivial ``set'' or $k$-scheme of connected components, but rather,
an infinitesimal one. More specifically, the suggestion which comes to
mind here, is to call now ``bundle'' over $k$ any scheme $X$ over $k$
admitting a subscheme
\[X_*\subset X\]
in such a way that $X_0$ should be a $V$-bundle (namely isomorphic to
a vector bundle) over $k$, and $X$ should be an infinitesimal
neighborhood of $X_0$, i.e., $X_0$ should be definable by a
quasi-coherent ideal on $X$ which is nilpotent. Equivalently, in terms
of the affine ring $A$ of $X$, we are demanding that $A$ should admit
a nilpotent ideal $J$ (which is of course not part of its structure),
such that $A/J$ should be isomorphic to a symmetric algebra over $k$
(with respect to some $k$-module $M$). Possibly, we may have moreover
to impose further flatness restrictions.

When working with this extended notion of bundles, there is no problem
for describing for the corresponding semisimplicial
models\pspage{550} the three functors $\LtH_\bullet$, $\widetilde K$,
$\widetilde S$. Indeed, as we just recalled, the first of the three
functors is well-defined and has an evident description for all
semisimplicial affine schemes over $k$. As for $\widetilde K$ and
$\widetilde S$, they are obtained in terms of a variable
co-semisimplicial $k$-module $L^*$ (representing the semisimplicial
vector bundle $L_*=V(L^*)$) by applying componentwise the functor
$\Sym_k$ and the first-order truncation $\Sym_k({-})(1)$, respectively
-- one may hardly imagine something simpler! This brings to my
attention that in terms of this description, we get a canonical
functorial map
\begin{equation}
  \label{eq:131.13}
  \widetilde S(L_*) \hookrightarrow \widetilde K(L_*)\tag{13}
\end{equation}
when working with the semisimplicial models, and hence presumably a
corresponding map for the functors between the relevant derived
categories $\HotabOf_0(k)$ and $\HotOf_0(k)$. Working either in the
model or in the derived categories, this map, as a matter of fact, may
be deduced from the basic formulaire of section~\ref{sec:124}, where
it had by then escaped my attention. Indeed, $\widetilde S$ is a left
adjoint of $\Lpi_\bullet$, and $\widetilde K$ a right adjoint of
$\LtH_\bullet$, to give such a map \eqref{eq:131.13} is equivalent
with giving either one of two maps
\begin{equation}
  \label{eq:131.14}
  \left\{
    \begin{tabular}{@{}lll@{}}
      a) & $L_*\to \Lpi_\bullet(\widetilde K(L_*))$ & \\
      b) & $\LtH_\bullet(\widetilde S(L_*)) \to L_*$ & ,
    \end{tabular}
  \right.\tag{14}
\end{equation}
and the formulaire provides for two such maps, namely the ``inversion
isomorphisms'' (\eqref{eq:124.9}, p.~\ref{p:525}). Thus, there is an
extra property which was forgotten in the formulaire, namely that the
two maps \eqref{eq:131.13} associated to the two inversion
isomorphisms should be the same. A nicer way, then, to state the
formulaire is to consider the map \eqref{eq:131.13} as a basic datum,
and say that the two maps in \eqref{eq:131.14} deduced from it by the
adjunction property should be isomorphisms. The situation is
reminiscent of the two ways by which we could obtain the Hurewicz map
(p.~\ref{p:525}, \ref{subsec:124.C}) -- presumably, the basic data for
the formulaire of section~\ref{sec:124} should be the functors
$\widetilde S$ and $\widetilde K$ and the map \eqref{eq:131.13}
between them, with the property that the relevant adjoint functors
$\Lpi_\bullet$ and $\LtH_\bullet$ exist, and that the corresponding
maps in \eqref{eq:131.14} should be isomorphisms, which then will
allow to define a \emph{unique} Hurewicz map
$\Lpi_\bullet\to\LtH_\bullet$.

As long as we are sticking to the purely formal aspect, and even when
working in the larger context of semisimplicial affine schemes over
$k$ satisfying merely $X_0=e$, or more generally still, dropping the
last restriction and taking ``$k$-pointed'' semisimplicial affine
schemes instead, the whole ``four functors formalism'' (including even
$\Lpi_\bullet$) as contemplated in section~\ref{sec:124} (and
with\pspage{551} the extra feature \eqref{eq:131.13} above as just
notice) goes over very smoothly, in an essentially tautological
way. As recalled on p.~\ref{p:547}, the $\Lpi_\bullet$-functor, when
interpreted on the co-semisimplicial side of the dualizing functor,
appears as a quotient of the $\LtH_\bullet$-functor, the latter
identified to the functor obtained by taking augmentation ideals of
co-semisimplicial algebras -- the quotient being obtained by dividing
out by the squares of the latter ideals. Dually, we get the Hurewicz
map for semisimplicial vector bundles, which is always an
inclusion. Again, imagine something simpler! The only trouble (but an
extremely serious one indeed!) is that in this general set-up, the
relation of the so-called $\Lpi_\bullet$-functor to homotopy
groups-or-sheaves becomes a very dim one. Definitely, the only firm
hope here is that the relationship between the two is OK (as
contemplated in section~\ref{sec:118}) whenever the components $X_n$
are actual flat vector bundles, satisfying moreover $X_0=e$ (at the
very least) -- plus possibly even some extra Kan type conditions
(sorry for the vagueness of even this one ``firm hope''!). If we take
already the ``next best'' set of assumptions, namely essentially that
the $X_n$ be flat ``bundles'' in the sense above (not necessarily
vector bundles, though), then the hoped-for relationship again seems
to vanish. The first case of interest, of course, is the case when
$X_*$ is of the form $\widetilde S(L_*)$, which includes (if our
$\widetilde S$ functor is ``the right one'' indeed) the $n$-spheres
over $k$. We get in this case (namely when $X_*$ is a first-order
neighborhood of the marked section) the trivial, and really stupid
relation
\[\Lpi_\bullet(\widetilde S(L_*)) \simeq L_*\quad\text{(!!!),}\]
which translates into: the homotopy groups of a sphere, computed in
the most naive ``Lie'' way, are canonically isomorphic to its homology
groups! Not much of a success\dots

% 132
\hangsection[A crazy tentative wrong-quadrant (bi)complex for \dots]{A
  crazy tentative wrong-quadrant
  \texorpdfstring{\textup(bi\textup)}{(bi)}complex for the homotopy
  groups of a sphere.}\label{sec:132}%
This makes it very clear that, while the functors $\widetilde K$,
$\widetilde S$, $\LtH_\bullet$ in our new context of semisimplicial
``bundles'' make perfectly good sense as they are, the $\Lpi_\bullet$
functor computed naively (taking tangent spaces) definitely doesn't,
except when actually working with flat (hence, essentially ``smooth'')
vector bundles as components of our semisimplicial models. This, after
all, shouldn't be too much of a surprise, if we remember the way
differentials and tangent spaces fit into a sweeping homology or
cohomology formalism. It has become quite familiar to people ``in the
know'' that taking the sheaf of $1$-differentials, say, or its dual,
or a sheaf of $1$-differentials or a tangent sheaf along a section,
behaves\pspage{552} as ``the'' good object in terms of homological
algebra and obstruction theory in various geometric situations,
\emph{only} in the case when the relative scheme ($X$ say) we are
working with is \emph{smooth} over the base scheme $S$ -- which in the
present case amounts to saying (when $X=X_n$ is a component of a
semisimplicial ``bundle'') that $X$ is indeed a flat vector bundle
over $S$. In more general cases, the work of
André-Quillen-Illusie\scrcomment{\textcite{Andre1967,Andre1974,Quillen1970,Illusie1971};
  see also \textcite{SGA6}\dots} tells us that the relevant object
which replaces $\Omega^1_{X/S}$ or its dual $\mathrm T_{X/S}$ is the
relative tangent or cotangent \emph{complex} $\mathrm L_\bullet^{X/S}$
or $\mathrm L^\bullet_{X/S}$, the second being the dual of the other
\begin{equation}
  \label{eq:132.15}
  \mathrm L^\bullet_{X/S} = \RbHom_{\scrO_X}(\mathrm L^{X/S}_\bullet,
  \scrO_X),\tag{15} 
\end{equation}
these objects being viewed, respectively, as objects in the derived
categories $\D_\bullet(\scrO_X)$ and $\D^\bullet(\scrO_X)$ (deduced
from chain and cochain complexes of $\scrO_X$-modules). When $X$ is
endowed with a section over $S$, the naive differentials and
codifferentials along this section should in the same way be replaced
by the co-Lie and Lie complexes
\begin{equation}
  \label{eq:132.16}
  \ell_\bullet(X/S,s) \quad\text{and}\quad
  \ell^\bullet(X/S,s)\simeq
  \RbHom(\ell_\bullet(X/S,s), \scrO_S),\tag{16}
\end{equation}
where $s$ is the given section, obtained from the previous complexes
by taking its inverse images $\mathrm Ls^*$ by $s$. As a matter of
fact, the chain complex $\mathrm L_\bullet^{X/S}$ can be realized
canonically, up to unique isomorphism, via a semisimplicial module
\[\mathrm L_*^{X/S}\]
on $X$, whose components are free $\scrO_X$-modules. Accordingly, we
get \eqref{eq:132.16} in terms of a well-defined semisimplicial
$\scrO_S$-module,
\begin{equation}
  \label{eq:132.17}
  \ell_*(X/S,s)\tag{17}
\end{equation}
whose components are free -- and as $S=\Spec(k)$, we may interpret
this more simply as a semisimplicial $k$-module with free
components. When we apply this to the components $X_n$ of a
semisimplicial bundle $X_*$, we get however the co-Lie invariants; in
order to get the relevant Lie invariants we'll have to take the duals
\begin{equation}
  \label{eq:132.17prime}
  \ell^*(X/S,s) = \Hom_k(\ell_*(X/S,s),k),\tag{17'}
\end{equation}
where $X$ is any one among the $X_n$'s, and $s$ its marked
section. Thus, the ``corrected'' description of $\Lpi_\bullet$, by
using the André-Quillen-Illusie version of the ``Lie-functor along a
section'', would seem to be
\begin{equation}
  \label{eq:132.18}
  \Lpi_\bullet(X_*) = \ell^*(X_*/k,e_*)\quad
  \text{(?),} \tag{18}
\end{equation}
where\pspage{553} now the second member appears as a mixed complex of
$k$-modules
\[(n,p) \mapsto \ell^p(X_n/k,e_n) : \Simplexop\times\Simplex \to
\AbOf_k,\]
contravariant with respect to the index $n$, covariant with respect to
$p$. Translating this via Kan-Dold-Puppe, we get a bicomplex of
$k$-modules, which we'll write in cohomological notation (with the two
partial differential operators of degree $+1$)
\[(C^{n,p}) = (C^{n,p}(X_*))\]
situated in the ``quadrant''
\[n\le 0, p\ge 0.\]
As we finally want an object of the derived category $\D(\AbOf_k)$ of
the category of $k$-modules, and even an object in the subcategory
$\D_\bullet(\AbOf_k)$, the evident thing that seems to be done now is
to take the associated simple complex, which hopefully may prove to be
the ``correct'' expression of the looked-for $\Lpi_\bullet$ -- and
this (if any) should be the precise meaning of \eqref{eq:132.18}.

The associations for getting \eqref{eq:132.18} are very tempting,
indeed, the expression we got makes us feel a little uneasy,
though. The main point is that the quadrant where our bicomplex lies
in is one of the two ``awkward'' ones, which implies that a)\enspace
for a given total degree, there are an infinity of summands occurring
(and one has to be careful, therefore, if these should be ``summands''
indeed, or rather ``factors'', namely if we should take an infinite
direct sum, or an infinite product instead); and b)\enspace the total
complex will have components of any degree both positive and negative,
and it isn't clear at all that it should be (as an object of
$\D(\AbOf_k)$) of the nature of a chain complex, namely that its
cohomology modules vanish for (total) degree $d>0$. If it should turn
out that this is not so (I didn't yet check any particular case), this
would imply for the least that \eqref{eq:132.18} should be corrected,
by taking the relevant truncation of the second-hand side.

Working with the $\mathrm L^*_{X/S}$ and $\ell^*(X/S,s)$ invariants
brings in a slightly awkward feature of its own which we have been
silent about, namely (except under suitable finiteness conditions) it
brings in non-quasi-coherent $\scrO_X$ or $\scrO_S$-modules. This may
encourage us to dualize \eqref{eq:132.18}, which will amount to
working with the co-semisimplicial algebra $A^*$ expressing $X_*$, and
taking componentwise the reduced (via augmentations) André-Quillen
complexes (rather than their duals). At any rate,\pspage{554} the
would-be expression of ``total co-homotopy'' of $X_*$ we'll get this
way isn't so much more appealing than \eqref{eq:132.18} -- it lies
still in one of the wrong quadrants, which definitely makes us feel
uncomfortable.

In principle, the tentative formula \eqref{eq:132.18}, when applied
say to an object such as
\[S(n,k) \eqdef \widetilde S(k[n]),\]
gives a rather explicit (but for the time being highly hypothetical!)
expression of the homotopy modules of the $n$-sphere over $k$, which
in case $k=\bZ$ are hoped to be just the homotopy groups of the
ordinary $n$-sphere. To test whether this makes at all sense, we'll
have to understand first the structure of the André-Quillen ``Lie
complex'' of an algebra \eqref{eq:131.6} of the type $D(M)$, for
variable $k$-module $M$. I haven't started looking into this yet, and
I doubt I am going to do it presently.

At any rate, whether or not the formula \eqref{eq:132.18} we ended up
with is essentially correct, in order (among other things) to get a
method for computing the total homotopy $\Lpi_\bullet$ for a
semisimplicial bundle $X_*$ that isn't a flat vector bundle, we'll
have to find out some more or less explicit means of replacing $X_*$
by some $X_*'$ whose components \emph{are} flat vector bundles, and
which is isomorphic to $X_*$ in the relevant derived category
$\HotOf_0(k)$. When $k$ isn't a field, the question arises already
even when the components $X_*$ \emph{are} vector bundles, when these
are not flat. The first idea that comes to mind, from the
André-Quillen theory precisely, in terms of the co-semisimplicial
algebra $A^*$ expressing $X_*$, is to take ``projective resolutions''
of the various components $A^n$ by polynomial algebras. Again we end
up with a mixed complex, this time a complex of augmented algebras
depending on two indices $n,p$, covariant in $n$ and contravariant in
$p$, or the reverse if we replace those algebras $A^n_p$ by their
spectra $X_n^p$. In any case, we again end up in a ``wrong quadrant''!

The hesitating question that comes to mind now is whether it is at all
feasible to work with a category of models which isn't a category of
\emph{semisimplicial} bundles say, but one of such \emph{mixed
  wrong-quadrant} bundles; namely, use these as ``models'' for getting
hold of a reasonable derived category of ``schematic homotopy types''?
I never heard of anything such yet, and I confess that at this point
my (anyhow rather poor!) formal intuition of the situation breaks down
completely -- maybe the suggestion is complete nonsense, for some
wholly trivial reason! Maybe Larry Breen could tell me at once -- or
someone else who has more feeling than I for semisimplicial and
cosemisimplicial models?

%%% Local Variables:
%%% mode: latex
%%% TeX-master: "main.tex"
%%% End:
